\documentclass[oneside,12pt]{book}
\usepackage[german]{babel}%required, alt.: \usepackage{german}

% comment (delete) the following five lines for a non-fraktur version
\usepackage{yfonts}%required for fraktur version
\usepackage{soul}%recommended
\newenvironment{antiqua}{\normalfont}{}%
\protect\renewcommand{\thepage}{\textfrak{\arabic{page}}}%
\newcommand{\s}{s:}%

% un-comment the following five lines for a non-fraktur version
%\newcommand{\s}{s}
%\newcommand{\so}[1]{\textbf{#1}}
%\newcommand{\frakfamily}{\null}
%\newenvironment{antiqua}{}{}
%\sloppy

\begin{document}
\thispagestyle{empty}

\begin{verbatim}

The Project Gutenberg EBook of Frau Bovary, by Gustave Flaubert

This eBook is for the use of anyone anywhere at no cost and with
almost no restrictions whatsoever.  You may copy it, give it away or
re-use it under the terms of the Project Gutenberg License included
with this eBook or online at www.gutenberg.net


Title: Frau Bovary

Author: Gustave Flaubert

Translator: Arthur Schurig

Release Date: April 26, 2005 [EBook #15711]

Language: German

Character set encoding: TeX

*** START OF THIS PROJECT GUTENBERG EBOOK FRAU BOVARY ***




Produced by Gunter Hille, K.F. Greiner and the Online
Distributed Proofreading Team.




\end{verbatim}


\newpage
\frakfamily

\thispagestyle{empty}

\bigskip\bigskip\bigskip\bigskip\bigskip\bigskip
\begin{center}
\Huge \so{Frau Bovary}

\bigskip\bigskip\bigskip\bigskip\bigskip\bigskip
\large

\so{von} \\
\bigskip\bigskip\bigskip\bigskip
\Huge \so{Gu{st}ave Flaubert} \\
\bigskip\bigskip\bigskip\bigskip\bigskip
\large
\vfill
Die "Ubertragung de{\s} Roman{\s} \begin{antiqua}Madame
Bovary\end{antiqua} au{\s} dem Franz"osischen besorgte Arthur
Schurig. \\ \bigskip\bigskip
Insel-Verlag zu Leipzig
\end{center}
\newpage
\thispagestyle{empty}


\begin{center}
\vspace{5cm}
{\Huge \so{Er{st}e{\s} Bu{ch}}}
\end{center}


\newpage\begin{center}
{\large \so{Er{st}e{\s} Kapitel}}\bigskip\bigskip
\end{center}

E{\s} war Arbeit{\s}stunde. Da trat der Rektor ein, ihm zur Seite
ein "`Neuer"', in gew"ohnlichem Anzuge. Der Pedell hinter den
beiden, Schulstubenger"at in den H"anden. Alle Sch"uler erhoben
sich von ihren Pl"atzen, wobei man so tat, al{\s} sei man au{\s}
seinen Studien aufgescheucht worden. Wer eingenickt war, fuhr mit
auf.

Der Rektor winkte ab. Man setzte sich wieder hin. Darauf wandte er
sich zu dem die Aufsicht f"uhrenden Lehrer.

"`Herr Roger!"' lispelte er. "`Diesen neuen Z"ogling hier empfehle
ich Ihnen besonder{\s}. Er kommt zun"achst in die Quinta. Bei
l"oblichem Flei"s und Betragen wird er aber in die Quarta versetzt,
in die er seinem Alter nach geh"ort."'

Der Neuling blieb in dem Winkel hinter der T"ure stehen. Man
konnte ihn nicht ordentlich sehen, aber offenbar war er ein
Bauernjunge, so ungef"ahr f"unfzehn Jahre alt und gr"o"ser al{\s}
alle andern. Die Haare trug er mit Simpelfransen in die Stirn
hinein, wie ein Dorfschulmeister. Sonst sah er gar nicht dumm
au{\s}, nur war er h"ochst verlegen. So schm"achtig er war,
beengte ihn sein gr"uner Tuchrock mit schwarzen Kn"opfen doch
sichtlich, und durch den Schlitz in den "Armelaufschl"agen
schimmerten rote Handgelenke hervor, die zweifello{\s} die freie
Luft gew"ohnt waren. Er hatte gelbbraune, durch die Tr"ager
"uberm"a"sig hochgezogene Hosen an und blaue Str"umpfe. Seine
Stiefel waren derb, schlecht gewichst und mit N"ageln beschlagen.

Man begann die fertigen Arbeiten vorzulesen. Der Neuling h"orte
aufmerksamst zu, mit wahrer Kirchenandacht, wobei er e{\s} nicht
einmal wagte, die Beine "ubereinander zu schlagen noch den
Ellenbogen aufzust"utzen. Um zwei Uhr, al{\s} die Schulglocke
l"autete, mu"ste ihn der Lehrer erst besonder{\s} auffordern, ehe
er sich den andern anschlo"s.

E{\s} war in der Klasse Sitte, beim Eintritt in da{\s}
Unterricht{\s}zimmer die M"utzen wegzuschleudern, um die H"ande
frei zu bekommen. E{\s} kam darauf an, seine M"utze gleich von der
T"ur au{\s} unter die richtige Bank zu facken, wobei sie unter
einer t"uchtigen Staubwolke laut aufklatschte. Da{\s} war so
Schuljungenart.

Sei e{\s} nun, da"s ihm diese{\s} Verfahren entgangen war oder
da"s er nicht gewagt hatte, e{\s} ebenso zu machen, kurz und gut:
al{\s} da{\s} Gebet zu Ende war, hatte der Neuling seine M"utze
noch immer vor sich auf den Knien. Da{\s} war ein wahrer
Wechselbalg von Kopfbedeckung. Bestandteile von ihr erinnerten an
eine B"arenm"utze, andre an eine Tschapka, wieder andre an einen
runden Filzhut, an ein Pelzbarett, an ein wollne{\s} K"appi, mit
einem Worte: an allerlei armselige Dinge, deren stumme
H"a"slichkeit tiefsinnig stimmt wie da{\s} Gesicht eine{\s}
Bl"odsinnigen. Sie war eif"ormig, und Fischbeinst"abchen verliehen
ihr den inneren Halt; zu unterst sah man drei runde W"ulste,
dar"uber (voneinander durch ein rote{\s} Band getrennt) Rauten
au{\s} Samt und Kaninchenfell und zu oberst eine Art Sack, den ein
vieleckiger Pappdeckel mit kunterbunter Schnurenstickerei kr"onte
und von dem herab an einem ziemlich d"unnen Faden eine kleine
goldne Troddel hing. Diese Kopfbedeckung war neu, wa{\s} man am
Glanze de{\s} Schirme{\s} erkennen konnte.

"`Steh auf!"' befahl der Lehrer.

Der Junge erhob sich. Dabei entglitt ihm sein Turban, und die
ganze Klasse fing an zu kichern. Er b"uckte sich, da{\s}
M"utzenunget"um aufzuheben. Ein Nachbar stie"s mit dem Ellenbogen
daran, so da"s e{\s} wiederum zu Boden fiel. Ein abermalige{\s}
Sich-darnach-b"ucken.

"`Leg doch deinen Helm weg!"' sagte der Lehrer, ein Witzbold.

Da{\s} schallende Gel"achter der Sch"uler brachte den armen Jungen
g"anzlich au{\s} der Fassung, und nun wu"ste er gleich gar nicht,
ob er seinen "`Helm"' in der Hand behalten oder auf dem Boden
liegen lassen oder aufsetzen sollte. Er nahm Platz und legte die
M"utze "uber seine Knie.

"`Steh auf!"' wiederholte der Lehrer, "`und sag mir deinen Namen!"'

Der Neuling stotterte einen unverst"andlichen Namen her.

"`Noch mal!"'

Da{\s}selbe Silbengestammel machte sich h"orbar, von dem Gel"achter
der Klasse "ubert"ont.

"`Lauter!"' rief der Lehrer. "`Lauter!"'

Nunmehr nahm sich der Neuling fest zusammen, ri"s den Mund weit
auf und gab mit voller Lungenkraft, al{\s} ob er jemanden rufen
wollte, da{\s} Wort von sich: "`Kabovary!"'

H"ollenl"arm erhob sich und wurde immer st"arker; dazwischen
gellten Rufe. Man br"ullte, heulte, gr"olte wieder und wieder:
"`Kabovary! Kabovary!"' Nach und nach verlor sich der Spektakel in
vereinzelte{\s} Brummen, kam m"uhsam zur Ruhe, lebte aber in den
Bankreihen heimlich weiter, um da und dort pl"otzlich al{\s}
halberstickte{\s} Gekicher wieder aufzukommen, wie eine Rakete,
die im Verl"oschen immer wieder noch ein paar Funken spr"uht.

W"ahrenddem ward unter einem Hagel von Strafarbeiten die Ordnung
in der Klasse allm"ahlich wiedergewonnen, und e{\s} gelang dem
Lehrer, den Namen "`Karl Bovary"' fest\/zustellen, nachdem er sich
ihn hatte diktieren, buchstabieren und dann noch einmal im ganzen
wiederholen lassen. Al{\s}dann befahl er dem armen Schelm, sich
auf die Strafbank dicht vor dem Katheder zu setzen. Der Junge
wollte den Befehl au{\s}f"uhren, aber kaum hatte er sich in Gang
gesetzt, al{\s} er bereit{\s} wieder stehen blieb.

"`Wa{\s} suchst du?"' fragte der Lehrer.

"`Meine M"u..."', sagte er sch"uchtern, indem er mit scheuen
Blicken Umschau hielt.

"`F"unfhundert Verse die ganze Klasse!"'

Wie da{\s} \begin{antiqua}Quos ego\end{antiqua} b"andigte
die Stimme, die diese Worte w"utend au{\s}rief, einen neuen
Sturm im Entstehen.

"`Ich bitte mir Ruhe au{\s}!"' fuhr der emp"orte Schulmeister
fort, w"ahrend er sich mit seinem Taschentuche den Schwei"s von
der Stirne trocknete. "`Und du, du Rekrut du, du schreibst mir
zwanzigmal den Satz auf: \begin{antiqua}Ridiculus
sum\end{antiqua}!"' Sein Zorn lie"s nach. "`Na, und deine M"utze
wirst du schon wiederfinden. Die hat dir niemand gestohlen."'

Alle{\s} ward wieder ruhig. Die K"opfe versanken in den Heften,
und der Neuling verharrte zwei Stunden lang in musterhafter
Haltung, obgleich ihm von Zeit zu Zeit mit einem Federhalter
abgeschwuppte kleine Papierkugeln in{\s} Gesicht flogen. Er wischte
sich jede{\s}mal mit der Hand ab, ohne sich weiter zu bewegen noch
die Augen aufzuschlagen.

Abend{\s}, im Arbeit{\s}saal, holte er seine "Armelschoner au{\s}
seinem Pult, brachte seine Habseligkeiten in Ordnung und liniierte
sich sorgsam sein Schreibpapier. Die andern beobachteten, wie er
gewissenhaft arbeitete; er schlug alle W"orter im W"orterbuche
nach und gab sich viel M"uhe. Zweifello{\s} verdankte er e{\s} dem
gro"sen Flei"se, den er an den Tag legte, da"s man ihn nicht in
der Quinta zur"uckbehielt; denn wenn er auch die Regeln ganz
leidlich wu"ste, so verstand er sich doch nicht gewandt
au{\s}zudr"ucken. Der Pfarrer seine{\s} Heimatdorfe{\s} hatte ihm
kaum ein bi"schen Latein beigebracht, und au{\s} Sparsamkeit war
er von seinen Eltern so sp"at wie nur m"oglich auf da{\s}
Gymnasium geschickt worden.

Sein Vater, Karl Diony{\s} Barthel Bovary, war Stab{\s}arzt a.D.;
er hatte sich um 1812 bei den Au{\s}hebungen etwa{\s} zuschulden
kommen lassen, worauf er den Abschied nehmen mu"ste. Er setzte
nunmehr seine k"orperlichen Vorz"uge in bare M"unze um und
ergatterte sich im Handumdrehen eine Mitgift von sechzigtausend
Franken, die ihm in der Person der Tochter eine{\s} Hutfabrikanten
in den Weg kam. Da{\s} M"adchen hatte sich in den h"ubschen Mann
verliebt. Er war ein Schweren"oter und Prahlhan{\s}, der
sporenklingend einherstolzierte, Schnurr- und Backenbart trug, die
H"ande voller Ringe hatte und in seiner Kleidung auff"allige
Farben liebte. Neben seinem Haudegentum besa"s er da{\s} gewandte
Getue eine{\s} Ellenreiter{\s}. Sobald er verheiratet war, begann
er zwei, drei Jahre auf Kosten seiner Frau zu leben, a"s und trank
gut, schlief bi{\s} in den halben Tag hinein und rauchte au{\s}
langen Porzellanpfeifen. Nacht{\s} pflegte er sehr sp"at
heimzukommen, nachdem er sich in Kaffeeh"ausern herumgetrieben
hatte. Al{\s} sein Schwiegervater starb und nur wenig hinterlie"s,
war Bovary emp"ort dar"uber. Er "ubernahm die Fabrik, b"u"ste aber
Geld dabei ein, und so zog er sich schlie"slich auf da{\s} Land
zur"uck, wovon er sich goldne Berge ertr"aumte. Aber er verstand
von der Landwirtschaft auch nicht mehr al{\s} von der Hutmacherei,
ritt lieber spazieren, al{\s} da"s er seine Pferde zur Arbeit
einspannen lie"s, trank seinen Apfelwein flaschenweise selber,
anstatt ihn in F"assern zu verkaufen, lie"s da{\s} fetteste
Gefl"ugel in den eignen Magen gelangen und schmierte sich mit dem
Speck seiner Schweine seine Jagdstiefel. Auf diesem Wege sah er zu
guter Letzt ein, da"s e{\s} am tunlichsten f"ur ihn sei, sich in
keinerlei Gesch"afte mehr einzulassen.

F"ur zweihundert Franken Jahre{\s}pacht mietete er nun in einem
Dorfe im Grenzgebiete von Caux und der Pikardie ein Grundst"uck,
halb Bauernhof, halb Herrenhau{\s}. Dahin zog er sich zur"uck,
f"unfundvierzig Jahre alt, mit Gott und der Welt zerfallen, gallig
und mi"sg"unstig zu jedermann. Von den Menschen angeekelt, wie er
sagte, wollte er in Frieden f"ur sich hinleben.

Seine Frau war dereinst toll verliebt in ihn gewesen. Aber unter
tausend Dem"utigungen starb ihre Liebe doch rettung{\s}lo{\s}.
Ehedem heiter, mitteilsam und herzlich, war sie allm"ahlich (just
wie sich abgestandner Wein zu Essig wandelt) m"urrisch, z"ankisch
und nerv"o{\s} geworden. Ohne zu klagen, hatte sie viel gelitten,
wenn sie immer wieder sah, wie ihr Mann hinter allen Dorfdirnen
her war und abend{\s} m"ude und nach Fusel stinkend au{\s}
irgendwelcher Spelunke zu ihr nach Hau{\s} kam. Ihr Stolz hatte
sich zun"achst m"achtig geregt, aber schlie"slich schwieg sie,
w"urgte ihren Grimm in stummem Stoizi{\s}mu{\s} hinunter und
beherrschte sich bi{\s} zu ihrem letzten St"undlein. Sie war
unabl"assig t"atig und immer auf dem Posten. Sie war e{\s}, die zu
den Anw"alten und Beh"orden ging. Sie wu"ste, wenn Wechsel f"allig
waren; sie erwirkte ihre Verl"angerung. Sie machte alle
Hau{\s}arbeiten, n"ahte, wusch, beaufsichtigte die Arbeiter und
f"uhrte die B"ucher, w"ahrend der Herr und Gebieter sich um
nicht{\s} k"ummerte, au{\s} seinem Zustande grie{\s}gr"amlicher
Schl"afrigkeit nicht herau{\s}kam und sich h"ochsten{\s} dazu
ermannte, seiner Frau garstige Dinge zu sagen. Meist hockte er am
Kamin, qualmte und spuckte ab und zu in die Asche.

Al{\s} ein Kind zur Welt kam, mu"ste e{\s} einer Amme gegeben
werden; und al{\s} e{\s} wieder zu Hause war, wurde da{\s}
schw"achliche Gesch"opf grenzenlo{\s} verw"ohnt. Die Mutter
n"ahrte e{\s} mit Zuckerzeug. Der Vater lie"s e{\s} barfu"s
herumlaufen und meinte h"ochst weise obendrein, der Kleine k"onne
eigentlich ganz nackt gehen wie die Jungen der Tiere. Im Gegensatz
zu den Bestrebungen der Mutter hatte er sich ein bestimmte{\s}
m"annliche{\s} Erziehung{\s}ideal in den Kopf gesetzt, nach
welchem er seinen Sohn zu modeln sich M"uhe gab. Er sollte rauh
angefa"st werden wie ein junger Spartaner, damit er sich t"uchtig
abh"arte. Er mu"ste in einem ungeheizten Zimmer schlafen, einen
ordentlichen Schluck Rum vertragen und auf den "`kirchlichen
Klimbim"' schimpfen. Aber der Kleine war von friedfertiger Natur
und widerstrebte allen diesen Bem"uhungen. Die Mutter schleppte
ihn immer mit sich herum. Sie schnitt ihm Pappfiguren au{\s} und
erz"ahlte ihm M"archen; sie unterhielt sich mit ihm in endlosen
Selbstgespr"achen, die von schwerm"utiger Fr"ohlichkeit und
wortreicher Z"artlichkeit "uberquollen. In ihrer Verlassenheit
pflanzte sie in da{\s} Herz ihre{\s} Jungen alle ihre eigenen
unerf"ullten und verlorenen Sehns"uchte. Im Traume sah sie ihn
erwachsen, hochangesehen, sch"on, klug, al{\s} Beamten beim
Stra"sen- und Br"uckenbau oder in einer Rat{\s}stellung. Sie
lehrte ihn Lesen und brachte ihm sogar an dem alten Klavier,
da{\s} sie besa"s, da{\s} Singen von ein paar Liedchen bei. Ihr
Mann, der von gelehrten Dingen nicht viel hielt, bemerkte zu
alledem, e{\s} sei blo"s schade um die M"uhe; sie h"atten doch
niemal{\s} die Mittel, den Jungen auf eine h"ohere Schule zu
schicken oder ihm ein Amt oder ein Gesch"aft zu kaufen. Zu wa{\s}
auch? Dem Kecken geh"ore die Welt! Frau Bovary schwieg still, und
der Kleine trieb sich im Dorfe herum. Er lief mit den
Feldarbeitern hinau{\s}, scheuchte die Kr"ahen auf, schmauste
Beeren an den Rainen, h"utete mit einer Gerte die Truth"ahne und
durchstreifte Wald und Flur. Wenn e{\s} regnete, spielte er unter
dem Kirchenportal mit kleinen Steinchen, und an den Feiertagen
best"urmte er den Kirchendiener, die Glocken l"auten zu d"urfen.
Dann h"angte er sich mit seinem ganzen Gewicht an den Strang der
gro"sen Glocke und lie"s sich mit emporziehen. So wuch{\s} er auf
wie eine Lilie auf dem Felde, bekam kr"aftige Glieder und frische
Farben.

Al{\s} er zw"olf Jahre alt geworden war, setzte e{\s} seine Mutter
durch, da"s er endlich etwa{\s} Gescheite{\s} lerne. Er bekam
Unterricht beim Pfarrer, aber die Stunden waren so kurz und so
unregelm"a"sig, da"s sie nicht viel Erfolg hatten. Sie fanden
statt, wenn der Geistliche einmal gar nicht{\s} ander{\s} zu tun
hatte, in der Sakristei, im Stehen, in aller Hast in den Pausen
zwischen den Taufen und Begr"abnissen. Mitunter, wenn er keine
Lust hatte au{\s}zugehen, lie"s der Pfarrer seinen Sch"uler nach
dem Ave-Maria zu sich holen. Die beiden sa"sen dann oben im
St"ubchen. M"ucken und Nachtfalter tanzten um die Kerze; aber
e{\s} war so warm drin, da"s der Junge schl"afrig wurde, und e{\s}
dauerte nicht lange, da schnarchte der biedere Pfarrer, die H"ande
"uber dem Schmerbauche gefaltet. E{\s} kam auch vor, da"s der
Seelensorger auf dem Heimwege von irgendeinem Kranken in der
Umgegend, dem er da{\s} Abendmahl gereicht hatte, den kleinen
Vagabunden im Freien erwischte; dann rief er ihn heran, hielt ihm
eine viertelst"undige Strafpredigt und benutzte die Gelegenheit,
ihn im Schatten eine{\s} Baume{\s} seine Lektion hersagen zu
lassen. Entweder war e{\s} der Regen, der den Unterricht st"orte,
oder irgendein Bekannter, der vor"uberging. "Ubrigen{\s} war der
Lehrer durchweg mit seinem Sch"uler zufrieden, ja er meinte sogar,
der "`junge Mann"' habe ein gar treffliche{\s} Ged"achtni{\s}.

So konnte e{\s} nicht weitergehen. Frau Bovary ward energisch, und
ihr Mann gab widerstand{\s}lo{\s} nach, vielleicht weil er sich
selber sch"amte, wahrscheinlicher aber au{\s} Ohnmacht. Man wollte
nur noch ein Jahr warten; der Junge sollte erst gefirmelt werden.

Dar"uber hinau{\s} verstrich abermal{\s} ein halbe{\s} Jahr, dann
aber wurde Karl wirklich auf da{\s} Gymnasium nach Rouen
geschickt. Sein Vater brachte ihn selber hin. Da{\s} war Ende
Oktober.

Die meisten seiner damaligen Kameraden werden sich kaum noch
deutlich an ihn erinnern. Er war ein ziemlich phlegmatischer
Junge, der in der Freizeit wie ein Kind spielte, in den
Arbeit{\s}stunden eifrig lernte, w"ahrend de{\s} Unterricht{\s}
aufmerksam dasa"s, im Schlafsaal vorschrift{\s}m"a"sig schlief und
bei den Mahlzeiten ordentlich zulangte. Sein Verkehr au"serhalb
der Schule war ein Eisengro"sh"andler in der Handschuhmachergasse,
der aller vier Wochen einmal mit ihm au{\s}ging, an Sonntagen nach
Ladenschlu"s. Er lief mit ihm am Hafen spazieren, zeigte ihm die
Schiffe und brachte ihn abend{\s} um sieben Uhr vor dem Abendessen
wieder in da{\s} Gymnasium. Jeden Donnerstag abend schrieb Karl
mit roter Tinte an seine Mutter einen langen Brief, den er immer
mit drei Oblaten zuklebte. Hernach vertiefte er sich wieder in
seine Geschicht{\s}hefte, oder er la{\s} in einem alten Exemplar
von Barthelemy{\s} "`Reise de{\s} jungen Anacharsi{\s}"', da{\s}
im Arbeit{\s}saal herumlag. Bei Au{\s}fl"ugen plauderte er mit dem
Pedell, der ebenfall{\s} vom Lande war.

Durch seinen Flei"s gelang e{\s} ihm, sich immer in der Mitte der
Klasse zu halten; einmal errang er sich sogar einen Prei{\s} in
der Naturkunde. Aber gegen Ende de{\s} dritten Schuljahre{\s}
nahmen ihn seine Eltern vom Gymnasium fort und lie"sen ihn Medizin
studieren. Sie waren der festen Zuversicht, da"s er sich bi{\s}
zum Staat{\s}examen schon durchw"urgen w"urde.

Die Mutter mietete ihm ein St"ubchen, vier Stock hoch, nach der
Eau-de-Robec zu gelegen, im Hause eine{\s} F"arber{\s}, eine{\s}
alten Bekannten von ihr. Sie traf Vereinbarungen "uber die
Verpflegung ihre{\s} Sohne{\s}, besorgte ein paar M"obelst"ucke,
einen Tisch und zwei St"uhle, wozu sie von zu Hause noch eine
Bettstelle au{\s} Kirschbaumholz kommen lie"s. De{\s} weiteren
kaufte sie ein Kanonen"ofchen und einen kleinen Vorrat von Holz,
damit ihr armer Junge nicht frieren sollte. Acht Tage darnach
reiste sie wieder heim, nachdem sie ihn tausend- und
abertausendmal ermahnt hatte, ja h"ubsch flei"sig und solid zu
bleiben, sintemal er nun ganz allein auf sich selbst angewiesen sei.

Vor dem Verzeichni{\s} der Vorlesungen auf dem schwarzen Brette
der medizinischen Hochschule vergingen dem neubackenen Studenten
Augen und Ohren. Er la{\s} da von anatomischen und pathologischen
Kursen, von Kollegien "uber Physiologie, Pharmazie, Chemie,
Botanik, Therapeutik und Hygiene, von Kursen in der Klinik, von
praktischen "Ubungen usw. Alle diese vielen Namen, "uber deren
Herkunft er sich nicht einmal klar war, standen so recht vor ihm
wie geheimni{\s}volle Pforten in da{\s} Heiligtum der
Wissenschaft.

Er lernte gar nicht{\s}. So aufmerksam er auch in den Vorlesungen
war, er begriff nicht{\s}. Um so mehr b"uffelte er. Er schrieb
flei"sig nach, vers"aumte kein Kolleg und fehlte in keiner "Ubung.
Er erf"ullte sein t"agliche{\s} Arbeit{\s}pensum wie ein Gaul im
Hippodrom, der in einem fort den Hufschlag hintrottet, ohne zu
wissen, wa{\s} f"ur ein Gesch"aft er eigentlich verrichtet.

Zu seiner pekuni"aren Unterst"utzung schickte ihm seine Mutter
all\-w"ochentlich durch den Botenmann ein St"uck Kalb{\s}braten.
Da{\s} war sein Fr"uh\-st"uck, wenn er au{\s} dem Krankenhause auf
einen Husch nach Hause kam. Sich erst hinzusetzen, dazu langte die
Zeit nicht, denn er mu"ste al{\s}bald wieder in ein Kolleg oder
zur Anatomie oder Klinik eilen, durch eine Unmenge von Stra"sen
hindurch. Abend{\s} nahm er an der kargen Hauptmahlzeit seiner
Wirt{\s}leute teil. Hinterher ging er hinauf in seine Stube und
setzte sich an seine Lehrb"ucher, oft in nassen Kleidern, die ihm
dann am Leibe bei der Rotglut de{\s} kleinen Ofen{\s} zu dampfen
begannen.

An sch"onen Sommerabenden, wenn die schw"ulen Gassen leer wurden
und die Dienstm"adchen vor den Haust"uren Ball spielten, "offnete
er sein Fenster und sah hinau{\s}. Unten flo"s der Flu"s vor"uber,
der au{\s} diesem Viertel von Rouen ein h"a"sliche{\s}
Klein-Venedig machte. Seine gelben, violett und blau schimmernden
Wasser krochen tr"ag zu den Wehren und Br"ucken. Arbeiter kauerten
am Ufer und wuschen sich die Arme in der Flut. An Stangen, die
au{\s} Speichergiebeln lang hervorragten, trockneten B"undel von
Baumwolle in der Luft. Gegen"uber, hinter den D"achern, leuchtete
der weite klare Himmel mit der sinkenden roten Sonne. Wie herrlich
mu"ste e{\s} da drau"sen im Freien sein! Und dort im Buchenwald
wie frisch! Karl holte tief Atem, um den k"ostlichen Duft der
Felder einzusaugen, der doch gar nicht bi{\s} zu ihm drang.

Er magerte ab und sah sehr schm"achtig au{\s}. Sein Gesicht bekam
einen leidvollen Zug, der e{\s} beinahe interessant machte. Er
ward tr"age, wa{\s} gar nicht zu verwundern war, und seinen guten
Vors"atzen mehr und mehr untreu. Heute vers"aumte er die Klinik,
morgen ein Kolleg, und allm"ahlich fand er Genu"s am Faulenzen und
ging gar nicht mehr hin. Er wurde Stammgast in einer Winkelkneipe
und ein passionierter Dominospieler. Alle Abende in einer
schmutzigen Spelunke zu hocken und mit den beinernen Spielsteinen
auf einem Marmortische zu klappern, da{\s} d"unkte ihn der
h"ochste Grad von Freiheit zu sein, und da{\s} st"arkte ihm sein
Selbstbewu"stsein. E{\s} war ihm da{\s} so etwa{\s} wie der Anfang
eine{\s} weltm"annischen Leben{\s}, diese{\s} Kosten verbotener
Freuden. Wenn er hinkam, legte er seine Hand mit geradezu
sinnlichem Vergn"ugen auf die T"urklinke. Eine Menge Dinge, die
bi{\s} dahin in ihm unterdr"uckt worden waren, gewannen nunmehr
Leben und Gestalt. Er lernte Gassenhauer au{\s}wendig, die er
gelegentlich zum besten gab. B\'eranger, der Freiheit{\s}s"anger,
begeisterte ihn. Er lernte eine gute Bowle brauen, und zu guter
Letzt entdeckte er die Liebe. Dank diesen Vorbereitungen fiel er
im medizinischen Staat{\s}examen gl"anzend durch.

Man erwartete ihn am n"amlichen Abend zu Hau{\s}, wo sein Erfolg
bei einem Schmau{\s} gefeiert werden sollte. Er machte sich zu
Fu"s auf den Weg und erreichte gegen Abend seine Heimat. Dort
lie"s er seine Mutter an den Dorfeingang bitten und beichtete ihr
alle{\s}. Sie entschuldigte ihn, schob den Mi"serfolg der
Ungerechtigkeit der Examinatoren in die Schuhe und richtete ihn
ein wenig auf, indem sie ihm versprach, die Sache in{\s} Lot zu
bringen. Erst volle f"unf Jahre darnach erfuhr Herr Bovary die
Wahrheit. Da war die Geschichte verj"ahrt, und so f"ugte er sich
drein. "Ubrigen{\s} h"atte er e{\s} niemal{\s} zugegeben, da"s
sein leiblicher Sohn ein Dummkopf sei.

Karl widmete sich von neuem seinem Studium und bereitete sich
hart\-n"ackigst auf eine nochmalige Pr"ufung vor. Alle{\s}, wa{\s}
er gefragt werden konnte, lernte er einfach au{\s}wendig. In der
Tat bestand er da{\s} Examen nunmehr mit einer ziemlich guten
Note. Seine Mutter erlebte einen Freudentag. E{\s} fand ein
gro"se{\s} Festmahl statt.

Wo sollte er seine "arztliche Praxi{\s} nun au{\s}"uben? In
Toste{\s}. Dort gab e{\s} nur einen und zwar sehr alten Arzt.
Mutter Bovary wartete schon lange auf sein Hinscheiden, und kaum
hatte der alte Herr da{\s} Zeitliche gesegnet, da lie"s sich Karl
Bovary auch bereit{\s} al{\s} sein Nachfolger daselbst nieder.

Aber nicht genug, da"s die Mutter ihren Sohn erzogen, ihn Medizin
studieren lassen und ihm eine Praxi{\s} au{\s}findig gemacht
hatte: nun mu"ste er auch eine Frau haben. Selbige fand sie in der
Witwe de{\s} Gericht{\s}vollzieher{\s} von Dieppe, die neben
f"unfundvierzig J"ahrlein zw"olfhundert Franken Rente ihr eigen
nannte. Obgleich sie h"a"slich war, d"urr wie eine Hopfenstange
und im Gesicht so viel Pickel wie ein Kirschbaum Bl"uten hatte,
fehlte e{\s} der Witwe Dubuc keine{\s}weg{\s} an Bewerbern. Um zu
ihrem Ziele zu gelangen, mu"ste Mutter Bovary erst alle diese
Nebenbuhler au{\s} dem Felde schlagen, wa{\s} sie sehr geschickt
fertig brachte. Sie triumphierte sogar "uber einen
Fleischermeister, dessen Anwartschaft durch die Geistlichkeit
unterst"utzt wurde.

Karl hatte in die Heirat eingewilligt in der Erwartung, sich
dadurch g"unstiger zu stellen. Er hoffte, pers"onlich wie
pekuni"ar unabh"angiger zu werden. Aber Heloise nahm die Z"ugel in
ihre H"ande. Sie drillte ihm ein, wa{\s} er vor den Leuten zu
sagen habe und wa{\s} nicht. Alle Freitage wurde gefastet. Er
durfte sich nur nach ihrem Geschmacke kleiden, und die Patienten,
die nicht bezahlten, mu"ste er auf ihren Befehl hin kujonieren.
Sie erbrach seine Briefe, "uberwachte jeden Schritt, den er tat,
und horchte an der T"ure, wenn weibliche Wesen in seiner
Sprechstunde waren. Jeden Morgen mu"ste sie ihre Schokolade haben,
und die R"ucksichten, die sie erheischte, nahmen kein Ende.
Unaufh"orlich klagte sie "uber Migr"ane, Brustschmerzen oder
Verdauung{\s}st"orungen. Wenn viel Leute durch den Hau{\s}flur
liefen, ging e{\s} ihr auf die Nerven. War Karl au{\s}w"art{\s},
dann fand sie die Einsamkeit gr"a"slich; kehrte er heim, so war
e{\s} zweifello{\s} blo"s, weil er gedacht habe, sie liege im
Sterben. Wenn er nacht{\s} in da{\s} Schlafzimmer kam, streckte
sie ihm ihre mageren langen Arme au{\s} ihren Decken entgegen,
umschlang seinen Hal{\s} und zog ihn auf den Rand ihre{\s}
Bette{\s}. Und nun ging die Jeremiade lo{\s}. Er vernachl"assige
sie, er liebe eine andre! Man habe e{\s} ihr ja gleich gesagt,
diese Heirat sei ihr Ungl"uck. Schlie"slich bat sie ihn um einen
L"offel Arznei, damit sie gesund werde, und um ein bi"schen mehr
Liebe.



\newpage\begin{center}
{\large \so{Zweite{\s} Kapitel}}\bigskip\bigskip
\end{center}

Einmal nacht{\s} gegen elf Uhr wurde da{\s} Ehepaar durch da{\s}
Getrappel eine{\s} Pferde{\s} geweckt, da{\s} gerade vor der
Haust"ure zum Stehen kam. Anastasia, da{\s} Dienstm"adchen,
klappte ihr Bodenfenster auf und verhandelte eine Weile mit einem
Manne, der unten auf der Stra"se stand. Er wolle den Arzt holen.
Er habe einen Brief an ihn.

Anastasia stieg frierend die Treppen hinunter und schob die Riegel
auf, einen und dann den andern. Der Bote lie"s sein Pferd stehen,
folgte dem M"adchen und betrat ohne weitere{\s} da{\s}
Schlafgemach. Er entnahm seinem wollnen K"appi, an dem eine graue
Troddel hing, einen Brief, der in einen Lappen eingewickelt war,
und "uberreicht ihn dem Arzt mit h"oflicher Geb"arde. Der richtete
sich im Bett auf, um den Brief zu lesen. Anastasia stand dicht
daneben und hielt den Leuchter. Die Frau Doktor kehrte sich
versch"amt der Wand zu und zeigte den R"ucken.

In dem Briefe, den ein niedliche{\s} blaue{\s} Siegel verschlo"s,
wurde Herr Bovary dringend gebeten, unverz"uglich nach dem
Pachtgut Le{\s} Bertaux zu kommen, ein gebrochene{\s} Bein zu
behandeln. Nun braucht man von Toste{\s} "uber Longueville und
Sankt Victor bi{\s} Bertaux zu Fu"s sech{\s} gute Stunden. Die
Nacht war stockfinster. Frau Bovary sprach die Bef"urchtung
au{\s}, e{\s} k"onne ihrem Manne etwa{\s} zusto"sen. Infolgedessen
ward beschlossen, da"s der Stallknecht vorau{\s}reiten, Karl aber
erst drei Stunden sp"ater, nach Mondaufgang, folgen solle. Man
w"urde ihm einen Jungen entgegenschicken, der ihm den Weg zum Gute
zeige und ihm den Hof aufschl"osse.

Fr"uh gegen vier Uhr machte sich Karl, fest in feinen Mantel
geh"ullt, auf den Weg nach Bertaux. Noch ganz verschlafen
"uberlie"s er sich dem Zotteltrab seine{\s} Gaule{\s}. Wenn dieser
von selber vor irgendeinem im Wege liegenden Hinderni{\s} zum
Halten parierte, wurde der Reiter jede{\s}mal wach, erinnerte sich
de{\s} gebrochnen Beine{\s} und begann in seinem Ged"achtnisse
alle{\s} au{\s}zukramen, wa{\s} er von Knochenbr"uchen wu"ste.

Der Regen h"orte auf. E{\s} d"ammerte. Auf den laublosen "Asten
der Apfelb"aume hockten regung{\s}lose V"ogel, da{\s} Gefieder ob
de{\s} k"uhlen Morgenwinde{\s} gestr"aubt. So weit da{\s} Auge
sah, dehnte sich flache{\s} Land. Auf dieser endlosen grauen
Fl"ache hoben sich hie und da in gro"sen Zwischenr"aumen
tiefviolette Flecken ab, die am Horizonte mit de{\s} Himmel{\s}
tr"uben Farben zusammenflossen; da{\s} waren Baumgruppen um G"uter
und Meiereien herum. Von Zeit zu Zeit ri"s Karl seine Augen auf,
bi{\s} ihn die M"udigkeit von neuem "uberw"altigte und der Schlaf
von selber wiederkam. Er geriet in einen traumartigen Zustand, in
dem sich frische Empfindungen mit alten Erinnerungen paarten, so
da"s er ein Doppelleben f"uhrte. Er war noch Student und
gleichzeitig schon Arzt und Ehemann. Im n"amlichen Moment glaubte
er in seinem Ehebette zu liegen und wie einst durch den
Operation{\s}saal zu schreiten. Der Geruch von hei"sen Umschl"agen
mischte sich in seiner Phantasie mit dem frischen Dufte de{\s}
Morgentau{\s}. Dazu h"orte er, wie die Messingringe an den Stangen
der Bettvorh"ange klirrten und wie seine Frau im Schlafe atmete~...

Al{\s} er durch da{\s} Dorf Vassonville ritt, bemerkte er einen
Jungen, der am Rande de{\s} Stra"sengraben{\s} im Grase sa"s.

"`Sind Sie der Herr Doktor?"'

Al{\s} Karl diese Frage bejahte, nahm der Kleine seine
Holzpantoffeln in die H"ande und begann vor dem Pferde
herzurennen. Unterweg{\s} h"orte Bovary au{\s} den Reden seine{\s}
F"uhrer{\s} herau{\s}, da"s Herr Rouault, der Patient, der ihn
erwartete, einer der wohlhabendsten Landwirte sei. Er hatte sich
am vergangenen Abend auf dem Heimwege von einem Nachbar, wo man
da{\s} Dreik"onig{\s}fest gefeiert hatte, ein Bein gebrochen.
Seine Frau war schon zwei Jahre tot. Er lebte ganz allein mit
"`dem gn"adigen Fr"aulein"', da{\s} ihm den Hau{\s}halt f"uhrte.

Die Radfurchen wurden tiefer. Man n"aherte sich dem Gute.
Pl"otzlich verschwand der Junge in der L"ucke einer Gartenhecke,
um hinter der Mauer eine{\s} Vorhofe{\s} wieder aufzutauchen, wo
er ein gro"se{\s} Tor "offnete. Da{\s} Pferd trat in nasse{\s}
rutschige{\s} Gra{\s}, und Karl mu"ste sich ducken, um nicht vom
Baumgezweig au{\s} dem Sattel gerissen zu werden. Hofhunde fuhren
au{\s} ihren H"utten, schlugen an und rasselten an den Ketten.
Al{\s} der Arzt in den eigentlichen Gut{\s}hof einritt, scheute
der Gaul und machte einen gro"sen Satz zur Seite.

Da{\s} Pachtgut Bertaux war ein ansehnliche{\s} Besitztum. Durch
die offenstehenden T"uren konnte man in die St"alle blicken, wo
kr"aftige Ackerg"aule gem"achlich au{\s} blanken Raufen ihr Heu
kauten. L"ang{\s} der Wirtschaft{\s}geb"aude zog sich ein
dampfender Misthaufen hin. Unter den H"uhnern und Truth"ahnen
machten sich f"unf bi{\s} sech{\s} Pfauen mausig, der Stolz der
G"uter jener Gegend. Der Schafstall war lang, die Scheune hoch und
ihre Mauern spiegelglatt. Im Schuppen standen zwei gro"se
Leiterwagen und vier Pfl"uge, dazu die n"otigen Pferdegeschirre,
Kumte und Peitschen; auf den blauen Woilach{\s} au{\s} Schafwolle
hatte sich feiner Staub gelagert, der von den Kornb"oden
heruntersickerte. Der Hof, der nach dem Wohnhause zu etwa{\s}
anstieg, war auf beiden Seiten mit einer Reihe B"aume bepflanzt.
Vom T"umpel her erscholl da{\s} fr"ohliche Geschnatter der G"anse.

An der Schwelle de{\s} Hause{\s} erschien ein junge{\s}
Frauenzimmer in einem mit drei Volant{\s} besetzten blauen
Merinokleide und begr"u"ste den Arzt. Er wurde nach der K"uche
gef"uhrt, wo ein t"uchtige{\s} Feuer brannte. Auf dem Herde kochte
in kleinen T"opfen von verschiedener Form da{\s} Fr"uhst"uck
de{\s} Gesinde{\s}. Oben im Rauchfang hingen na"sgewordene
Kleidung{\s}st"ucke zum Trocknen. Kohlenschaufel, Feuerzange und
Blasebalg, alle miteinander von riesiger Gr"o"se, funkelten wie
von blankem Stahl, w"ahrend l"ang{\s} der W"ande eine Unmenge
K"uchenger"at hing, "uber dem die helle Herdflamme um die Wette
mit den ersten Strahlen der durch die Fenster huschenden
Morgensonne spielte und glitzerte.

Karl stieg in den ersten Stock hinauf, um den Kranken aufzusuchen.
Er fand ihn in seinem Bett, schwitzend unter seinen Decken. Seine
Nachtm"utze hatte er in die Stube geschleudert. E{\s} war ein
st"ammiger kleiner Mann, ein F"unfziger, mit wei"sem Haar, blauen
Augen und kahler Stirn. Er trug Ohrringe. Neben ihm auf einem
Stuhle stand eine gro"se Karaffe voll Branntwein, au{\s} der er
sich von Zeit zu Zeit ein Gl"a{\s}chen einschenkte, um "`Mumm in
die Knochen zu kriegen"'. Angesicht{\s} de{\s} Arzte{\s} legte
sich seine Erregung. Statt zu fluchen und zu wettern -- wa{\s} er
seit zw"olf Stunden getan hatte -- fing er nunmehr an zu "achzen
und zu st"ohnen.

Der Bruch war einfach, ohne jedwede Komplikation. Karl h"atte sich
einen leichteren Fall nicht zu w"unschen gewagt. Al{\s}bald
erinnerte er sich der Al"l"uren, die seine Lehrmeister an den
Krankenlagern zur Schau getragen harten, und spendete dem
Patienten ein reichliche{\s} Ma"s der "ublichen guten Worte,
jene{\s} Chirurgenbalsam{\s}, der an da{\s} "Ol gemahnt, mit dem
die Seziermesser eingefettet werden. Er lie"s sich au{\s} dem
Holzschuppen ein paar Latten holen, um Holz zu Schienen zu
bekommen. Von den gebrachten St"ucken w"ahlte er ein{\s} au{\s},
schnitt die Schienen darau{\s} zurecht und gl"attete sie mit einer
Gla{\s}scherbe. W"ahrenddem stellte die Magd Leinwandbinden her,
und Fr"aulein Emma, die Tochter de{\s} Hause{\s}, versuchte
Polster anzufertigen. Al{\s} sie ihren N"ahkasten nicht gleich
fand, polterte der Vater lo{\s}. Sie sagte kein Wort. Aber beim
N"ahen stach sie sich in den Finger, nahm ihn in den Mund und sog
da{\s} Blut au{\s}.

Karl war erstaunt, wa{\s} f"ur blendendwei"se N"agel sie hatte.
Sie waren mandelf"ormig geschnitten und sorglich gepflegt, und so
schimmerten sie wie da{\s} feinste Elfenbein. Ihre H"ande freilich
waren nicht gerade sch"on, vielleicht nicht wei"s genug und ein
wenig zu mager in den Fingern; dabei waren sie allzu schlank,
nicht besonder{\s} weich und in ihren Linien ungrazi"o{\s}. Wa{\s}
jedoch sch"on an ihr war, da{\s} waren ihre Augen. Sie waren
braun, aber im Schatten der Wimpern sahen sie schwarz au{\s}, und
ihr offener Blick traf die Menschen mit der K"uhnheit der
Unschuld.

Al{\s} der Verband fertig war, lud Herr Rouault den Arzt feierlich
"`einen Bissen zu essen"', ehe er wieder aufbr"ache. Karl ward in
da{\s} E"szimmer gef"uhrt, da{\s} zu ebener Erde lag. Auf einem
kleinen Tische war f"ur zwei Personen gedeckt; neben den Gedecken
blinkten silberne Becher. Au{\s} dem gro"sen Eichenschranke,
gegen"uber dem Fenster, str"omte Geruch von Iri{\s} und feuchtem
Leinen. In einer Ecke standen aufrecht in Reih und Glied mehrere
S"acke mit Getreide; sie hatten auf der Kornkammer nebenan keinen
Platz gefunden, zu der drei Steinstufen hinauff"uhrten. In der
Mitte der Wand, deren gr"uner Anstrich sich stellenweise
abbl"atterte, hing in einem vergoldeten Rahmen eine
Bleistift\/zeichnung: der Kopf einer Minerva. In schn"orkeliger
Schrift stand darunter geschrieben. "`Meinem lieben Vater!"'

Sie sprachen zuerst von dem Unfall, dann vom Wetter, vom starken
Frost, von den W"olfen, die nacht{\s} die Umgegend unsicher
machen. Fr"aulein Rouault schw"armte gar nicht besonder{\s} von
dem Leben auf dem Lande, zumal jetzt nicht, wo die ganze Last der
Gut{\s}wirtschaft fast allein auf ihr ruhe. Da e{\s} im Zimmer
kalt war, fr"ostelte sie w"ahrend der ganzen Mahlzeit. Beim Essen
fielen ihre vollen Lippen etwa{\s} auf. Wenn da{\s} Gespr"ach
stockte, pflegte sie mit den Oberz"ahnen auf die Unterlippe zu
bei"sen.

Ihr Hal{\s} wuch{\s} au{\s} einem wei"sen Umlegekragen herau{\s}.
Ihr schwarze{\s}, hinten zu einem reichen Knoten vereinte{\s} Haar
war in der Mitte gescheitelt; beide H"alften lagen so glatt auf
dem Kopfe, da"s sie wie zwei Fl"ugel au{\s} je einem St"ucke
au{\s}sahen und kaum die Ohrl"appchen blicken lie"sen. "Uber den
Schl"afen war da{\s} Haar gewellt, wa{\s} der Landarzt noch nie in
seinem Leben gesehen hatte. Ihre Wangen waren rosig. Zwischen zwei
Kn"opfen ihrer Taille lugte -- wie bei einem Herrn -- ein Lorgnon
au{\s} Schildpatt hervor.

Nachdem sich Karl oben beim alten Rouault verabschiedet hatte,
trat er nochmal{\s} in da{\s} E"szimmer. Er fand Emma am Fenster
stehend, die Stirn an die Scheiben gedr"uckt. Sie schaute in den
Garten hinau{\s}, wo der Wind die Bohnenstangen umgeworfen hatte.
Sich umwendend, fragte sie:

"`Suchen Sie etwa{\s}?"'

"`Meinen Reitstock, wenn Sie gestatten!"'

Er fing an zu suchen, hinter den T"uren und unter den St"uhlen.
Der Stock war auf den Fu"sboden gefallen, gerade zwischen die
S"acke und die Wand. Emma entdeckte ihn. Al{\s} sie sich "uber die
S"acke beugte, wollte Karl ihr galant zuvorkommen. Wie er seinen
Arm in der n"amlichen Absicht wie sie au{\s}streckte, ber"uhrte
seine Brust den geb"uckten R"ucken de{\s} jungen M"adchen{\s}. Sie
f"uhlten e{\s} beide. Emma fuhr rasch in die H"ohe. Ganz rot
geworden, sah sie ihn "uber die Schulter weg an, indem sie ihm
seinen Reitstock reichte.

Er hatte versprochen, in drei Tagen wieder nachzusehen; statt
dessen war er bereit{\s} am n"achsten Tag zur Stelle, und von da
ab kam er regelm"a"sig zweimal in der Woche, ungerechnet die
gelegentlichen Besuche, die er hin und wieder machte, wenn er
"`zuf"allig in der Gegend"' war. "Ubrigen{\s} ging alle{\s}
vorz"uglich; die Heilung verlief regelrecht, und al{\s} man nach
sech{\s} und einer halben Woche Vater Rouault ohne Stock wieder in
Hau{\s} und Hof herumstiefeln sah, hatte sich Bovary in der ganzen
Gegend den Ruf einer Kapazit"at erworben. Der alte Herr meinte,
besser h"atten ihn die ersten "Arzte von Yvetot oder selbst von
Rouen auch nicht kurieren k"onnen.

Karl dachte gar nicht daran, sich zu befragen, warum er so gern
nach dem Rouaultschen Gute kam. Und wenn er auch dar"uber
nachgesonnen h"atte, so w"urde er den Beweggrund seine{\s}
Eifer{\s} zweifello{\s} in die Wichtigkeit de{\s} Falle{\s} oder
vielleicht in da{\s} in Au{\s}sicht stehende hohe Honorar gelegt
haben. Waren die{\s} aber wirklich die Gr"unde, die ihm seine
Besuche de{\s} Pachthofe{\s} zu k"ostlichen Abwechselungen in dem
armseligen Einerlei seine{\s} t"atigen Leben{\s} machten? An
solchen Tagen stand er zeitig auf, ritt im Galopp ab und lie"s den
Gaul die ganze Strecke lang kaum zu Atem kommen. Kurz vor seinem
Ziele aber pflegte er abzusitzen und sich die Stiefel mit Gra{\s}
zu reinigen; dann zog er sich die braunen Reithandschuhe an, und
so ritt er kreuzvergn"ugt in den Gut{\s}hof ein. E{\s} war ihm ein
Wonnegef"uhl, mit der Schulter gegen den nachgebenden Fl"ugel
de{\s} Hoftore{\s} anzureiten, den Hahn auf der Mauer kr"ahen zu
h"oren und sich von der Dorfjugend umringt zu sehen. Er liebte die
Scheune und die St"alle; er liebte den Papa Rouault, der ihm so
treuherzig die Hand sch"uttelte und ihn seinen Leben{\s}retter
nannte; er liebte die niedlichen Holzpantoffeln de{\s}
Gut{\s}fr"aulein{\s}, die auf den immer sauber gescheuerten
Fliesen der K"uche so allerliebst schl"urften und klapperten. In
diesen Schuhen sah Emma viel gr"o"ser au{\s} denn sonst. Wenn Karl
wieder ging, gab sie ihm jede{\s}mal da{\s} Geleit bi{\s} zur
ersten Stufe der Freitreppe. War sein Pferd noch nicht
vorgef"uhrt, dann wartete sie mit. Sie hatten schon Abschied
voneinander genommen, und so sprachen sie nicht mehr. Wenn e{\s}
sehr windig war, kam ihr flaumige{\s} Haar im Nacken in wehenden
Wirrwarr, oder die Sch"urzenb"ander begannen ihr um die H"uften zu
flattern. Einmal war Tauwetter. An den Rinden der B"aume rann
Wasser in den Hof hinab, und auf den D"achern der Geb"aude schmolz
aller Schnee. Emma war bereit{\s} auf der Schwelle, da ging sie
wieder in{\s} Hau{\s}, holte ihren Sonnenschirm und spannte ihn
auf. Die Sonnenlichter stahlen sich durch die taubengraue Seide
und tupften tanzende Reflexe auf die wei"se Haut ihre{\s}
Gesicht{\s}. Da{\s} gab ein so warme{\s} und wohlige{\s} Gef"uhl,
da"s Emma l"achelte. Einzelne Wassertropfen prallten auf da{\s}
Schirmdach, laut vernehmbar, einer, wieder einer, noch einer~...

Im Anfang hatte Frau Bovary h"aufig nach Herrn Rouault und seiner
Krankheit gefragt, auch hatte sie nicht verfehlt, f"ur ihn in
ihrer doppelten Buchf"uhrung ein besondre{\s} Konto einzurichten.
Al{\s} sie aber vernahm, da"s er eine Tochter hatte, zog sie
n"ahere Erkundigungen ein, und da erfuhr sie, da"s Fr"aulein
Rouault im Kloster, bei den Ursulinerinnen, erzogen worden war,
sozusagen also "`eine feine Erziehung genossen"' hatte, da"s sie
infolgedessen Kenntnisse im Tanzen, in der Erdkunde, im Zeichnen,
Sticken und Klavierspielen haben mu"ste. Da{\s} ging ihr "uber die
Hutschnur, wie man zu sagen pflegt.

"`Also darum!"' sagte sie sich. "`Darum also lacht ihm da{\s}
ganze Gesicht, wenn er zu ihr hinreitet! Darum zieht er die neue
Weste an, gleichg"ultig, ob sie ihm vom Regen verdorben wird! Oh
diese{\s} Weib, diese{\s} Weib!"'

Instinktiv ha"ste sie Emma. Zuerst tat sie sich eine G"ute in
allerhand Anspielungen. Karl verstand da{\s} nicht. Darauf
versuchte sie e{\s} mit anz"uglichen Bemerkungen, die er au{\s}
Angst vor einer h"au{\s}lichen Szene "uber sich ergehen lie"s.
Schlie"slich aber ging sie im Sturm vor. Karl wu"ste nicht, wa{\s}
er sagen sollte. We{\s}halb renne er denn ewig nach Bertaux, wo
doch der Alte l"angst geheilt sei, wenn die Rasselbande auch noch
nicht berappt habe? Na freilich, weil e{\s} da "`eine Person"'
g"abe, die fein zu schwatzen verst"unde, ein Weib{\s}bild, da{\s}
sticken k"onne und weiter nicht{\s}, ein Blaustrumpf! In die sei
er verschossen! Ein Stadtd"amchen, da{\s} sei ihm ein
gefundene{\s} Fressen.

"`Bl"odsinn!"' polterte sie weiter. "`Die Tochter de{\s} alten
Rouault, die und eine feine Dame! O jeh! Ihr Gro"svater hat noch
die Schafe geh"utet, und ein Vetter von ihr ist beinahe vor den
Staat{\s}anwalt gekommen, weil er bei einem Streite jemanden
halbtot gedroschen hat! So wa{\s} hat gar keinen Anla"s, sich
wa{\s} Besonder{\s} einzubilden und Sonntag{\s} aufgedonnert in
die Kirche zu schw"anzeln, in seidnen Kleidern wie eine
Prinzessin. Und der Alte, der arme Schluder! Wenn im vergangenen
Jahre die Rap{\s}ernte nicht so unversch"amt gut au{\s}gefallen
w"are, h"atte er seinen lumpigen Pacht nicht mal blechen
k"onnen!"'

Die Freude war Karl verdorben. Er stellte seine Ritte nach Bertaux
ein. Seine Frau hatte ihn nach einer Flut von Tr"anen und K"ussen
und unter tausend Z"artlichkeiten auf ihr Me"sbuch schw"oren
lassen, nicht mehr hinzugehen. Er gehorchte. Aber in seiner
heimlichen Sehnsucht war er k"uhner; da war er emp"ort "uber seine
tats"achliche eigne Feigheit. Und in naivem Machiavelli{\s}mu{\s}
sagte er sich, gerade ob diese{\s} Verbot{\s} habe er ein Recht
auf seine Liebe. Wa{\s} war die ehemalige Witwe auch f"ur ein
Weib: sie war spindeld"urr und hatte h"a"sliche Z"ahne; Sommer wie
Winter trug sie denselben schwarzen Schal mit dem "uber den
R"ucken herabh"angenden langen Zipfel; ihre steife Figur stak in
den immer zu kurzen Kleidern wie in einem Futteral, und wa{\s}
f"ur plumpe Schuhe trug sie "uber ihren grauen Str"umpfen.

Karl{\s} Mutter kam von Zeit zu Zeit zu Besuch. Dann wurde e{\s}
noch schlimmer; dann hackten sie alle beide auf ihn ein. Da{\s}
viele Essen bek"ame ihm schlecht. Warum er dem ersten besten immer
gleich ein Gla{\s} Wein vorsetze? Und e{\s} sei blo"s
Dickk"opfigkeit von ihm, keine Flanellw"asche zu tragen.

Zu Beginn de{\s} Fr"uhling{\s} begab e{\s} sich, da"s der
Verm"ogen{\s}verwalter der Frau verwitweten Dubuc, ein Notar in
Ingouville, samt allen ihm anvertrauten Geldern "uber{\s} Meer
da{\s} Weite suchte. Nun besa"s sie allerding{\s} au"serdem einen
Schiff{\s}anteil in der H"ohe von sechstausend Franken und ein
Hau{\s} in Dieppe. Aber von allen diesen vielgepriesenen
Besitzt"umern hatte man nie etwa{\s} Ordentliche{\s} zu sehen
bekommen. Die Witwe hatte nicht{\s} mit in die Ehe gebracht al{\s}
ein paar M"obel und etliche Nippsachen. Nunmehr ging man der Sache
auf den Grund, und da stellte sich denn herau{\s}, da"s
besagte{\s} Hau{\s} bi{\s} an die Feueresse mit Hypotheken
belastet, da"s kein Mensch wu"ste, wieviel Geld wirklich mit dem
Notar zum Teufel gegangen, und da"s die Schiff{\s}hypothek keine
tausend Taler wert war. Folglich hatte die liebe Frau Heloise
geflunkert. In seinem Zorn warf der alte Bovary einen Stuhl gegen
die Wand, da"s er in tausend St"ucke ging, und machte seiner Frau
den Vorwurf, sie habe den Jungen in da{\s} Ungl"uck gest"urzt und
ihn mit einer alten Kracke eingespannt, die de{\s} Futter{\s}
nicht einmal mehr wert sei.

Sie fuhren nach Toste{\s}. E{\s} kam zu einer Au{\s}einandersetzung
und zu heftigen Szenen. Heloise warf sich weinend in die Arme
ihre{\s} Gatten und beschwor ihn, sie den Eltern gegen"uber in
Schutz zu nehmen. Karl wollte die Partei seiner Frau ergreifen.
Aber da{\s} nahmen ihm die Alten "ubel. Sie reisten ab.

Diesen Schlag vermochte Heloise nicht zu verwinden. Acht Tage
darnach, al{\s} sie dabei war, W"asche im Hofe aufzuh"angen, bekam
sie einen Blutsturz, und am andern Morgen war sie tot.

Al{\s} Karl vom Friedhofe zur"uckkam, fand er im Erdgescho"s
keinen Menschen. Er stieg die Treppe hinauf. Wie er in da{\s}
Schlafzimmer trat, fiel sein Blick auf einen Rock Heloisen{\s},
der am Bette hing. Er lehnte sich gegen da{\s} Schreibpult und
blieb da hocken, bi{\s} e{\s} dunkel wurde, in schmerzliche
Tr"aumereien versunken. Alle{\s} in allem hatte sie ihn doch
geliebt~...



\newpage\begin{center}
{\large \so{Dritte{\s} Kapitel}}\bigskip\bigskip
\end{center}

Eine{\s} Vormittag{\s} erschien Vater Rouault und brachte da{\s}
Honorar f"ur den behandelten Beinbruch: f"unfundsiebzig Franken in
blanken Talern und eine Truthenne. Er hatte Karl{\s} Ungl"uck
erfahren und tr"ostete ihn, so gut er konnte.

"`Ich wei"s, wie einem da zumute ist!"' sagte er, indem er dem
Witwer auf die Schulter klopfte. "`Hab{\s} ja selber mal
durchgemacht, ganz so wie Sie! Al{\s} ich meine Selige begraben
hatte, da lief ich hinau{\s} in{\s} Freie, um allein f"ur mich zu
sein. Ich warf mich im Walde hin und weinte mich au{\s}. Fing an,
mit dem lieben Gott zu hadern, und machte ihm die d"ummsten
Vorw"urfe. An einem Aste sah ich einen verreckten Maulwurf
h"angen, dem der Bauch von W"urmern wimmelte. Ich beneidete den
Kadaver! Und wenn ich daran dachte, da"s im selben Augenblicke
andre M"anner mit ihren netten kleinen Frauen zusammen waren und
sie an sich dr"uckten, schlug ich mit meinem Stocke wild um mich.
E{\s} war sozusagen nicht mehr ganz richtig mit mir. Ich a"s nicht
mehr. Der blo"se Gedanke, in ein Kaffeehau{\s} zu gehn, ekelte
mich an. Glauben Sie mir da{\s}! Na, und so nach und nach im Gang
der Zeiten, wie so der Fr"uhling dem Winter und der Herbst dem
Sommer folgte, da ging{\s} ein{\s}, zwei, drei, und weg war der
Jammer! Weg! Hinunter! Da{\s} ist da{\s} richtige Wort: hinunter!
Denn ganz kriegt man ja so wa{\s} im ganzen Leben nicht lo{\s}. Da
tief drinnen in der Brust bleibt immer wa{\s} stecken. Aber Luft
kriegt man wieder! Sehen Sie, da{\s} ist nun einmal unser aller
Schicksal, und de{\s}halb darf man nicht gleich die Flinte in{\s}
Korn werfen. Man darf nicht sterben wollen, weil andere gestorben
sind. Auch Sie m"ussen sich aufrappeln, Herr Bovary! E{\s} geht
alle{\s} vor"uber! Besuchen Sie un{\s}! Sie wissen ja, meine Emma
denkt oft an Sie. Sie h"atten un{\s} vergessen, meint sie. E{\s}
wird nun Fr"uhling. Zerstreuen Sie sich ein bi"schen bei un{\s}.
Schie"sen Sie ein paar Karnickel auf meinem Revier!"'

Karl befolgte seinen Rat. Er kam wieder nach Bertaux und fand da
alle{\s} wie einst, da{\s} hei"st wie vor f"unf Monaten. Die
Birnb"aume hatten schon Bl"uten, und der treffliche Vater Rouault
war wieder mord{\s}gesund und von fr"uh bi{\s} abend auf den
Beinen. Und im ganzen Gut war m"achtiger Betrieb.

E{\s} war ihm eine Ehrensache, den Arzt mit der erdenklichsten
R"ucksicht auf sein Leid zu behandeln. Er bat ihn, sich{\s} so
bequem wie nur m"oglich zu machen, sprach im Fl"ustertone mit ihm
wie mit einem Genesenden, und er war sichtlich au"ser sich, wenn
man de{\s} Gaste{\s} wegen nicht, wie befohlen, die
leichtverdaulichsten Gerichte auf den Tisch brachte, zum Beispiel
feine Eierspeisen oder ged"unstete Birnen. Er erz"ahlte Anekdoten
und Abenteuer. Zu seiner eignen Verwunderung lachte Karl. Aber mir
einem Male erinnerte er sich seiner Frau und wurde nachdenklich.
Der Kaffee ward gebracht, und da verga"s er sie wieder.

Je mehr er sich an sein Witwertum gew"ohnte, um so weniger
gedachte er der Verstorbenen. Da{\s} angenehme, ihm neue
Bewu"stsein, unabh"angig zu sein, machte ihm die Einsamkeit bald
ertr"aglicher. Jetzt durfte er die Stunden der Mahlzeiten selber
bestimmen, konnte gehen und kommen, ohne Rechenschaft dar"uber
geben zu m"ussen, und wenn er m"ude war, alle vier von sich
strecken und sich in seinem Bette breit machen. Er hegte und
pflegte sich und lie"s alle Tr"ostungen "uber sich ergehen.
"Ubrigen{\s} hatte der Tod seiner Frau keine ung"unstige Wirkung
auf seinen Beruf al{\s} Arzt. Indem man wochenlang in einem fort
sagte: "`Der arme Doktor. Wie traurig!"' blieb sein Name im Munde
der Leute. Seine Praxi{\s} vergr"o"serte sich. Und dann konnte er
nun nach Bertaux reiten, wann e{\s} ihm beliebte. Eine
unbestimmbare Sehnsucht wuch{\s} in ihm auf, ein namenlose{\s}
Gl"uck{\s}gef"uhl. Wenn er sich im Spiegel betrachtete und sich
den Bart strich, fand er sich gar nicht "ubel.

Eine{\s} sch"onen Tage{\s} kam er nachmittag{\s} gegen drei Uhr im
Gute angeritten. Alle{\s} war drau"sen auf dem Felde. Er betrat
die K"uche. Emma war drinnen, aber er bemerkte sie zun"achst
nicht. Die Fensterl"aden waren geschlossen. Durch die Ritzen
de{\s} Holze{\s} stachen die Sonnenstrahlen mit langen d"unnen
Nadeln auf die Fliesen, oder sie brachen sich an den Kanten der
M"obel ent\/zwei und wirbelten hinauf zur Decke. Auf dem
K"uchentische krabbelten Fliegen an den Gl"asern hinauf, purzelten
summend in die Apfelweinneigen und ertranken. Da{\s} Sonnenlicht,
da{\s} durch den Kamin eindrang, verwandelte die ru"sige
Herdplatte in eine Samtfl"ache und f"arbte den Aschehaufen blau.
Emma sa"s zwischen dem Fenster und dem Herd und n"ahte. Sie hatte
kein Hal{\s}tuch um, und auf ihren entbl"o"sten Schultern
gl"anzten kleine Schwei"sperlen.

Nach l"andlichem Brauch bot sie dem Ank"ommling einen Trunk an.
Al{\s} er ihn au{\s}schlug, n"otigte sie ihn, und schlie"slich bat
sie ihn lachend, ein Gl"a{\s}chen Lik"or mit ihr zu trinken. Sie
holte au{\s} dem Schranke eine Flasche Cura\c{c}ao, suchte zwei
Gl"aser herau{\s}, f"ullte da{\s} eine bi{\s} zum Rande und go"s
in da{\s} andre ein paar Tropfen. Sie stie"s mit Karl an und
f"uhrte dann ihr Gla{\s} zum Munde. Da soviel wie nicht{\s} drin
war, mu"ste sie sich beim Trinken zur"uckbiegen. Den Kopf nach
hinten gelegt, die Lippen zugespitzt, den Hal{\s} gestrafft, so
stand sie da und lachte dar"uber, da"s ihr nicht{\s} auf die Zunge
lief, obgleich diese mit der Spitze au{\s} den feinen Z"ahnen
herau{\s}spazierte und bi{\s} an den Boden de{\s} Glase{\s}
mehreremal{\s} suchend vorstie"s.

Emma nahm wieder Platz und begann sich von neuem ihrer Handarbeit
zu widmen. Ein wei"ser baumwollener Strumpf war zu stopfen. Mit
gesenkter Stirn sa"s sie da. Sie sagte nicht{\s} und Karl erst
recht nicht{\s}. Der Luft\/zug, der sich zwischen T"ur und Schwelle
eindr"angte, wirbelte ein wenig Staub von den Fliesen auf. Karl
sah diesem Tanze der Atome zu. Dabei h"orte er nicht{\s} al{\s}
da{\s} H"ammern seine{\s} Blute{\s} im eignen Hirne und au{\s} der
Ferne da{\s} Gackern einer Henne, die irgendwo im Hofe ein Ei
gelegt hatte. Hin und wieder hielt Emma die Handfl"achen ihrer
H"ande auf den kalten Knauf der Herdstange und pre"ste sie dann an
ihre Wangen, um diese zu k"uhlen.

Sie klagte "uber die Schwindelanf"alle, von denen sie seit
Fr"uhjahr{\s}anfang heimgesucht wurde, und fragte, ob ihr wohl
Seeb"ader dienlich w"aren. Dann plauderte sie von ihrem Aufenthalt
im Kloster und er von seiner Gymnasiastenzeit. So gerieten sie in
ein Gespr"ach. Sie f"uhrte ihn in ihr Zimmer und zeigte ihm ihre
Notenhefte von damal{\s} und die niedlichen B"ucher, die sie
al{\s} Schulpr"amien bekommen hatte, und die Eichenlaubkr"anze,
die im untersten Schrankfache ihr Dasein fristeten. Dann erz"ahlte
sie von ihrer Mutter, von deren Grabe, und zeigte ihm sogar im
Garten da{\s} Beet, wo die Blumen w"uchsen, die sie der Toten
jeden ersten Freitag im Monat hintrug. Der G"artner, den sie
hatten, verst"unde nicht{\s}. Mit dem seien sie schlecht dran. Ihr
Wunsch w"are e{\s}, wenigsten{\s} w"ahrend der Wintermonate in der
Stadt zu wohnen. Dann aber meinte sie wieder, an den langen
Sommertagen sei da{\s} Leben auf dem Lande noch langweiliger. Und
je nachdem, wa{\s} sie sagte, klang ihre Stimme hell oder scharf;
oder sie nahm pl"otzlich einen matten Ton an, und wenn sie wie mit
sich selbst plauderte, ward sie wieder ganz ander{\s}, wie
fl"usternd und murmelnd. Bald war Emma lustig und hatte gro"se
unschuldige Augen, dann wieder schlossen sich ihre Lider zur
H"alfte, und ihr schimmernder Blick sah teilnahm{\s}lo{\s} und
traumverloren au{\s}.

Abend{\s} auf dem Heimritt wiederholte sich Karl alle{\s}, wa{\s}
sie geredet hatte, bi{\s} in{\s} einzelne, und versuchte den
vollen Sinn ihrer Worte zu erfassen. Er wollte sich damit eine
Vorstellung von der Existenz schaffen, die Emma gef"uhrt, ehe er
sie kennen gelernt hatte. Aber e{\s} gelang ihm nicht, sie in
seinen Gedanken ander{\s} zu erschauen al{\s} so, wie sie
au{\s}gesehen hatte, al{\s} er sie zum ersten Male erblickt, oder
so, wie er sie eben vor sich gehabt hatte. Dann fragte er sich,
wie e{\s} wohl w"urde, wenn sie sich verheiratete, aber mit wem?
Ja, ja, mit wem? Ihr Vater war so reich und sie ... so sch"on!

Und immer wieder sah er Emma{\s} Gesicht vor seinen geistigen
Augen, und eine Art eint"onige Melodie summte ihm durch die Ohren
wie da{\s} Surren eine{\s} Kreisel{\s}: "`Emma, wenn du dich
verheiratetest! Wenn du dich nun verheiratetest!"' In der Nacht
konnte er keinen Schlaf finden. Die Kehle war ihm wie
zugeschn"urt. Er versp"urte Durst, stand auf, trank ein Gla{\s}
Wasser und machte da{\s} Fenster auf. Der Himmel stand voller
Sterne. Der laue Nachtwind strich in da{\s} Zimmer. Fern bellten
Hunde. Er wandte den Blick in die R"otung nach Bertaux.

Endlich kam er auf den Gedanken, da"s e{\s} den Hal{\s} nicht
kosten k"onne, und so nahm er sich vor, bei der ersten besten
Gelegenheit um Emma{\s} Hand zu bitten. Aber sooft sich diese
Gelegenheit bot, wollten ihm vor lauter Angst die passenden Worte
nicht "uber die Lippen. Vater Rouault h"atte l"angst nicht{\s}
dagegen gehabt, wenn ihm jemand seine Tochter geholt h"atte. Im
Grunde n"utzte sie ihm in Hau{\s} und Hof nicht viel. Er machte
ihr keinen Vorwurf darau{\s}: sie war eben f"ur die Landwirtschaft
zu geweckt. "`Ein gottverdammte{\s} Gewerbe!"' pflegte er zu
schimpfen. "`Da{\s} hat auch noch keinen zum Million"ar gemacht!"'
Ihm hatte e{\s} in der Tat keine Reicht"umer gebracht; im
Gegenteil, er setzte alle Jahre zu. Denn wenn er auch auf den
M"arkten zu seinem Stolz al{\s} gerissener Kerl bekannt war, so
war er eigentlich doch f"ur Ackerbau und Viehzucht durchau{\s}
nicht geschaffen. Er verstand nicht zu wirtschaften. Er nahm nicht
gern die H"ande au{\s} den Hosentaschen, und seinem eigenen Leibe
war er kein Stiefvater. Er hielt auf gut Essen und Trinken, einen
warmen Ofen und au{\s}giebigen Schlaf. Ein gute{\s} Gla{\s}
Landwein, ein halb durchgebratene{\s} Hammelkotelett und ein
T"a"schen Mokka mit Kognak geh"orten zu den Idealen seine{\s}
Leben{\s}. Er nahm seine Mahlzeiten in der K"uche ein und zwar
allein f"ur sich, in der N"ahe de{\s} Herdfeuer{\s} an einem
kleinen Tische, der ihm -- wie auf der B"uhne -- fix und fertig
gedeckt hereingebracht werden mu"ste.

Al{\s} er die Entdeckung machte, da"s Karl einen roten Kopf bekam,
wenn er Emma sah, war er sich sofort klar, da"s fr"uher oder
sp"ater ein Heirat{\s}antrag zu erwarten war. Alsobald "uberlegte
er sich die Geschichte. Besonder{\s} schneidig sah ja Karl Bovary
nicht gerade au{\s}, und Rouault hatte sich ehedem seinen
k"unftigen Schwiegersohn ein bi"schen ander{\s} gedacht, aber er
war doch al{\s} anst"andiger Kerl bekannt, sparsam und t"uchtig in
seinem Berufe. Und zweifello{\s} w"urde er wegen der Mitgift nicht
lange feilschen. Vater Rouault hatte gerade eine Menge gro"ser
Au{\s}gaben. Um allerlei Handwerker zu bezahlen, sah er sich
gezwungen, zweiundzwanzig Acker von seinem Grund und Boden zu
verkaufen. Die Kelter mu"ste auch erneuert werden. Und so sagte er
sich: "`Wenn er um Emma anh"alt, soll er sie kriegen!"'

Zur Weinlese war Karl drei Tage lang da. Aber Tag verging auf Tag
und Stunde auf Stunde, ohne da"s Karl{\s} Wille zur Tat ward.
Rouault gab ihm ein kleine{\s} St"uck Weg{\s} da{\s} Geleite; am
Ende de{\s} Hohlweg{\s} vor dem Dorfe pflegte er sich von seinem
Gaste zu verabschieden. Da{\s} war also der Moment! Karl nahm sich
noch Zeit bi{\s} zuallerletzt. Erst al{\s} die Hecke hinter ihnen
lag, stotterte er lo{\s}:

"`Verehrter Herr Rouault, ich m"ochte Ihnen gern etwa{\s} sagen!"'

Weiter brachte er nicht{\s} herau{\s}. Die beiden M"anner blieben
stehen.

"`Na, rau{\s} mit der Sprache! Ich kann mir{\s} schon denken!"'
Rouault lachte gem"utlich.

"`Vater Rouault! Vater Rouault!"' stammelte Karl.

"`Meinen Segen sollen Sie haben!"' fuhr der Gut{\s}p"achter fort.
"`Meine Kleine denkt gewi"s nicht ander{\s} al{\s} ich, aber
gefragt werden mu"s sie. Reiten Sie getrost nach Hause. Ich werde
sie gleich mal in{\s} Gebet nehmen. Wenn sie Ja sagt, --
wohlverstanden! -- brauchen Sie jedoch nicht umzukehren. Wegen der
Leute nicht, und auch weil sie sich erst ein bi"schen beruhigen
soll. Damit Sie aber nicht zu lange Blut schwitzen, will ich Ihnen
ein Zeichen geben: ich werde einen Fensterladen gegen die Mauer
klappen lassen. Wenn Sie da oben "uber die Hecke gucken, k"onnen
Sie da{\s} ungesehen beobachten!"'

Damit ging er.

Karl band seinen Schimmel an einen Baum; kletterte die B"oschung
hinauf und stellt sich auf die Lauer, die Taschenuhr in der Hand.
Eine halbe Stunde verstrich -- und dann noch neunzehn Minuten ...
Da gab e{\s} mit einem Male einen Schlag gegen die Mauer. Der
Laden blieb sperrangelweit offen und wackelte noch eine Weile.

Am andern Morgen war Karl vor neun Uhr in Bertaux. Emma wurde
"uber und "uber rot, al{\s} sie ihn sah. Sie l"achelte gezwungen
ein wenig, um ihre Fassung zu bewahren. Rouault umarmte seinen
k"unftigen Schwiegersohn. Die Besprechung der gesch"aftlichen
Punkte wurde verschoben. "Ubrigen{\s} war noch viel Zeit dazu, da
die Hochzeit anstand{\s}halber vor Ablauf von Karl{\s} Trauerjahr
nicht stattfinden konnte, da{\s} hie"s, nicht vor dem n"achsten
Fr"uhjahr.

In dieser Erwartung verging der Winter. Fr"aulein Rouault
besch"aftigte sich mit ihrer Au{\s}steuer. Ein Teil davon wurde in
Rouen bestellt. Die Hemden und Hauben stellte sie nach Schnitten,
die sie sich lieh, selbst her. Wenn Karl zu Besuch kam, plauderte
da{\s} Brautpaar von den Vorbereitungen zur Hochzeit{\s}feier.
E{\s} wurde "uberlegt, in welchem Raume da{\s} Festmahl
stattfinden, wieviel Platten und Sch"usseln auf die Tafel kommen
und wa{\s} f"ur Vorspeisen e{\s} geben solle.

Am liebsten h"atte e{\s} Emma gehabt, wenn die Trauung auf
nacht{\s} zw"olf Uhr bei Fackelschein festgesetzt worden w"are;
aber f"ur solche Romantik hatte Vater Rouault kein Verst"andni{\s}.
Man einigte sich also auf eine Hochzeit{\s}feier, zu der
dreiundvierzig G"aste Einladungen bekamen. Sechzehn Stunden wollte
man bei Tisch sitzen bleiben. Am n"achsten Tage und an den
folgenden sollte e{\s} so weitergehen.



\newpage\begin{center}
{\large \so{Vierte{\s} Kapitel}}\bigskip\bigskip
\end{center}

Die Hochzeit{\s}g"aste stellten sich p"unktlich ein, in Kutschen,
Landauern, Einsp"annern, Gig{\s}, Kremsern mit Ledervorh"angen, in
allerlei Fuhrwerk moderner und vorsintflutlicher Art. Da{\s} junge
Volk au{\s} den n"achsten Nachbard"orfern kam t"uchtig
durchger"uttelt im Trabe in einem Heuwagen angefahren, aufrecht in
einer Reihe stehend, die H"ande an den Seitenstangen, um nicht
umzufallen. Etliche eilten zehn Wegstunden weit herbei, au{\s}
Goderville, Normanville und Cany. Die Verwandten beider Familien
waren samt und sonder{\s} geladen. Freunde, mit denen man
unein{\s} gewesen, vers"ohnte man, und e{\s} war an Bekannte
geschrieben worden, von denen man wer wei"s wie lange nicht{\s}
geh"ort hatte.

Immer wieder vernahm man hinter der Gartenhecke Peitschengeknall.
Eine Weile sp"ater erschien der Wagen im Hoftor. Im Galopp ging
e{\s} bi{\s} zur Freitreppe, wo mit einem Rucke gehalten wurde.
Die Insassen stiegen nach beiden Seiten au{\s}. Man rieb sich die
Knie und turnte mit den Armen. Die Damen, Hauben auf dem Kopfe,
trugen st"adtische Kleider, goldne Uhrketten, Umh"ange mit langen
Enden, die sie sich kreuzweise umgeschlagen hatten, oder
Schal{\s}, die mit einer Nadel auf dem R"ucken festgesteckt waren,
damit sie hinten den Hal{\s} frei lie"sen. Die Knaben, genau so
angezogen wie ihre V"ater, f"uhlten sich in ihren R"ocken
sichtlich unbehaglich; viele hatten an diesem Tage gar zum ersten
Male richtige Stiefel an. Ihnen zur Seite gewahrte man vierzehn-
bi{\s} sechzehnj"ahrige M"adchen, offenbar ihre Basen oder
"alteren Schwestern, in ihren wei"sen Firmelkleidern, die man zur
Feier de{\s} Tage{\s} um ein St"uck l"anger gemacht hatte, alle
mit roten versch"amten Gesichtern und pomadisiertem Haar, voller
Angst, sich die Handschuhe nicht zu beschmutzen. Da nicht Knechte
genug da waren, um all die Wagen gleichzeitig abzuspannen,
streiften die Herren die Rock"armel hoch und stellten ihre Pferde
eigenh"andig ein. Je nach ihrem gesellschaftlichen Range waren sie
in Fr"acken, R"ocken oder Jackett{\s} erschienen. Manche in
ehrw"urdigen Bratenr"ocken, die nur bei ganz besonderen
Festlichkeiten feierlich au{\s} dem Schranke geholt wurden; ihre
langen Sch"o"se flatterten im Winde, die Kragen daran sahen au{\s}
wie Hal{\s}panzer, und die Taschen hatten den Umfang von S"acken.
E{\s} waren auch Jacken au{\s} derbem Tuch zum Vorschein gekommen,
meist im Verein mit messingumr"anderten M"utzen; fernerhin ganz
kurze R"ocke mit zwei dicht nebeneinandersitzenden gro"sen
Kn"opfen hinten in der Taille und mit Sch"o"sen, die so
au{\s}schauten, al{\s} habe sie der Zimmermann mit einem Beile
au{\s} dem Ganzen herau{\s}gehackt. Ein paar (einige wenige)
G"aste -- und da{\s} waren solche, die dann an der Festtafel
gewi"s am alleruntersten Ende zu sitzen kamen -- trugen nur
Sonntag{\s}blusen mit breitem Umlegekragen und R"uckenfalten unter
dem G"urtel.

Die steifen Hemden w"olbten sich "uber den Br"usten wie K"urasse.
Durchweg hatte man sich unl"angst da{\s} Haar schneiden lassen (um
so mehr standen die Ohren von den Sch"adeln ab!), und alle waren
ordentlich rasiert. Manche, die noch im Dunkeln aufgestanden
waren, hatten offenbar beim Rasieren nicht Licht genug gehabt und
hatten sich unter der Nase die Kreuz und die Quer geschnitten oder
hatten am Kinn L"ocher in der Haut bekommen, gro"s wie
Talerst"ucke. Unterweg{\s} hatten sich diese Wunden in der
frischen Morgenluft ger"otet, und so leuchteten auf den breiten
blassen Bauerngesichtern gro"se rote Flecke.

Da{\s} Gemeindeamt lag eine halbe Stunde vom Pachthofe entfernt.
Man begab sich zu Fu"s dahin und ebenso zur"uck, nachdem die
Zeremonie in der Kirche stattgefunden hatte. Der Hochzeit{\s}zug
war anfang{\s} wohlgeordnet gewesen. Wie ein bunte{\s} Band hatte
er sich durch die gr"unen Felder geschl"angelt. Aber bald lockerte
er sich und zerfiel in verschiedene Gruppen, von denen sich die
letzten plaudernd versp"ateten. Ganz vorn schritt ein Spielmann
mit einer buntbeb"anderten Fiedel. Dann kamen die Brautleute,
darauf die Verwandten, dahinter ohne besondre Ordnung die Freunde
und zuletzt die Kinder, die sich damit vergn"ugten, "Ahren au{\s}
den Kornfeldern zu rupfen oder sich zu jagen, wenn e{\s} niemand
sah. Emma{\s} Kleid, da{\s} etwa{\s} zu lang war, schleppte ein
wenig auf der Erde hin. Von Zeit zu Zeit blieb sie stehen, um den
Rock aufzuraffen. Dabei la{\s} sie behutsam mit ihren
behandschuhten H"anden die kleinen stacheligen Distelbl"atter ab,
die an ihrem Kleide h"angen geblieben waren. W"ahrenddem stand
Karl mit leeren H"anden da und wartete, bi{\s} sie fertig war.
Vater Rouault trug einen neuen Zylinderhut und einen schwarzen
Rock, dessen "Armel ihm bi{\s} an die Fingern"agel reichten. Am
Arm f"uhrte er Frau Bovary senior. Der alte Herr Bovary, der im
Grunde seine{\s} Herzen{\s} die ganze Sippschaft um sich herum
verachtete, war einfach in einem uniform"ahnlichen einreihigen
Rock erschienen. Ihm zur Seite schritt eine junge blonde B"auerin,
die er mir derben Galanterien traktierte. Sie h"orte ihm
respektvoll zu, wu"ste aber in ihrer Verlegenheit gar nicht,
wa{\s} sie sagen sollte. Die "ubrigen G"aste sprachen von ihren
Gesch"aften oder ulkten sich gegenseitig an, um sich in fidele
Stimmung zu bringen. Wer aufhorchte, h"orte in einem fort da{\s}
Tirilieren de{\s} Spielmanne{\s}, der auch im freien Felde
weitergeigte. Sooft er bemerkte, da"s die Gesellschaft weit hinter
ihm zur"uckgeblieben war, machte er Halt und sch"opfte Atem.
Umst"andlich rieb er seinen Fiedelbogen mit Kolophonium ein, damit
die Saiten sch"oner quietschen sollten, und dann setzte er sich
wieder in Bewegung. Er hob und senkte den Hal{\s} seine{\s}
Instrument{\s}, um recht h"ubsch im Takte zu bleiben. Die Fidelei
verscheuchte die V"ogel schon von weitem.

Die Festtafel war unter dem Schutzdache de{\s} Wagenschuppen{\s}
aufgestellt. E{\s} prangten darauf vier Lendenbraten, sech{\s}
Sch"usseln mit H"uhnerfrikassee, eine Platte mit gekochtem
Kalbfleisch, drei Hammelkeulen und in der Mitte, umgeben von vier
Leberw"ursten in Sauerkraut, ein k"ostlich knusprig gebratene{\s}
Spanferkel. An den vier Ecken de{\s} Tische{\s} br"usteten sich
Karaffen mit Branntwein, und in einer langen Reihe von Flaschen
wirbelte perlender Apfelweinsekt, w"ahrend auf der Tafel
bereit{\s} alle Gl"aser im vorau{\s} bi{\s} an den Rand
vollgeschenkt waren. Gro"se Teller mit gelber Creme, die beim
leisesten Sto"s gegen den Tisch zitterte und bebte,
vervollst"andigten die Augenweide. Auf der glatten Oberfl"ache
diese{\s} Dessert{\s} prangten in umschn"orkelten Monogrammen von
Zuckergu"s die Anfang{\s}buchstaben der Namen von Braut und
Br"autigam. F"ur die Torten und Kuchen hatte man einen Konditor
au{\s} Yvetot kommen lassen. Da die{\s} sein Deb"ut in der Gegend
war, hatte er sich ganz besondre M"uhe gegeben. Beim Nachtisch
trug er eigenh"andig ein Prunkst"uck seiner Kunst auf, da{\s} ein
allgemeine{\s} "`Ah!"' hervorrief. Der Unterbau au{\s} blauer
Pappe stellte ein von Sternen au{\s} Goldpapier "ubers"ate{\s}
Tempelchen dar, mit einem S"aulenumgang und Nischen, in denen
Statuen au{\s} Marzipan standen. Im zweiten Stockwerk rundete sich
ein Festung{\s}turm au{\s} Pfefferkuchen, umbaut von einer
Brustwehr au{\s} Bonbon{\s}, Mandeln, Rosinen und
Apfelsinenschnitten. Die oberste Plattform aber kr"onte "uber
einer gr"unen Landschaft au{\s} Wiesen, Felsen und Teichen mit
Nu"sschalenschiffchen darauf (alle{\s} Zuckerwerk): ein niedlicher
Amor, der sich auf einer Schaukel au{\s} Schokolade wiegte. In den
beiden kugelgeschm"uckten Schn"abeln der Schaukel steckten zwei
lebendige Rosenknospen.

Man schmauste bi{\s} zum Abend. Wer von dem zu langen Sitzen
erm"udet war, ging im Hof oder im Garten spazieren oder machte
eine Partie de{\s} in jener Gegend beliebten Pfropfenspiel{\s} mit
und setzte sich dann wieder an den Tisch. Ein paar G"aste
schliefen gegen da{\s} Ende de{\s} Mahle{\s} ein und schnarchten
ganz laut. Aber beim Kaffee war alle{\s} wieder munter. Man sang
Lieder, vollf"uhrte allerlei Kraftleistungen, stemmte schwere
Steine, scho"s Purzelb"aume, hob Schubkarren bi{\s} zur
Schulterh"ohe, erz"ahlte gepfefferte Geschichten und scharwenzelte
mit den Damen.

Vor dem Aufbruch war e{\s} kein leichte{\s} St"uck Arbeit, den
Pferden, die allesamt der allzu reichlich vertilgte Hafer stach,
die Kumte und Geschirre aufzulegen. Die "uberm"utigen Tiere
stiegen, bockten und schlugen au{\s}, w"ahrend die Herren und
Kutscher fluchten und lachten. Die ganze Nacht hindurch gab e{\s}
auf den mondbegl"anzten Landstra"sen in Karriere "uber Stock und
Stein heimrasende Fuhrwerke.

Die nacht"uber in Bertaux bleibenden G"aste zechten am K"uchentische
bi{\s} zum fr"uhen Morgen weiter, w"ahrend die Kinder unter den
B"anken schliefen.

Die junge Frau hatte ihren Vater besonder{\s} gebeten, sie vor den
herk"ommlichen Sp"a"sen zu bewahren. Indessen machte sich ein
Vetter -- ein See\-fisch\-h"and\-ler, der al{\s}
Hoch\-zeit{\s}\-ge\-schenk selbstverst"andlich ein paar Seezungen
gestiftet hatte -- doch daran, einen Mund voll Wasser durch da{\s}
Schl"usselloch de{\s} Brautgemach{\s} zu spritzen. Vater Rouault
erwischte ihn gerade noch rechtzeitig, um ihn daran zu hindern. Er
machte ihm klar, da"s sich derartige Scherze mit der W"urde
seine{\s} Schwiegersohne{\s} nicht vertr"ugen. Der Vetter lie"s
sich durch diese Einw"ande nur widerwillig von seinem Vorhaben
abbringen. In{\s}geheim hielt er den alten Rouault f"ur
aufgeblasen. Er setzte sich unten in eine Ecke mir vier bi{\s}
f"unf andern Unzufriedenen, die w"ahrend de{\s} Mahle{\s} bei der
Wahl der Fleischst"ucke Mi"sgriffe getan hatten. Diese
Ungl"uck{\s}menschen r"asonierten nun alle untereinander auf den
Gastgeber und w"unschten ihm ungeniert alle{\s} "Uble.

Die alte Frau Bovary war den ganzen Tag "uber au{\s} ihrer
Verbissenheit nicht herau{\s}gekommen. Man hatte sie weder bei der
Toilette ihrer Schwiegertochter noch bei den Vorbereitungen zur
Hochzeit{\s}feier um Rat gefragt. Darum zog sie sich zeitig
zur"uck. Ihrem Manne aber fiel e{\s} nicht ein, mit zu verschwinden;
er lie"s sich Zigarren holen und paffte bi{\s} zum Morgen, wozu er
Grog von Kirschwasser trank. Da diese Mischung den Dabeisitzenden
unbekannt war, staunte man ihn erst recht al{\s} Wundertier an.

Karl war kein witziger Kopf, und so hatte er w"ahrend de{\s}
Feste{\s} gar keine gl"anzende Rolle gespielt. Gegen alle die
Neckereien, Sp"a"se, Kalauer, Zweideutigkeiten, Komplimente und
Anulkungen, die ihm der Sitte gem"a"s bei Tische zuteil geworden
waren, hatte er sich alle{\s} andre denn schlagfertig gezeigt. Um
so m"achtiger war seine innere Wandlung. Am andern Morgen war er
offensichtlich wie neugeboren. Er und nicht Emma war tag{\s} zuvor
sozusagen die Jungfrau gewesen. Die junge Frau beherrschte sich
v"ollig und lie"s sich nicht da{\s} geringste anmerken. Die
gr"o"sten Schandm"auler waren sprachlo{\s}; sie standen da wie vor
einem Wundertier. Karl freilich machte au{\s} seinem Gl"uck kein
Hehl. Er nannte Emma "`mein liebe{\s} Frauchen"', duzte sie, lief
ihr "uberallhin nach und zog sie mehrfach abseit{\s}, um allein
mit ihr im Hofe unter den B"aumen ein wenig zu plaudern, wobei er
den Arm vertraulich um ihre Taille legte. Beim Hin- und Hergehen
kam er ihr mit seinem Gesicht ganz nahe und zerdr"uckte mit seinem
Kopfe ihr Hal{\s}tuch.

Zwei Tage nach der Hochzeit brachen die Neuverm"ahlten auf. Karl
konnte seiner Patienten wegen nicht l"anger verweilen. Vater
Rouault lie"s da{\s} Ehepaar in seinem Wagen nach Hau{\s} fahren
und gab ihm pers"onlich bi{\s} Vassonville da{\s} Geleite. Beim
Abschied k"u"ste er seine Tochter noch einmal, dann stieg er
au{\s} und machte sich zu Fu"s auf den R"uckweg.

Nachdem er hundert Schritte gegangen war, blieb er stehen, um dem
Wagen nachzuschauen, der die sandige Stra"se dahinrollte. Dabei
seufzte er tief auf. Er dachte zur"uck an seine eigne Hochzeit, an
l"angstvergangne Tage, an die Zeit der ersten Mutterschaft seiner
Frau. Wie froh war er damal{\s} gewesen. Er erinnerte sich de{\s}
Tage{\s}, wo er mit ihr da{\s} Hau{\s} de{\s} Schwiegervater{\s}
verlassen hatte. Auf dem Ritt in da{\s} eigne Heim, durch den
tiefen Schnee, da hatte er seine Frau hinten auf die Kruppe
seine{\s} Pferde{\s} gesetzt. E{\s} war so um Weihnachten herum
gewesen, und die ganze Gegend war verschneit. Mit der einen Hand
hatte sie sich an ihm festgehalten, in der andern ihren Korb
getragen. Die langen B"ander ihre{\s} normannischen Kopfputze{\s}
hatten im Winde geflattert, und manchmal waren sie ihm um die Nase
geflogen. Und wenn er sich umdrehte, sah er "uber seine Schulter
weg ganz dicht hinter sich ihr niedliche{\s} rosige{\s} Gesicht,
da{\s} unter der Goldborte ihrer Haube still vor sich
hinl"achelte. Wenn sie an die Finger fror, steckte sie die Finger
eine Weile in seinen Rock, ihm dicht an die Brust ... Wie lange
war da{\s} nun her! Wenn ihr Sohn am Leben geblieben w"are, dann
w"are er jetzt drei"sig Jahre alt!

Er blickte sich nochmal{\s} um. Auf der Stra"se war nicht{\s} mehr
zu sehen. Da ward ihm unsagbar traurig zumute. In seinem von dem
vielen Essen und Trinken beschwerten Hirne mischten sich die
z"artlichen Erinnerungen mit schwerm"utigen Gedanken. Einen
Augenblick lang versp"urte er da{\s} Verlangen, den Umweg "uber
den Friedhof zu machen. Aber er f"urchtete sich davor, da"s ihn
die{\s} nur noch tr"ubseliger stimmte, und so ging er auf dem
k"urzesten Wege nach Hause.

Karl und Emma erreichten Toste{\s} gegen sech{\s} Uhr. Die
Nachbarn st"urzten an die Fenster, um die junge Frau Doktor zu
ersp"ahen. Die alte Magd empfing sie unter Gl"uckw"unschen und bat
um Entschuldigung, da"s da{\s} Mittagessen noch nicht ganz fertig
sei. Sie lud die gn"adige Frau ein, einstweilen ihr neue{\s} Heim
in Augenschein zu nehmen.


\newpage\begin{center}
{\large \so{F"unfte{\s} Kapitel}}\bigskip\bigskip
\end{center}

Die Backsteinfassade de{\s} Hause{\s} stand gerade in der
Fluchtlinie der Stra"se, genauer gesagt: der Landstra"se. In der
Hau{\s}flur, gleich an der Hau{\s}t"ure, hingen an einem Halter
ein Kragenmantel, ein Z"ugel, eine M"utze au{\s} schwarzem Leder,
und in einem Winkel auf dem Fu"sboden lagen ein paar Gamaschen,
voll von trocken gewordnem Stra"senschmutz. Rechter Hand lag die
"`Gro"se Stube"', da{\s} hei"st der Raum, in dem die Mahlzeiten
eingenommen wurden und der zugleich al{\s} Wohnzimmer diente. An
den W"anden bauschte sich allenthalben die schlecht aufgeklebte
zeisiggr"une Papiertapete, die an der Decke durch eine Girlande
von blassen Blumen abgeschlossen ward. An den Fenstern
"uberschnitten sich wei"se Kattunvorh"ange, die rote Borten
hatten. Auf dem schmalen Sim{\s} de{\s} Kamin{\s} funkelte eine
Stutzuhr mit dem Kopfe de{\s} Hippokrate{\s} zwischen zwei
versilberten Leuchtern, die unter ovalen Gla{\s}glocken standen.

Auf der andern Seite der Flur lag Karl{\s} Sprechzimmer, ein
kleine{\s} Gemach, etwa sech{\s} Fu"s in der Breite. Drinnen ein
Tisch, drei St"uhle und ein Schreibtischsessel. Die sech{\s}
F"acher eine{\s} B"uchergestell{\s} au{\s} Tannenholz wurden in
der Hauptsache durch die B"ande de{\s} "`Medizinischen
Lexikon{\s}"' au{\s}gef"ullt, die unaufgeschnitten geblieben waren
und durch den mehrfachen Besitzerwechsel, den sie bereit{\s}
erlebt hatten, zerfledderte Umschl"age bekommen hatten. Durch die
d"unne Wand drang Buttergeruch au{\s} der benachbarten K"uche in
da{\s} Sprechzimmer, w"ahrend man dort h"oren konnte, wenn die
Patienten husteten und ihre langen Leiden{\s}geschichten
erz"ahlten.

Nach dem Hofe zu, wo da{\s} Stallgeb"aude stand, lag ein
gro"se{\s} verwahrloste{\s} Gemach, ehemal{\s} Backstube, da{\s}
jetzt al{\s} Holzraum, Keller und Rumpelkammer diente und
vollgepfropft war mit altem Eisen, leeren F"assern, abgetanenem
Ackerger"at und einer Menge andrer verstaubter Dinge, deren
einstigen Zweck man ihnen kaum mehr ansehen konnte.

Der Garten, der mehr in die L"ange denn in die Breite ging, dehnte
sich zwischen zwei Lehmmauern mit Aprikosenspalieren; hinten
begrenzte ihn eine Dornhecke und trennte ihn vom freien Felde.
Mitten im Garten stand ein gemauerter Sockel mit einer Sonnenuhr
darauf, auf einer Schieferplatte. Vier Felder mit d"urftigen
Heckenrosen umg"urteten symmetrisch ein Mittelbeet mit
n"utzlicherem Gew"ach{\s}. Ganz am Ende de{\s} Garten{\s}, in
einer Fichtengruppe, stand eine Tonfigur: ein M"onch, in sein
Brevier vertieft.

Emma stieg die Treppe hinauf. Da{\s} erste Zimmer oben war
"uberhaupt nicht m"obliert, aber im zweiten, der gemeinsamen
Schlafstube, stand in einer Nische mir roten Vorh"angen ein
Himmelbett au{\s} Mahagoniholz. Auf einer Kommode thronte eine mit
Muscheln besetzte kleine Truhe, und auf dem Schreibpult am Fenster
leuchtete in einer Kristallvase ein Strau"s von Orangenbl"uten,
umwunden von einem Seidenbande: ein Hochzeit{\s}bukett, die
Brautblumen der andern! Emma betrachtete sie. Karl bemerkte e{\s},
nahm den Strau"s au{\s} der Vase und trug ihn auf den Oberboden.
W"ahrenddem sa"s sie in einem Lehnstuhl. Ihr eigene{\s}
Brautbukett kam ihr in den Sinn, da{\s} in einer Schachtel
verpackt war. Eben trug man ihr ihre Sachen in da{\s} Zimmer und
baute sie um sie herum auf. Nachdenklich fragte sie sich, wa{\s}
wohl mit ihrem Strau"se gesch"ahe, wenn sie zuf"allig auch bald
st"urbe.

In den ersten Tagen besch"aftigte sich Emma damit, sich allerlei
"Anderungen in ihrem Hause au{\s}zudenken. Sie nahm die
Gla{\s}glocken von den Leuchtern, lie"s neu tapezieren, die Treppe
streichen und B"anke im Garten aufstellen, um die Sonnenuhr herum.

Sie erkundigte sich, ob nicht ein Wasserbassin mit einem
Springbrunnen und Fischen darin angelegt werden k"onnte. Karl
wu"ste, da"s sie gern spazieren fuhr, und da sich gerade eine
Gelegenheit bot, kaufte er ihr einen Wagen. Nach Anbringung von
neuen Laternen und gesteppten Spritzledern sah er ganz au{\s} wie
ein Dogcart.

So war Karl der gl"ucklichste und sorgenloseste Mensch auf der
Welt. Die Mahlzeiten zu zweit, die Abendpromenaden auf der
Landstra"se, die Gesten von Emma{\s} Hand, wenn sie sich da{\s}
Band im Haar zurechtstrich, der Anblick ihre{\s} an einem
Fensterkreuze h"angenden Strohhute{\s} und noch allerhand andre
kleine Dinge, von denen er nie geglaubt h"atte, da"s sie einen
erfreuen k"onnten, all da{\s} trug dazu bei, da"s sein Gl"uck
nicht aufh"orte. Fr"uhmorgen{\s} im Bette, Seite an Seite mit ihr
auf demselben Kopfkissen, sah er zu, wie die Sonnenlichter durch
den blonden Flaum ihrer von den Haubenb"andern halbverdeckten
Wangen huschten. So au{\s} der N"ahe kamen ihm ihre Augen viel
gr"o"ser vor, besonder{\s} beim Erwachen, wenn sich ihre Lider
mehrere Male hintereinander hoben und wieder senkten. Im Schatten
sahen diese Augen schwarz au{\s} und dunkelblau am lichten Tage;
in ihrer Tiefe wurden sie immer dunkler, w"ahrend sie sich nach
der schimmernden Oberfl"ache zu aufhellten. Sein eigene{\s} Auge
verlor sich in diese Tiefe; er sah sich darin gespiegelt, ganz
klein, bi{\s} an die Schultern, mit dem Seidentuche, da{\s} er
sich um den Kopf geschlungen hatte, und dem Kragen seine{\s} offen
stehenden Nachthemde{\s}.

Wenn er aufgestanden war, schaute sie ihm vom Fenster au{\s} nach,
um ihn fortreiten zu sehen. Eine Weile blieb sie, auf da{\s}
Fensterbrett gest"utzt, so stehen, in ihrem Morgenkleide, da{\s}
sie leicht umflo"s, zwischen zwei Geranienst"ocken. Karl unten auf
der Stra"se schnallte sich an einem Prellsteine seine Sporen an.
Emma sprach in einem fort zu ihm von oben herunter, w"ahrenddem
sie mit ihrem Munde eine Bl"ute oder ein Bl"attchen von den
Geranien abzupfte und ihm zublie{\s}. Da{\s} Abgerupfte schwebte
und schaukelte sich in der Luft, flog in kleinen Kreisen wie ein
Vogel und blieb schlie"slich im Fallen in der ungepflegten M"ahne
der alten Schimmelstute h"angen, die unbeweglich vor der
Hau{\s}t"ure wartete. Karl sa"s auf und warf seiner Frau eine
Ku"shand zu. Sie antwortete winkend und schlo"s da{\s} Fenster. Er
ritt ab.

Dann, auf der endlo{\s} sich hinwindenden staubigen Landstra"se,
in den Hohlwegen, "uber denen sich die B"aume zu einem Laubdache
schlossen, auf den Feldwegen, wo ihm da{\s} Korn zu beiden Seiten
die Knie streifte, die warme Sonne auf dem R"ucken, die frische
Morgenluft in der Nase und da{\s} Herz noch voll von den Freuden
der Nacht, friedsamen Gem"ut{\s} und befriedigter Sinne, -- da
geno"s er all sein Gl"uck abermal{\s}, just wie einer, der nach
einem Schlemmermahle den Wohlgeschmack der Tr"uffeln, die er
bereit{\s} verdaut, noch auf der Zunge hat.

Wa{\s} hatte er bi{\s}her an Gl"uck in seinem Leben erfahren? War
er denn im Gymnasium gl"ucklich gewesen, wo er sich in der Enge
hoher Mauern so einsam gef"uhlt hatte, unter seinen Kameraden, die
reicher und st"arker waren al{\s} er, "uber seine b"auerische
Au{\s}sprache lachten, sich "uber seinen Anzug lustig machten und
zur Besuch{\s}zeit mit ihren M"uttern plauderten, die mit Kuchen
in der Tasche kamen? Oder etwa sp"ater al{\s} Student der Medizin,
wo er niemal{\s} Geld genug im Beutel gehabt hatte, um irgendein
kleine{\s} M"adel zum Tanz f"uhren zu k"onnen, da{\s} seine
Geliebte geworden w"are? Oder gar w"ahrend der vierzehn Monate, da
er mit der Witwe verheiratet war, deren F"u"se im Bett kalt wie
Ei{\s}klumpen gewesen waren? Aber jetzt, jetzt besa"s er f"ur
immerdar seine h"ubsche Frau, in die er vernarrt war. Seine Welt
fand ihre Grenzen mit der Saumlinie ihre{\s} seidnen
Unterrock{\s}, und doch machte er sich den Vorwurf, er liebe sie
nicht genug. Und so "uberkam ihn unterweg{\s} die Sehnsucht nach
ihr. Spornstreich{\s} ritt er heimw"art{\s}, rannte die Treppe
hinauf, mit klopfendem Herzen ... Emma sa"s in ihrem Zimmer bei
der Toilette. Er schlich sich auf den Fu"sspitzen von hinten an
sie heran und k"u"ste ihr den Nacken. Sie stie"s einen Schrei
au{\s}.

Er konnte e{\s} nicht lassen, immer wieder ihren Kamm, ihre Ringe,
ihr Hal{\s}tuch zu bef"uhlen. Manchmal k"u"ste er sie t"uchtig auf
die Wangen, oder er reihte eine Menge kleiner K"usse gleichsam
aneinander, die ihren nackten Arm in seiner ganzen L"ange von den
Fingerspitzen bi{\s} hinauf zur Schulter bedeckten. Sie wehrte ihn
ab, l"achelnd und gelangweilt, wie man ein kleine{\s} Kind
zur"uckdr"angt, da{\s} sich an einen anklammert.

Vor der Hochzeit hatte sie fest geglaubt, Liebe zu ihrem Karl zu
empfinden. Aber al{\s} da{\s} Gl"uck, da{\s} sie au{\s} dieser
Liebe erwartete, au{\s}blieb, da mu"ste sie sich doch get"auscht
haben. So dachte sie. Und sie gab sich M"uhe, zu ergr"ubeln, wo
eigentlich in der Wirklichkeit all da{\s} Sch"one sei, da{\s} in
den Romanen mit den Worten Gl"uckseligkeit, Leidenschaft und
Rausch so verlockend geschildert wird.


\newpage\begin{center}
{\large \so{Se{ch}{st}e{\s} Kapitel}}\bigskip\bigskip
\end{center}

Emma hatte "`Paul und Virginia"' gelesen und in ihren Tr"aumereien
alle{\s} vor sich gesehen: die Bambu{\s}h"utte, den Neger Domingo,
den Hund Fideli{\s}. In{\s}besondre hatte sie sich in die
z"artliche Freundschaft irgendeine{\s} guten Kameraden
hineingelebt, der f"ur sie rote Fr"uchte auf "uberturmhohen
B"aumen pfl"uckte und barfu"s durch den Sand gelaufen kam, ihr ein
Vogelnest zu bringen.

Al{\s} sie dreizehn Jahre alt war, brachte ihr Vater sie zur
Stadt, um sie in da{\s} Kloster zu geben. Sie stiegen in einem
Gasthofe im Viertel Saint-Gervai{\s} ab, wo sie beim Abendessen
Teller vorgesetzt bekamen, auf denen Szenen au{\s} dem Leben
de{\s} Fr"aulein{\s} von Lavalli\`ere gemalt waren. Alle diese
legendenhaften Bilder, hier und da von Messerkritzeln besch"adigt,
verherrlichten Fr"ommigkeit, Gef"uhl{\s}"uberschwang und
h"ofischen Prunk.

In der ersten Zeit ihre{\s} Klosteraufenthalt{\s} langweilte sie
sich nicht im geringsten. Sie f"uhlte sich vielmehr in der
Gesellschaft der g"utigen Schwestern ganz behaglich, und e{\s} war
ihr ein Vergn"ugen, wenn man sie mit in die Kapelle nahm, wohin
man vom Refektorium durch einen langen Kreuzgang gelangte. In den
Freistunden spielte sie nur h"ochst selten, im Katechi{\s}mu{\s}
war sie al{\s}bald sehr bewandert, und auf schwierige Fragen war
sie e{\s}, die dem Herrn Pfarrer immer zu antworten wu"ste. So
lebte sie, ohne in die Welt hinau{\s}zukommen, in der lauen
Atmosph"are der Schulstuben und unter den blassen Frauen mit ihren
Rosenkr"anzen und Messingkreuzchen, und langsam versank sie in den
mystischen Traumzustand, der sich um die Weihrauchd"ufte, die
K"uhle der Weihwasserbecken und den Kerzenschimmer webt. Statt der
Messe zuzuh"oren, betrachtete sie die frommen himmelblau
umr"anderten Vignetten ihre{\s} Gebetbuche{\s} und verliebte sich
in da{\s} kranke Lamm Gotte{\s}, in da{\s} von Pfeilen durchbohrte
Herz Jesu und in den armen Christu{\s} selber, der, sein Kreuz
schleppend, zusammenbricht. Um sich zu kasteien, versuchte sie,
einen ganzen Tag lang ohne Nahrung au{\s}zuhalten. Sie zerbrach
sich den Kopf, um irgendein Gel"ubde zu ersinnen, da{\s} sie auf
sich nehmen wollte.

Wenn sie zur Beichte ging, erfand sie allerlei kleine S"unden, nur
damit sie l"anger im Halbdunkel knien durfte, die H"ande gefaltet,
da{\s} Gesicht an{\s} Gitter gepre"st, unter dem fl"usternden
Priester. Die Gleichnisse vom Br"autigam, vom Gemahl, vom
himmlischen Geliebten und von der ewigen Hochzeit, die in den
Predigten immer wiederkehrten, erweckten im Grunde ihrer Seele
geheimni{\s}volle s"u"se Schauer.

Abend{\s}, vor dem Ave-Maria, ward im Arbeit{\s}saal au{\s} einem
frommen Buche vorgelesen. An den Wochentagen la{\s} man au{\s} der
Biblischen Geschichte oder au{\s} den "`Stunden der Andacht"'
de{\s} Abb\'e Frayssinou{\s} und Sonntag{\s} zur Erbauung au{\s}
Chateaubriand{\s} "`Geist de{\s} Christentum{\s}"'. Wie
andacht{\s}voll lauschte sie bei den ersten Malen den klangreichen
Klagen romantischer Schwermut, die wie ein Echo au{\s} Welt und
Ewigkeit erschallten! W"are Emma{\s} Kindheit im Hinterst"ubchen
eine{\s} Kramladen{\s} in einem Gesch"aft{\s}viertel
dahingeflossen, dann w"are da{\s} junge M"adchen vermutlich der
Naturschw"armerei verfallen, die zumeist in literarischer Anregung
ihre Quelle hat. So aber kannte sie da{\s} Land zu gut: da{\s}
Bl"oken der Herden, die Milch- und Landwirtschaft. An friedsame
Vorg"ange gew"ohnt, gewann sie eine Vorliebe f"ur da{\s} dem
Entgegengesetzte: da{\s} Abenteuerliche. So liebte sie da{\s} Meer
einzig um der wilden St"urme willen und da{\s} Gr"un, nur wenn
e{\s} zwischen Ruinen sein Dasein fristete. E{\s} war ihr ein
Bed"urfni{\s}, au{\s} den Dingen einen egoistischen Genu"s zu
sch"opfen, und sie warf alle{\s} al{\s} unn"utz beiseite, wa{\s}
nicht unmittelbar zum Labsal ihre{\s} Herzen{\s} diente. Ihre
Eigenart war eher sentimental al{\s} "asthetisch; sie sp"urte
lieber seelischen Erregungen al{\s} Landschaften nach.

Im Kloster gab e{\s} nun eine alte Jungfer, die sich alle vier
Wochen auf acht Tage einstellte, um die W"asche au{\s}zubessern.
Da sie einer alten Adel{\s}familie entstammte, die in der
Revolution zugrunde gegangen war, wurde sie von der Geistlichkeit
beg"onnert. Sie a"s mit im Refektorium, an der Tafel der frommen
Schwestern, und pflegte mit ihnen nach Tisch ein Plauderst"undchen
zu machen, bevor sie wieder an ihre Arbeit ging. Oft geschah e{\s}
auch, da"s sich die Pension"arinnen au{\s} der Arbeit{\s}stube
stahlen und die Alte aufsuchten. Sie wu"ste galante Chanson{\s}
au{\s} dem \begin{antiqua}ancien r\'egime\end{antiqua}
au{\s}wendig und sang ihnen welche halbleise vor, ohne dabei ihre
Flickarbeit zu vernachl"assigen. Sie erz"ahlte Geschichten, wu"ste
stet{\s} Neuigkeiten, "ubernahm allerhand Besorgungen in der Stadt
und lieh den gr"o"seren M"adchen Romane, von denen sie immer ein
paar in den Taschen ihrer Sch"urze bei sich hatte. In den
Ruhepausen ihrer T"atigkeit verschlang da{\s} gute Fr"aulein
selber schnell ein paar Kapitel. Darin wimmelte e{\s} von
Liebschaften, Liebhabern, Liebhaberinnen, von verfolgten Damen,
die in einsamen Pavillonen ohnm"achtig, und von Postillionen, die
an allen Ecken und Enden gemordet wurden, von edlen Rossen, die
man auf Seite f"ur Seite zuschanden ritt, von d"usteren W"aldern,
Herzen{\s}k"ampfen, Schw"uren, Schluchzen, Tr"anen und K"ussen,
von Gondelfahrten im Mondenschein, Nachtigallen in den B"uschen,
von hohen Herren, die wie L"owen tapfer und sanft wie Bergschafe
waren, dabei tugendsam bi{\s} in{\s} Wunderbare, immer k"ostlich
gekleidet und ganz unbeschreiblich tr"anenselig. Ein halbe{\s}
Jahr lang beschmutzte sich die f"unfzehnj"ahrige Emma ihre Finger
mit dem Staube dieser alten Scharteken. Dann geriet ihr Walter
Scott in die H"ande, und nun berauschte sie sich an
geschichtlichen Begebenheiten im Banne von Burgzinnen,
Ritters"alen und Minnes"angern. Am liebsten h"atte sie in einem
alten Herrensitze gelebt, geh"ullt in schlanke Gew"ander wie jene
Edeldamen, die, den Ellenbogen auf den Fensterstein gest"utzt und
da{\s} Kinn in der Hand, unter Kleeblattbogen ihre Tage
vertr"aumten und in die Fernen der Landschaft hinau{\s}schauten,
ob nicht ein Ritter{\s}mann mit wei"ser Helmzier dahergest"urmt
k"ame auf einem schwarzen Ro"s. Damal{\s} trieb sie einen wahren
Kult mit Maria Stuart; ihre Verehrung von ber"uhmten oder
ungl"ucklichen Frauen ging bi{\s} zur Schw"armerei. Die Jungfrau
von Orlean{\s}, Heloise, Agne{\s} Sorel, die sch"one Ferronni\`ere
und Clemence Isaure leuchteten wie strahlende Meteore in dem
grenzenlosen Dunkel ihrer Geschicht{\s}unkenntnisse. Fast ganz im
Lichtlosen und ohne Beziehungen zueinander schwebten ferner in
ihrer Vorstellung: der heilige Ludwig mit seiner Eiche, der
sterbende Ritter Bayard, ein paar grausame Taten Ludwig{\s} de{\s}
Elften, irgendeine Szene au{\s} der Bartholom"au{\s}nacht, der
Helmbusch Heinrich{\s} de{\s} Vierten, dazu unau{\s}l"oschlich die
Erinnerung an die gemalten Teller mit den Verherrlichungen
Ludwig{\s} de{\s} Vierzehnten.

In den Romanzen, die Emma in den Musikstunden sang, war immer die
Rede von Englein mit goldenen Fl"ugeln, von Madonnen, Lagunen und
Gondolieren. Sie waren musikalisch nicht{\s} wert, aber so banal
ihr Text und so reizlo{\s} ihre Melodien auch sein mochten: die
Realit"aten de{\s} Leben{\s} hatten in ihnen den phantastischen
Zauber der Sentimentalit"at. Etliche ihrer Kameradinnen
schmuggelten lyrische Almanache in da{\s} Kloster ein, die sie
al{\s} Neujahr{\s}geschenke bekommen hatten. Da"s man sie heimlich
halten mu"ste, war die Hauptsache dabei. Sie wurden im Schlafsaal
gelesen. Emma nahm die sch"onen Atla{\s}einb"ande nur behutsam in
die Hand und lie"s sich von den Namen der unbekannten Autoren
fa{\s}\/zinieren, die ihre Beitr"age zumeist al{\s} Grafen und
Barone signiert hatten. Da{\s} Herz klopfte ihr, wenn sie da{\s}
Seidenpapier von den Kupfern darin leise aufblie{\s}, bi{\s} e{\s}
sich bauschte und langsam auf die andre Seite sank. Auf einem der
Stiche sah man einen jungen Mann in einem M"antelchen, wie er
hinter der Br"ustung eine{\s} Altan{\s} ein wei"s gekleidete{\s}
junge{\s} M"adchen mit einer Tasche am G"urtel an sich dr"uckte;
auf anderen waren Bildnisse von ungenannten blondlockigen
englischen Lady{\s}, die unter runden Strohh"uten mit gro"sen
hellen Augen hervorschauten. Andre sah man in flotten Wagen durch
den Park fahren, wobei ein Windspiel vor den Pferden hersprang,
die von zwei kleinen Groom{\s} in wei"sen Hosen kutschiert wurden.
Andre tr"aumten auf dem Sofa, ein offene{\s} Briefchen neben sich,
und himmelten durch da{\s} halb offene, schwarz umh"angte Fenster
den Mond an. Wieder andre, Unschuld{\s}kinder, krauten, eine
Tr"ane auf der Wange, durch da{\s} Gitter eine{\s} gotischen
K"afig{\s} ein Turtelt"aubchen oder zerzupften, den Kopf
versch"amt geneigt, mit koketten Fingern, die wie
Schnabelschuhspitzen nach oben gebogen waren, eine Marguerite.
Alle{\s} m"ogliche andre zeigten die "ubrigen Stiche: Sultane mit
langen Pfeifen, unter Lauben gelagert, Bajaderen in den Armen;
Giaur{\s}, T"urkens"abel, phrygische M"utzen, nicht zu vergessen
die faden heroischen Landschaften, auf denen Palmen und Fichten,
Tiger und L"owen friedlich beieinanderstehen, und Minarett{\s} am
Horizonte und r"omische Ruinen im Vordergrunde eine Gruppe
lagernder Kamele "uberragen, w"ahrend auf der einen Seite ein
wohlgepflegte{\s} St"uck Urwald steht, auf der andern ein See,
eine Riesensonne mit stechenden Strahlen dar"uber und auf seiner
stahlblauen, hie und da wei"s aufsch"aumenden Flut, in die Ferne
verstreut, gleitende Schw"ane~...

Da{\s} matte Licht der Lampe, die zu Emma{\s} H"aupten an der Wand
hing, blinzelte auf alle diese weltlichen Bilder, die ein{\s} nach
dem andern an ihr vor"uberzogen, in de{\s} Schlafsaale{\s} Stille,
in die kein Ger"ausch drang, h"ochsten{\s} da{\s} ferne Rollen
eine{\s} sp"aten Fuhrwerk{\s}.

Al{\s} ihr die Mutter starb, weinte Emma die ersten Tage viel. Sie
lie"s sich eine Locke der Verstorbenen in einen Gla{\s}rahmen
fassen, schrieb ihrem Vater einen Brief ganz voller wehm"utiger
Betrachtungen "uber da{\s} Leben und bat ihn, man m"oge sie
dereinst in demselben Grabe bestatten. Der gute Mann dachte, sie
sei krank, und besuchte sie. Emma empfand eine innere Befriedigung
darin, da"s sie mit einem Male emporgehoben worden war in die
hohen Regionen einer seltenen Gef"uhl{\s}welt, in die
Alltag{\s}herzen niemal{\s} gelangen. Sie verlor sich in
Lamartinischen R"uhrseligkeiten, h"orte Harfenkl"ange "uber den
Weihern und Schwanenges"ange, die Klagen de{\s} fallenden
Laube{\s}, die Himmelfahrten jungfr"aulicher Seelen und die Stimme
de{\s} Ewigen, die in den Tiefen fl"ustert.

Eine{\s} Tage{\s} jedoch ward ihr alle{\s} da{\s} langweilig, aber
ohne sich{\s} einzugestehen, und so blieb sie dabei zun"achst
au{\s} Gewohnheit, dann au{\s} Eitelkeit, und schlie"slich war sie
"uberrascht, da"s sie den inneren Frieden wiedergefunden hatte und
da"s ihr Herz ebensowenig schwerm"utig war wie ihre jugendliche
Stirne runzelig.

Die frommen Schwestern, die stark auf Emma{\s} heilige Mission
gehofft hatten, bemerkten zu ihrem h"ochsten Befremden, da"s
Fr"aulein Rouault ihrem Einflu"s zu entschl"upfen drohte. Man
hatte ihr allzu reichliche Gebete, Andacht{\s}lieder, Predigten
und Fasten angedeihen lassen, ihr zu trefflich vorgeredet, welch
gro"se Verehrung die Heiligen und M"artyrer gen"ossen, und ihr zu
vorz"ugliche Ratschl"age gegeben, wie man den Leib kasteie und die
Seele der ewigen Seligkeit zuf"uhre; und so ging e{\s} mit ihr wie
mit einem Pferd, da{\s} man zu straff an die Kandare genommen hat:
sie blieb pl"otzlich stehen und machte nicht mehr mit.

Bei aller Schw"armerei war sie doch eine Verstande{\s}natur; sie
hatte die Kirche wegen ihrer Blumen, die Musik wegen der
Liedertexte und die Dichterwerke wegen ihrer sinnlichen Wirkung
geliebt. Ihr Geist emp"orte sich gegen die Mysterien de{\s}
Glauben{\s}, und noch mehr lehnte sie sich nunmehr gegen die
Klosterzucht auf, die ihrem tiefsten Wesen v"ollig zuwider war.
Al{\s} ihr Vater sie au{\s} dem Kloster nahm, hatte man
durchau{\s} nicht{\s} dagegen; die Oberin fand sogar, Emma habe
e{\s} in der letzten Zeit an Ehrfurcht vor der Schwesternschaft
recht fehlen lassen.

Wieder zu Hause, gefiel sich da{\s} junge M"adchen zun"achst
darin, da{\s} Gesinde zu kommandieren, bald jedoch ward sie de{\s}
Landleben{\s} "uberdr"ussig, und nun sehnte sie sich nach dem
Kloster zur"uck. Al{\s} Karl zum ersten Male da{\s} Gut betrat,
war sie just "uberzeugt, da"s sie alle Illusionen verloren habe,
da"s e{\s} nicht{\s} mehr auf der Welt g"abe, wa{\s} ihr Hirn oder
Herz r"uhren k"onne. Dann aber waren da{\s} mit jedem neuen
Zustande verbundene wirre Gef"uhl und die Unruhe, die sich ihrer
diesem Manne gegen"uber bem"achtigte, stark genug, um in ihr den
Glauben zu erwecken: endlich sei jene wunderbare Leidenschaft in
ihr erstanden, die bi{\s}her nicht ander{\s} al{\s} wie ein
Riesenvogel mit rosigem Gefieder hoch in der Herrlichkeit
himmlischer Traumfernen geschwebt hatte. Doch jetzt, in ihrer Ehe,
hatte sie keine Kraft zu glauben, da"s die Friedsamkeit, in der
sie hinlebte, da{\s} ertr"aumte Gl"uck sei.


\newpage\begin{center}
{\large \so{Siebente{\s} Kapitel}}\bigskip\bigskip
\end{center}

Zuweilen machte sie sich Gedanken, ob da{\s} wirklich die
sch"onsten Tage ihre{\s} Leben{\s} sein sollten: ihre
Flitterwochen, wie man zu sagen pflegt. Um ihre Wonnen zu sp"uren,
h"atten sie wohl in jene L"ander mit klangvollen Namen reisen
m"ussen, wo der Morgen nach der Hochzeit in s"u"sem Nicht{\s}tun
verrinnt. Man f"ahrt gem"achlich in einer Postkutsche mit
blauseidnen Vorh"angen die Gebirg{\s}stra"sen hinauf und lauscht
dem Lied de{\s} Postillion{\s}, da{\s} in den Bergen zusammen mit
den Herdenglocken und dem dumpfen Rauschen de{\s} Gie"sbach{\s}
sein Echo findet. Wenn die Sonne sinkt, atmet man am Golf den Duft
der Limonen, und dann nacht{\s} steht man auf der Terrasse einer
Villa am Meere, einsam zu zweit, mit verschlungenen H"anden,
schaut zu den Gestirnen empor und baut Luftschl"osser. E{\s} kam
ihr vor, al{\s} seien nur gewisse Erdenwinkel Heimst"atten de{\s}
Gl"uck{\s}, genau so wie bestimmte Pflanzen nur an sonnigen Orten
gedeihen und nirgend{\s} ander{\s}. Warum war e{\s} ihr nicht
beschieden, sich auf den Altan eine{\s} Schweizerh"au{\s}chen{\s}
zu lehnen oder ihre Tr"ubsal in einem schottischen Landhause zu
vergessen, an der Seite eine{\s} Gatten, der einen langen
schwarzen Gehrock, feine Schuhe, einen eleganten Hut und
Manschettenhemden tr"uge?

Alle diese Gr"ubeleien h"atte sie wohl irgendwem anvertrauen
m"ogen. H"atte sie aber ihr namenlose{\s} Unbehagen, da{\s} sich
aller Augenblicke neu formte wie leichte{\s} Gew"olk und da{\s}
wie der Wind wirbelte, in Worte zu fassen verstanden? Ach, e{\s}
fehlten ihr die Worte, die Gelegenheit, der Mut! Ja, wenn Karl
gewollt h"atte, wenn er eine Ahnung davon gehabt h"atte, wenn sein
Blick nur ein einzige{\s}mal ihren Gedanken begegnet w"are, dann
h"atte sich alle{\s} da{\s}, so meinte sie, sofort von ihrem
Herzen lo{\s}gel"ost wie eine reife Frucht vom Spalier, wenn eine
Hand daran r"uhrt. So aber ward die innere Entfremdung, die sie
gegen ihren Mann empfand, immer gr"o"ser, je intimer ihr
eheliche{\s} Leben wurde.

Karl{\s} Art zu sprechen war platt wie da{\s} Trottoir auf der
Stra"se: Allerwelt{\s}gedanken und Allt"aglichkeiten, die
niemanden r"uhrten, "uber die kein Mensch lachte, die nie einen
Nachklang erweckten. Solange er in Rouen gelebt hatte, sagte er,
h"atte er niemal{\s} den Drang versp"urt, ein Pariser Gastspiel im
Theater zu sehen. Er konnte weder schwimmen noch fechten; er war
auch kein Pistolensch"utze, und gelegentlich kam e{\s} zutage,
da"s er Emma einen Au{\s}druck de{\s} Reitsport{\s} nicht
erkl"aren konnte, der ihr in einem Romane begegnet war. Mu"s ein
Mann nicht vielmehr alle{\s} kennen, auf allen Gebieten bewandert
sein und seine Frau in die gro"sen Leidenschaften de{\s}
Leben{\s}, in seine erlesensten Gen"usse und in alle Geheimnisse
einweihen? Der ihre aber lehrte sie nicht{\s}, verstand von
nicht{\s} und erstrebte nicht{\s}. Er glaubte, sie sei gl"ucklich,
inde{\s} sie sich "uber seine satte Tr"agheit emp"orte, seinen
zufriedenen Stumpfsinn, ja selbst "uber die Wonnen, die sie ihm
gew"ahrte.

Manchmal zeichnete sie. E{\s} belustigte ihn ungemein,
dabeizustehen und zuzusehn, wie sie sich "uber da{\s} Blatt beugte
oder wie sie die Augen zukniff und ihr Werk kritisch betrachtete
oder wie sie mit den Fingern Brotk"ugelchen drehte, die sie zum
Verwischen brauchte. Wenn sie am Klavier sa"s, war sein Ent\/z"ucken
um so gr"o"ser, je geschwinder ihre H"ande "uber die Tasten
sprangen. Dann trommelte sie ordentlich auf dem Klavier herum und
machte ein H"ollenkonzert. Da{\s} alte Instrument dr"ohnte und
wackelte, und wenn da{\s} Fenster offen stand, h"orte man da{\s}
Spiel im ganzen Dorfe. Der Gemeindediener, der im blo"sen Kopfe
und in Pantoffeln, Akten unterm Arme, "uber die Stra"se humpelte,
blieb stehen und lauschte.

Dabei war Emma eine vorz"ugliche Hau{\s}frau. Sie schickte die
Liquidationen an die Patienten au{\s} und zwar in h"oflichster
Briefform, die gar nicht an Rechnungen erinnerte. Wenn sie
Sonntag{\s} irgendwen au{\s} der Nachbarschaft zu Gaste hatten,
wu"ste sie e{\s} immer einzurichten, da"s etwa{\s} Besondere{\s}
auf den Tisch kam. Sie schichtete auf Weinbl"attern Pyramiden von
Reineclauden auf und verstand, die eingezuckerten Fr"uchte so
au{\s} ihren B"uchsen zu st"urzen, da"s sie noch in der Form
serviert wurden. Demn"achst sollten auch kleine Waschschalen f"ur
den Nachtisch angeschafft werden. Mit alledem vermehrte sie da{\s}
"offentliche Ansehen ihre{\s} Manne{\s}. Schlie"slich fing er
selbst an, mehr und mehr Respekt vor sich zu bekommen, weil er
solch eine Frau besa"s. Mit Stolz zeigte er zwei kleine
Bleistift\/zeichnungen Emma{\s}, die er in ziemlich breite Rahmen
hatte fassen lassen und in der Gro"sen Stube an langen gr"unen
Schnuren an den W"anden aufgeh"angt hatte. Wenn die Kirche zu Ende
war, sah man Herrn Bovary in sch"ongestickten Hau{\s}schuhen vor
der Hau{\s}t"ure stehen.

Er kam sp"at heim, um zehn Uhr, zuweilen um Mitternacht. Dann a"s
er noch zu Abend, und da da{\s} Dienstm"adchen bereit{\s} Schlafen
gegangen war, bediente ihn Emma selber. Er pflegte seinen Rock
au{\s}zuziehen und sich{\s} zum Essen bequem zu machen. Kauend
z"ahlte er gewissenhaft alle Menschen auf, denen er tag{\s}"uber
begegnet war, nannte die Ortschaften, durch die er geritten, und
wiederholte die Rezepte, die er verschrieben hatte. Zufrieden mit
sich selbst, verzehrte er sein Gulasch bi{\s} auf den letzten
Rest, schabte sich den K"ase sauber, schmauste einen Apfel und
trank die Weinkaraffe leer, worauf er zu Bett ging, sich auf{\s}
Ohr legte und zu schnarchen begann. Wenn er fr"uhmorgen{\s}
aufmachte, hing ihm da{\s} Haar wirr "uber die Stirn.

Er trug stet{\s} derbe hohe Stiefel, die in der Kn"ochelgegend
zwei Falten hatten; in den Sch"aften waren sie steif und
geradlinig, al{\s} ob ein Holzbein drinnen st"ake. Er pflegte zu
sagen: "`Die sind hier auf dem Lande gut genug!"'

Seine Mutter best"arkte ihn in seiner Sparsamkeit. Wie vordem kam
sie zu Besuch, wenn e{\s} bei ihr zu Hause kleine Mi"slichkeiten
gegeben hatte. Allerding{\s} hegte die alte Frau Bovary gegen ihre
Schwiegertochter sichtlich ein Vorurteil. Sie war ihr "`f"ur ihre
Verh"altnisse ein bi"schen zu gro"sartig."' Mit Holz, Licht und
dergleichen werde "`wie in einem herrschaftlichen Hause
gew"ustet."' Und mit den Kohlen, die in der K"uche verbraucht
w"urden, k"onne man zwei Dutzend G"ange kochen! Sie ordnete ihr
den W"ascheschrank und hielt Vortr"age, wie man dem Fleischer auf
die Finger zu sehen habe, wenn er da{\s} Fleisch brachte. Emma
nahm diese guten Lehren hin, aber die Schwiegermutter erteilte sie
immer wieder von neuem. Die von beiden Seiten in einem fort
gewechselten Anreden "`Liebe Tochter"' und "`Liebe Mutter!"'
standen in Widerspruch zu den Mienen der Sprecherinnen. Beide
Frauen sagten sich Artigkeiten mit vor Groll zitternder Stimme.

Zu Lebzeiten von Frau Heloise hatte sich die alte Dame nicht in
den Hintergrund gedr"angt gef"uhlt, jetzt aber kam ihr Karl{\s}
Liebe zu Emma wie ein Abfall vor von ihr und ihrer Mutterliebe,
wie ein Einbruch in ihr Eigentum. Und so sah sie auf da{\s} Gl"uck
ihre{\s} Sohne{\s} mit stiller Trauer, just wie ein um Hab und Gut
Gekommener auf den neuen Besitzer{\s} eine{\s} ehemaligen
Hause{\s} blickt. Sie mahnte ihn durch Erinnerungen daran, wie sie
sich einst f"ur ihn gesorgt und abgem"uht und ihm Opfer gebracht
hatte. Im Vergleiche damit leiste Emma viel weniger f"ur ihn, und
darum w"are seine au{\s}schlie"sliche Anbetung durchau{\s} nicht
gerechtfertigt.

Karl wu"ste nicht, wa{\s} er dazu sagen sollte. Er verehrte seine
Mutter, und seine Frau liebte er auf seine Art "uber alle Ma"sen.
Wa{\s} die eine sagte, galt ihm f"ur unfehlbar; gleichwohl fand er
an der andern nicht{\s} au{\s}zusetzen. Wenn Frau Bovary wieder
abgereist war, machte er sch"uchterne Versuche, die oder jene
ihrer Bemerkungen w"ortlich zu wiederholen. Emma bewie{\s} ihm
dann mit wenigen Worten, da"s er im Irrtum sei, und meinte, er
solle sich lieber seinen Patienten widmen.

Immerhin versuchte sie nach Theorien, die ihr gut schienen,
Liebe{\s}stimmung nach ihrem Geschmack zu erregen. Wenn sie bei
Mondenschein zusammen im Garten sa"sen, sagte sie verliebte Verse
her, soviel sie nur au{\s}wendig wu"ste, oder sie sang eine
schwerm"utige gef"uhlvolle Weise. Aber hinterher kam sie sich
selber nicht aufgeregter al{\s} vorher vor, und auch Karl war
offenbar weder verliebter noch weniger stumpfsinnig denn erst.

Da{\s} waren vergebliche Versuche, eine gro"se Leidenschaft zu
entfachen. Im "ubrigen war Emma unf"ahig, etwa{\s} zu verstehen,
wa{\s} sie nicht an sich selber erlebte, oder an etwa{\s} zu
glauben, wa{\s} nicht offen zutage lag. Und so redete sie sich
ohne weitere{\s} ein, Karl{\s} Liebe sei nicht mehr "uberm"a"sig
stark. In der Tat gewannen seine Z"artlichkeiten eine gewisse
Regelm"a"sigkeit. Er schlo"s seine Frau zu ganz bestimmten Stunden
in seine Arme. E{\s} ward da{\s} eine Gewohnheit wie alle andern,
gleichsam der Nachtisch, der kommen mu"s, weil er auf der
Men"ukarte steht.

Ein Waldw"arter, den der Herr Doktor von einer Lungenent\/z"undung
geheilt hatte, schenkte der Frau Doktor ein junge{\s}
italienische{\s} Windspiel. Sie nahm e{\s} mit auf ihre
Spazierg"ange. Mitunter ging sie n"amlich au{\s}, um einmal eine
Weile f"ur sich allein zu sein und nicht in einem fort blo"s den
Garten und die staubige Landstra"se vor Augen zu haben.

Sie wanderte meist bi{\s} zum Buchenw"aldchen von Banneville,
bi{\s} zu dem leeren Lusth"au{\s}chen, da{\s} an der Ecke der
Parkmauer steht, wo die Felder beginnen. Dort wuch{\s} in einem
Graben zwischen gew"ohnlichen Gr"asern hohe{\s} Schilf mit langen
scharfen Bl"attern. Jede{\s}mal, wenn sie dahin kam, sah sie
zuerst nach, ob sich seit ihrem letzten Hiersein etwa{\s}
ver"andert habe. E{\s} war immer alle{\s} so, wie sie e{\s}
verlassen hatte. Alle{\s} stand noch auf seinem Platze: die
Heckenrosen und die wilden Veilchen, die Brennesseln, die in
B"uscheln die gro"sen Kieselsteine umwucherten, und die
Moo{\s}fl"achen unter den drei Pavillonfenstern mit ihren immer
geschlossenen morschen Holzl"aden und rostigen Eisenbeschl"agen.
Nun schweiften Emma{\s} Gedanken in{\s} Ziellose ab, wie die
Spr"unge ihre{\s} Windspiel{\s}, da{\s} sich in gro"sen
Krei{\s}linien tummelte, gelbe Schmetterlinge ankl"affte,
Feldm"ausen nachstellte und die Mohnblumen am Raine de{\s}
Kornfelde{\s} anknabberte. Allm"ahlich gerieten ihre Gr"ubeleien
in eine bestimmte Richtung. Wenn die junge Frau so im Grase sa"s
und e{\s} mit der Stockspitze ihre{\s} Sonnenschirme{\s} ein wenig
aufw"uhlte, sagte sie sich immer wieder: "`Mein Gott, warum habe
ich eigentlich geheiratet?"'

Sie legte sich die Frage vor, ob e{\s} nicht m"oglich gewesen
w"are durch irgendwelche andre F"ugung de{\s} Schicksal{\s}, da"s
sie einen andern Mann h"atte finden k"onnen. Sie versuchte sich
vorzustellen, wa{\s} f"ur ungeschehene Ereignisse dazu geh"ort
h"atten, wie diese{\s} andre Leben geworden w"are und wie der
ungefundne Gatte au{\s}gesehen h"atte. In keinem Falle so wie
Karl! Er h"atte elegant, klug, vornehm, verf"uhrerisch au{\s}sehen
m"ussen; so wie zweifello{\s} die M"anner, die ihre ehemaligen
Klosterfreundinnen alle geheiratet hatten ... Wie e{\s} denen wohl
jetzt erging? In der Stadt, im Get"ummel de{\s} Stra"senleben{\s},
im Stimmengewirr der Theater, im Lichtmeere der B"alle, da lebten
sie sich au{\s} und lie"sen die Herzen und Sinne nicht verdorren.
Sie jedoch, sie verk"ummerte wie in einem Ei{\s}keller, und die
Langeweile spann wie eine schweigsame Spinne ihre Weben in allen
Winkeln ihre{\s} sonnelosen Herzen{\s}.

Die Tage der Prei{\s}verteilung traten ihr in die Erinnerung. Sie
sah sich auf da{\s} Podium steigen, wo sie ihre kleinen
Au{\s}zeichnungen au{\s}geh"andigt bekam. Mit ihrem Zopf, ihrem
wei"sen Kleid und ihren Lack-Halbschuhen hatte sie allerliebst
au{\s}gesehen, und wenn sie zu ihrem Platze zur"uckging, hatten
ihr die anwesenden Herren galant zugenickt. Der Klosterhof war
voller Kutschen gewesen, und durch den Wagenschlag hatte man ihr
"`Auf Wiedersehn!"' zugerufen. Und der Musiklehrer, den
Violinkasten in der Hand, hatte im Vor"ubergehen den Hut vor ihr
gezogen ... Wie weit zur"uck war da{\s} alle{\s}! Ach, wie so
weit!

Sie rief Djali, nahm ihn auf den Scho"s und streichelte seinen
schmalen feinlinigen Kopf.

"`Komm!"' fl"usterte sie. "`Gib Frauchen einen Ku"s! Du, du hast
keinen Kummer!"'

Dabei betrachtete sie da{\s} ihr wie wehm"utig au{\s}sehende
Gesicht de{\s} schlanken Tiere{\s}. E{\s} g"ahnte behaglich. Aber
sie bildete sich ein, da{\s} Tier habe auch einen Kummer. Die
R"uhrung "uberkam sie, und sie begann laut mit dem Hunde zu
sprechen, genau so wie zu jemandem, den man in seiner
Betr"ubni{\s} tr"osten will.

Zuweilen blie{\s} ruckweiser Wind, der vom Meere herkam und
m"achtig "uber da{\s} ganze Hochland von Caux strich und weit in
die Lande hinein salzige Frische trug. Da{\s} Schilf bog sich
pfeifend zu Boden, fliehende Schauer raschelten durch da{\s}
Bl"atterwerk der Buchen, w"ahrend sich die Wipfel rastlo{\s}
wiegten und in einem fort laut rauschten. Emma zog ihr Tuch fester
um die Schultern und erhob sich.

In der Allee, "uber dem teppichartigen Moo{\s}, da{\s} unter
Emma{\s} Tritten leise knisterte, spielten Sonnenlichter mit den
gr"unen Reflexen de{\s} Laubdache{\s}. Da{\s} Tage{\s}gestirn war
im Versinken; der rote Himmel flammte hinter den braunen St"ammen,
die in Reih und Glied kerzengerade dastanden und den Eindruck
eine{\s} S"aulengange{\s} an einer goldnen Wand entlang erzeugten.

Emma ward bang zumute. Sie rief den Hund heran und beeilte sich,
auf die Landstra"se und heimzukommen. Zu Hause sank sie in einen
Lehnstuhl und sprach den ganzen Abend kein Wort.

Da, gegen Ende de{\s} September{\s}, geschah etwa{\s} ganz
Besondere{\s} in ihrem Leben. Bovary{\s} bekamen eine Einladung
nach Vaubyessard, zu dem Marqui{\s} von Andervillier{\s}. Der
Marqui{\s}, der unter der Restauration Staatssekret"ar gewesen
war, wollte von neuem eine politische Rolle spielen. Seit langem
bereitete er seine Wahl in da{\s} Abgeordnetenhau{\s} vor. Im
Winter lie"s er gro"se Mengen Holz verteilen, und im
Bezirk{\s}au{\s}schu"s trat er immer wieder mit dem h"ochsten
Eifer f"ur neue Stra"senbauten im Bezirk ein. W"ahrend de{\s}
letzten Hochsommer{\s} hatte er ein Geschw"ur im Munde bekommen,
von dem ihn Karl wunderbar schnell durch einen einzigen Einstich
befreit hatte. Der Privatsekret"ar de{\s} Marqui{\s} war bald
darauf nach Toste{\s} gekommen, um da{\s} Honorar f"ur die
Operation zu bezahlen, und hatte abend{\s} nach seiner R"uckkehr
erz"ahlt, da"s er in dem kleinen Garten de{\s} Arzte{\s} herrliche
Kirschen gesehen habe. Nun gediehen gerade die Kirschb"aume in
Vaubyessard schlecht. Der Marqui{\s} erbat sich von Bovary einige
Ableger und hielt e{\s} daraufhin f"ur seine Pflicht, sich
pers"onlich zu bedanken. Bei dieser Gelegenheit sah er Emma, fand
ihre Figur ent\/z"uckend und die Art, wie sie ihn empfing,
durchau{\s} nicht b"auerisch. Und so kam man im Schlosse zu der
Ansicht, e{\s} sei weder allzu entgegenkommend noch unangebracht,
wenn man da{\s} junge Ehepaar einmal einl"ude.

An einem Mittwoch um drei Uhr bestiegen Herr und Frau Bovary ihren
Dogcart und fuhren nach Vaubyessard. Hinterr"uck{\s} war ein
gro"ser Koffer angeschnallt und vorn auf dem Schutzleder lag eine
Hutschachtel. Au"serdem hatte Karl noch einen Pappkarton zwischen
den Beinen.

Bei Anbruch der Nacht, gerade al{\s} man im Schlo"spark die
Laternen am Einfahrt{\s}wege anz"undete, kamen sie an.


\newpage\begin{center}
{\large \so{A{ch}te{\s} Kapitel}}\bigskip\bigskip
\end{center}

Vor dem Schlo"s, einem modernen Baue im Renaissancestil mit zwei
vorspringenden Fl"ugeln und drei Freitreppen, dehnte sich eine
ungeheure Rasenfl"ache mit vereinzelten Baumgruppen, zwischen
denen etliche K"uhe weideten. Ein Kie{\s}weg lief in Windungen
hindurch, beschattet von allerlei Geb"usch in verschiedenem Gr"un,
Rhododendren, Flieder- und Schneeballstr"auchern. Unter einer
Br"ucke flo"s ein Bach. Weiter weg, verschwommen im Abendnebel,
erkannte man ein paar H"auser mit Strohd"achern. Die gro"se Wiese
ward durch l"angliche kleine H"ugel begrenzt, die bewaldet waren.
Versteckt hinter diesem Geh"olz lagen in zwei gleichlaufenden
Reihen die Wirtschaft{\s}geb"aude und Wagenschuppen, die noch vom
ehemaligen Schlo"sbau herr"uhrten.

Karl{\s} W"aglein hielt vor der mittleren Freitreppe. Dienerschaft
erschien. Der Marqui{\s} kam entgegen, bot der Arztfrau den Arm
und geleitete sie in die hohe, mit Marmorfliesen belegte Vorhalle.
Ger"ausch von Tritten und Stimmen hallte darin wider wie in einer
Kirche. Dem Eingange gegen"uber stieg geradeau{\s} eine breite
Treppe auf. Zur Linken begann eine Galerie, mit Fenstern nach dem
Garten hinau{\s}, die zum Billardzimmer f"uhrte; schon von weitem
vernahm man da{\s} Karambolieren der elfenbeinernen B"alle. Durch
da{\s} Billardzimmer kam man in den Empfang{\s}saal. Beim
Hindurchgehen sah Emma Herren in w"urdevoller Haltung beim Spiel,
da{\s} Kinn vergraben in den Krawatten, alle mit
Orden{\s}b"andchen. Schweigsam l"achelnd handhabten sie die
Queue{\s}.

Auf dem d"usteren Holzget"afel der W"ande hingen gro"se Bilder in
schweren vergoldeten Rahmen mit schwarzen Inschriften. Eine
lautete:

\begin{center}
\begin{tabular}{|c|}
\hline
Han{\s} Anton von Andervillier{\s} zu Yverbonville, \\
Graf von Vaubyessard und Edler Herr auf Fre{\s}naye, \\
gefallen in der Schlacht von Coutra{\s} \\
am 20. Oktober 1587. \\
\hline
\end{tabular}
\end{center}

Eine andre:

\begin{center}
\begin{tabular}{|c|}
\hline
Han{\s} Anton Heinrich Guy, Graf von Andervillier{\s} \\
und Vaubyessard, Admiral von Frankreich, \\
Ritter de{\s} Sankt-Michel-Orden{\s}, \\
verwundet bei Saint Vaast de la Hougue \\
am 29. Mai 1692, \\
gestorben zu Vaubyessard am 23. Januar 1693 \\
\hline
\end{tabular}
\end{center}

Die "ubrigen vermochte man kaum zu erkennen, weil sich da{\s}
Licht der Lampen auf da{\s} gr"une Tuch de{\s} Billard{\s}
konzentrierte und da{\s} Zimmer im Dunkeln lie"s. Nur ein
schwacher Schein hellte die Gem"aldefl"achen auf, deren
spr"ungiger Firni{\s} mit diesem feinen Schimmer spielte. Und so
traten au{\s} allen den gro"sen schwarzen goldumflossenen
Vierecken Partien der Malerei deutlicher und heller hervor, hier
eine blasse Stirn, da zwei starre Augen, dort eine gepuderte
Allongeper"ucke "uber der Schulter eine{\s} roten Rocke{\s} und
ander{\s}wo die Schnalle eine{\s} Kniebande{\s} "uber einer
strammen Wade.

Der Marqui{\s} "offnete die T"ur zum Salon. Eine der Damen --
e{\s} war die Schlo"sherrin selbst -- erhob sich, ging Emma
entgegen und bot ihr einen Sitz neben sich an, auf einem Sofa, und
begann freundschaftlich mit ihr zu plaudern, ganz al{\s} ob sie
eine alte Bekannte vor sich h"atte. Die Marquise war etwa
Vierzigerin; sie hatte h"ubsche Schultern, eine Adlernase und eine
etwa{\s} schleppende Art zu sprechen. An diesem Abend trug sie
"uber ihrem kastanienbraunen Haar ein einfache{\s} Spitzentuch,
da{\s} ihr dreieckig in den Nacken herabhing. Neben ihr, auf einem
hochlehnigen Stuhle, sa"s eine junge Blondine. Ein paar Herren,
kleine Blumen an den R"ocken, waren im Gespr"ache mit den Damen.
Alle sa"sen sie um den Kamin herum.

Um sieben Uhr ging man zu Tisch. Die Herren, die in der "Uberzahl
da waren, nahmen Platz an der einen Tafel in der Vorhalle; die
Damen, der Marqui{\s} und die Marquise an der andern im E"szimmer.
Al{\s} Emma eintrat, drang ihr ein warme{\s} Gemisch von D"uften
und Ger"uchen entgegen: von Blumen, Tischdamast, Wein und
Delikatessen. Die Flammen der Kandelaberkerzen lieb"augelten mit
dem Silberzeug, und in den geschliffenen Gl"asern und Schalen
tanzte der bunte Widerschein. Die Tafel entlang paradierte eine
Reihe von Blumenstr"au"sen. Au{\s} den Falten der Servietten, die
in der Form von Bischof{\s}m"utzen "uber den breitrandigen Tellern
lagen, lugten ovale Br"otchen. Hummern, die auf den gro"sen
Platten nicht Platz genug hatten, leuchteten in ihrem Rot. In
durchbrochenen K"orbchen waren riesige Fr"uchte aufget"urmt.
Kunstvoll zubereitete Wachteln wurden dampfend aufgetragen. Der
Hau{\s}hofmeister, in seidnen Str"umpfen, Kniehosen und wei"ser
Krawatte, reichte mit Grandezza und gro"sem Geschick die
Sch"usseln. Auf all die{\s} gesellschaftliche Treiben sah
regung{\s}lo{\s} die bi{\s} zum Kinn verh"ullte G"ottin herab, die
auf dem m"achtigen, bronzegeschm"uckten Porzellanofen thronte.

Am oberen Ende der Tafel, mitten unter all den Damen, sa"s, "uber
seinen vollen Teller gebeugt, ein alter Herr, der sich die
Serviette nach Kinderart um den Hal{\s} gekn"upft hatte. Die Sauce
tropfte ihm au{\s} dem Munde; seine Augen waren rotunterlaufen. Er
trug noch einen Zopf, um den ein schwarze{\s} Band geschlungen
war. Da{\s} war der Schwiegervater de{\s} Marqui{\s}, der alte
Herzog von Laverdi\`ere. Anno dazumal (zu den seligen Zeiten der
Jagdfeste in Vaudreuil beim Marqui{\s} von Conflan{\s}) war er ein
Busenfreund de{\s} Grafen Artoi{\s}. Auch munkelte man, er w"are
der Geliebte der K"onigin Marie-Antoinette gewesen, der Nachfolger
de{\s} Herrn von Coigny und der Vorg"anger de{\s} Herzog{\s} von
Lauzun. Er hatte ein w"uste{\s} Leben hinter sich, voller
Zweik"ampfe, toller Wetten und Frauengeschichten. Ob seiner
Verschwendung{\s}sucht war er ehedem der Schrecken seiner Familie.
Jetzt stand ein Diener hinter seinem Stuhle, der ihm in{\s} Ohr
br"ullen mu"ste, wa{\s} e{\s} f"ur Gerichte zu essen gab.

Emma{\s} Blicke kehrten immer wieder unwillk"urlich zu diesem
alten Manne mit den h"angenden Lippen zur"uck, al{\s} ob er
etwa{\s} ganz Besondere{\s} und Gro"sartige{\s} sei: war er doch
ein Favorit de{\s} K"onig{\s}hofe{\s} gewesen und hatte im Bette
einer K"onigin geschlafen!

E{\s} wurde frappierter Sekt gereicht. Emma "uberlief e{\s} am
ganzen K"orper, al{\s} sie da{\s} eisige Getr"ank im Munde
sp"urte. Zum erstenmal in ihrem Leben sah sie Granat"apfel und a"s
sie Anana{\s}. Selbst der gesto"sene Zucker, den e{\s} dazu gab,
kam ihr wei"ser und feiner vor denn ander{\s}wo.

Nach Tische zogen sich die Damen in ihre Zimmer zur"uck, um sich
zum Ball umzukleiden. Emma widmete ihrer Toilette die sorglichste
Gr"undlichkeit, wie eine Schauspielerin vor ihrem Deb"ut. Ihr Haar
ordnete sie nach den Ratschl"agen de{\s} Coiffeur{\s}. Dann
schl"upfte sie in ihr Barege-Kleid, da{\s} auf dem Bett
au{\s}gebreitet bereitlag.

Karl f"uhlte sich in seiner Sonntag{\s}hose am Bauche beengt.

"`Ich glaube, die Stege werden mich beim Tanzen st"oren!"' meinte
er.

"`Du willst tanzen?"' entgegnete ihm Emma.

"`Na ja!"'

"`Du bist nicht recht gescheit! Man w"urde dich blo"s au{\s}lachen.
Bleib du nur ruhig sitzen! "Ubrigen{\s} schickt sich da{\s} viel
besser f"ur einen Arzt"', f"ugte sie hinzu.

Karl schwieg. Er lief mit gro"sen Schritten im Zimmer hin und her
und wartete, bi{\s} Emma fertig w"are. Er sah sie "uber ihren
R"ucken weg im Spiegel, zwischen zwei brennenden Kerzen. Ihre
schwarzen Augen erschienen ihm noch dunkler denn sonst. Ihr Haar
war nach den Ohren zu ein wenig aufgebauscht; e{\s} schimmerte in
einem bl"aulichen Glanze, und "uber ihnen zitterte eine bewegliche
Rose, mit k"unstlichen Tauperlen in den Bl"attern. Ihr
mattgelbe{\s} Kleid ward durch drei Str"au"schen von Moo{\s}rosen
mit Gr"un darum belebt.

Karl k"u"ste sie von hinten auf die Schulter.

"`La"s mich!"' wehrte sie ab. "`Du zerkn"ullst mir alle{\s}!"'

Violinen- und Waldhornkl"ange drangen herauf. Emma stieg die
Treppe hinunter, am liebsten w"are sie gerannt.

Die Quadrille hatte bereit{\s} begonnen. Der Saal war gedr"angt
voller Menschen, und immer noch kamen G"aste. Emma setzte sich
unweit der T"ur auf einen Diwan.

Al{\s} der Kontertanz zu Ende war, blieben auf dem Parkett nur
Gruppen plaudernder Menschen und Diener in Livree, die gro"se
Platten herumtrugen. In der Linie der sitzenden Damen gingen die
bemalten F"acher auf und nieder; die Blumenbukette verdeckten zur
H"alfte die lachenden Gesichter, und die goldnen St"opsel der
Riechfl"aschchen funkelten hin und her in den wei"sen Handschuhen,
an denen die Konturen der Fingern"agel ihrer Tr"agerinnen
hervortraten, w"ahrend da{\s} eingepre"ste Fleisch nur in den
Handfl"achen schimmerte. Die Spitzen, die Brillantbroschen, die
Armb"ander mit Anh"angseln wogten an den Miedern, glitzerten an
den Br"usten und klapperten an den Handgelenken. Die Damen trugen
im Haar, da{\s} durchweg glatt und im Nacken geknotet war,
Vergi"smeinnicht, Ja{\s}min, Granatbl"uten, "Ahren und Kornblumen
in Kr"anzen, Str"au"sen oder Ranken. Bequem in ihren St"uhlen
lehnten die M"utter mit gelangweilten Mienen, etliche in roten
Turbanen.

Da{\s} Herz klopfte Emma ein wenig, al{\s} der erste T"anzer sie
an den Fingerspitzen fa"ste und in die Reihe der anderen f"uhrte.
Beim ersten Geigenton tanzten sie lo{\s}. Bald jedoch legte sich
ihre Aufregung. Sie begann sich im Flusse der Musik zu wiegen, und
mit einer leichten Biegung im Halse glitt sie sicher dahin. Bei
besonder{\s} z"artlichen Passagen de{\s} Violinsolo{\s} flog ein
s"u"se{\s} L"acheln um ihre Lippen. Wenn so die andern Instrumente
schwiegen, h"orte man im Tanzsaal da{\s} helle Klimpern der
Goldst"ucke auf den Spieltischen nebenan, bi{\s} da{\s} Orchester
mit einem Male wieder voll einsetzte. Dann ging{\s} im
wiedergewonnenen Takte weiter; die R"ocke der T"anzerinnen
bauschten sich und streiften einander, H"ande suchten und mieden
sich, und dieselben Blicke, die eben sch"uchtern gesenkt waren,
fanden ihr Ziel.

Unter den tanzenden oder plaudernd an den T"uren stehenden Herren
stachen etliche, etwa zw"olf bi{\s} f"unfzehn, bei allem
Alter{\s}- und sonstigem Unterschied durch einen gewissen
gemeinsamen Typ von den andern ab. Ihre Kleider waren von
eleganterem Schnitte und au{\s} feinerem Stoff. Ihr nach den
Schl"afen zu gewellte{\s} Haar verriet die beste Pflege. Sie
hatten den Teint de{\s} Grandseigneur{\s}, jene wei"se Hautfarbe,
die wie abgestimmt zu bleichem Porzellan, schillernder Seide und
feinpolierten M"obeln erscheint und durch sorgf"altige und
raffinierte Ern"ahrung erhalten wird. Ihre Bewegungen waren
ungezwungen. Ihren mit Monogrammen bestickten Taschent"uchern
entstr"omte leise{\s} Parf"um. Den "alteren unter diesen Herren
haftete Jugendlichkeit an, w"ahrend den Gesichtern der j"ungeren
eine gewisse Reife eigen war. In ihren gleichg"ultigen Blicken
spiegelte sich die Ruhe der immer wieder befriedigten Sinne, und
hinter ihren glatten Manieren schlummerte da{\s} brutale eitle
Herrentum, da{\s} sich im Umgange mit Rassepferden und leichten
Damen entwickelt und kr"aftigt.

Ein paar Schritte von Emma entfernt, plauderte ein Kavalier in
blauem Frack mit einer blassen, jungen, perlengeschm"uckten Dame
"uber Italien. Sie schw"armten von der Kuppel de{\s} Sankt Peter,
von Tivoli, vom Vesuv, von Castellammare, von Florenz, von den
Genueser Rosen und vom Kolosseum bei Mondenschein, mit ihrem
andern Ohre horchte Emma auf eine Unterhaltung, in der sie tausend
Dinge nicht verstand. Man umringte einen jungen Herrn, der in der
vergangnen Woche in England Mi"s Arabella und Romulu{\s}
"`geschlagen"' und durch einen "`famosen Grabensprung"'
vierzigtausend Franken gewonnen hatte. Ein andrer beklagte sich,
seine "`Rennschinder"' seien "`nicht im Training"', und ein
dritter jammerte "uber einen Druckfehler in der "`Sportwelt"', der
den Namen eine{\s} seiner "`Vollbl"uter"' verballhornt habe.

Die Luft im Ballsaale wurde schwer, die Lichter schimmerten
fahler. Man dr"angte nach dem Billardzimmer. Ein Diener, der auf
einen Stuhl gestiegen war, um die Fenster zu "offnen, zerbrach
au{\s} Ungeschicklichkeit eine Scheibe. Da{\s} Klirren der
Gla{\s}scherben veranla"ste Frau Bovary hinzublicken, und da
gewahrte sie von drau"sen herein gaffende Bauerngesichter. Die
Erinnerung an da{\s} elterliche Gut "uberkam sie. Im Geiste sah
sie den Hof mit dem Misthaufen, ihren Vater in Hemd{\s}"armeln
unter den Apfelb"aumen und sich selber ganz wie einst, wie sie in
der Milchkammer mit den Fingern die Milch in den Sch"usseln
abrahmte. Aber im Strahlenglanz der gegenw"artigen Stunde starb
die eben noch so klare Erinnerung an ihr fr"uhere{\s} Leben
schnell wieder; e{\s} je gelebt zu haben, kam ihr fast unm"oglich
vor. Hier, hier lebte sie, und wa{\s} "uber diesen Ballsaal
hinau{\s} existieren mochte, da{\s} lag f"ur sie im tiefsten
Dunkel~...

Sie schl"urfte von dem Mara{\s}chino-Ei{\s}, da{\s} sie in einer
vergoldeten Silberschale in der Hand hielt, wobei sie die Augen
halb schlo"s und den goldnen L"offel lange zwischen den Z"ahnen
behielt. Neben ihr lie"s eine Dame ihren F"acher zu Boden gleiten.
Ein T"anzer ging vor"uber.

"`Sie w"aren sehr g"utig, mein Herr,"' sagte die Dame, "`wenn Sie
mir meinen F"acher aufheben wollten. Er ist unter diese{\s} Sofa
gefallen."'

Der Herr b"uckte sich, und w"ahrend er mit dem Arm nach dem
F"acher langte, bemerkte Emma, da"s ihm die Dame etwa{\s}
wei"se{\s}, dreieckig Zusammengefaltete{\s} in den Hut warf. Er
"uberreichte ihr den aufgehobenen F"acher ehrerbietig. Sie dankte
mit einem leichten Neigen de{\s} Kopfe{\s} und barg schnell ihr
Gesicht in den Blumen ihre{\s} Strau"se{\s}.

Nach dem Souper, bei dem e{\s} verschiedene Sorten von S"ud- und
Rheinweinen gab, Kreb{\s}suppe, Mandelmilch, Pudding \`a la
Trafalgar und allerlei kalte{\s} Fleisch, mit zitterndem Gelee
garniert, begannen die Wagen einer nach dem andern vor- und
wegzufahren. Wer einen der Musselinvorh"ange am Fenster ein wenig
beiseiteschob, konnte die Laternenlichter in die Nacht
hinau{\s}ziehen sehen. E{\s} sa"sen immer weniger T"anzer im
Saale. Nur im Spielzimmer war noch Leben. Die Musikanten leckten
sich die hei"sen Finger ab. Karl stand gegen eine T"ur gelehnt,
dem Einschlafen nahe.

Um drei Uhr begann der Kotillon. Walzer tanzen konnte Emma nicht.
Aber alle Welt, sogar Fr"aulein von Andervillier{\s} und die
Marquise tanzten. E{\s} waren nur noch die im Schlosse zur Nacht
bleibenden G"aste da, etwa ein Dutzend Personen.

Da geschah e{\s}, da"s einer der T"anzer, den man schlechtweg
"`Vicomte"' nannte -- die weitau{\s}geschnittene Weste sa"s ihm
wie angegossen -- Frau Bovary zum Tanz aufforderte. Sie wagte
e{\s} nicht. Der Vicomte bat abermal{\s}, indem er versicherte, er
w"urde sie sicher f"uhren und e{\s} w"urde vortrefflich gehen.

Sie begannen langsam, um allm"ahlich rascher zu tanzen.
Schlie"slich wirbelten sie dahin. Alle{\s} drehte sich rund um
sie: die Lichter, die M"obel, die W"ande, der Parkettboden, al{\s}
ob sie in der Mitte eine{\s} Kreisel{\s} w"aren. Einmal, al{\s}
da{\s} Paar dicht an einer der T"uren vorbeitanzte, wickelte sich
Emma{\s} Schleppe um da{\s} Bein ihre{\s} T"anzer{\s}. Sie
f"uhlten sich beide und blickten sich einander in die Augen. Ein
Schwindel ergriff Emma. Sie wollte stehen bleiben. Aber e{\s} ging
weiter: der Vicomte raste nur noch rascher mit ihr dahin, bi{\s}
an da{\s} Ende der Galerie, wo Emma, v"ollig au"ser Atem, beinahe
umsank und einen Augenblick lang ihren Kopf an seine Brust lehnte.
Dann brachte er sie, von neuem, aber ganz langsam tanzend, an
ihren Platz zur"uck. E{\s} schwindelte ihr; sie mu"ste den R"ucken
anlehnen und ihr Gesicht mit der einen Hand bedecken.

Al{\s} sie die Augen wieder aufschlug, sah sie, da"s in der Mitte
de{\s} Saale{\s} eine der Damen auf einem Taburett sa"s, w"ahrend
drei der Herren vor ihr knieten. Der Vicomte war darunter. Er war
der Bevorzugte. Und von neuem setzten die Geigen ein.

Alle Blicke galten dem tanzenden Paare. E{\s} tanzte einmal und
noch einmal herum: sie regung{\s}lo{\s} in den Linien ihre{\s}
K"orper{\s}, da{\s} Kinn ein wenig gesenkt; er in immer der
n"amlichen Haltung, kerzengerade, die Arme elegant gerundet, den
Blick geradeau{\s} gerichtet. Da{\s} waren Walzert"anzer! Sie
fanden kein Ende. Eher erm"udeten die Zuschauer.

Nach dem Kotillon plauderte man noch eine kleine Weile. Dann sagte
man sich "`Gute Nacht"' oder vielmehr "`Guten Morgen"', und alle{\s}
ging schlafen.

Karl schleppte sich am Treppengel"ander hinauf. Er hatte sich
"`die Beine in den Bauch gestanden."' Ohne sich zu setzen, hatte
er sich f"unf Stunden hintereinander bei den Spieltischen
aufgehalten und den Whistspielern zugesehen, ohne etwa{\s} von
diesem Spiel zu verstehen. Und so stie"s er einen m"achtigen
Seufzer der Erleichterung au{\s}, al{\s} er sich endlich seiner
Stiefel entledigt hatte.

Emma legte sich ein Tuch um die Schultern, "offnete da{\s} Fenster
und lehnte sich hinau{\s}. Die Nacht war schwarz. Feiner
Spr"uhregen fiel. Sie atmete den feuchten Wind ein, der ihr die
Augenlider k"uhlte. Walzerkl"ange summten ihr noch in den Ohren.
Emma hielt sich gewaltsam wach, um den eben erlebten
M"archenglanz, ehe er ganz wieder verronnen, noch ein wenig zu
besitzen~...

Der Morgen d"ammerte. Sie schaute hin"uber nach den Fensterreihen
de{\s} Mittelbaue{\s}, lange, lange, und versuchte zu erraten, wo
die einzelnen Personen alle wohnten, die sie diesen Abend
beobachtet hatte. Sie sehnte sich darnach, etwa{\s} von ihrem
Leben zu wissen, eine Rolle darin zu spielen, selber darin
aufzugehen.

Schlie"slich begann sie zu fr"osteln. Sie entkleidete sich und
schmiegte sich in die Kissen, zur Seite ihre{\s} schlafenden
Gatten.

Zum Fr"uhst"uck erschienen eine Menge Menschen. E{\s} dauerte zehn
Minuten. E{\s} gab keinen Kognak, wa{\s} dem Arzt wenig behagte.

Beim Aufstehen sammelte Fr"aulein von Andervillier{\s} die
angebrochenen Br"otchen in einen kleinen Korb, um sie den
Schw"anen auf dem Schlo"steiche zu bringen. Nach der F"utterung
begab man sich in da{\s} Gew"ach{\s}hau{\s}, mit seinen seltsamen
Kakteen und Schlingpflanzen, und in die Orangerie. Von dieser
f"uhrte ein Au{\s}gang in den Wirtschaft{\s}hof.

Um der jungen Arztfrau ein Vergn"ugen zu bereiten, zeigte ihr der
Marqui{\s} die St"alle. "Uber den korbartigen Raufen waren
Porzellanschilder angebracht, auf denen in schwarzen Buchstaben
die Namen der Pferde standen. Man blieb an den einzelnen Boxen
stehen, und wenn man mit der Zunge schnalzte, scharrten die Tiere.
Die Dielen in der Sattel- und Geschirrkammer waren blank gewichst
wie Salonparkett. Die Wagengeschirre ruhten in der Mitte de{\s}
Raume{\s} auf drehbaren B"ocken, w"ahrend die Kandaren, Trensen,
Kinnketten, Steigb"ugel, Z"ugel und Peitschen wohlgeordnet zu
Reihen an den W"anden hingen.

Karl bat einen Stallburschen, sein Gef"ahrt zurechtzumachen.
Sodann fuhr er vor. Da{\s} ganze Gep"ack ward aufgepackt. Da{\s}
Ehepaar Bovary bedankte und verabschiedete sich bei dem Marqui{\s}
und der Marquise. Und heim ging e{\s} nach Toste{\s}.

Schweigsam sah Emma dem Drehen der R"ader zu. Karl sa"s auf dem
"au"sersten Ende de{\s} Sitze{\s} und kutschierte mit abstehenden
Ellbogen. Da{\s} kleine Pferd lief im Zotteltrab dahin, in seiner
Gabel, die ihm viel zu weit war. Die schlaffen Z"ugel tanzten auf
der Kruppe de{\s} Gaule{\s}. Gischt flatterte. Der Koffer, der
hinten angeschnallt war, sa"s nicht recht fest und polterte in
einem fort im Takte an den Wagenkasten.

Auf der H"ohe von Thibourville wurden sie pl"otzlich von ein paar
Reitern "uberholt. Lachende Gesichter und Zigarettenrauch. Emma
glaubte, den Vicomte zu bemerken. Sie schaute ihm nach, aber sie
vermochte nicht{\s} zu erkennen al{\s} die Konturen der Reiter,
die sich vom Himmel abhoben und sich im Rhythmu{\s} de{\s}
Trabe{\s} auf und nieder bewegten.

Wenige Minuten sp"ater mu"sten sie Halt machen, um die zerrissene
Hemmkette mit einem Strick fest\/zubinden. Al{\s} Karl da{\s} ganze
Geschirr noch einmal "uberblickte, gewahrte er zwischen den Beinen
seine{\s} Pferde{\s} einen Gegenstand liegen. Er hob eine
Zigarrentasche auf; sie war mit gr"uner Seide gestickt und auf der
Mitte der Oberseite mit einem Wappen geschm"uckt.

"`E{\s} sind sogar zwei Zigarren drin!"' sagte er. "`Die kommen
heute abend nach dem Essen dran!"'

"`Du rauchst demnach?"' fragte Emma.

"`Manchmal! Gelegentlich!"'

Er steckte seinen Fund in die Tasche und gab dem Gaul ein{\s} mit
der Peitsche.

Al{\s} sie zu Hause ankamen, war da{\s} Mittagessen noch nicht
fertig. Frau Bovary war unwillig dar"uber. Anastasia gab eine
dreiste Antwort.

"`Scheren Sie sich fort"' rief Emma. "`Sie machen sich "uber mich
lustig. Sie sind entlassen!"'

Zu Tisch gab e{\s} Zwiebelsuppe und Kalbfleisch mit Sauerkraut.
Karl sa"s seiner Frau gegen"uber. Er rieb sich die H"ande und
meinte vergn"ugt:

"`Zu Hause ist{\s} doch am sch"onsten!"'

Man h"orte, wie Anastasia drau"sen weinte. Karl hatte da{\s} arme
Ding gern. Ehedem, in der trostlosen Einsamkeit seiner Witwerzeit,
hatte sie ihm so manchen Abend Gesellschaft geleistet. Sie war
seine erste Patientin gewesen, seine "alteste Bekannte in der
ganzen Gegend.

"`Hast du ihr im Ernst gek"undigt?"' fragte er nach einer Weile.

"`Gewi"s! Warum soll ich auch nicht?"' gab Emma zur Antwort.

Nach Tisch w"armten sich die beiden in der K"uche, w"ahrend die
Gro"se Stube wieder in Ordnung gebracht wurde. Karl brannte sich
eine der Zigarren an. Er rauchte mit aufgeworfenen Lippen und
spuckte dabei aller Minuten, und bei jedem Zuge lehnte er sich
zur"uck, damit ihm der Rauch nicht in die Nase stieg.

"`Da{\s} Rauchen wird dir nicht bekommen!"' bemerkte Emma
ver"achtlich.

Karl legte die Zigarre weg, lief schnell an die Plumpe und trank
gierig ein Gla{\s} frische{\s} Wasser. W"ahrenddessen nahm Emma
die Zigarrentasche und warf sie rasch in einen Winkel de{\s}
Schranke{\s}.

Der Tag war endlo{\s}: dieser Tag nach dem Feste!

Emma ging in ihrem G"artchen spazieren. Immer dieselben Wege auf
und ab wandelnd, blieb sie vor den Blumenbeeten stehen, vor dem
Obstspalier, vor dem t"onernen M"onch, und betrachtete sich alle
diese ihr so wohlbekannten alten Dinge voll Verwunderung. Wie weit
hinter ihr der Ballabend schon lag! Und wa{\s} war e{\s}, da{\s}
sich zwischen vorgestern und heute abend wie eine breite Kluft
dr"angte? Diese Reise nach Vaubyessard hatte in ihr Leben einen
tiefen Ri"s gerissen, einen klaffenden Abgrund, wie ihn der Sturm
zuweilen in einer einzigen Nacht in den Bergen aufw"uhlt. Trotzdem
kam eine gewisse Resignation "uber sie. Wie eine Reliquie
verwahrte sie ihr sch"one{\s} Ballkleid in ihrem Schranke, sogar
die Atla{\s}schuhe, deren Sohlen vom Parkettwach{\s} eine
br"aunliche Politur bekommen hatten. Emma{\s} Herz ging e{\s} wie
ihnen. Bei der Ber"uhrung mit dem Reichtum war etwa{\s} daran
haften geblieben f"ur immerdar.

An den Ball zur"uckdenken, wurde f"ur Emma eine besondre
Besch"aftigung. An jedem Mittwoche wachte sie mit dem Gedanken
auf: "`Ach, heute vor acht Tagen war e{\s}!"' -- "`Heute vor
vierzehn Tagen war e{\s}!"' -- "`Heute vor drei Wochen war
e{\s}!"' Allm"ahlich aber verschwammen in ihrem Ged"achtnisse die
einzelnen Gesichter, die sie im Schlosse gesehen hatte. Die
Melodien der T"anze entfielen ihr. Sie verga"s, wie die Gem"acher
und die Livreen au{\s}gesehen hatten. Immer mehr schwanden ihr die
Einzelheiten, aber ihre Sehnsucht blieb zur"uck.


\newpage\begin{center}
{\large \so{Neunte{\s} Kapitel}}\bigskip\bigskip
\end{center}

Oft, wenn Karl unterweg{\s} war, holte Emma die gr"unseidene
Zigarrentasche au{\s} dem Schrank, wo sie unter gefalteter W"asche
verborgen lag. Sie betrachtete sie, "offnete sie und sog sogar den
Duft ihre{\s} Futter{\s} ein, da{\s} nach Lavendel und Tabak roch.
Wem mochte sie geh"ort haben? Dem Vicomte? Vielleicht war e{\s}
ein Geschenk seiner Geliebten. Gewi"s hatte sie die Stickerei auf
einem kleinen Rahmen von Polisanderholz angefertigt, ganz
heimlich, in vielen, vielen Stunden, und die weichen Locken der
tr"aumerischen Arbeiterin hatten die Seide gestreift. Ein Hauch
von Liebe wehte au{\s} den Stichen hervor. Mir jedem Faden war
eine Hoffnung oder eine Erinnerung eingestickt worden, und alle
diese kleinen Seidenkreuzchen waren da{\s} Denkmal einer langen
stummen Leidenschaft. Und dann, eine{\s} Morgen{\s}, hatte der
Vicomte die Tasche mitgenommen. Wovon hatten die beiden wohl
geplaudert, al{\s} sie noch auf dem breiten Simse de{\s}
Kamine{\s} zwischen Blumenvasen und Stutzuhren au{\s} den Zeiten
der Pompadour lag?

Jetzt war der Vicomte wohl in Pari{\s}. Weit weg von ihr und von
Toste{\s}! Wie mochte diese{\s} Pari{\s} sein? Welch
geheimni{\s}voller Name! Pari{\s}! Sie fl"usterte da{\s} Wort
immer wieder vor sich hin. E{\s} machte ihr Vergn"ugen. E{\s}
raunte ihr durch die Ohren wie der Klang einer gro"sen
Kirchenglocke. E{\s} flammte ihr in die Augen, wo e{\s} auch
stand, selbst von den Etiketten ihrer Pomadenb"uchsen.

Nacht{\s}, wenn die Seefischh"andler unten auf der Stra"se
vorbeifuhren mit ihren Karren und die "`Majorlaine"' sangen, ward
sie wach. Sie lauschte dem Rasseln der R"ader, bi{\s} die Wagen
au{\s} dem Dorfe hinau{\s} waren und e{\s} wieder still wurde.

"`Morgen sind sie in Pari{\s}!"' seufzte die Einsame. Und in ihren
Gedanken folgte sie den Fahrzeugen "uber Berg und Tal, durch
D"orfer und St"adte, immer die gro"se Stra"se hin in der lichten
Sternennacht. Aber weiter weg gab e{\s} ein verschwommene{\s}
Ziel, wo ihre Tr"aume versagten. Sie kaufte sich einen Plan von
Pari{\s} und machte mit dem Fingernagel lange Wanderungen durch
die Weltstadt. Sie lief auf den Boulevard{\s} hin, blieb an jeder
Stra"senecke stehen, an jedem Hause, da{\s} im Stadtplan
eingezeichnet war. Wenn ihr die Augen schlie"slich m"ude wurden,
schlo"s sie die Lider, und dann sah sie im Dunkeln, wie die
Flammen der Laternen im Winde flackerten und wie die Kutschen vor
dem Portal der Gro"sen Oper donnernd vorfuhren.

Sie abonnierte auf den "`Bazar"' und die "`Modenwelt"' und
studierte auf da{\s} gewissenhafteste alle Berichte "uber die
Premieren, Rennen und Abendgesellschaften. Sie war unterrichtet,
wenn ber"uhmte S"angerinnen Gastspiele gaben oder neue
Warenh"auser er"offnet wurden; sie kannte die neuesten Moden, die
Adressen der guten Schneider; sie wu"ste, an welchen Tagen die
vornehme Gesellschaft im Boi{\s} und in der Oper zu finden war.
Au{\s} den Moderomanen lernte sie, wie die Pariser Wohnungen
eingerichtet waren. Sie la{\s} Balzac und die George Sand, um
wenigsten{\s} in der Phantasie ihre Begehrlichkeit zu befriedigen.
Sie brachte diese B"ucher sogar mit zu den Mahlzeiten und la{\s}
darin, w"ahrend Karl a"s und ihr erz"ahlte. Und wa{\s} sie auch
la{\s}, "uberallhinein drangen ihre Reminis\/zenzen an den
Vicomte. Zwischen ihm und den Romangestalten fand sie allerhand
Beziehungen. Aber allm"ahlich erweiterte sich der Ideenkrei{\s},
dessen Mittelpunkt er war, und der Heiligenschein, den er getragen
hatte, erblich schlie"slich, um auf andren Idealgesch"opfen wieder
aufzuflammen.

Unerme"slich wie da{\s} Weltmeer, in der Sonne eine{\s}
Wunderhimmel{\s}, so stand Pari{\s} vor Emma{\s} Phantasie. Da{\s}
tausendf"altige Leben, da{\s} sich in diesem Babylon abspielt, war
gleichwohl f"ur sie auf ganz bestimmte Einzelheiten beschr"ankt,
die sie im Geiste in deutlichen Bildern sah. Neben diesen -- man
k"onnte sagen -- Symbolen de{\s} mond"anen Leben{\s} trat alle{\s}
andre in Dunkel und D"ammerung zur"uck.

Da{\s} Dasein der Hofmenschen, so wie sie sich{\s} vorstellte,
spielte sich auf gl"anzendem Parkett ab, in Spiegels"alen, um
ovale Tische, auf denen Samtdecken mit goldnen Fransen liegen.
Dazu Schleppkleider, Staat{\s}geheimnisse und tausend Qualen
hinter heuchlerischem L"acheln. Da{\s} Milieu de{\s} h"ochsten
Adel{\s} bildete sie sich folgenderma"sen ein: Vornehme bleiche
Gesichter; man steht fr"uh um vier Uhr auf; die Damen, allesamt
ungl"uckliche Engel, tragen Unterr"ocke au{\s} irischen Spitzen;
die M"anner, verkannte Genie{\s}, kokettierend mit der Ma{\s}ke
der Oberfl"achlichkeit, reiten au{\s} "Ubermut ihre Vollbl"uter
zuschanden, die Sommersaison verbringen sie in Baden-Baden, und
wenn sie vierzig Jahre alt geworden sind, heiraten sie zu guter
Letzt reiche Erbinnen. Die dritte Welt, von der Emma tr"aumte, war
da{\s} bunte Leben und Treiben der K"unstler, Schriftsteller und
Schauspielerinnen, da{\s} sich in den separierten Zimmern der
Restaurant{\s} abspielt, wo man nach Mitternacht bei Kerzenschein
soupiert und sich au{\s}tollt. Diese Menschen sind die
Verschwender de{\s} Leben{\s}, K"onige in ihrer Art, voller Ideale
und Phantastereien. Ihr Dasein verl"auft hoch "uber dem Alltag,
zwischen Himmel und Erde, in Sturm und Drang.

Alle{\s} andre in der Welt war f"ur Emma verloren,
wesen{\s}lo{\s}, so gut wie nicht vorhanden. Je n"aher ihr die
Dinge "ubrigen{\s} standen, um so weniger ber"uhrten sie ihr
Innenleben. Alle{\s}, wa{\s} sie unmittelbar umgab: die eint"onige
Landschaft, die kleinlichen armseligen Spie"sb"urger, ihr
ganze{\s} Durchschnitt{\s}dasein kam ihr wie ein Winkel der
eigentlichen Welt vor. Er existierte zuf"allig, und sie war in ihn
verbannt. Aber drau"sen vor seinen Toren, da begann da{\s} weite,
weite Reich der Seligkeiten und Leidenschaften. In der Sehnsucht
ihre{\s} Traumleben{\s} flossen Wollust und Luxu{\s} mit den
Freuden de{\s} Herzen{\s}, erlesene Leben{\s}f"uhrung mit
Gef"uhl{\s}feinheiten ineinander. Bedarf die Liebe, "ahnlich wie
die Pflanzen der Tropen, nicht ihre{\s} eigenen Boden{\s} und
ihrer besondren Sonne? Seufzer bei Mondenschein, innige K"usse,
Tr"anen, vergossen auf hingebung{\s}volle H"ande, Fleische{\s}lust
und schmachtende Z"artlichkeit, alle{\s} da{\s} war ihr
unzertrennlich von stolzen Schl"ossern voll m"u"sigen Leben{\s},
von Boudoiren mit seidnen Vorh"angen und dicken Teppichen, von
blumengef"ullten Vasen, von Himmelbetten, von funkelnden
Brillanten und goldstrotzender Dienerschaft.

Der Postkutscher, der allmorgentlich in seiner zerrissenen
Stalljacke, die blo"sen F"u"se in Holzpantoffeln, kam, um die
Stute zu f"uttern und zu putzen, klapperte jede{\s}mal durch die
Hau{\s}flur. Da{\s} war der Groom in Kniehosen. Mit dem mu"ste sie
zufrieden sein. Wenn er fertig war, lie"s er sich den ganzen Tag
"uber nicht wieder blicken. Karl pflegte n"amlich sein Pferd, wenn
er e{\s} geritten hatte, selbst einzustellen. W"ahrend er Sattel
und Z"aumung aufhing, warf die Magd dem Tiere ein Bund Heu vor.

Nachdem Anastasia unter tausend Tr"anen wirklich da{\s} Hau{\s}
verlassen hatte, nahm Emma an ihrer Stelle ein junge{\s} M"adchen
in Dienst, eine Waise von vierzehn Jahren, ein sanftm"utige{\s}
Wesen. Sie zog sie nett an, brachte ihr h"ofliche Manieren bei,
lehrte sie, ein Gla{\s} Wasser auf dem Teller zu reichen, vor dem
Eintreten in ein Zimmer anzuklopfen, unterrichtete sie im Pl"atten
und B"ugeln der W"asche und lie"s sich von ihr beim Ankleiden
helfen. Mit einem Worte, sie bildete sich eine Kammerzofe au{\s}.
Felicie -- so hie"s da{\s} neue M"adchen -- gehorchte ihr ohne
Murren. E{\s} gefiel ihr im Hause. Die Hau{\s}frau pflegte den
B"ufettschl"ussel stecken zu lassen. Felicie nahm sich alle Abende
einige St"ucke Zucker und verzehrte sie, wenn sie allein war, im
Bett, nachdem sie ihr Gebet gesprochen hatte. Nachmittag{\s}, wenn
Frau Bovary wie gew"ohnlich oben in ihrem Zimmer blieb, ging sie
ein wenig in die Nachbarschaft klatschen.

Emma kaufte sich eine Schreibunterlage, Briefbogen, Umschl"age und
einen Federhalter, obgleich sie niemanden hatte, an den sie h"atte
schreiben k"onnen. H"aufig besah sie sich im Spiegel. Mitunter
nahm sie ein Buch zur Hand, aber beim Lesen verfiel sie in
Tr"aumereien und lie"s da{\s} Buch in den Scho"s sinken. Am
liebsten h"atte sie eine gro"se Reise gemacht oder w"are wieder in
da{\s} Kloster gegangen. Der Wunsch zu sterben und die Sehnsucht
nach Pari{\s} beherrschten sie in der gleichen Minute.

Karl trabte indessen bei Wind und Wetter seine Landstra"sen hin.
Er fr"uhst"uckte in den Geh"often, griff in feuchte Krankenbetten,
lie"s sich beim Aderlassen da{\s} Gesicht voll Blut spritzen,
h"orte dem R"ocheln Sterbender zu, pr"ufte den Inhalt von
Nachtt"opfen und zog so und so oft schmutzige Hemden hoch.
Abend{\s} aber fand er immer ein gem"utliche{\s} Feuer im Kamin,
einen nett gedeckten Tisch, den zurechtgesetzten Gro"svaterstuhl
und eine allerliebst angezogene Frau. Ein Duft von Frische ging
von ihr au{\s}; wer wei"s, wa{\s} da{\s} war, ein Odeur, ihre
W"asche oder ihre Haut?

Eine Menge andrer seltsamer Kleinigkeiten war sein Ent\/z"ucken. Sie
erfand neue Papiermanschetten f"ur die Leuchter, oder sie besetzte
ihren Rock mit einem koketten Volant, oder sie taufte ein ganz
gew"ohnliche{\s} Gericht mit einem putzigen Namen, weil e{\s} ihm
herrlich geschmeckt und er e{\s} bi{\s} auf den letzten Rest
vertilgt hatte, obgleich e{\s} dem M"adchen greulich mi"sraten
war. Einmal sah sie in Rouen, da"s die Damen an ihren Uhrketten
allerlei Anh"angsel trugen; sie kaufte sich auch welche. Ein
andermal war e{\s} ihr Wunsch, auf dem Kamine ihre{\s} Zimmer{\s}
zwei gro"se Vasen au{\s} blauem Porzellan stehen zu haben, oder
sie wollte ein N"ahk"astchen au{\s} Elfenbein mit einem
vergoldeten Fingerhut. So wenig Karl diese eleganten Neigungen
begriff, so sehr "ubten sie doch auch auf ihn eine verf"uhrerische
Wirkung au{\s}. Sie erh"ohten die Freuden seiner Sinnlichkeit und
verliehen seinem Heim einen s"u"sen Reiz mehr. E{\s} war, al{\s}
ob Goldstaub auf den Pfad seine{\s} Leben{\s} fiel.

Er sah gesund und w"urdevoll au{\s}, und sein Ansehen al{\s} Arzt
stand l"angst fest. Die Bauern mochten ihn gern, weil er gar nicht
stolz war. Er streichelte die Kinder, ging niemal{\s} in ein
Wirt{\s}hau{\s} und fl"o"ste jedermann durch seine Solidit"at
Vertrauen ein. Er war Spezialist f"ur Hal{\s}- und Lungenleiden.
In Wirklichkeit r"uhrten seine Erfolge daher, da"s er Angst hatte,
die Leute zu Tode zu kurieren, und ihnen darum mit Vorliebe nur
beruhigende Arzneien verschrieb und ihnen hin und wieder ein
Abf"uhrmittel, ein Fu"sbad oder einen Blutegel verordnete. In der
Chirurgie war er allerding{\s} ein St"umper. Er schnitt
drauflo{\s} wie ein Fleischermeister, und Z"ahne zog er wie der
Satan.

Um sich in seinem Handwerk "`auf dem laufenden zu halten"', war er
auf die "`Medizinische Wochenschrift"' abonniert, von der ihm
einmal ein Prospekt zugegangen war. Abend{\s} nach der
Hauptmahlzeit nahm er sie gew"ohnlich zur Hand, aber die warme
Zimmerluft und die Verdauung{\s}m"udigkeit brachten ihn
regelm"a"sig nach f"unf Minuten zum Einschlafen. Da{\s} Haupt sank
ihm dann auf den Tisch, und sein Haar fiel wie eine L"owenm"ahne
vorn"uber nach dem Fu"se der Tischlampe zu. Emma sah sich
diese{\s} Bild ver"achtlich an. Wenn ihr Mann nur wenigsten{\s}
eine der stillen Leuchten der Wissenschaft gewesen w"are, die
nacht{\s} "uber ihren B"uchern hocken und mit sechzig Jahren, wenn
sich da{\s} Zipperlein einstellt, den Verdienstorden in da{\s}
Knopfloch ihre{\s} schlecht sitzenden schwarzen Rocke{\s} geh"angt
bekommen! Der Name Bovary, der ja auch der ihre war, h"atte
Bedeutung haben m"ussen in der Fachliteratur, in den Zeitungen, in
ganz Frankreich! Aber Karl hegte so gar keinen Ehrgeiz. Ein Arzt
au{\s} Yvetot, mit dem er unl"angst gemeinsam konsultiert worden
war, hatte ihn in Gegenwart de{\s} Kranken und im Beisein der
Verwandten blamiert. Al{\s} Karl ihr abend{\s} die Geschichte
erz"ahlte, war Emma ma"slo{\s} emp"ort "uber den Kollegen. Karl
k"u"ste ihr ger"uhrt die Stirn. Die Tr"anen standen ihm in den
Augen. Sie war au"ser sich vor Scham ob der Dem"utigung ihre{\s}
Manne{\s} und h"atte ihn am liebsten verpr"ugelt. Um sich zu
beruhigen, eilte sie auf den Gang hinau{\s}, "offnete da{\s}
Fenster und sog die k"uhle Nachtluft ein.

"`Ach, wa{\s} habe ich f"ur einen erb"armlichen Mann!"' klagte sie
leise vor sich hin und bi"s sich auf die Lippen.

Er wurde ihr auch sonst immer widerw"artiger. Mit der Zeit nahm er
allerlei unmanierliche Gewohnheiten an. Beim Nachtisch
zerschnippselte er den Kork der leeren Flasche; nach dem Essen
leckte er sich die Z"ahne mit der Zunge ab, und wenn er die Suppe
l"offelte, schmatzte er bei jedem Schlucke. Er ward immer
beleibter, und seine an und f"ur sich schon winzigen Augen drohten
allm"ahlich g"anzlich hinter seinen feisten Backen zu
verschwinden.

Zuweilen schob ihm Emma den roten Saum seine{\s}
Trikotunterhemde{\s} wieder unter den Kragen, zupfte die Krawatte
zurecht oder beseitigte ein Paar abgetragener Handschuhe, die er
sonst noch l"anger angezogen h"atte. Aber dergleichen tat sie
nicht, wie er w"ahnte, ihm zuliebe. E{\s} geschah einzig und
allein au{\s} nerv"oser Reizbarkeit und egoistischem
Sch"onheit{\s}drang. Mitunter erz"ahlte sie ihm Dinge, die sie
gelesen hatte, etwa au{\s} einem Roman oder au{\s} einem neuen
St"ucke, oder Vorkommnisse au{\s} dem Leben der oberen
Zehntausend, die sie im Feuilleton einer Zeitung erhascht hatte.
Schlie"slich war Karl wenigsten{\s} ein aufmerksamer und geneigter
Zuh"orer, und sie konnte doch nicht immer nur ihr Windspiel,
da{\s} Feuer im Kamin und den Perpendikel ihrer Kaminuhr zu ihren
Vertrauten machen!

Im tiefsten Grunde ihrer Seele harrte sie freilich immer de{\s}
gro"sen Erlebnisse{\s}. Wie der Schiffer in Not, so suchte sie mit
verzweifelten Augen den einsamen Horizont ihre{\s} Dasein{\s} ab
und sp"ahte in die dunstigen Fernen nach einem wei"sen Segel.
Dabei hatte sie gar keine bestimmte Vorstellung, ob ihr der
richtige Kur{\s} oder der Zufall da{\s} ersehnte Schiff zuf"uhren
solle, nach welchem Gestade sie dann auf diesem Fahrzeuge steuern
w"urde, welcher Art diese{\s} Schiff "uberhaupt sein solle, ob ein
schwache{\s} Boot oder ein gro"ser Ozeandampfer, und mit welcher
Fracht er fahre, mit tausend "Angsten oder mit Gl"uckseligkeiten
beladen bi{\s} hinauf in die Wimpel. Aber jeden Morgen, wenn sie
erwachte, rechnete sie bestimmt darauf, heute m"usse e{\s} sich
ereignen. Bei jedem Ger"ausch zuckte sie zusammen, fuhr sie empor
und war dann betroffen, da"s e{\s} immer noch nicht kam, da{\s}
gro"se Erlebni{\s}. Wenn die Sonne sank, war sie jede{\s}mal
tieftraurig, aber sie hoffte von neuem auf den n"achsten Tag.

Der Fr"uhling zog wieder in da{\s} Land. Al{\s} die Tage w"armer
wurden und die Birnb"aume zu bl"uhen begannen, litt Emma an
Beklemmungen. Dann ward e{\s} Sommer. Bereit{\s} Anfang Juli
z"ahlte sie sich an den Fingern ab, wieviel Wochen e{\s} noch
bi{\s} zum Oktober seien. Vielleicht g"abe der Marqui{\s} von
Andervillier{\s} wieder einen Ball. Aber der ganze September
verstrich, ohne da"s ein Brief oder ein Besuch au{\s} Vaubyessard
kam. Nach dieser Entt"auschung war ihr Herz wieder leer, und
da{\s} ewige Einerlei ihre{\s} Leben{\s} hub von neuem an.

Also sollten sich denn fortan ihre Tage aneinanderreihen wie die
Perlen an einer Schnur, jeder immer wieder gleich dem andern,
sollten kommen und gehen und nie etwa{\s} Neue{\s} bringen! So
flach auch da{\s} Leben andrer Leute war, sie hatten doch immerhin
die M"oglichkeit eine{\s} au"sergew"ohnlichen Geschehnisse{\s}.
Ein Abenteuer zieht h"aufig die unglaublichsten Umw"alzungen nach
sich und ver"andert rasch die ganze Szene. Aber in ihrem Dasein
blieb alle{\s} beim alten. Da{\s} war ihr Schicksal! Die Zukunft
lag vor ihr wie ein langer stockfinsterer Gang, und die T"ur ganz
am Ende war fest verriegelt.

Sie vernachl"assigte die Musik. Wozu Klavier spielen? Wer h"orte
ihr denn zu? E{\s} war ihr doch niemal{\s} verg"onnt, in einem
Gesellschaft{\s}kleid mit kurzen "Armeln auf einem Konzertfl"ugel
vor einer gro"sen Zuh"orerschaft vorzutragen, ihre flinken Finger
"uber die Elfenbeintasten hinst"urmen zu lassen und da{\s} Murmeln
der Verz"uckung um sich zu h"oren wie da{\s} Rauschen de{\s}
Zephir{\s}. Wozu also da{\s} m"uhevolle Einstudieren? Ebenso
packte sie ihr Zeichenger"at und den Stickrahmen in den Schrank.
Wozu da{\s} alle{\s}? Wem zuliebe? Auch da{\s} N"ahen ward ihr
widerlich, und selbst da{\s} Lesen lie"s sie. "`E{\s} ist immer
wieder da{\s}selbe!"' sagte sie sich.

Und so tr"aumte sie vor sich hin, starrte in die Glut de{\s}
Kamin{\s} oder sah zu, wie drau"sen der Regen herniederfiel.

Am traurigsten waren ihr die Sonntag{\s}nachmittage. Wenn e{\s}
zur Vesper l"autete, h"orte sie, vor sich hinbr"utend, den dumpfen
Glockenschl"agen zu. Eine Katze schlich "uber die D"acher,
gem"achlich und langsam, und wo ein bi"schen Sonne war, machte sie
einen Buckel. Auf der Landstra"se blie{\s} der Wind Staubwirbel
auf. In der Ferne heulte ein Hund. Und zu allem dem, in einem
fort, in gleichen Zeitr"aumen, der monotone Glockenklang, der
"uber den Feldern verhallte.

Inzwischen kamen die Leute au{\s} der Kirche. Die Frauen in
Lackschuhen, die Bauern in ihren Sonntag{\s}blusen, die hin und
her laufenden Kinder in blo"sen K"opfen. Alle{\s} ging
heimw"art{\s}. Nur f"unf bi{\s} sech{\s} M"anner, immer dieselben,
blieben vor dem Hoftor de{\s} Gasthofe{\s} beim St"opselspiel,
bi{\s} e{\s} dunkel wurde.

E{\s} kam ein kalter Winter. Jeden Morgen waren die
Fensterscheiben mit Ei{\s}blumen bedeckt, und da{\s}
Tage{\s}licht, da{\s} wie durch mattgeschliffene{\s} Gla{\s}
hereindrang, blieb mitunter den ganzen Tag "uber tr"ub. Von
nachmittag{\s} vier Uhr an mu"sten die Lampen brennen.

An sch"onen Tagen ging Emma in den Garten hinunter. Der Rauhfrost
hatte "uber die Gr"aser ein silberne{\s} Netz gewoben, dessen
glitzernde Maschen von Halm zu Halm gesponnen waren. Kein Vogel
sang. Die Natur schien zu schlafen. Da{\s} Spalier war mit Stroh
umwickelt, und die Weinst"ocke hingen an der Mauer wie vereiste
Schlangen. Der lesende M"onch unter den Fichten an der Hecke hatte
den rechten Fu"s verloren. Im Frost war die Glasur abgesprungen,
und graue Flecke entstellten ihm nun da{\s} Gesicht.

Nach einer Weile stieg sie wieder hinauf in ihr Zimmer, schlo"s
die T"ur ab und sch"urte da{\s} Feuer im Kamine. In der W"arme
de{\s} Zimmer{\s} ward sie matt, und die Langeweile lastete
schwerer auf ihr. Gern w"are sie hinuntergelaufen, um mit dem
Dienstm"adchen zu plaudern, aber dazu war sie zu stolz.

Alle Morgen um die n"amliche Stunde "offnete dr"uben der
Schulmeister, sein schwarzseidne{\s} K"appchen auf dem Kopfe, die
Fensterl"aden seiner Behausung. Dann marschierte der Landgendarm
mit seinem S"abel vor"uber. Morgen{\s} und abend{\s} wurden die
Postpferde, immer drei auf einmal, zur Tr"anke nach dem Dorfteiche
vorbeigef"uhrt. Von Zeit zu Zeit schellte die T"urklingel
irgendeine{\s} Laden{\s}; und wenn der Wind ging, h"orte man die
Messingbecken, die al{\s} Au{\s}h"angeschilder vor dem
Barbiergesch"afte hingen, an ihre Stange klirren. Da{\s}
Schaufenster schm"uckten ein alte{\s} auf Pappe au{\s}geklebte{\s}
Modenkupfer und eine weibliche Wach{\s}b"uste mit einer gelben
Per"ucke. Der Friseur pflegte "uber seinen brotlosen Beruf und
seine jammervolle Zukunft zu lamentieren; sein h"ochster Traum war
ein Laden in einer gro"sen Stadt, etwa in Rouen, am Kai, in der
N"ahe de{\s} Theater{\s}. M"urrisch wanderte er den ganzen Tag
"uber zwischen dem Gemeindeamt und der Kirche hin und her und
lauerte auf Kundschaft. Sooft Frau Bovary durch ihr Fenster
blickte, sah sie ihn jede{\s}mal in seinem braunen Rock, die
Zipfelm"utze auf dem Haupte, wie einen Wachtposten hin und her
patrouillieren.

Am Nachmittag erschien zuweilen vor den Fenstern de{\s}
E"szimmer{\s} ein sonnengebr"aunter M"annerkopf mit einem
schwarzen Schnurrbarte und einem tr"agen L"acheln um den Mund, in
dem die Z"ahne leuchteten. Al{\s}bald begann eine Walzermelodie
au{\s} einem Leierkasten, auf dessen Deckel ein kleiner Ballsaal
aufgebaut war mit daumenhohen Figuren darin: Frauen in roten
Kopft"uchern, Tiroler in Lodenjacken, Affen in schwarzen R"ocken,
Herren in Kniehosen; alle tanzten sie zwischen den Sofa{\s} und
Lehnst"uhlen und Tischen, wobei sie sich in Spiegelst"ucken
vervielf"altigten, die mit Goldpapier aneinandergereiht waren. Der
Leierkastenmann drehte die Kurbel und sp"ahte dabei nach recht{\s}
und link{\s} nach allen Fenstern. Hin und wieder spie er einen
langen Strahl tabakbraunen Speichel{\s} gegen die Prellsteine oder
stie"s mit dem Knie seinen Kasten in die H"ohe, dessen Gurt ihm
die Schultern dr"uckte. In einem fort, bald schwerm"utig und
schleppend, bald flott und lustig, dudelte die Musik hinter dem
roten Taftbezug, der unter einer schn"orkelhaft au{\s}gestanzten
Messingleiste an den Leierkasten angenagelt war. E{\s} waren
Melodien, die gerade Mode waren und die man "uberall h"orte, in
den Theatern, Salon{\s} und Tanzs"alen, Kl"ange au{\s} der fernen
Welt, die auf diese Weise die einsame Frau erreichten. Diese
Kl"ange im Dreivierteltakt wollten dann nicht wieder au{\s} ihrem
Kopfe weichen. Wie die Bajadere "uber den Blumen ihre{\s}
Teppich{\s}, tanzten ihre Gedanken im Rhythmu{\s} dieser Melodien
und wiegten sich von Traum zu Traum und von Tr"ubsal zu Tr"ubsal.
Wenn der Mann die milden Gaben in seiner M"utze gesammelt hatte,
umh"ullte er seinen Kasten mit einem blauwollnen "Uberzug, nahm
ihn auf den R"ucken und verlie"s da{\s} Dorf schweren
Schritte{\s}. Emma schaute ihm lange nach.

Am unertr"aglichsten waren ihr die Mahlzeiten im E"szimmer unten
im Erdgescho"s. Der Ofen rauchte, die T"ure knarrte, die W"ande
waren feucht und der Fu"sboden kalt. Die ganze Bitterni{\s}
ihre{\s} Dasein{\s} schien ihr da auf ihrem Teller zu liegen, und
au{\s} dem Dampf de{\s} au{\s}gekochten Rindfleische{\s} wehte ihr
gleichsam der Brodem ihre{\s} ihr so widerw"artig gewordenen
Leben{\s} entgegen. Karl a"s und a"s, w"ahrend sie ein paar N"usse
knackte oder, auf die Ellenbogen gest"utzt, sich damit vergn"ugte,
mit der Messerspitze allerlei Linien in da{\s} Wach{\s}tuch zu
kritzeln.

In der Wirtschaft lie"s sie jetzt alle{\s} gehen, wie e{\s} ging.
Ihre Schwiegermutter, die einen Teil der Fastenzeit zu Besuch nach
Toste{\s} kam, war ob diese{\s} Wandel{\s} arg verdutzt. Emma, die
erst in ihrem "Au"seren so akkurat und adrett gewesen war, lief
nunmehr tagelang in ihrem Morgenkleide umher, trug graue
baumwollne Str"umpfe und fing an zu knausern und zu geizen. Sie
meinte, man m"usse sich einschr"anken, da sie nicht reich seien,
f"ugte aber hinzu, sie sei h"ochst zufrieden und "uberau{\s}
gl"ucklich, und in Toste{\s} gefalle e{\s} ihr "uber alle Ma"sen.
Mit solch wunderlichen Reden beschwichtigte sie die alte Frau
Bovary. Im "ubrigen zeigte sie sich f"ur die guten Lehren der
Schwiegermutter nicht empf"anglicher denn fr"uher. Al{\s} diese
gelegentlich die Bemerkung machte, die Herrschaft sei f"ur die
Gotte{\s}furcht der Dienstboten verantwortlich, ward Emma{\s}
Antwort von einem so zornigen Blick und einem so ei{\s}kalten
L"acheln begleitet, da"s die gute Frau ihr nicht wieder zu nahe
kam.

Emma wurde unzug"anglich und launisch. Sie lie"s sich besondre
Gerichte zubereiten, die sie dann aber nicht anr"uhrte; an dem
einen Tage trank sie nicht{\s} al{\s} Milch und am andern ein
Dutzend Tassen Tee. Oft war sie nicht au{\s} dem Hause zu
bekommen, und bald war ihr wieder die Stubenluft zum Ersticken.
Sie sperrte alle Fenster auf und konnte sich nicht leicht genug
anziehen. Wenn sie da{\s} Dienstm"adchen angefahren hatte, machte
sie ihr im n"achsten Augenblicke Geschenke oder lie"s sie in die
Nachbarschaft au{\s}gehen. Au{\s} "ahnlicher Bizarrerie warf sie
bi{\s}weilen armen Leuten alle{\s} Kleingeld hin, da{\s} sie bei
sich hatte, obgleich sie eigentlich gar nicht weichherzig und
mitleidig war, just wie alle Menschen, die auf dem Lande gro"s
geworden sind und leben{\s}lang etwa{\s} von der H"arte der
v"aterlichen H"ande in ihrem Herzen behalten.

Gegen Ende de{\s} Februar{\s} brachte Vater Rouault in Erinnerung
an seine Heilung pers"onlich eine pr"achtige Truthenne und blieb
drei Tage im Hause seine{\s} Schwiegersohne{\s}. W"ahrend Karl auf
Praxi{\s} war, leistete ihm seine Tochter Gesellschaft. Er rauchte
in ihrem Zimmer, spuckte in den Kamin, schwatzte von
Ernteau{\s}sichten, K"albern, K"uhen, H"uhnern und von den
Gemeinderat{\s}sitzungen. Wenn er wieder hinau{\s}gegangen war,
schlo"s sie ihre T"ur mit einem Gef"uhl der Befriedigung ab,
da{\s} ihr selber sonderbar vorkam.

Ihre Verachtung aller Menschen und Dinge verhehlte sie fortan
immer weniger. Bi{\s}weilen gefiel sie sich darin, die
merkw"urdigsten Ansichten zu "au"sern. Sie tadelte, wa{\s} andre
f"ur gut hielten, und billigte Dinge, die f"ur unnat"urlich oder
unmoralisch erkl"art wurden. Karl machte mitunter verwunderte
Augen dazu.

Sollte diese{\s} Jammerdasein ewig dauern? So fragte sie sich
immer wieder. Sollte sie niemal{\s} von hier fortkommen? Sie war
doch ebensoviel wert wie alle die Menschen, die gl"ucklich waren!
In Vaubyessard hatte sie Herzoginnen gesehen, die plumper im
Wuch{\s} waren al{\s} sie und ein gew"ohnlichere{\s} Benehmen
hatten. Sie verw"unschte die Ungerechtigkeit ihre{\s}
Sch"opfer{\s} und dr"uckte ihr Haupt weinend an die W"ande vor
lauter Sehnsucht nach dem Tumult der Welt, ihren n"achtlichen
Ma{\s}keraden und frechen Freuden und allen den Tollheiten, die
sie nicht kannte und die e{\s} doch gab.

Sie wurde immer blasser und litt an Herzklopfen. Karl verordnete
ihr Baldriantropfen und Kampferb"ader. Da{\s} machte sie nur noch
reizsamer.

An manchen Tagen redete sie ohne Unterla"s wie eine Fieberkranke.
Dieser Aufgeregtheit folgte ein pl"otzlicher Umschlag in einen
Zustand von Empfindung{\s}losigkeit. Dann lag sie stumm da, ohne
sich zu r"uhren, und e{\s} wirkte bei ihr nur ein
Belebung{\s}mittel: da{\s} "Ubergie"sen mit K"olnischem Wasser.

Dieweil sie sich fortw"ahrend "uber Toste{\s} beklagte, bildete
sich Karl ein, ihr Leiden sei zweifello{\s} durch irgendwelchen
"ortlichen Einflu"s verursacht, und so begann er ernstlich daran
zu denken, sich in einer andren Gegend niederzulassen.

Um diese Zeit fing Emma an, Essig zu trinken, weil sie mager
werden wollte. Sie bekam einen leichten trocknen Husten und verlor
jegliche E"slust.

E{\s} fiel Karl sehr schwer, Toste{\s} aufzugeben, wo er gerade
jetzt, nach vierj"ahriger Praxi{\s}, ein gemachter Mann war.
Indessen, e{\s} mu"ste sein! Er lie"s Emma in Rouen von seinem
ehemaligen Lehrmeister untersuchen. E{\s} sei ein nerv"ose{\s}
Leiden; Luftver"anderung w"are vonn"oten.

Karl zog nun allerort{\s} Erkundigungen ein, und da brachte er in
Erfahrung, da"s im Bezirk von Neufch\^atel in einem gr"o"seren
Marktflecken namen{\s} Abtei Yonville der bi{\s}herige Arzt, ein
polnischer Ref"ugi\'e, in der vergangenen Nacht da{\s} Weite
gesucht hatte. Er schrieb an den dortigen Apotheker und erkundigte
sich, wieviel Einwohner der Ort habe, wie weit die n"achsten
Kollegen entfernt s"a"sen und wie hoch die Jahre{\s}einnahme
de{\s} Verschwundenen gewesen sei. Die Antwort fiel befriedigend
au{\s}, und infolgedessen entschlo"s sich Bovary, zu Beginn de{\s}
kommenden Fr"uhjahre{\s} nach Abtei Yonville "uberzusiedeln,
fall{\s} sich Emma{\s} Zustand noch nicht gebessert habe.

Eine{\s} Tage{\s} kramte Emma de{\s} bevorstehenden Umzuge{\s}
wegen in einem Schubfache. Da ri"s sie sich in den Finger und zwar
an einem der Dr"ahte ihre{\s} Hochzeit{\s}strau"se{\s}. Die
Orangenknospen waren grau vor Staub, und da{\s} Atla{\s}band mit
der silbernen Franse war au{\s}gefranst. Sie warf den Strau"s in
da{\s} Feuer. Er flackerte auf wie trockne{\s} Stroh. Eine Weile
gl"uhte er noch wie ein feuriger Busch "uber der Asche, dann sank
er langsam in sich zusammen. Nachdenklich sah Emma zu. Die kleinen
Beeren au{\s} Pappmasse platzten, die Dr"ahte kr"ummten sich, die
Silberfransen schmolzen. Die verkohlte Papiermanschette zerfiel,
und die St"ucke flatterten im Kamine hin und her wie schwarze
Schmetterlinge, bi{\s} sie in den Rauchfang hinaufflogen~...

Bei dem Weggange von Toste{\s}, im M"arz, ging Frau Bovary einer
guten Hoffnung entgegen.


\newpage
\thispagestyle{empty}
\begin{center}
\vspace{5cm}
{\Huge \so{Zweite{\s} Bu{ch}}}
\end{center}


\newpage\begin{center}
{\large \so{Er{st}e{\s} Kapitel}}\bigskip\bigskip
\end{center}

Abtei Yonville (so genannt nach einer ehemaligen Kapuzinerabtei,
von der indessen nicht einmal mehr die Ruinen stehen) ist ein
Marktflecken, acht Wegstunden "ostlich von Rouen, zwischen der
Stra"se von Abbeville und der von Beauvai{\s}. Der Ort liegt im
Tale der Rieule, eine{\s} Nebenfl"u"schen{\s} der Andelle. Nahe
seiner Einm"undung treibt der Bach drei M"uhlen. Er hat Forellen,
nach denen die Dorfjungen reihenweise an den Sonntagen zu ihrer
Belustigung angeln.

Man verl"a"st die Heere{\s}stra"se bei La Boissi\`ere und geht auf
der Hochebene bi{\s} zur H"ohe von Leux, wo man da{\s} Tiefland
offen vor sich liegen sieht. Der Flu"s teilt e{\s} in zwei
deutlich unterscheidbare H"alften: zur Linken Weideland, recht{\s}
ist alle{\s} bebaut. Diese Pr"arie, die sich bi{\s} zu den Triften
der Landschaft Pray hinzieht, wird von einer ganz niedrigen
H"ugelkette begrenzt, w"ahrend die Ebene gegen Osten allm"ahlich
ansteigt und sich im Unerme"slichen verliert. So weit da{\s} Auge
reicht, schweift e{\s} "uber meilenweite Kornfelder. Da{\s}
Gew"asser sondert wie mit einem langen wei"sen Strich da{\s} Gr"un
der Wiesen von dem Blond der "Acker, und so liegt da{\s} ganze
Land unten au{\s}gebreitet da wie ein riesiger gelber Mantel mit
einem gr"unen silbernges"aumten Samtkragen.

Fern am Horizont erkennt man geradeau{\s} den Eichwald von Argueil
und die steilen Abh"ange von Sankt Johann mit ihren eigent"umlichen,
senkrechten, ungleichm"a"sigen roten Strichen. Da{\s} sind die
Wege, die sich da{\s} Regenwasser sucht; und die roten Streifen
auf dem Grau der Berge r"uhren von den vielen eisenhaltigen
Quellen drinnen im Gebirge her, die ihr Wasser nach allen Seiten
hinab in{\s} Land schicken.

Man steht auf der Grenzscheide der Normandie, der Pikardie und der
Ile-de-France, inmitten eine{\s} von der Natur stiefm"utterlich
behandelten Ge\-l"ande{\s}, da{\s} weder im Dialekt seiner Bewohner
noch in seinem Landschaft{\s}bilde besondre Eigenheiten aufweist.
Von hier kommen die allerschlechtesten K"ase de{\s} ganzen
Bezirk{\s} von Neufch\^atel. Allerding{\s} ist die Bewirtschaftung
dieser Gegend kostspielig, da der trockene steinige Sandboden viel
D"unger verlangt.

Bi{\s} zum Jahre 1835 f"uhrte keine brauchbare Stra"se nach
Yonville. Erst um diese Zeit wurde ein sogenannter
"`Hauptvizinalweg"' angelegt, der die beiden gro"sen
Heere{\s}stra"sen von Abbeville und von Amien{\s} untereinander
verbindet und bi{\s}weilen von den Fuhrleuten benutzt wird, die
von Rouen nach Flandern fahren. Aber trotz dieser "`neuen
Verbindungen"' gelangte Yonville zu keiner rechten Entwicklung.
Anstatt sich mehr auf den Getreidebau zu legen, blieb man
hartn"ackig immer noch bei der Weidebewirtschaftung, so kargen
Gewinn sie auch brachte; und die tr"age Bewohnerschaft baut sich
auch noch heute lieber nach dem Berge statt nach der Ebene zu an.
Schon von weitem sieht man den Ort am Ufer lang hingestreckt
liegen, wie einen Kuhhirten, der sich faulenzend am Bache
hingeworfen hat.

Von der Br"ucke, die "uber die Rieule f"uhrt, geht der mit Pappeln
bes"aumte Fahrweg in schnurgerader Linie nach den ersten Geh"often
de{\s} Orte{\s}. Alle sind sie von Hecken umschlossen. Neben den
Hauptgeb"auden sieht man allerhand ordnung{\s}lo{\s} angelegte
Nebenh"au{\s}chen, Keltereien, Schuppen und Brennereien,
dazwischen buschige B"aume, an denen Leitern, Stangen, Sensen und
andre{\s} Ger"at h"angen oder lehnen. Die Strohd"acher sehen wie
bi{\s} an die Augen in{\s} Gesicht hereingezogene Pelzm"utzen
au{\s}; sie verdecken ein Drittel der niedrigen
Butzenscheibenfenster. Da und dort rankt sich d"urre{\s}
Spalierobst an den wei"sen, von schwarzem Geb"alk durchquerten
Kalkw"anden der H"auser empor. Die Eing"ange im Erdgescho"s haben
drehbare Halbt"uren, damit die H"uhner nicht eindringen, die auf
den Schwellen in Apfelwein aufgeweichte Brotkrumen aufpicken.

Allm"ahlich werden die H"ofe enger, die Geb"aude r"ucken n"aher
aneinander, und die Hecken verschwinden. An einem der H"auser
h"angt, schaukelnd an einem Besenstiel zum Fenster herau{\s}, ein
B"undel Farnkraut. Hier ist die Schmiede; ein Wagen und zwei oder
drei neue Karren stehen davor und versperren die Stra"se.
Weiterhin leuchtet durch die offene Pforte der Gartenmauer ein
wei"se{\s} Landhau{\s}, eine runde Rasenfl"ache davor mit einem
Amor in der Mitte, der sich den Finger vor den Mund h"alt. Die
Freitreppe flankieren zwei Vasen au{\s} Bronze. Ein Amt{\s}schild
mit Wappen gl"anzt am Tore. E{\s} ist da{\s} Hau{\s} de{\s}
Notar{\s}, da{\s} sch"onste der ganzen Gegend.

Zwanzig Schritte weiter, auf der andern Seite der Stra"se, beginnt
der Marktplatz mit der Kirche. In dem kleinen Friedhofe um sie
herum, den eine niedrige Mauer von Ellbogenh"ohe umschlie"st,
liegt Grabplatte an Grabplatte. Diese alten Steine bilden geradezu
ein Pflaster, auf da{\s} au{\s} den Ritzen hervorschie"sende{\s}
Gra{\s} gr"une Rechtecke gezeichnet hat. Die Kirche selbst ist ein
Neubau au{\s} der letzten Zeit der Regierung Karl{\s} de{\s}
Zehnten. Da{\s} h"olzerne Dach beginnt bereit{\s} morsch zu
werden. Auf dem blauen Anstrich der Decke "uber dem Schiff zeigen
sich stellenweise schwarze Flecken. "Uber dem Eingang befindet
sich da, wo gew"ohnlich sonst in der Kirche die Orgel ist, eine
Empore f"ur die M"anner, zu der eine Wendeltreppe hinauff"uhrt,
die laut dr"ohnt, wenn man sie betritt.

Da{\s} Tage{\s}licht flutet in schr"agen Strahlen durch die
farblosen Scheiben auf die Bankreihen hernieder, die sich von
L"ang{\s}wand zu L"ang{\s}wand hinziehen. Vor manchen Sitzen sind
Strohmatten befestigt, und Namen{\s}schilder verk"unden weithin
sichtbar: "`Platz de{\s} Herrn Soundso."' Wo sich da{\s} Schiff
verengert, steht der Beichtstuhl und ihm gegen"uber ein Standbild
der Madonna, die ein Atla{\s}gewand und einen Schleier, mit lauter
silbernen Sternen bes"at, tr"agt. Ihre Wangen sind genau so
knallrot angemalt wie die eine{\s} G"otzenbilde{\s} auf den
Sandwichinseln. Im Chor "uber dem Hochaltar schimmert hinter vier
hohen Leuchtern die Kopie einer Heiligen Familie von Pietro
Perugino, eine Stiftung der Regierung. Die Chorst"uhle au{\s}
Fichtenholz sind ohne Anstrich.

Fast die H"alfte de{\s} Marktplatze{\s} von Yonville nehmen "`die
Hallen"' ein: ein Ziegeldach auf etlichen zwanzig Holzs"aulen.
Da{\s} Rathau{\s}, nach dem Entwurfe eine{\s} Pariser Architekten
in antikem Stil erbaut, steht in der jenseitigen Ecke de{\s}
Platze{\s} neben der Apotheke. Da{\s} Erdgescho"s hat eine
dorische S"aulenhalle, der erste Stock eine offene Galerie, und
dar"uber im Giebelfelde haust ein gallischer Hahn, der mit der
einen Klaue da{\s} Gesetzbuch umkrallt und in der andern die Wage
der Gerechtigkeit h"alt.

Da{\s} Augenmerk de{\s} Fremden f"allt immer zuerst auf die
Apotheke de{\s} Herrn Homai{\s}, schr"ag gegen"uber vom "`Gasthof
zum goldnen L"owen"'. Zumal am Abend, wenn die gro"se Lampe im
Laden brennt und ihr helle{\s}, durch die bunten Fl"ussigkeiten in
den dickbauchigen Flaschen, die da{\s} Schaufenster schm"ucken
sollen, rot und gr"un gef"arbte{\s} Licht weit hinau{\s} "uber
da{\s} Stra"senpflaster f"allt, dann sieht man den Schattenri"s
de{\s} "uber sein Pult gebeugten Apotheker{\s} wie in bengalischer
Beleuchtung. Au"sen ist sein Hau{\s} von oben bi{\s} unten mit
Reklameschildern bedeckt, die in allen m"oglichen Schriftarten
au{\s}schreien: "`Mineralwasser von Vichy"', "`Sauerbrunnen"',
"`Selter{\s}wasser"', "`Kamillentee"', "`Kr"auterlik"or"',
"`Kraftmehl"', "`Hustenpastillen"', "`Zahnpulver"', "`Mundwasser"',
"`Bandagen"', "`Badesalz"', "`Gesundheit{\s}schokolade"' usw. usw.
Auf der Firma, die so lang ist wie der ganze Laden, steht in
m"achtigen goldnen Buchstaben: "`Homai{\s}, Apotheker"'. Drinnen,
hinter den hohen, auf der Ladentafel festgeschraubten Wagen, liest
man "uber einer Gla{\s}t"ure da{\s} Wort "`Laboratorium"' und auf
der T"ur selbst noch einmal in goldnen Lettern auf schwarzem
Grunde den Namen "`Homai{\s}"'.

Weitere Sehen{\s}w"urdigkeiten gibt e{\s} in Yonville nicht. Die
Hauptstra"se (die einzige) reicht einen B"uchsenschu"s weit und
hat zu beiden Seiten ein paar Kraml"aden. An der Stra"senbiegung
ist der Ort zu Ende. Wenn man vorher nach link{\s} abwendet und
dem Hange folgt, gelangt man hinab zum Gemeindefriedhof.

Zur Zeit der Cholera wurde ein St"uck der Kirchhof{\s}mauer
niedergelegt und der Friedhof durch Ankauf von drei Morgen Land
vergr"o"sert, aber dieser ganze neue Teil ist so gut wie noch
unbenutzt geblieben. Wie vordem dr"angen sich die Grabh"ugel nach
dem Eingang{\s}tor zu zusammen. Der Pf"ortner, der zugleich auch
Totengr"aber und Kirchendiener ist und somit au{\s} den Leichen
der Gemeinde eine doppelte Einnahme zieht, hat sich da{\s}
unbenutzte Land angeeignet, um darauf Kartoffeln zu erbauen. Aber
von Jahr zu Jahr vermindert sich sein bi"schen Boden, und e{\s}
brauchte blo"s wieder einmal eine Epidemie zu kommen, so w"u"ste
er nicht, ob er sich "uber die vielen Toten freuen oder "uber ihre
neuen Gr"aber "argern solle.

"`Lestiboudoi{\s}, Sie leben von den Toten!"' sagte eine{\s}
Tage{\s} der Pfarrer zu ihm.

Diese gruselige Bemerkung stimmte den K"uster nachdenklich. Eine
Zeitlang enthielt er sich der Landwirtschaft. Dann aber und bi{\s}
auf den heutigen Tag zog er seine Erd"apfel weiter. Ja, er
versichert sogar mit Nachdruck, sie w"uchsen ganz von selber.

Seit den Ereignissen, die hier erz"ahlt werden, hat sich in
Yonville wirklich nicht{\s} ver"andert. Noch immer dreht sich auf
der Kirchturmspitze die wei"s-rot-blaue Fahne au{\s} Blech, noch
immer flattern vor dem Laden de{\s} Modewarenh"andler{\s} zwei
Kattunwimpel im Winde, noch immer schwimmen im Schaufenster der
Apotheke h"a"sliche Pr"aparate in Gla{\s}b"uchsen voll
tr"ubgewordnem Alkohol, und ganz wie einst zeigt der alte, von
Wind und Wetter ziemlich entgoldete L"owe "uber dem Tore de{\s}
Gasthofe{\s} den Vor"ubergehenden seine Pudelm"ahne.

An dem Abend, da da{\s} Ehepaar Bovary in Yonville eintreffen
sollte, war die L"owenwirtin, die Witwe Franz, derartig
besch"aftigt, da"s ihr beim Hantieren mit ihren T"opfen der
Schwei"s von der Stirne perlte. Am folgenden Tag war n"amlich
Markttag im St"adtchen. Da mu"ste Fleisch zurechtgehackt,
Gefl"ugel au{\s}genommen, Bouillon gekocht und Kaffee gebrannt
werden. Daneben die regelm"a"sigen Tischteilnehmer und heute
obendrein der neue Doktor nebst Frau Gemahlin und Dienstm"adchen!
Am Billard lachten G"aste, und in der kleinen Gaststube riefen
drei M"ullerburschen nach Schnap{\s}. Im Herde prasselte und
schmorte e{\s}, und auf dem langen K"uchentische paradierten neben
einer rohen Hammelkeule St"o"se von Tellern, die nach dem Takte
de{\s} Wiegemesser{\s} tanzten, mit dem die K"ochin Spinat
zerkleinerte. Vom Hofe au{\s} ert"onte da{\s} "angstliche Gegacker
der H"uhner, die von der Magd gejagt wurden, weil sie etlichen die
K"opfe abschneiden wollte.

Ein Herr in gr"unledernen Pantoffeln, eine goldne Troddel an
seinem schwarz\-samt\-nen K"appchen, w"armte sich am Kamin de{\s}
Gast\/zimmer{\s} den R"ucken. Im Gesicht hatte er ein paar
Blatternarben. Sein ganze{\s} Wesen strahlte f"ormlich von
Selbst\/zufriedenheit. Offenbar lebte er genau so gleichm"utig dahin
wie der Stieglitz, der oben an der Decke in seinem Weidenbauer
herumh"upfte. Dieser Herr war der Apotheker.

"`Artemisia!"' rief die Wirtin. "`Leg noch ein bi"schen Reisig
in{\s} Feuer! F"ulle die Wasserflaschen! Schaff den Schnap{\s}
hinein! Und mach schnell! Ach, wenn ich nur w"u"ste, wa{\s} ich
den Herrschaften, die heute eintreffen, zum Nachtisch vorsetzen
soll? Heiliger Bimbam! Die Leute von der Spedition{\s}gesellschaft
h"oren mit ihrem Geklapper auf dem Billard auch gar nicht auf! Und
der M"obelwagen steht drau"sen immer noch mitten auf der Stra"se,
gerade vor der Hofeinfahrt! Wenn die Post kommt, wird e{\s} eine
Karambolage geben. Ruf mir mal Hippolyt! Er soll den Wagen
beiseiteschieben ... Wa{\s} ich sagen wollte, Herr Apotheker,
diese Leute spielen schon den ganzen Vormittag. Jetzt sind sie bei
der f"unfzehnten Partie und beim achten Schoppen Apfelwein! Man
wird mir noch ein Loch in{\s} Tuch sto"sen!"'

Sie war auf einen Augenblick, den Kochl"offel in der Hand, in{\s}
Gast\/zimmer gelaufen.

"`Da{\s} w"ar auch weiter kein Malheur!"' meinte Homai{\s}. "`Dann
schaffen Sie gleich ein neue{\s} Billard an!"'

"`Ein neue{\s} Billard!"' jammerte die Witwe.

"`Nu freilich, Frau Franz! Da{\s} alte Ding da taugt nicht mehr
viel! Ich hab{\s} Ihnen schon tausendmal gesagt. E{\s} ist Ihr
eigner Schaden! Und ein gro"ser Schaden! Heutzutage verlangen
passionierte Spieler gro"se B"alle und schwere Queue{\s}. Mit
solchen B"allchen spielt man nicht mehr. Die Zeiten "andern sich!
Man mu"s modern sein! Sehen Sie sich mal bei Tellier im Caf\'e
Fran\c{c}ai{\s}~..."'

Die Wirtin wurde rot vor "Arger, aber der Apotheker fuhr fort:

"`Sie k"onnen sagen, wa{\s} Sie wollen! Sein Billard ist
handlicher al{\s} Ihr{\s}. Und wenn e{\s} hei"st, eine
patriotische Poule zu entrieren, sagen wir: zum Besten der
vertriebenen Polen oder f"ur die "Uberschwemmten von Lyon~..."'

"`Ach wa{\s}!"' unterbrach ihn die L"owenwirtin ver"achtlich.
"`Vor dem Bettelvolk hat unsereiner noch lange keine Angst! Lassen
Sie{\s} nur gut sein, Herr Apotheker! Solange der Goldne L"owe
bestehen wird, sitzen auch G"aste drin! Wir verhungern nicht! Aber
Ihr geliebte{\s} Caf\'e Fran\c{c}ai{\s}, da{\s} wird eine{\s}
sch"onen Tage{\s} die Bude zumachen! Oder vielmehr der
Gericht{\s}vollzieher! Ich soll mir ein andre{\s} Billard
anschaffen? Wo mein{\s} so bequem ist zum W"aschefalten! Und wenn
Jagdg"aste da sind, k"onnen gleich sechse drauf "ubernachten! Nee,
nee ... Wo bleibt nur eigentlich der langweilige Kerl, der
Hivert!"'

"`Sollen denn Ihre Tischg"aste mit dem Essen warten, bi{\s} die
Post gekommen ist?"' fragte Homai{\s} ungeduldig.

"`Warten? Herr Binet ist ja noch nicht da! Der kommt Schlag
sech{\s}, einen wie alle Tage! So ein Muster von P"unktlichkeit
gibt{\s} auf der ganzen Welt nicht wieder. Er hat seit
urdenklichen Zeiten seinen Stammplatz in der kleinen Stube. Er
lie"se sich eher totschlagen, al{\s} da"s er wo ander{\s} "a"se.
Wa{\s} Schlechte{\s} darf man dem nicht vorsetzen. Und auf den
Apfelwein versteht er sich au{\s} dem ff. Er ist nicht wie Herr
Leo, der heute um sieben und morgen um halb acht erscheint und
alle{\s} i"st, wa{\s} man ihm vorsetzt! "Ubrigen{\s} ein feiner
junger Mann! Ich hab noch nie ein laute{\s} Wort von ihm
geh"ort."'

"`Da sehen Sie eben den Unterschied zwischen jemandem, der eine
Kinderstube hinter sich hat, und einem ehemaligen K"urassier und
jetzigen Steuereinnehmer!"'

E{\s} schlug sech{\s}. Binet trat ein.

Er hatte einen blauen Rock an, der schlaff an seinem mageren
K"orper herunterhing. Unter dem Schirm seiner Lederm"utze blickte
ein Kahlkopf hervor, der um die Stirn eingedr"uckt von dem
langj"ahrigen Tragen de{\s} schweren Helm{\s} au{\s}sah. Er trug
eine Weste au{\s} schwarzem Stoff, einen Pelzkragen, graue Hosen
und tadello{\s} blankgewichste Schuhe, die vorn besonder{\s}
au{\s}gearbeitet waren, weil er dauernd an geschwollenen Zehen
litt. Sein blonder Backenbart war peinlichst gestutzt und umrahmte
ihm da{\s} lange bleiche Gesicht mit den kleinen Augen und der
Adlernase wie eine Hecke den Garten. Er war ein Meister in
jeglichem Kartenspiel und ein guter J"ager, hatte eine h"ubsche
Handschrift und besa"s zu Hause eine Drehbank, auf der er zu
seinem Vergn"ugen Serviettenringe drechselte. Er hatte ihrer schon
eine Unmenge, die er mit der Eifersucht eine{\s} K"unstler{\s} und
dem Geiz de{\s} Spie"ser{\s} h"utete.

Binet schritt nach der kleinen Stube zu. Erst mu"sten dort aber
die drei M"ullerburschen hinau{\s}komplimentiert werden. W"ahrend
man drin f"ur ihn deckte, blieb er in der gro"sen Gaststube stumm
in der N"ahe de{\s} Ofen{\s} stehen, dann ging er hinein, klinkte
die T"ure ein und nahm seine M"utze ab. Da{\s} hatte alle{\s} so
seine Ordnung.

"`An "uberm"a"siger H"oflichkeit wird der mal nicht sterben!"'
bemerkte der Apotheker, al{\s} er wieder mit der Wirtin allein
war.

"`Er redet nie viel,"' entgegnete diese. "`Vergangene Woche waren
zwei Tuchreisende hier, lustige Kerle, die un{\s} den ganzen Abend
Schnurren erz"ahlt haben. Ich w"are beinahe umgekommen vor Lachen.
Der aber hat wie ein Stockfisch dabeigesessen und keine Miene
verzogen."'

"`Ja, ja,"' sagte der Apotheker, "`der Mensch hat keine Phantasie,
keinen Witz, keinen geselligen Sinn!"'

"`Er soll aber wohlhabend sein,"' warf die Wirtin ein.

"`Wohlhabend?"' echote Homai{\s}. "`Der und wohlhabend!"' Und
gelassen f"ugte er hinzu: "`Gott ja, so f"ur seine Verh"altnisse.
Da{\s} ist schon m"oglich!"'

Nach einer kleinen Weile fuhr er fort: "`Hm! Wenn ein Kaufmann,
der ein gro"se{\s} Gesch"aft hat, oder ein Recht{\s}anwalt, ein
Arzt, ein Apotheker derartig in seinem Beruf aufgeht, da"s er zum
Grie{\s}gram oder Sonderling wird, so verstehe ich da{\s}. Davor
gibt e{\s} Beispiele und Exempel. Solche Leute haben immerhin
Gedanken im Kopfe. Wie oft ist{\s} mir nicht selber passiert, da"s
ich meinen Federhalter auf meinem Schreibtische gesucht habe, um
ein Schildchen au{\s}zuf"ullen oder so wa{\s}, -- und wei"s der
Kuckuck, schlie"slich hatte ich ihn hinterm rechten Ohre
stecken!"'

Frau Franz ging indessen an die Hau{\s}t"ur, um nachzusehen, ob
die Post noch nicht angekommen sei. Sie war ganz aufgeregt. Da
trat ein schwarz gekleideter Mann in die K"uche. Da{\s}
D"ammerlicht beleuchtete sein kupferrote{\s} Antlitz und umflo"s
seine herkulischen Linien.

"`Wa{\s} steht dem Herrn Pfarrer zu Diensten?"' fragte die Wirtin
und nahm vom Kaminsim{\s} einen der Messingleuchter, die mit ihren
wei"sen Kerzen in einer wohlgeordneten Reihe dastanden. "`Haben
Ehrw"urden einen Wunsch? Ein Gl"a{\s}chen Wacholder oder einen
Schoppen Wein?"'

Der Priester dankte verbindlich. Er kam wegen seine{\s}
Regenschirme{\s}, den er tag{\s} zuvor im Kloster Ernemont hatte
stehen lassen. Nachdem er Frau Franz gebeten hatte, ihn
gelegentlich holen und im Pfarrhause abgeben zu lassen, empfahl er
sich, um nach der Kirche zu gehen, wo schon da{\s} Ave-Maria
gel"autet ward.

Al{\s} die Tritte de{\s} Geistlichen drau"sen verklungen waren,
machte der Apotheker die Bemerkung, der Pfarrer habe sich eben
sehr ungeb"uhrlich benommen. Eine angebotene Erfrischung
abzuschlagen, sei seiner Ansicht nach eine ganz abscheuliche
Heuchelei. Die Pfaffen s"offen in{\s}geheim alle miteinander. Am
liebsten m"ochten sie den Zehnten wieder einf"uhren.

Die L"owenwirtin verteidigte ihren Beichtvater.

"`Na, "ubrigen{\s} nimmt er{\s} mit vier Mannsen von Eurem Kaliber
zugleich auf!"' meinte sie. "`Vorige{\s} Jahr hat er unsern Leuten
beim Strohaufladen geholfen. Er hat immer sech{\s} Sch"utten auf
einmal getragen. So stark ist er!"'

"`Nat"urlich!"' rief Homai{\s} au{\s}. "`Schickt nur Eure
M"adel{\s} solchen Krafthubern zur Beichte! Wenn ich im Staate
wa{\s} zu sagen h"atte, dann kriegte jeder Pfaffe aller vier
Wochen einen Blutegel angesetzt. Jawohl, Frau Wirtin, aller vier
Wochen einen ordentlichen Aderla"s zur Hebung von Sicherheit und
Sittlichkeit im Lande!"'

"`Aber Herr Apotheker! Sie sind gottlo{\s}! Sie haben keine Religion!"'

Homai{\s} erwiderte:

"`Ich habe eine Religion: meine Religion! Und die ist mehr wert
al{\s} die dieser Leute mit all dem Firlefanz und Mummenschanz.
Ich verehre Gott. Erst recht tue ich da{\s}. Ich glaube an eine
h"ohere Macht, an einen Sch"opfer. Sein Wesen kommt hierbei nicht
in Frage. Wir Menschen sind hienieden da, damit wir unsre
Pflichten al{\s} Staat{\s}b"urger und Familienv"ater erf"ullen.
Aber ich habe kein Bed"urfni{\s}, in die Kirche zu gehen,
silberne{\s} Ger"at zu k"ussen und eine Bande von Possenrei"sern
au{\s} meiner Tasche zu m"asten, die sich besser hegen und pflegen
al{\s} ich mich selber. Gott kann man viel sch"oner verehren im
Walde, im freien Felde oder meinetwegen nach antiker Anschauung
angesicht{\s} der Gestirne am Himmel. Mein Gott ist der Gott der
Philosophen und K"unstler. Ich bin f"ur Rousseau{\s}
Glauben{\s}bekenntni{\s} de{\s} savoyischen Vikar{\s}. F"ur die
unsterblichen Ideen von Anno 1789! Und da glaube ich nicht an den
sogenannten lieben Gott, der mit einem Spazierst"ockchen in der
Hand gem"utlich durch seinen Erdengarten bummelt, seine Freunde in
einem Walfischbauch einquartiert, jammernd am Kreuze stirbt und am
dritten Tage wieder aufersteht von den Toten. Da{\s} ist schon an
und f"ur sich Bl"odsinn und obendrein wider alle Naturgesetze!
E{\s} beweist aber nebenbei, da"s sich die Pfaffen in der
schmachvollen Ignoranz, mit der sie die Menschheit verdummen
m"ochten, mir Wollust selber herumsielen."'

Er schwieg und "uberschaute seine Zuh"orerschaft. Er hatte sich
in{\s} Zeug gelegt, al{\s} spr"ache er vor versammeltem
Gemeinderat. Die Wirtin war l"angst au{\s} der Gaststube gelaufen.
Sie lauschte drau"sen und vernahm ein ferne{\s} rollende{\s}
Ger"ausch. Bald h"orte sie deutlich da{\s} Rasseln der R"ader und
da{\s} Klappern eine{\s} lockeren Eisen{\s} auf dem Pflaster.
Endlich hielt die Postkutsche vor der Hau{\s}t"ure.

E{\s} war ein gelblackierter Kasten auf zwei Riesenr"adern, die
bi{\s} an da{\s} Wagendeck hinaufreichten. Sie raubten dem
Reisenden jegliche Au{\s}sicht und bespritzten ihn fortw"ahrend.
Die winzigen Scheiben in den Wagenfenstern klirrten in ihrem
Rahmen. Wenn man sie heraufzog, sah man, da"s sie vor Staub und
Stra"senschmutz starrten. Der st"arkste Platzregen h"atte sie
nicht rein gewaschen. Da{\s} Fahrzeug war mit drei Pferden
bespannt: zwei Stangen- und einem Vorderpferde.

Vor dem Gasthofe entstand ein kleiner Menschenauflauf. Alle{\s}
redete durcheinander. Der eine fragte nach Neuigkeiten, ein andrer
wollte irgendwelche Au{\s}kunft, ein dritter erwartete eine
Postsendung. Hivert, der Postkutscher, wu"ste gar nicht, wem er
zuerst Bescheid geben sollte. Er pflegte n"amlich allerlei
Auftr"age f"ur die Landleute in der Stadt zu "ubernehmen. Er
machte Eink"aufe, brachte dem Schuster Leder und dem Schmied
alte{\s} Eisen mit; er besorgte der Posthalterin eine Tonne
Heringe, holte von der Modistin Hauben und vom Friseur
Lockenwickel. Auf dem R"uckwege verteilte er dann die Pakete
l"ang{\s} seiner Fahrstra"se. Wenn er am Geh"oft eine{\s}
Auftraggeber{\s} vorbeifuhr, schrie er au{\s} voller Kehle und
warf da{\s} Paket "uber den Zaun in da{\s} Grundst"uck, wobei er
sich von seinem Kutscherbocke erhob und die Pferde eine Strecke
ohne Z"ugel laufen lie"s.

Heute kam er mit Versp"atung. Unterweg{\s} war Frau Bovary{\s}
Windspiel querfeldein weggelaufen. Eine Viertelstunde lang pfiff
man nach ihm. Hivert lief sogar ein paar Kilometer zur"uck; aller
Augenblicke glaubte er, den Hund von weitem zu sehen. Schlie"slich
aber mu"ste weitergefahren werden.

Emma weinte und war ganz au"ser sich. Karl sei an diesem Ungl"uck
schuld. Herr Lheureux, der Modewarenh"andler, der mit in der Post
fuhr, versuchte sie zu tr"osten, indem er ein Schock Geschichten
von Hunden erz"ahlte, die entlaufen waren und sich nach langen
Jahren bei ihren einstigen Herren wieder eingestellt hatten. Unter
anderem wu"ste er von einem Dackel zu berichten, der von
Konstantinopel au{\s} den Weg nach Pari{\s} zur"uckgefunden haben
sollte. Ein andrer Hund war hinter einander drei"sig Meilen
gelaufen und hatte dabei vier Fl"usse durchschwommen. Und sein
eigner Vater hatte einen Pudel besessen; der war volle zw"olf
Jahre weg. Eine{\s} Abend{\s}, al{\s} der alte Lheureux durch die
Stadt nach dem Gasthau{\s} ging, sprang der Hund an ihm hoch.


\newpage\begin{center}
{\large \so{Zweite{\s} Kapitel}}\bigskip\bigskip
\end{center}

Emma stieg zuerst au{\s}, nach ihr Felicie, dann Herr Lheureux und
eine Amme. Karl mu"ste man erst aufwecken. Er war in seiner Ecke
beim Einbruch der Dunkelheit fest eingeschlafen.

Homai{\s} stellte sich vor. Er ersch"opfte sich der "`gn"adigen
Frau"' und dem "`Herrn Doktor"' gegen"uber in Galanterien und
H"oflichkeiten. Er sei ent\/z"uckt, sagte er, bereit{\s} Gelegenheit
gehabt zu haben, ihnen gef"allig sein zu d"urfen. Und in
herzlichem Tone f"ugte er hinzu, er l"ude sich f"ur heute bei
ihnen zu Tisch ein. Er sei Strohwitwer.

Frau Bovary begab sich in die K"uche und an den Herd. Mit den
Fingerspitzen fa"ste sie ihr Kleid in der Kniegegend, zog e{\s}
bi{\s} zu den Kn"ocheln herauf und w"armte ihre mit
schwarzledernen Stiefeletten bekleideten F"u"se an der Glut, in
der die Hammelkeule am Spie"s gedreht wurde. Da{\s} Feuer
beleuchtete ihre ganze Gestalt und warf grelle Lichter auf den
Stoff ihre{\s} Kleide{\s}, auf ihre por"ose wei"se Haut und in die
Wimpern ihrer Augen, die sich von Zeit zu Zeit schl"ossen. Der
Luft\/zug strich durch die halboffene T"ur und r"otete die Flammen.
Hochrote Reflexe umflossen die Frau am Herd. Am andern Ende
de{\s}selben stand ein junger Mann mit blondem Haar, der sie stumm
betrachtete.

E{\s} war Leo D"upui{\s}, der Adjunkt de{\s} Notar{\s} Guillaumin,
einer der Stamm\-g"aste im Goldnen L"owen. Er langweilte sich
geh"orig in Yonville, und de{\s}halb kam er zu Tisch "ofter{\s}
absichtlich zu sp"at, in der Hoffnung, mit irgendeinem Reisenden
den Abend im Wirt{\s}hause verplaudern zu k"onnen. Wenn er aber in
der Kanzlei gerade gar nicht{\s} zu tun hatte, mu"ste er au{\s}
Langeweile wohl oder "ubel p"unktlich erscheinen und von der Suppe
bi{\s} zum K"ase Binet{\s} Gesellschaft erdulden. Frau Franz hatte
ihm den Vorschlag gemacht, heute mit den neuen G"asten zusammen zu
essen; er war mit Vergn"ugen darauf eingegangen. Zur Feier de{\s}
Tage{\s} war im Saal f"ur vier Personen gedeckt worden.

Man versammelte sich daselbst. Homai{\s} bat um Erlaubni{\s}, sein
K"appchen aufbehalten zu d"urfen. Er erk"alte sich leicht.

Frau Bovary sa"s ihm beim Essen zur Rechten.

"`Gn"adige Frau sind zweifello{\s} ein wenig m"ude?"' begann er.
"`In un{\s}rer alten Postkutsche wird man schauderhaft
durchger"uttelt."'

"`Freilich!"' gab Emma zur Antwort. "`Aber diese{\s} Dr"uber und
Drunter macht mir gerade Spa"s. Ich liebe die Abwechselung."'

"`Ach ja, immer auf demselben Platze hocken ist gr"a"slich!"'
seufzte der Adjunkt.

"`Wenn Sie wie ich den ganzen Tag auf dem Gaule sitzen
m"u"sten~..."', warf Karl ein.

Leo wandte sich an Emma:

"`Grade da{\s} denke ich mir k"ostlich. Nat"urlich mu"s man ein
guter Reiter sein."'

"`Ein praktizierender Arzt hat{\s} "ubrigen{\s} in hiesiger Gegend
ziemlich bequem"', meinte der Apotheker. "`Die Wege sind n"amlich
soweit imstand, da"s man ein Kabriolett verwenden kann. Im
allgemeinen lohnt sich die Praxi{\s} auch. Die Bauern sind
wohlhabend. Nach den statistischen Feststellungen haben wir,
abgesehen von den gew"ohnlichen Diarrh"oen, Rachenkatarrhen und
Magenbeschwerden, hin und wieder w"ahrend der Erntezeit wohl
F"alle von Wechselfieber, aber im gro"sen und ganzen selten
schwere Krankheiten. Besonder{\s} zu erw"ahnen sind die
zahlreichen skroful"osen Leiden, die zweifello{\s} von den
kl"aglichen hygienischen Verh"altnissen in den Bauernh"ausern
herr"uhren. Ja, ja, Herr Bovary, Sie werden "ofter{\s} mit
altmodischen Ansichten zu k"ampfen haben, und vielfach werden
Dickk"opfigkeit und alter Schlendrian alle Anstrengungen Ihrer
Kunst zunichte machen. Denn die Leute hierzulande versuchen e{\s}
in ihrer Dummheit immer noch erst mit Beten, mit Reliquien und mit
dem Pfarrer, statt da"s sie von vornherein zum Arzt oder in die
Apotheke gingen. Im "ubrigen ist da{\s} Klima wirklich nicht
schlecht. Wir haben sogar etliche Neunzigj"ahrige in der Gemeinde.
Nach meinen Beobachtungen ist die Maximalk"alte im Winter
4${}^{\circ}$ Celsiu{\s}, w"ahrend wir im Hochsommer auf
25${}^{\circ}$, h"ochsten{\s} 30${}^{\circ}$ kommen. Da{\s}
w"are ein Maximum von 24${}^{\circ}$ Reaumur. Da{\s} ist
nicht viel. Da{\s} kommt aber daher, da"s wir einerseit{\s} vor den
Nordwinden durch die W"alder von Argueil, andrerseit{\s} vor den
Westwinden durch die H"ohe von Sankt Johann gesch"utzt sind. Diese
W"arme, die ihre Ursachen auch in der Wasserverdunstung de{\s}
Flusse{\s} und in den zahlreich vorhandenen Viehherden in den
Weidegebieten hat, die, wie Sie wissen, viel Ammoniak produzieren
(also Stickstoff, Wasserstoff und Sauerstoff, ach nein, nur
Stickstoff und Sauerstoff!), -- diese W"arme, die den Humu{\s}
au{\s}saugt und alle D"unste de{\s} Boden{\s} aufnimmt, sich
gleichsam zu einer Wolke zusammenballt und sich mit der
Elektrizit"at der Atmosph"are verbindet, die k"onnte schlie"slich
(wie in den Tropenl"andern) gesundheit{\s}sch"adliche Mia{\s}men
erzeugen --, diese W"arme, sag ich, wird gerade dort, wo sie
herkommt, oder vielmehr, wo sie herkommen k"onnte, da{\s} hei"st
im S"uden, durch die S"udostwinde abgek"uhlt, die ihre K"uhle
"uber der Seine erlangen und bei un{\s} bi{\s}weilen pl"otzlich
al{\s} sanfte{\s} Mail"ufterl wehen~..."'

"`Gibt e{\s} denn wenigsten{\s} ein paar Spazierwege in der
Umgegend?"' fragte Frau Bovary im Laufe ihre{\s} Gespr"ache{\s}
mit dem jungen Manne.

"`Leider nur sehr wenige"', entgegnete er. "`Einen h"ubschen Ort
gibt e{\s} auf der H"ohe, am Waldrande, der
{\glq}Futterplatz{\grq} genannt. Dort sitze ich manchmal
Sonntag{\s} und vertiefe mich in ein Buch und seh mir den
Sonnenuntergang an."'

"`E{\s} gibt nicht{\s} Wunderbarere{\s} al{\s} den Sonnenuntergang,"'
schw"armte Emma, "`zumal am Gestade de{\s} Meere{\s}!"'

"`Ach, ich bete da{\s} Meer an!"' stimmte Leo bei.

"`Haben Sie nicht auch die Empfindung,"' fuhr Frau Bovary fort,
"`da"s die Seele beim Anblicke dieser unerme"slichen Weite Fl"ugel
bekommt, die Fl"ugel der Andacht, die in{\s} Reich der Ewigkeiten
emporheben, in die Sph"are der Ideen, der Ideale?"'

"`Im Hochgebirge ergeht e{\s} einem ebenso"', meinte Leo. "`Ich
habe einen Vetter, der im vergangnen Jahre eine Schweizerreise
gemacht hat. Der hat mir erz"ahlt: ohne sie selber zu sehen,
k"onne man sich den romantischen Reiz der Seen gar nicht
vorstellen, den Zauber der Wasserf"alle und den gro"sartigen
Eindruck der Gletscher. "Uber Gie"sb"achen h"angen riesige
Fichten, und am Rande von tiefen Abgr"unden kleben Alpenh"utten;
und wenn die Wolken einmal zerrei"sen, erblickt man tausend Fu"s
unten in der Tiefe die langen T"aler. Wer da{\s} schaut, mu"s in
Begeisterung geraten, in Andacht{\s}stimmung, in Ekstase! Jetzt
begreife ich auch jenen ber"uhmten Musiker, der nur angesicht{\s}
von erhabenen Landschaften arbeiten konnte."'

"`Treiben Sie Musik?"' fragte Emma.

"`Nein, aber ich liebe die Musik!"' antwortete er.

"`Glauben Sie ihm da{\s} nicht, Frau Doktor!"' mischte sich
Homai{\s} ein. "`Da{\s} sagt er nur au{\s} purer Bescheidenheit
... Aber gewi"s, mein Verehrter! Gestern, in Ihrem Zimmer, da
haben Sie doch da{\s} \so{Engellied} wundervoll gesungen. Ich hab
e{\s} von meinem Laboratorium au{\s} geh"ort. Sie haben eine
Stimme wie ein Operns"anger!"'

Leo D"upui{\s} bewohnte n"amlich im Hause de{\s} Apotheker{\s} im
zweiten Stock ein kleine{\s} Zimmer, da{\s} nach dem Markt
hinau{\s}ging. Bei dem Komplimente seine{\s} Hau{\s}wirte{\s}
wurde er "uber und "uber rot.

Homai{\s} widmete sich bereit{\s} wieder dem Arzte, dem er die
bemerken{\s}werten Einwohner von Yonville einzeln aufz"ahlte. Er
wu"ste tausend Anekdoten und Einzelheiten. Nur "uber da{\s}
Verm"ogen de{\s} Notar{\s} k"onne er nicht{\s} Genaue{\s} sagen.
Auch "uber die Familie T"uvache munkele man so allerlei.

Emma fuhr fort:

"`Da{\s} ist ja ent\/z"uckend! Und welche Musik lieben Sie am
meisten?"'

"`Die deutsche! Die ist da{\s} wahre Traumland~..."'

"`Kennen Sie die Italiener?"'

"`Noch nicht. Aber ich werde sie n"achste{\s} Jahr h"oren. Ich
habe die Absicht, nach Pari{\s} zu gehen, um mein juristische{\s}
Studium zu vollenden."'

"`Wie ich bereit{\s} die Ehre hatte, Ihrem Herrn Gemahl
mit\/zuteilen,"' sagte wiederum der Apotheker, "`al{\s} ich ihm von
dem armen Stryien{\s}ki berichtete, der auf und davon gegangen
ist: dank den Dummheiten, die der begangen hat, werden Sie sich
eine{\s} der komfortabelsten H"auser von Yonville erfreuen. Eine
ganz besondre Bequemlichkeit gerade f"ur einen Arzt ist da{\s}
Vorhandensein einer Hinterpforte nach dem Bach und der Allee zu.
Man kann dadurch unbeobachtet ein und au{\s} gehen. Die Wohnung
selbst besitzt alle denkbaren Annehmlichkeiten; sie hat ein
gro"se{\s} E"szimmer, eine K"uche mit Speisekammer, eine
Waschk"uche, einen Obstkeller usw. Ihr Vorg"anger war ein flotter
Kerl, dem e{\s} auf ein paar Groschen nicht ankam. Hinten in
seinem Garten, mit dem Blick auf unser Fl"u"schen, da hat er sich
ein Lusth"au{\s}chen bauen lassen, lediglich, um an Sommerabenden
sein Bier drin zu s"uffeln. Wenn die gn"adige Frau die Blumenzucht
liebt~..."'

"`Meine Frau gibt sich damit nicht weiter ab"', unterbrach ihn
Karl. "`Obgleich ihr k"orperliche Bewegung verordnet ist, bleibt
sie lieber dauernd in ihrem Zimmer und liest."'

"`Ganz wie ich!"' fiel Leo ein. "`Wa{\s} w"are wohl auch
gem"utlicher, al{\s} abend{\s} beim Schein der Lampe mit einem
Buche am Kamine zu sitzen, w"ahrend drau"sen der Wind gegen die
Fensterscheiben schl"agt?"'

"`So ist e{\s}!"' stimmte sie zu und blickte ihn mit ihren gro"sen
schwarzen Augen voll an.

Er fuhr fort:

"`Dann denkt man an nicht{\s}, und die Stunden verrinnen. Ohne
da"s man sich bewegt, wandert man mit dem Erz"ahler durch ferne
Lande. Man w"ahnt sie vor Augen zu haben. Man tr"aumt sich in die
fremden Erlebnisse hinein, bi{\s} in alle Einzelheiten; man
verstrickt sich in allerhand Abenteuer; man lebt und webt unter
den Gestalten der Dichtung, und e{\s} kommt einem zuletzt vor,
al{\s} schl"uge da{\s} eigne Herz in ihnen."'

"`Wie wahr! Wie wahr!"' rief Emma au{\s}.

"`Haben Sie e{\s} nicht zuweilen erlebt, in einem Buche einer
bestimmten Idee zu begegnen, die man verschwommen und unklar
l"angst in sich selbst tr"agt? Wie au{\s} der Ferne schwebt sie
nun mit einem Male auf einen zu, gewinnt feste Umrisse, und e{\s}
ist einem, al{\s} stehe man vor einer Offenbarung seine{\s}
tiefsten Ich{\s}~..."'

"`Da{\s} hab ich schon erlebt!"' fl"usterte sie.

"`Und darum"', fuhr er fort, "`liebe ich die Dichter "uber
alle{\s}. Ich finde, Verse sind zarter al{\s} Prosa. Sie r"uhren
so sch"on zu Tr"anen!"'

"`Aber sie erm"uden auf die Dauer,"' wandte Emma ein, "`und daher
ziehe ich jetzt mehr die Romane vor, aber sie m"ussen spannend und
aufregend sein. Widerlich sind mir Alltag{\s}leute und lauwarme
Gef"uhle. Die hat man doch schon genug in der Wirklichkeit."'

"`Gewi"s,"' bemerkte der Adjunkt, "`die naturalistischen Romane
haben dem Herzen nicht{\s} zu sagen und entfernen sich damit,
meiner Ansicht nach, von dem wahren Ziele der Kunst. E{\s} ist so
s"u"s, sich au{\s} den H"a"slichkeiten de{\s} Dasein{\s}
herau{\s}zuz"uchten, wenigsten{\s} in Gedanken: zu edlen
Charakteren, zu hehren Leidenschaften und zu gl"uckseligen
Zust"anden. F"ur mich, der ich hier fern der gro"sen Welt lebe,
ist da{\s} die einzige Erholung. Nur hat man in Yonville wenig
Gelegenheit~..."'

"`Jedenfall{\s} genau so wie in Toste{\s}!"' bemerkte Emma. "`Drum
war ich st"andig in einer Leihbibliothek abonniert."'

Der Apotheker hatte diese letzten Worte geh"ort. "`Wenn gn"adige
Frau mir die Ehre erweisen wollen,"' sagte er, "`meine Bibliothek
zu benutzen, so steht sie Ihnen zur Verf"ugung. Sie enth"alt die
besten Autoren: Voltaire, Rousseau, Delille, Walter Scott,
au"serdem ein paar Zeitschriften und Zeitungen, unter andern den
"`Leuchtturm von Rouen"', ein Tage{\s}blatt, dessen Korrespondent
f"ur Buchy, Forge{\s}, Neufch\^atel, Yonville und Umgegend ich
bin."'

Man sa"s bereit{\s} zwei und eine halbe Stunde bei Tisch, nicht
ohne Mitverschulden der bedienenden Artemisia, die in ihren
Holzschuhen saumselig "uber die Dielen schl"urfte, jeden Teller
einzeln hereinbrachte, allerlei verga"s, jeden Auftrag "uberh"orte
und immer wieder die T"ure zum Billardzimmer offen lie"s, die dann
krachend von selber zuklappte.

Ohne e{\s} zu bemerken, hatte Leo, w"ahrend er so eifrig
plauderte, einen Fu"s auf eine der Querleisten de{\s} Stuhle{\s}
gesetzt, auf dem Frau Bovary sa"s. Sie trug einen gefalteten
steifen Batistkragen und einen blauseidnen Schlip{\s}, und je nach
den Bewegungen, die sie mit ihrem Kopfe machte, ber"uhrte ihr Kinn
den Batist oder entfernte sich grazi"o{\s} davon. So kamen Leo und
Emma, w"ahrend sich Karl mit dem Apotheker unterhielt, in ein{\s}
jener uferlosen Gespr"ache, die um tausend oberfl"achliche Dinge
kreisen und keinen andern Sinn haben, al{\s} die gegenseitige
Sympathie einander zu bekunden. Pariser Theaterereignisse,
Romantitel, moderne T"anze, die ihnen fremde gro"se Gesellschaft,
Toste{\s}, wo Emma gelebt hatte, und Yonville, wo sie sich
gefunden, alle{\s} da{\s} ber"uhrten sie in ihrer Plauderei,
bi{\s} die Mahlzeit zu Ende war.

Al{\s} der Kaffee gebracht wurde, ging Felicie fort, um in der
neuen Wohnung da{\s} Schlafzimmer zurechtzumachen. Bald darauf
brach die kleine Tischgesellschaft auf. Frau Franz war l"angst am
erloschenen Herdfeuer eingenickt. Aber der Hau{\s}knecht war
wachgeblieben. Eine Laterne in der einen Hand, begleitete er Herrn
und Frau Bovary nach Hau{\s}. In seinem roten Haar hing H"acksel,
und auf einem Beine war er lahm. Den Schirm de{\s} Pfarrer{\s},
den er ihm noch hintragen sollte, in der andern Hand, ging er
voran.

Der Ort lag in tiefem Schlafe. Die S"aulen der Hallen auf dem
Markte warfen lange Schatten "uber da{\s} Pflaster. Der Boden war
hellgrau wie in einer Sommernacht. Da da{\s} Hau{\s} de{\s}
Arzte{\s} nur f"unfzig Schritte vom Goldnen L"owen entfernt lag,
w"unschte man sich al{\s}bald gegenseitig Gute Nacht, und so
schied man voneinander.

Al{\s} Emma die Hau{\s}flur ihre{\s} neuen Heim{\s} betrat, hatte
sie die Empfindung, al{\s} lege sich ihr die K"uhle der W"ande wie
feuchte Leinwand um die Schultern. Der Kalkbewurf war frisch. Die
Holztreppen knarrten. In ihrem Zimmer, im ersten Stock, fiel
fahle{\s} Licht durch die gardinenlosen Fenster. Sie sah drau"sen
Baumwipfel und weiterhin in der Niederung da{\s} Wiesenland, ein
Nebelmeer dar"uber. Da{\s} Mondlicht sickerte durch die
aufwallenden D"ampfe.

Im Zimmer standen Kommodenk"asten, Flaschen, Gardinenstangen,
M"o\-bel\-st"ucke und Geschirr kunterbunt umher. Die beiden
Packer hatten alle{\s} so stehen und liegen lassen.

Zum vierten Male schlief Emma an einem ihr noch fremden Orte.
Da{\s} erstemal war e{\s} am Tage ihre{\s} Eintritt{\s} in{\s}
Kloster gewesen, da{\s} zweitemal an dem ihrer Ankunft in
Toste{\s}, da{\s} drittemal im Schlo"s Vaubyessard und da{\s}
vierte hier in Yonville. Jede{\s}mal hatte ein neuer Abschnitt in
ihrem Leben begonnen. Darum glaubte sie, da"s sich die gleichen
Dinge an verschiedenen Orten nicht wiederholen k"onnten; und da
ihr bi{\s}herige{\s} St"uck Leben h"a"slich gewesen war, so m"usse
da{\s}, wa{\s} sie noch zu erleben hatte, zweifello{\s} sch"oner
sein.


\newpage\begin{center}
{\large \so{Dritte{\s} Kapitel}}\bigskip\bigskip
\end{center}

Am andern Morgen, al{\s} Emma kaum aufgestanden war, sah sie den
Adjunkt "uber den Markt gehen. Sie war im Morgenkleid. Er schaute
zu ihr herauf und gr"u"ste. Sie nickte hastig mit dem Kopfe und
schlo"s da{\s} Fenster.

Den ganzen Tag "uber konnte e{\s} Leo D"upui{\s} kaum erwarten,
da"s e{\s} sech{\s} schlug. Al{\s} er aber endlich in den Goldnen
L"owen kam, fand er niemanden vor al{\s} den Steuereinnehmer, der
bereit{\s} am Tische sa"s.

Da{\s} gestrige Mahl war f"ur Leo ein bedeutung{\s}volle{\s}
Ereigni{\s}. Bi{\s} dahin hatte er noch niemal{\s} zwei Stunden
lang mit einer "`Dame"' geplaudert. Wie hatte er e{\s} nur
fertiggebracht, ihr eine solche Menge von Dingen und in so guter
Form zu sagen? Da{\s} war ihm vordem unm"oglich gewesen. Er war
von Natur sch"uchtern und wahrte eine gewisse Zur"uckhaltung, die
sich au{\s} Schamhaftigkeit und Heuchelei zusammensetzt. Die
Yonviller fanden sein Benehmen tadello{\s}. Er h"orte still zu,
wenn "altere Herren di{\s}putierten, und zeigte sich in
politischen Dingen keine{\s}weg{\s} radikal, wa{\s} an einem
jungen Manne eine seltene Sache ist. Dazu besa"s er allerlei
Talent: er aquarellierte, er war musikalisch, er besch"aftigte
sich in seinen Mu"sestunden gern mit der Literatur, -- wenn er
nicht gerade Karten spielte. Der Apotheker sch"atzte ihn wegen
seiner Kenntnisse, und Frau Homai{\s} war ihm wohlgewogen, weil er
h"oflich und gef"allig war; "ofter{\s} widmete er sich n"amlich im
Garten ihren Kindern, kleinem Volk, da{\s} immer schmutzig
au{\s}sah und sehr schlecht erzogen war und dessen Beaufsichtigung
einmal dem Dienstm"adchen und dann noch besonder{\s} dem Lehrling
oblag, einem jungen Burschen, namen{\s} Justin. Er war ein
entfernter Verwandter de{\s} Apotheker{\s}, von diesem au{\s}
Mitleid in seinem Hau{\s} aufgenommen, wo er eine Art "`Mann f"ur
alle{\s}"' geworden war.

Homai{\s} spielte die Rolle de{\s} guten Nachbar{\s}. Er gab Frau
Bovary die besten Adressen f"ur ihre Eink"aufe, lie"s seinen
Apfelweinlieferanten eigen{\s} f"ur sie herkommen, beteiligte sich
an der Weinprobe und gab pers"onlich acht, da"s da{\s} bestellte
Fa"s einen geeigneten Platz im Keller erhielt. Er verriet ihr die
beste und billigste Butterquelle und bestellte ihr
Lestiboudoi{\s}, den Kirchendiener, al{\s} G"artner; neben seinen
"Amtern in Kirche und Gotte{\s}acker hielt dieser n"amlich die
G"arten der Honoratioren von Yonville instand; man engagierte ihn
"`stundenweise"' oder "`auf{\s} Jahr"', ganz wie e{\s} gew"unscht
wurde.

Diese Hilf{\s}bereitschaft de{\s} Apotheker{\s} entsprang weniger
einem Herzen{\s}bed"urfni{\s} al{\s} schlauer Berechnung.
Homai{\s} hatte n"amlich fr"uher einmal gegen da{\s} Gesetz vom
19. Vent\^ose de{\s} Jahre{\s} \begin{antiqua}XI\end{antiqua}~
versto"sen, wonach die "arztliche Praxi{\s} jedem verboten ist,
der sich nicht im Besitze eine{\s} staatlichen Diplom{\s}
befindet. Eine{\s} Tage{\s} war er auf eine geheimni{\s}volle
Anzeige hin nach Rouen vor den Staat{\s}anwalt geladen worden.
Dieser Vertreter der Justiz hatte ihn in seinem Amt{\s}zimmer,
stehend und in Amt{\s}robe, da{\s} Barett auf dem Kopfe,
vernommen. E{\s} war am Vormittag, unmittelbar vor einer
Gericht{\s}sitzung gewesen. Von drau"sen, vom Gange her, waren dem
Apotheker die schweren Tritte der Schutzleute in{\s} Ohr gehallt.
E{\s} war ihm, al{\s} h"orte er fern da{\s} Aufschnappen wuchtiger
Schl"osser. Er bekam Ohrensausen und glaubte, der Schlag w"urde
ihn r"uhren. Schon sah er sich im Kerker sitzen, seine Familie in
Tr"anen, die Apotheke unter dem Hammer und seine Arzneiflaschen in
alle vier Winde verstreut. Hinterher mu"ste er seine
Leben{\s}geister in einem Kaffeehause mit einem Kognak in
Selter{\s} wieder auf die Beine bringen.

Allm"ahlich verbla"ste die Erinnerung an diese Vermahnung, und
Homai{\s} hielt von neuem in seinem Hinterst"ubchen "arztliche
Sprechstunden ab. Da aber der B"urgermeister nicht sein Freund war
und seine Kollegen in der Umgegend brotneidisch waren, bebte er in
ewiger Angst vor einer neuen Anzeige. Indem er sich nun Bovary
durch kleine Gef"alligkeiten verpflichtete, wollte er sich damit
ein Recht auf dessen Dankbarkeit erwerben und ihn mundtot machen,
fall{\s} die Kurpfuschereien in der Apotheke abermal{\s} ruchbar
w"urden. Er brachte dem Arzt alle Morgen den "`Leuchtturm"', und
oft verlie"s er nachmittag{\s} auf ein Viertelst"undchen sein
Gesch"aft, um ein wenig mit ihm zu schwatzen.

Karl war mi"sgestimmt. E{\s} kamen keine Patienten. Ganze Stunden
lang sa"s er vor sich hinbr"utend da, ohne ein Wort zu sprechen.
Er machte in seinem Sprechzimmer ein Schl"afchen oder sah seiner
Frau beim N"ahen zu. Um sich ein wenig Besch"aftigung zu machen,
verrichtete er allerhand grobe Hau{\s}arbeit. Er versuchte sogar,
die Bodent"ure mit dem Rest von "Olfarbe anzupinseln, den die
Anstreicher dagelassen hatten.

Am meisten dr"uckte ihn seine Geldverlegenheit. Er hatte in
Toste{\s} eine betr"achtliche Summe au{\s}gegeben f"ur neue
Anschaffungen im Hause, f"ur die Kleider seiner Frau und
neuerding{\s} f"ur den Umzug. Die ganze Mitgift, mehr al{\s}
dreitausend Taler, war in zwei Jahren daraufgegangen. Bei der
"Ubersiedelung von Toste{\s} nach Yonville war viele{\s}
besch"adigt worden oder verloren gegangen, unter anderm der
t"onerne M"onch, der unterweg{\s} vom Wagen heruntergefallen und
in tausend St"ucke zerschellt war.

Eine zartere Sorge lenkte ihn ab: die Mutterhoffnungen seiner
Frau. Je n"aher diese ihrer Erf"ullung entgegengingen, um so
liebevoller behandelte er Emma. Diese sich kn"upfenden neuen Bande
von Fleisch und Blut machten da{\s} Gef"uhl der ewigen
Zusammengeh"origkeit in ihm immer inniger. Wenn er ihrem tr"agen
Gange zusah, wenn er da{\s} allm"ahliche Vollerwerden ihrer
miederlosen H"uften bemerkte, wenn sie m"ude ihm gegen"uber auf
dem Sofa sa"s, dann strahlten seine Blicke, und er konnte sich in
seinem Gl"ucke nicht fassen. Er sprang auf, k"u"ste sie,
streichelte ihr Gesicht, nannte sie "`Mammchen"', wollte mit ihr
im Zimmer herumtanzen und sagte ihr unter Lachen und Weinen
tausend z"artliche, drollige Dinge, die ihm gerade in den Sinn
kamen. Der Gedanke, Vater zu werden, war ihm etwa{\s}
K"ostliche{\s}. Jetzt fehlte ihm nicht{\s} mehr auf der Welt. Nun
hatte er alle{\s} erlebt, wa{\s} Menschen erleben k"onnen, und er
durfte zufrieden und vergn"ugt sein.

In der ersten Zeit war Emma "uber sich selbst arg verwundert. Dann
kam die Sehnsucht, von ihrem Zustande wieder befreit zu sein. Sie
wollte wissen, wie e{\s} sein w"urde, wenn da{\s} Kind da war.
Aber al{\s} sie kein Geld dazu hatte, eine Wiege mit rosa-seidnen
Vorh"angen und gestickte Kinderh"aubchen zu kaufen, da "uberkam
sie eine pl"otzliche Erbitterung; sie verlor die Lust, die
Baby-Au{\s}stattung selber sorglich au{\s}zuw"ahlen, und
"uberlie"s die Herstellung in Bausch und Bogen einer N"aherin. So
lernte sie die stillen Freuden dieser Vorbereitungen nicht kennen,
die andre M"utter so z"artlich stimmen, und vielleicht war die{\s}
der Grund, da"s ihre Mutterliebe von Anfang an gewisser Elemente
entbehrte. Weil aber Karl bei allen Mahlzeiten immer wieder von
dem Kinde sprach, begann auch Emma mehr daran zu denken.

Sie w"unschte sich einen Sohn. Braun sollte er sein, und stark
sollte er werden, und Georg m"u"ste er hei"sen! Der Gedanke, einem
m"annlichen Wesen da{\s} Leben zu schenken, kam ihr vor wie eine
Entsch"adigung f"ur alle{\s} da{\s}, wa{\s} sich in ihrem eigenen
Dasein nicht erf"ullt hatte. Ein Mann ist doch wenigsten{\s} sein
freier Herr. Ihm stehen alle Leidenschaften und alle Lande offen,
er darf gegen alle Hindernisse anrennen und sich auch die
allerfernsten Gl"uckseligkeiten erobern. Ein Weib liegt an tausend
Ketten. Tatenlo{\s} und doch genu"sfreudig, steht sie zwischen den
Verf"uhrungen ihrer Sinnlichkeit und dem Zwang der Konvenienz. Wie
den flatternden Schleier ihre{\s} Hute{\s} ein feste{\s} Band
h"alt, so gibt e{\s} f"ur die Frau immer ein Verlangen, mit dem
sie hinwegfliegen m"ochte, und immer irgendwelche herk"ommliche
Moral, die sie nicht lo{\s}l"a"st.

An einem Sonntag kam da{\s} Kind zur Welt, fr"uh gegen sech{\s}
Uhr, al{\s} die Sonne aufging.

"`E{\s} ist ein M"adchen!"' verk"undete Karl.

Emma fiel im Bett zur"uck und ward ohnm"achtig. Schon stellten
sich auch Frau Homai{\s} und die L"owenwirtin ein, um die
W"ochnerin zu umarmen. Der Apotheker rief ihr di{\s}kret ein paar
vorl"aufige Gl"uckw"unsche durch die T"urspalte zu. Er wollte die
neue Erdenb"urgerin besichtigen und fand sie wohlgeraten.

W"ahrend der Genesung gr"ubelte Emma nach, welchen Namen da{\s}
Kind bekommen sollte. Zun"achst dachte sie an einen italienisch
klingenden Namen: an Amanda, Rosa, Joconda, Beatrice. Sehr
gefielen ihr Ginevra oder Leocadia, noch mehr Isolde. Karl
"au"serte den Wunsch, die Kleine solle nach der Mutter getauft
werden, aber davon wollte Emma nicht{\s} wissen. Man nahm alle
Kalendernamen durch und bat jeden Besucher um einen Vorschlag.

"`Herr Leo,"' berichtete der Apotheker, "`mit dem ich neulich
dar"uber gesprochen habe, wundert sich dar"uber, da"s Sie nicht
den Namen Magdalena w"ahlen. Der sei jetzt sehr in Mode."' Aber
gegen die Patenschaft einer solchen S"underin str"aubte sich die
alte Frau Bovary gewaltig. Homai{\s} f"ur seine Person hegte eine
Vorliebe f"ur Namen, die an gro"se M"anner, ber"uhmte Taten und
hohe Werke erinnerten. Nach dieser Theorie habe er seine vier
eigenen Spr"o"slinge getauft: Napoleon (der Ruhm!), Franklin (die
Freiheit!), Irma (ein Zugest"andni{\s} an die Romantik!) und
Athalia (zu Ehren de{\s} Meisterst"uck{\s} de{\s} franz"osischen
Drama{\s}!). Seine philosophische "Uberzeugung, sagte er, stehe
seiner Bewunderung der Kunst nicht im Wege. Der Denker in ihm
ersticke durchau{\s} nicht den Gef"uhl{\s}menschen. Er verst"unde
sich darauf, da{\s} eine vom andern zu scheiden und sich vor
fanatischer Einseitigkeit zu bewahren.

Zu guter Letzt fiel Emma ein, da"s sie im Schlo"s Vaubyessard
geh"ort hatte, wie eine junge Dame von der Marquise mit
"`Berta-Luise"' angeredet worden war. Von diesem Augenblick an
stand die Namen{\s}wahl fest. Da Vater Rouault zu kommen
verhindert war, wurde Homai{\s} gebeten, Gevatter zu stehen. Er
stiftete al{\s} Patengeschenk allerlei Gegenst"ande au{\s} seinem
Gesch"aft, al{\s} wie: sech{\s} Schachteln Brusttee, eine Dose
Kraftmehl, drei B"uchsen Marmelade und sech{\s} P"ackchen
Malzbonbon{\s}.

Am Taufabend gab e{\s} ein Festessen, zu dem auch der Pfarrer
erschien. Man geriet in Stimmung. Beim Lik"or gab der Apotheker
ein patriotische{\s} Lied zum besten, worauf Leo D"upui{\s} eine
Barkarole vortrug und die alte Frau Bovary (Patin de{\s}
Kinde{\s}) eine Romanze au{\s} der Napoleonischen Zeit sang. Der
alte Herr Bovary bestand darauf, da"s da{\s} Kind heruntergebracht
wurde, und taufte die Kleine "`Berta"', indem er ihr ein Gla{\s}
Sekt von oben "uber den Kopf go"s. Den Abb\'e Bournisien "argerte
diese Profanation einer kirchlichen Handlung, und al{\s} der alte
Bovary ihm gar noch ein sp"ottische{\s} Zitat vorhielt, wollte der
Geistliche fortgehen. Aber die Damen baten ihn inst"andig zu
bleiben, und auch der Apotheker legte sich in{\s} Mittel. So
gelang e{\s}, den Priester wieder zu beruhigen. Friedlich langte
er von neuem nach seiner halbgeleerten Kaffeetasse.

Bovary senior blieb noch volle vier Wochen in Yonville und
verbl"uffte die Yonviller durch da{\s} pr"achtige
Stab{\s}arzt{\s}k"appi mit Silbertressen, da{\s} er vormittag{\s}
trug, wenn er seine Pfeife auf dem Marktplatze schmauchte. Al{\s}
gewohnheit{\s}m"a"siger starker Schnapstrinker schickte er da{\s}
Dienstm"adchen h"aufig in den Goldnen L"owen, um seine Feldflasche
f"ullen zu lassen, wa{\s} selbstverst"andlich auf Rechnung
seine{\s} Sohne{\s} erfolgte. Um seine Hal{\s}t"ucher zu
parf"umieren, verbrauchte er den gesamten Vorrat an K"olnischem
Wasser, den seine Schwiegertochter besa"s.

Ihr selbst war seine Anwesenheit keine{\s}weg{\s} unangenehm. Er
war in der Welt herumgekommen. Er erz"ahlte von Berlin, Wien,
Stra"sburg, von seiner Soldatenzeit, seinen Liebschaften, den
Festlichkeiten, die er dereinst mitgemacht hatte. Dann war er
wieder ganz der alte Schweren"oter, und zuweilen, im Garten oder
auf der Treppe, fa"ste er Emma um die Taille und rief au{\s}:
"`Karl, nimm dich in acht!"'

Die alte Frau Bovary sah dergleichen voller Angst um da{\s}
Ehegl"uck ihre{\s} Sohne{\s}. Sie f"urchtete, ihr Mann k"onne am
Ende einen unsittlichen Einflu"s auf die Gedankenwelt der jungen
Frau au{\s}"uben, und so betrieb sie die Abreise. Vielleicht war
ihre Besorgni{\s} noch schlimmer. Dem alten Herrn war alle{\s}
zuzutrauen.

Emma hatte da{\s} Kind zu der Frau eine{\s} Tischler{\s} namen{\s}
Rollet in die Pflege gegeben. Eine{\s} Tage{\s} empfand sie
pl"otzlich Sehnsucht, da{\s} kleine M"adchen zu sehen.
Unverz"uglich machte sie sich auf den Weg zu diesen Leuten, deren
H"au{\s}chen ganz am Ende de{\s} Orte{\s}, zwischen der
Landstra"se und den Wiesen, in der Tiefe lag.

E{\s} war Mittag. Die Fensterl"aden der H"auser waren alle
geschlossen. Die sengende Sonne br"utete "uber den Schieferd"achern,
deren Giebellinien richtige Funken spr"uhten. Ein schw"uler Wind
wehte. Emma fiel da{\s} Gehen schwer. Da{\s} spitzige Pflaster tat
ihren F"u"sen weh. Sie ward sich unschl"ussig, ob sie umkehren
oder irgendwo eintreten und sich au{\s}ruhen sollte.

In diesem Augenblick trat Leo au{\s} dem n"achsten Hause
herau{\s}, eine Aktenmappe unter dem Arme. Er kam auf sie zu,
begr"u"ste sie und stellte sich mit ihr in den Schatten der
Leinwandmarkise vor dem Lheureuxschen Modewarenladen.

Frau Bovary erz"ahlte ihm, da"s sie nach ihrem Kinde sehen wollte,
aber m"ude zu werden beginne.

"`Wenn~..."', fing Leo an, wagte aber nicht weiterzusprechen.

"`Haben Sie etwa{\s} vor?"' fragte Emma. Auf die Verneinung de{\s}
Adjunkten hin bat sie ihn, sie zu begleiten. (Bereit{\s} am Abend
de{\s}selben Tage{\s} war die{\s} stadtbekannt, und Frau T"uvache,
die B"urgermeister{\s}gattin, erkl"arte in Gegenwart ihre{\s}
Dienstm"adchen{\s}, Frau Bovary habe sich kompromittiert.)

Um zu der Amme zu gelangen, mu"sten die beiden am Ende der
Hauptstra"se link{\s} abgehen und einen kleinen Fu"sweg
einschlagen, der zwischen einzelnen kleinen H"ausern und Geh"often
in der Richtung auf den Gemeindefriedhof hinlief. Die Weiden, die
den Pfad ums"aumten, bl"uhten, und e{\s} bl"uhten die Veroniken,
die wilden Rosen, die Glockenblumen und die Brombeerstr"aucher.
Durch L"ucken in den Hecken erblickte man hie und da auf den
Misthaufen der kleinen Geh"ofte ein Schwein oder eine angebundne
Kuh, die ihre H"orner an den St"ammen der B"aume wetzte.

Seite an Seite wandelten sie gem"achlich weiter. Emma st"utzte
sich auf Leo{\s} Arm, und er verk"urzte seine Schritte nach den
ihren. Vor ihnen her tanzte ein M"uckenschwarm und erf"ullte die
warme Luft mit ganz leisem Summen.

Emma erkannte da{\s} Hau{\s} an einem alten Nu"sbaum wieder, der
e{\s} umschattete. E{\s} war niedrig und hatte braune Ziegel auf
dem Dache. Au{\s} der Luke de{\s} Oberboden{\s} hing ein Kranz von
Zwiebeln. Eine Dornenhecke umfriedigte ein viereckige{\s}
G"artlein mit Salat, Lavendel und bl"uhenden Schoten, die an
Stangen gezogen waren. An der Hecke waren Reisigbunde
aufgeschichtet. Ein tr"ube{\s} W"asserchen rann sich verzettelnd
durch da{\s} Gra{\s}; allerhand kaum noch verwendbare Lumpen, ein
gestrickter Strumpf und eine rote baumwollene Jacke lagen auf dem
Rasen umher, und "uber der Hecke flatterte ein gro"se{\s} St"uck
Leinwand.

Beim Knarren der Gartent"ure erschien die Tischler{\s}frau, ein
Kind an der Brust, ein andre{\s} an der Hand, ein armselige{\s},
schw"achlich au{\s}sehende{\s}, skroful"ose{\s} J"ungelchen. E{\s}
war da{\s} Kind eine{\s} M"utzenmacher{\s} in Rouen, da{\s} die
von ihrem Gesch"aft zu sehr in Anspruch genommenen Eltern auf
da{\s} Land gegeben hatten.

"`Kommen Sie nur herein!"' sagte die Frau. "`Ihre Kleine schl"aft
drinnen."'

In der einzigen Stube im Erdgescho"s stand an der hinteren Wand
ein gro"se{\s} Bett ohne Vorh"ange. Die Seite am Fenster, in dem
eine der Scheiben mit blauem Papier verklebt war, nahm ein
Backtrog ein. In der Ecke hinter der T"ure standen unter der Gosse
Stiefel mit blanken N"ageln, daneben eine Flasche "Ol, au{\s}
deren Hal{\s} eine Feder herau{\s}ragte. Auf dem verstaubten
Kaminsim{\s} lagen ein Wetterkalender, Feuersteine, Kerzenst"umpfe
und ein paar Fetzen Z"undschwamm. Ein weitere{\s} Schmuckst"uck
diese{\s} Gemach{\s} war eine "`trompetende Fama"', offenbar
da{\s} Reklameplakat einer Parf"umfabrik, da{\s} mit sech{\s}
Schuhzwecken an die Wand genagelt war.

Emma{\s} T"ochterchen schlief in einer Wiege au{\s}
Weidengeflecht. Sie nahm e{\s} mit der Decke, in die e{\s}
gewickelt war, empor und begann e{\s} im Arme hin und her zu
wiegen, wobei sie leise sang.

Leo ging im Zimmer auf und ab. Die sch"one Frau in ihrem hellen
Sommerkleide in dieser elenden Umgebung zu sehen, kam ihm seltsam
vor. Sie ward pl"otzlich rot. Er wandte sich weg, weil er dachte,
sein Blick sei vielleicht zudringlich gewesen. Sie legte da{\s}
Kind wieder in die Wiege. E{\s} hatte sich erbrochen, und die
Mutter am Hal{\s}kragen beschmutzt. Die Amme eilte herbei, um die
Flecke abzuwischen. Sie beteuerte, man s"ahe nicht{\s} mehr davon.

"`Mir kommt sie noch ganz ander{\s}!"' meinte die Frau. "`Ich habe
weiter nicht{\s} zu tun, al{\s} sie immer wieder zu s"aubern. Wenn
Sie doch so gut sein wollten und den Kaufmann Calmu{\s}
beauftragten, da"s ich mir bei ihm ein bi"schen Seife holen kann,
wenn ich welche brauche. Da{\s} w"are auch f"ur Sie da{\s}
bequemste. Ich brauche Sie dann nicht immer zu st"oren."'

"`Meinetwegen!"' sagte Emma. "`Auf Wiedersehn, Frau Rollet!"'

Beim Hinau{\s}gehen sch"uttelte sie sich.

Die Frau begleitete die beiden bi{\s} zum Ende de{\s} Hofe{\s},
wobei sie in einem fort davon sprach, wie beschwerlich e{\s} sei,
nacht{\s} so h"aufig aufstehen zu m"ussen. "`Manchmal bin ich
fr"uh so zerschlagen, da"s ich im Sitzen einschlafe. Drum sollten
Sie mir ein Pf"undchen gemahlenen Kaffee zukommen lassen. Wenn ich
ihn fr"uh mit Milch trinke, reiche ich damit vier Wochen."'

Nachdem Frau Bovary die Danke{\s}beteuerungen der Frau "uber sich
hatte ergehen lassen, verabschiedete sie sich. Aber kaum war sie
mit ihrem Begleiter ein St"uck auf dem Fu"swege gegangen, al{\s}
sie da{\s} Klappern von Holzpantoffeln hinter sich vernahm. Sie
drehte sich um. E{\s} war die Amme.

"`Wa{\s} wollen Sie noch?"'

Die Frau zog Emma bi{\s} hinter eine Ulme beiseite und fing an,
von ihrem Manne zu erz"ahlen. "`Bei seinem Handwerke und seinen
sech{\s} Franken Pension im Jahre~..."'

"`Machen Sie rasch!"' unterbrach Emma ihren Wortschwall.

"`Ach, liebste Frau Doktor,"' fuhr die Frau fort, indem sie
zwischen jede{\s} ihrer Worte einen Seufzer schob, "`ich habe
Angst, er wird b"ose, wenn er sieht, da"s ich allein f"ur mich
Kaffee trinke. Sie wissen, wie die M"anner sind~..."'

"`Sie sollen ja welchen haben, ich will Ihnen ja welchen schicken!
Sie langweilen mich."'

"`Ach, meine liebe, gute Frau Doktor, '{\s} ist ja blo"s f"ur die
schrecklichen Brustschmerzen, die er immer von wegen der alten
Wunde kriegt. Der Apfelwein bekommt ihm gar nicht gut~..."'

"`Na, wa{\s} wollen Sie denn noch?"' fragte Emma.

"`Wenn e{\s} also,"' fuhr die Frau fort, indem sie einen Knick{\s}
machte, "`wenn e{\s} also nicht zuviel verlangt ist~..."' Sie
machte abermal{\s} einen tiefen Knick{\s}. "`Wenn Sie so gut sein
wollen~..."'

Ihre Augen bettelten gott{\s}j"ammerlich. Endlich bekam sie e{\s}
herau{\s}:

"`Ein Bullchen Branntwein! Ich k"onnte damit auch die F"u"se Ihrer
Kleinen ein bi"schen einreiben. Sie sind so riesig zart~..."'

Nachdem sich Emma endlich von der Frau lo{\s}gemacht hatte, nahm
sie Leo{\s} Arm. Eine Zeitlang schritten sie flott vorw"art{\s}.
Dann wurde sie langsamer, und Emma{\s} Blick, der bi{\s}her
geradeau{\s} gegangen war, glitt "uber die Schulter ihre{\s}
Begleiter{\s}. Er hatte einen schwarzen Samtkragen auf seinem
Rocke, auf den sein kastanienbraune{\s} wohlgepflegte{\s} Haar
schlicht herabwallte. Die N"agel an seiner Hand fielen ihr auf;
sie waren l"anger, al{\s} man sie in Yonville sonst trug. Ihre
Pflege war eine der Hauptbesch"aftigungen de{\s} Adjunkten; er
besa"s dazu besondre Instrumente, die er in seinem Schreibtische
aufbewahrte.

Am Ufer de{\s} Bache{\s} gingen sie nach dem St"adtchen zur"uck.
Jetzt in der hei"sen Jahre{\s}zeit war der Wasserstand so niedrig,
da"s man dr"uben die Gartenmauern bi{\s} auf ihre Grundlage sehen
konnte. Von den Gartenpforten f"uhrten kleine Treppen in da{\s}
Wasser. E{\s} flo"s lautlo{\s} und rasch dahin, K"uhle
verbreitend. Hohe, d"unne Gr"aser neigten sich zur klaren Flut und
lie"sen sich von der Str"omung treiben; da{\s} sah au{\s} wie
au{\s}gel"oste{\s}, lange{\s}, gr"une{\s} Haar. Hin und wieder
liefen oder schliefen Insekten auf den Spitzen der Binsen und auf
den Bl"attern der Wasserrosen. In den kleinen blauen Wellen, im
Zerflie"sen schon wieder neugeboren, glitzerte die Sonne. Die
verschnittenen alten Weiden spiegelten ihre grauen St"amme auf dem
Wasser. Und h"uben die weiten Wiesen lagen so verlassen~...

E{\s} war die Stunde, da man in den Gut{\s}h"ofen zu Mittag i"st.
Die junge Frau und ihr Begleiter vernahmen jetzt nicht{\s} al{\s}
den Klang ihrer eignen Tritte auf dem harten Pfade und die Worte,
die sie redeten, und da{\s} leise Rascheln von Emma{\s} Kleid.

Die oben mit Gla{\s}scherben bespickten Gartenmauern, an denen sie
nach "Uberschreitung eine{\s} Steg{\s} hingingen, gl"uhten wie die
Scheiben eine{\s} Treibhause{\s}. Zwischen den Steinen sprossen
Mauerblumen. Im Vor"ubergehen stie"s Frau Bovary mit dem Rande
ihre{\s} Sonnenschirme{\s} an die welken Bl"uten; gelber Staub
rieselte herab. Ab und zu streifte eine "uberh"angende
Jel"anger-jelieber- oder Klemati{\s}-Ranke die Seide ihre{\s}
Schirme{\s} und blieb einen Augenblick in den Spitzen h"angen.

Sie plauderten von einer Truppe spanischer T"anzer, die demn"achst
im Rouener Theater gastieren sollte.

"`Werden Sie hinfahren?"' fragte Emma.

"`Wenn ich kann, ja!"'

Hatten sie sich wirklich nicht{\s} andre{\s} zu sagen? Ihre Augen
sprachen eine viel ernstere Sprache, und w"ahrend sie sich mit so
banalen Reden{\s}arten abqu"alten, f"uhlten sie sich alle beide im
Banne der n"amlichen schw"ulen Sehnsucht. Ein leiser, seelentiefer
Unterton dominierte heimlich ohne Unterla"s in ihrem
oberfl"achlichen Gespr"ach. Betroffen von diesem ungewohnten
s"u"sen Zauber, dachten sie aber gar nicht daran, einander ihre
Empfindungen zu offenbaren oder ihnen auf den Grund zu gehen.
K"unftige{\s} Gl"uck ist wie ein tropische{\s} Gestade: e{\s}
sendet weit "uber den Ozean, der noch dazwischen liegt, seinen
lauen Erdgeruch her"uber, balsamischen Duft, von dem man sich
berauschen l"a"st, ohne den Horizont nach dem Woher zu fragen.

An einer Stelle de{\s} Wege{\s} stand Regenwasser in den
Wagengeleisen und Hufspuren; man mu"ste ein paar gro"se
moo{\s}bewachsene Steine, die Inseln in diesem Morast bildeten,
begehen. Auf jedem blieb Emma eine Weile stehen, um zu ersp"ahen,
wohin sie den n"achsten Schritt zu machen hatte. Wenn der Stein
wackelte, zog sie die Ellbogen hoch und beugte sich vorn"uber.
Aber bei aller Hilflosigkeit und Angst, in den T"umpel zu treten,
lachte sie doch.

Vor ihrem Garten angelangt, stie"s Frau Bovary die kleine Pforte
auf, stieg die Stufen hinauf und verschwand. Leo begab sich in
seine Kanzlei. Der Notar war abwesend. Der Adjunkt bl"atterte in
einem Aktenhefte, schnitt sich eine Feder zurecht, schlie"slich
ergriff er aber seinen Hut und ging wieder. Er stieg die H"ohe von
Argueil ein St"uck hinauf, nach dem "`Futterplatz"' am Waldrande.
Dort legte er sich unter eine Tanne und starrte in da{\s}
Himmel{\s}blau, die H"ande locker "uber den Augen.

"`Ach, ist da{\s} langweilig! Ist da{\s} langweilig!"' seufzte er.

Er fand da{\s} Dasein in diesem Neste jammervoll, mit Homai{\s}
al{\s} Freund und Guillaumin al{\s} Chef. Dem letzteren, diesem
gr"a"slichen Kanzleimenschen mit seiner goldnen Brille, seinem
roten Backenbart, seiner ewigen wei"sen Krawatte, dem mangelte
auch der geringste Sinn f"ur h"ohere Dinge. E{\s} war nur in der
ersten Zeit gewesen, da"s er dem Adjunkten mit seinen formellen
Diplomatenmanieren imponiert hatte. Wen gab e{\s} weiter in
Yonville? Die Frau de{\s} Apotheker{\s}. Die war weit und breit
die beste Gattin, sanft wie ein Lamm, brav und treu zu Kindern,
Vater, Mutter, Vettern und Basen. Keinen Menschen konnte sie
leiden sehen, und in der Wirtschaft lie"s sie alle{\s} drunter und
dr"uber gehn. Sie war eine Feindin de{\s} Korsett{\s}, sah sehr
gew"ohnlich au{\s} und war in ihrer Unterhaltung h"ochst
beschr"ankt. Alle{\s} in allem war sie eine ebenso harmlose wie
langweilige Dame. Obgleich sie drei"sig Jahre alt war und er
zwanzig, obwohl er T"ur an T"ur mit ihr schlief und obgleich er
t"aglich mit ihr sprach, war e{\s} ihm doch noch nie in den Sinn
gekommen, da"s sie irgendjemande{\s} Frau sein k"onne und mit
ihren Geschlecht{\s}genossinnen mehr gemeinsam habe al{\s} die
R"ocke.

Und wen gab e{\s} au"serdem noch? Den Steuereinnehmer Binet, ein
paar Kaufleute, zwei oder drei Kneipwirte, den Pfaffen, dann den
B"urgermeister T"uvache und seine beiden S"ohne, gro"sprotzige,
m"urrische, stumpfsinnige Kerle, die ihre "Acker selber pfl"ugten,
unter sich Gelage veranstalteten, scheinheilige Duckm"auser, mit
denen zu verkehren glatt unm"oglich war.

Von dieser Masse allt"aglicher Leute hob sich Emma{\s} Gestalt ab,
einsam und doch unerreichbar. Ihm wenigsten{\s} war e{\s}, al{\s}
l"agen tiefe Abgr"unde zwischen ihr und ihm. In der ersten Zeit
hatte er Bovary{\s} hin und wieder zusammen mit Homai{\s} besucht,
aber er hatte die Empfindung, al{\s} sei der Arzt durchau{\s}
nicht davon erbaut, ihn bei sich zu sehen, und so schwebte Leo
immer zwischen der Furcht, f"ur aufdringlich gehalten zu werden,
und dem Verlangen nach einem vertraulichen Umgang, der ihm so gut
wie unm"oglich schien.


\newpage\begin{center}
{\large \so{Vierte{\s} Kapitel}}\bigskip\bigskip
\end{center}

Sobald e{\s} herbstlich zu werden begann, siedelte Emma au{\s}
ihrem Zimmer in die Gro"se Stube "uber, einem l"anglichen
niedrigen Raume im Erdgeschosse. Gew"ohnlich sa"s sie am Fenster
in ihrem Lehnstuhle und betrachtete die Leute, die drau"sen
vor"ubergingen.

Leo kam t"aglich zweimal vorbei, auf seinem Wege nach dem Goldnen
L"owen und zur"uck. Seine Tritte erkannte Emma schon von weitem.
Sie neigte sich jede{\s}mal vor und lauschte, und der junge Mann
glitt an der Scheibengardine vor"uber, immer tadello{\s} gekleidet
und ohne den Kopf zu wenden. Oft aber in der D"ammerung, wenn sie,
auf dem Scho"se die begonnene Stickerei, vertr"aumt dasa"s,
"uberlief sie ein Schauer beim pl"otzlichen Vor"ubergleiten
seine{\s} Schatten{\s}. Dann fuhr sie auf und befahl da{\s} Essen.

Der Apotheker kam mitunter w"ahrend der Tischzeit. Sein K"appchen
in der Hand, trat er ger"auschlo{\s} ein, um ja niemanden zu
st"oren, jede{\s}mal mit derselben Reden{\s}art: "`Guten Abend,
die Herrschaften!"' Er setzte sich an den Tisch zwischen da{\s}
Ehepaar und fragte den Arzt, ob er neue Patienten habe, worauf
sich Bovary seinerseit{\s} erkundigte, ob diese auch
zahlung{\s}f"ahig seien. Sodann unterhielten sich die beiden "uber
da{\s}, wa{\s} in der Zeitung gestanden hatte. Um diese Stunde
wu"ste Homai{\s} sie bereit{\s} au{\s}wendig. Er rekapitulierte
sie von Anfang bi{\s} zu Ende: den Leitartikel genau so wie alle
darin berichteten merkw"urdigen Vorg"ange de{\s} In- und
Au{\s}land{\s}. Wenn auch dieser Gespr"ach{\s}stoff ersch"opft
war, konnte er ein paar Bemerkungen "uber die Gerichte auf dem
Tische nicht unterdr"ucken. Manchmal erhob er sich sogar ein wenig
und machte Frau Bovary artig auf da{\s} zarteste St"uck Fleisch
aufmerksam, oder er wandte sich an da{\s} Dienstm"adchen und gab
ihr Ratschl"age "uber die Zubereitung eine{\s} Ragout{\s} oder
"uber die richtige Verwendung der Gew"urze. Er verstand mit
erstaunlicher Fachkenntni{\s} "uber aromatische Zutaten,
Fleischertrakte, Saucen und S"afte zu sprechen. Er hatte in seinem
Kopfe mehr Rezepte al{\s} Arzneiflaschen in seiner Apotheke. In
der Herstellung von Konfit"uren, Weinessig und s"u"sen Lik"oren
war er ein Meister. Ferner kannte er auch alle neuen Erfindungen
auf dem Gebiete der K"uchen"okonomie, nicht minder da{\s} beste
Verfahren, K"ase zu konservieren und verdorbne Weine wieder
verwendbar zu machen.

Um acht Uhr erschien Justin, der Lehrling, um seinen Herrn zum
Schlie"sen de{\s} Laden{\s} zu holen. Homai{\s} pflegte ihm einen
pfiffigen Blick zuzuwerfen, zumal wenn Felicie zuf"allig im Zimmer
war. Er kannte n"amlich die Vorliebe seine{\s} Famulusse{\s} f"ur
da{\s} Hau{\s} de{\s} Arzte{\s}.

"`Der Schlingel setzt sich Allotria in den Kopf!"' meinte er.
"`Der Teufel soll mich holen: ich glaub, er hat sich in Ihr
Dienstm"adel verguckt!"'

"Ubrigen{\s} machte er ihm noch einen schwereren Vorwurf: er
horche auf alle{\s}, wa{\s} in seinem Hause gesprochen w"urde.
Beispiel{\s}weise sei er an den Sonntagen nicht au{\s} dem Salon
hinau{\s}zubringen, wenn er die schon halb eingeschlafenen Kinder
hole, um sie in{\s} Bett zu schaffen.

An diesen Sonntag{\s}abenden erschienen "ubrigen{\s} nur wenige
G"aste. Homai{\s} hatte sich nach und nach mit verschiedenen
Hauptpers"onlichkeiten de{\s} Orte{\s} wegen seiner Klatschsucht
und seiner politischen Ansichten "uberworfen. Aber der Adjunkt
stellte sich regelm"a"sig ein. Sobald er die Hau{\s}t"urklingel
h"orte, eilte er Frau Bovary entgegen, nahm ihr da{\s}
Umschlagetuch ab und die "Uberschuhe, die sie bei Schnee trug.

Zun"achst machte man ein paar Partien Dreiblatt, sodann spielten
Emma und der Apotheker Ecart\'e. Leo stand hinter ihr und half
ihr. Die H"ande auf die R"uckenlehne ihre{\s} Stuhle{\s}
gest"utzt, betrachtete er sich die Zinken de{\s} Kamme{\s}, der
ihr Haar zusammenhielt. Bei jeder ihrer Bewegungen w"ahrend de{\s}
Kartenspiel{\s} raschelte ihr Kleid. Im Nacken, unterhalb de{\s}
heraufgesteckten Haare{\s}, hatte ihre Haut einen br"aunlichen
Farbenton, der sich nach dem R"ucken zu aufhellte und im Schatten
de{\s} Kragen{\s} verschwamm. Ihr Rock bauschte sich zu beiden
Seiten de{\s} Stuhlsitze{\s} auf; er schlug eine Menge Falten und
bedeckte ein St"uck de{\s} Boden{\s}. Wenn Leo hin und wieder
au{\s} Versehen mit der Sohle seine{\s} Schuhe{\s} darauf geriet,
zog er den Fu"s rasch zur"uck, al{\s} habe er einen Menschen
getreten.

Wenn die Partie zu Ende war, begannen Homai{\s} und Karl Domino zu
spielen. Emma setzte sich dann an da{\s} andre Ende de{\s}
Tische{\s} und sah sich, die Ellbogen aufgest"utzt, die
"`Illustrierte Zeitung"' an. Oft hatte sie auch ihren "`Bazar"'
mitgebracht. Leo nahm neben ihr Platz. Sie betrachteten zusammen
die Holzschnitte und warteten mit dem Umbl"attern aufeinander.
Manchmal bat sie ihn, Gedichte vorzulesen. Leo trug mit langsamer
Stimme vor, die bei verliebten Stellen fl"usternd wurde. Da{\s}
Klappern der Dominosteine st"orte ihn. Der Apotheker war ein
gerissener Spieler und hatte dabei auch noch unversch"amte{\s}
Gl"uck. Wenn die dreihundert Point{\s} erreicht waren, setzten
sich die Spieler an den Kamin, und e{\s} dauerte nicht lange, da
waren sie alle beide eingenickt. Da{\s} Feuer im Kamin war im
Erl"oschen, die Teekanne leer. Leo la{\s} weiter, und Emma h"orte
ihm zu, wobei sie halb unbewu"st in einem fort den Lampenschirm
herumdrehte, auf dessen d"unnen Kattun Pierrot{\s} in einer
Kutsche und Seilt"anzerinnen mit Balancierstangen aufgedruckt
waren. Mit einem Male hielt der Leser inne und wie{\s} durch eine
Geste auf die eingeschlafene Zuh"orerschaft, und nun sprachen sie
lispelnd miteinander. Diese leise Plauderei d"unkte beide um so
s"u"ser, al{\s} niemand ihrer lauschte.

So bestand zwischen ihnen eine gewisse Gemeinschaft und ein
fortw"ahrender Au{\s}tausch von Romanen und Gedichtb"uchern. Karl,
der keine Neigung zur Eifersucht besa"s, hatte nicht{\s} dagegen.
Zu seinem Geburt{\s}tage bekam er einen phrenologischen Sch"adel,
der "uber und "uber mit blauen Linien und Zeichen bedeckt war,
eine Aufmerksamkeit Leo{\s}. Andre folgten. Er fuhr sogar mitunter
nach Rouen, um dort Besorgungen f"ur da{\s} Ehepaar zu machen.
Al{\s} infolge eine{\s} Moderoman{\s} die Kakteen in Beliebtheit
kamen, brachte er ein Exemplar, da{\s} er w"ahrend der Fahrt in
der Post vor sich auf den Knien hielt. Da{\s} stachlige Ding
zerstach ihm alle Finger.

Emma lie"s vor ihrem Fenster ein kleine{\s} Blumenbrett f"ur ihre
Blu\-men\-t"opfe anbringen, ganz so, wie der Adjunkt ein{\s}
hatte. Beim Begie"sen ihrer Blumen sahen sich die beiden.

Eine{\s} Abend{\s}, al{\s} Leo nach Hau{\s} kam, fand er in seinem
Zimmer eine Reisedecke au{\s} mattfarbenem Samt, auf dem mir Seide
und Wolle Blumen und Bl"atter gestickt waren. Er zeigte sie Frau
Homai{\s}, dem Apotheker, dem Lehrling, den Kindern und der
K"ochin; sogar seinem Chef erz"ahlte er davon. Alle Welt wollte
nun die Decke sehen. Aber warum machte die Frau de{\s} Doktor{\s}
dem Adjunkten so kostbare Geschenke? Da{\s} war doch sonderbar.
Und alsobald stand e{\s} unumst"o"slich fest: sie war "`seine gute
Freundin."'

Leo verst"arkte unvorsichtigerweise diesen Klatsch, weil er
unaufh"orlich und vor jedermann von Emma{\s} Sch"onheit und
Klugheit schw"armte. Binet wurde ihm de{\s}halb einmal geh"orig
grob:

"`Wa{\s} geht mich denn da{\s} an? Ich geh"ore nicht zu der
Clique!"'

Der Verliebte marterte sich mit Gr"ubeleien ab, wie er sich Emma
erkl"aren k"onne. Er schwankte fortw"ahrend zwischen der Furcht,
sich ihren Unwillen zuzuziehen, und der Scham "uber seine
Feigheit. Er vergo"s Tr"anen ob seiner Mutlosigkeit und seiner
Sehnsucht. Oft genug entschlo"s er sich zu k"uhner Entscheidung.
Er schrieb Briefe, die er wieder zerri"s; nahm sich Tage der Tat
vor, die er dann doch verstreichen lie"s. Manchmal ging er mir dem
festen Vorsatz zu ihr, alle{\s} zu wagen; aber in ihrer Gegenwart
verlor er al{\s}bald den Mut, und wenn gar Karl dazukam und ihn
einlud, sich mit in den Dogcart zu setzen, um irgendeinen
Patienten in der Umgegend zu besuchen, war er sofort dazu bereit.
Dann sagte er der "`gn"adigen Frau"' adieu und fuhr mit. War nicht
ihr Mann auch ein St"uck von ihr?

Emma ihrerseit{\s} fragte sich gar nicht, ob sie Leo liebe. E{\s}
war ihr Glaube, da"s die Liebe mit einem Male dasein m"usse, unter
Donner und Blitz, wie ein Sturm au{\s} blauem Himmel, der die
Menschen packt und ersch"uttert, ihnen den freien Willen
entrei"st, wie einem Baum da{\s} Laub, und da{\s} ganze Herz in
den Abgrund schwemmt. Sie wu"ste nicht, da"s der Regen auf den
flachen D"achern der H"auser Seen bildet, wenn die Traufen
verstopft sind. Und so w"are sie in ihrem Selbstbetrug verblieben,
wenn sie nicht mit einem Male den Ri"s in der Mauer bemerkt
h"atten.


\newpage\begin{center}
{\large \so{F"unfte{\s} Kapitel}}\bigskip\bigskip
\end{center}

E{\s} war an einem Sonntag nachmittag im Februar. E{\s} schneite.

Herr und Frau Bovary, der Apotheker und Leo hatten zusammen einen
Au{\s}flug unternommen, um eine neu errichtete Leineweberei, eine
halbe Stunde talabw"art{\s} von Yonville, zu besichtigen. Napoleon
und Athalia waren mitgenommen worden, weil sie Bewegung haben
sollten; und auch Justin war dabei, ein B"undel Regenschirme auf
der Schulter.

Die neue Sehen{\s}w"urdigkeit war eigentlich nicht{\s} weniger
al{\s} sehen{\s}wert. Um einen gro"sen "oden Platz, auf dem
zwischen Sand- und Steinhaufen bereit{\s} ein paar verrostete
Maschinenr"ader lagen, zog sich im Viereck ein Geb"aude mit einer
Menge kleiner Fenster hin. E{\s} war noch nicht ganz vollendet;
durch den ungedeckten Dachstuhl erblickte man den grauen Himmel.
An einem Giebelhaken hing ein Hebefestkranz au{\s} Stroh und
"Ahren mit einem im Winde flatternden wei"s-rot-blauen Wimpel.

Homai{\s} machte den F"uhrer. Er erkl"arte der Gesellschaft die
k"unftige Bedeutung de{\s} Etablissement{\s} und sch"atzte die
St"arke der Balken und die Dicke der Mauern, wobei er sehr
bedauerte, kein Meterma"s bei sich zu haben.

Emma hatte sich bei ihm eingeh"angt. Sie st"utzte sich ein wenig
auf seinen Arm und schaute tr"aumerisch in die Ferne nach der
Sonnenscheibe, deren matte{\s} rote{\s} Licht mit dem Nebel
k"ampfte. Pl"otzlich wandte sie sich ab. Da stand ihr Mann. Er
hatte seine M"utze bi{\s} auf die Augenbrauen in{\s} Gesicht
hereingezogen. Seine dicken Lippen zitterten vor Frost, wa{\s} ihm
einen bl"oden Zug verlieh. Sogar seine Hinteransicht, sein
beh"abiger R"ucken "argerte sie. Sie fand, die breite Fl"ache
seine{\s} Mantel{\s} kennzeichne die ganze Plattheit von Karl{\s}
Pers"onlichkeit.

W"ahrend sie ihn so ver"achtlich musterte, geno"s sie eine gewisse
perverse Wollust. Da kam Leo an sie heran. Die K"alte machte ihn
bleich, wa{\s} in sein Gesicht etwa{\s} Schmachtende{\s},
Sanfte{\s} brachte. Sein vorn offener Kragen lie"s zwischen
Krawatte und Hal{\s} ein St"uck Haut sehen; von seinem Ohr lugte
ein Teilchen zwischen den Str"ahnen seine{\s} Haar{\s} hervor, und
seine gro"sen blauen Augen, die zu den Wolken aufschauten, kamen
Emma viel klarer und sch"oner vor al{\s} in den Gedichten die
Bergseen, in denen sich der Himmel spiegelt.

"`Rabenkind!"' schrie pl"otzlich der Apotheker und scho"s auf
seinen Jungen lo{\s}, der eben in ein Kalkloch gesprungen war, um
sch"one wei"se Schuhe zu bekommen. Al{\s} er t"uchtig
au{\s}gescholten wurde, begann er laut zu heulen. Justin
versuchte, ihm die Stiefelchen mit einem Strohwisch zu reinigen,
aber ohne Messer ging da{\s} nicht. Karl bot ihm sein{\s} an.

"`Unerh"ort!"' dachte Emma bei sich. "`Er tr"agt ein Messer in der
Tasche wie ein Bauer!"'

Die neblige Luft wurde immer feuchter. Man machte sich auf den
Heimweg nach Yonville.

An diesem Abend ging Emma nicht mit zu den Nachbar{\s}leuten
hin"uber. Al{\s} ihr Mann fort war und sie sich allein wu"ste,
begann sie die beiden M"anner von neuem zu vergleichen, und der
andere stand in geradezu sinnlicher Deutlichkeit vor ihr, mit der
eigent"umlichen Linienver"anderung, die da{\s} menschliche
Ged"achtni{\s} vornimmt. Von ihrem Bette au{\s} sah sie die lichte
Glut im Kamin und daneben -- ganz so wie vor ein paar Stunden --
Leo, den Freund. Er stand da, in gerader Haltung, in der rechten
Hand den Spazierstock, und f"uhrte an der andern Athalia, die
bed"achtig an einem Ei{\s}zapfen saugte. Diese Szene hatte ihr
gefallen, und sie konnte von diesem Bilde nicht lo{\s}kommen. Sie
versuchte sich vorzustellen, wie er an andern Tagen au{\s}gesehen
hatte, welche Worte er gesagt, in welchem Tone. Wie sein Wesen
"uberhaupt sei~...

Die Lippen wie zum Kusse gerundet, fl"usterte sie immer wieder vor
sich hin: "`Ach, s"u"s, s"u"s!"' Und dann fragte sie sich: "`Ob er
eine liebt? Aber wen? Ach, mich, mich!"'

Mit einem Male sprach alle{\s} daf"ur. Da{\s} Herz schlug ihr vor
Freude. Die Flammen im Kamin warfen auf die Decke fr"ohliche
Lichter. Emma legte sich auf den R"ucken und breitete ihre Arme
weit au{\s}.

Dann aber hob sie ihr alte{\s} Klagelied an: "`Ach, warum hat
e{\s} der Himmel so gewollt? Warum nicht ander{\s}? Au{\s} welchem
Grunde?"'

Al{\s} Karl um Mitternacht heimkam, stellte sie sich so, al{\s}
wache sie auf; und al{\s} er sich etwa{\s} ger"auschvoll
au{\s}zog, klagte sie "uber Kopfschmerzen. Ganz nebenbei fragte
sie aber, wie der Abend verlaufen sei.

"`Leo ist heute zeitig gegangen"', erz"ahlte Karl.

Sie mu"ste l"acheln, und mit dem Gef"uhl einer ungeahnten
Gl"uckseligkeit schlummerte sie ein.

Am andern Tage, gegen Abend, empfing sie den Besuch de{\s} Herrn
Lheureux, de{\s} Modewarenh"andler{\s}. Der war, wie man zu sagen
pflegt, mit allen Hunden gehetzt. Obgleich ein geborener
Ga{\s}cogner, war er doch ein vollkommener Normanne geworden; er
einte in sich die lebhafte Redseligkeit de{\s} S"udl"ander{\s} und
die n"uchterne Verschlagenheit seiner neuen Land{\s}leute. Sein
feiste{\s}, aufgeschwemmte{\s} und bartlose{\s} Gesicht sah
au{\s}, al{\s} sei e{\s} mit S"u"sholztinktur gef"arbt, und sein
wei"se{\s} Haar brachte den scharfen Glanz seiner munteren
schwarzen Augen noch mehr zur Wirkung. Wa{\s} er fr"uher
getrieben, wu"ste man nicht. Manche munkelten, er sei Hausierer
gewesen, andre sagten, Geldwech{\s}ler in Routot. Etwa{\s} aber
stand fest: er konnte im Kopfe die schwierigsten Berechnungen
au{\s}f"uhren. Selbst Binet kam die{\s} unheimlich vor. Dabei war
er kriechend h"oflich; er lief in immer halb geb"uckter Haltung
herum, al{\s} ob er jemanden gr"u"sen oder einladen wollte.

Seinen mit einem Trauerflor versehenen Hut legte er an der T"ure
ab, stellte einen gr"unen Pappkasten auf den Tisch und begann sich
dann unter tausend Flo{\s}keln bei Frau Bovary zu beklagen, da"s
er ihre Kundschaft noch immer nicht gewonnen habe. Allerding{\s}
sei eine "`armselige Butike"' wie die seine nicht gerade
verlockend f"ur eine "`elegante Dame"'. Diese beiden Worte betonte
er ganz besonder{\s}. Aber sie brauche nur zu befehlen, er mache
sich anheischig, ihr alle{\s} nach Wunsch zu besorgen, Kurzwaren,
W"asche, Str"umpfe, Modewaren, wa{\s} sie brauche. Er fahre
regelm"a"sig viermal im Monat nach der Stadt und stehe mit den
ersten Firmen in Verbindung. Sie k"onne sich "uberall nach ihm
erkundigen. Heute komme er nur ganz im Vor"ubergehen, um der
gn"adigen Frau ein paar feine Sachen zu zeigen, die er durch einen
ganz besonder{\s} g"unstigen Gelegenheit{\s}kauf erworben h"atte.
Dabei packte er au{\s} dem Kasten ein halbe{\s} Dutzend gestickter
Hal{\s}kragen.

Frau Bovary besah sie sich.

"`Ich brauche nicht{\s}"', bemerkte sie.

Nunmehr kramte der H"andler behutsam drei algerische Seident"ucher
au{\s}, mehrere Pakete englischer N"ahnadeln, ein paar
strohgeflochtne Pantoffeln und schlie"slich vier Eierbecher au{\s}
Koko{\s}nu"sschale, filigranartige Schnitzarbeiten von
Str"aflingen. Sich mit beiden H"anden auf den Tisch st"utzend, mit
langem Hal{\s} und offnem Mund, beobachtete er Emma{\s} Augen, die
unentschlossen in all diesen Gegenst"anden herumsuchten. Von Zeit
zu Zeit strich er mit dem Fingernagel "uber die lang
hingebreiteten T"ucher, al{\s} wolle er ein St"aubchen entfernen;
die Seide knisterte leise, und da{\s} gr"unliche D"ammerlicht
glitzerte auf den Goldf"aden de{\s} Gewebe{\s} in sternigen
Funken.

"`Wa{\s} kostet so ein Tuch?"' fragte Emma.

"`Ein paar Groschen!"' antwortete er. "`Ein paar Groschen! Aber
da{\s} eilt ja nicht. Ganz wann{\s} Ihnen pa"st! Unsereiner ist ja
kein Jude!"'

Sie dachte einen Augenblick nach, schlie"slich dankte sie dem
H"andler, der gelassen erwiderte:

"`Na ja, dann ein andermal! Ich habe mich bi{\s}her mit allen
Damen vertragen, mit meiner nur nicht."'

Emma l"achelte. Er sah e{\s} und fuhr mit der Ma{\s}ke de{\s}
Biedermanne{\s} fort:

"`Ich wollte damit nur gesagt haben, da"s Geld Nebensache ist.
Wenn Sie mal welche{\s} brauchten, k"onnten Sie e{\s} von mir
haben."'

Sie machte eine erstaunte Miene.

Schnell fl"usterte er:

"`Oh! Ich verschaffte e{\s} Ihnen auf der Stelle! Darauf k"onnen
Sie sich verlassen!"'

Davon abspringend, erkundigte er sich flug{\s} nach dem alten
Tellier, dem Wirt vom Caf\'e Fran\c{c}ai{\s}, den Bovary gerade in
Behandlung hatte.

"`Wa{\s} fehlt ihm denn eigentlich, dem alten Freunde? Er hustet,
da"s sein ganze{\s} Hau{\s} wackelt. Ich f"urchte, ich f"urchte,
er l"a"st sich eher zu einem "Uberzieher au{\s} Fichtenholz Ma"s
nehmen al{\s} zu einem au{\s} Wintertuch. Na, solange er auf dem
Damme war, da hat er sch"one Zicken gemacht! Die Sorte, gn"adige
Frau, die wird nie vern"unftig! Und dann der Schnap{\s}, da{\s}
ist allemal der Ruin! Aber e{\s} ist immer betr"ubend, wenn man
sieht, wie e{\s} mit einem alten Bekannten zu Ende geht."'

W"ahrend er seine Siebensachen wieder in den Pappkasten packte,
schwatzte er so von allen m"oglichen Patienten de{\s} Arzte{\s}.

"`Da{\s} liegt am Wetter, ganz zweifello{\s}!"' erh"arte er, indem
er verdrie"slich durch die Fensterscheiben sah. "`Da{\s} bringt
alle diese Krankheiten. E{\s} geht mir ja selber so: ich f"uhle
mich gar nicht recht \begin{antiqua}au fait\end{antiqua}. Werde
wohl demn"achst auch mal zu Ihrem Herrn Gemahl in die Sprechstunde
kommen m"ussen. Meiner Kreuzschmerzen wegen. Na, auf Wiedersehen,
Frau Doktor! Stehe immer zu Ihrer Verf"ugung! Gehorsamster
Diener!"'

Und er schlo"s die T"ure sacht hinter sich.

Emma lie"s sich da{\s} Essen in ihrem Zimmer servieren, auf einem
Tischchen am Kamin. Sie nahm sich mehr Zeit denn sonst, und e{\s}
schmeckte ihr alle{\s} vorz"uglich.

"`Wie vern"unftig ich doch war!"' sagte sie bei sich und dachte an
die Seident"ucher.

Da h"orte sie Tritte auf der Treppe. E{\s} war Leo. Sie stand
schnell auf und nahm von der Kommode von einem Sto"s Wischt"ucher,
die ges"aumt werden sollten, da{\s} oberste zur Hand. Al{\s} der
junge Mann eintrat, tat sie sehr besch"aftigt.

Die Unterhaltung wollte nicht recht in Gang kommen. Frau Bovary
schwieg immer wieder, und Leo war au{\s} Sch"uchternheit
einsilbig. Er sa"s nahe am Kamin auf einem niedrigen Sessel und
spielte mit ihrem elfenbeinernen Nadelb"uchschen.

Emma n"ahte oder gl"attete von Zeit zu Zeit mit dem Fingernagel
den umgelegten Saum. Sie verstummte ganz, und er sagte nicht{\s},
weil ihn ihr Schweigen ebenso nachdenklich machte, al{\s} ob sie
wer wei"s wa{\s} gesprochen h"atte.

"`Armer Junge!"' dachte sie.

"`Warum bin ich bei ihr in Ungnade?"' fragte er sich.

Schlie"slich fing er an zu reden. Er m"usse in den n"achsten Tagen
nach Rouen fahren. In einer Beruf{\s}angelegenheit.

"`Ihr Musikalienabonnement ist abgelaufen. Darf ich e{\s}
erneuern?"'

"`Nein"', entgegnete sie.

"`Warum nicht?"'

"`Weil~..."'

Emma bi"s sich auf die Lippen. Umst"andlich zog sie den grauen
Zwirn hoch. Leo "argerte sich "uber ihre Emsigkeit. "`Warum
zersticht sie sich die Finger?"' dachte er. Eine galante Bemerkung
fuhr ihm durch den Sinn, aber er wagte nicht, sie
au{\s}zusprechen.

"`So wollen Sie e{\s} also aufgeben?"'

"`Wa{\s}?"' fragte sie nerv"o{\s}. "`Die Musik? Ach, du mein Gott!
Ich habe soviel in der Wirtschaft zu tun, meinen Mann zu versorgen
und tausend andre Dinge. Mit einem Wort: erst die Pflicht!"'

Sie blickte nach der Uhr. Karl h"atte schon l"angst heim sein
m"ussen. Sie stellte sich beunruhigt. Zwei- oder dreimal meinte
sie im Gespr"ache:

"`Mein Mann ist so gut!"'

Der Adjunkt mochte Herrn Bovary sehr gut leiden. Aber diese
Z"artlichkeit befremdete ihn auf da{\s} unangenehmste. Gleichwohl
stimmte er in ihr Lob ein.

"`Dar"uber sind wir un{\s} alle einig; der Apotheker sagt{\s} auch
immer!"' erkl"arte er.

"`Ja, ja, er ist ein pr"achtiger Mensch!"' wiederholte sie.

"`Gewi"s!"' best"atigte der Adjunkt.

Er begann dann von Frau Homai{\s} zu sprechen, "uber deren sehr
nachl"assige Kleidung sich die beiden sonst h"aufig am"usierten.

"`So schlimm ist e{\s} gar nicht!"' behauptete Emma heute. "`Eine
gute Hau{\s}frau kann sich nicht blo"s um ihre Toilette k"ummern."'

Dann versank sie in ihr fr"uhere{\s} Stillschweigen.

So blieb sie auch an den folgenden Tagen. Ihre Sprache, ihr
Benehmen, ihr ganze{\s} Wesen waren wie verwandelt. Sie k"ummerte
sich um ihr Hau{\s}, ging wieder regelm"a"sig in die Kirche und
hielt ihr Dienstm"adchen strenger.

Die kleine Berta wurde au{\s} der Ziehe zur"uckgeholt. Wenn Besuch
kam, brachte Felicie da{\s} Kind herein, und Frau Bovary zeigte,
wa{\s} f"ur stramme Beinchen e{\s} hatte. Sie beteuerte, Kinder
h"atte sie "uber alle{\s} gern; da{\s} ihre sei ihr Trost, ihre
Freude, ihr Gl"uck. Dabei liebkoste sie e{\s} unter einem Schwall
von schw"armerischen Tiraden, die jeden Literaturfreund -- die
biederen Yonviller waren keine! -- an die Sachette in Viktor
H"ugo{\s} "`Notre-Dame"' erinnert h"atten.

Wenn Karl heimkam, fand er seine Hau{\s}schuhe gew"armt am Kamine
stehen, seine Westen hatten kein zerrissene{\s} Futter mehr, und
an seinen Hemden waren die Kn"opfe immer vollz"ahlig. Er hatte
sogar da{\s} Vergn"ugen, seine H"ute und M"utzen wohlgeordnet im
Schranke h"angen zu sehen. Emma lehnte e{\s} mit einem Male nicht
mehr ab, ihn zu einem kleinen Rundgang in den Garten zu begleiten.
Sie war mit jedem Vorschlage, den Karl machte, sofort
einverstanden; selbst wenn sie den Zweck nicht recht einsah,
f"ugte sie sich ohne Murren. Wenn Leo die beiden nach Tisch so
sah: ihn am Kamin, die H"ande "uber dem Bauche gefaltet, die
F"u"se behaglich gegen die Glut gestemmt, die Backen noch rot vom
Mahle und die "Auglein in eitel Wonne schwimmend, vor sich da{\s}
Kind, da{\s} auf dem Teppich herumrutschte, und daneben die
feinlinige schlanke Frau, wie sie sich "uber die Lehne seine{\s}
Gro"svaterstuhl{\s} beugte und ihm einen Ku"s auf die Stirn gab,
-- dann sagte er sich:

"`Ich Narr! Nie wird sie die meine werden!"'

Sie kam ihm ebenso vollkommen wie unnahbar vor, und ihm schwand
jede, auch die leiseste Hoffnung. In seiner Resignation begann er
sie zu verg"ottern. Allm"ahlich verlor sie in seinen Augen ihre
K"orperlichkeit, die nun einmal doch f"ur ihn nicht da war. Vor
seiner Phantasie schwebte sie immer h"oher, umstrahlt von einer
Gloriole. Seine reine Liebe hatte nicht{\s} mehr mit seinem
Alltag{\s}leben zu tun; sie ward zu einem Heiligenkult, dessen
Verlust mehr Schmerz bereitet, al{\s} der k"orperliche Besitz der
Geliebten Genu"s gew"ahrt.

Emma magerte ab, ihre Wangen verloren die Farbe, ihr Gesicht wurde
schm"achtiger. Mit ihrem schwarzen gescheitelten Haar, ihren
gro"sen Augen, ihrer gerade geschnittenen Nase, ihrem Vogelgange
und ihrer jetzigen Schweigsamkeit schien sie durch{\s} Leben zu
schreiten, ohne den Erdboden zu ber"uhren, und e{\s} war, al{\s}
tr"uge sie auf der Stirne da{\s} geheimni{\s}volle Mal einer
h"oheren Bestimmung. Sie war so traurig und so still, so sanft und
dabei so unnahbar, da"s man ihre Gegenwart wie eine ei{\s}kalte
Wonne empfand. Geradeso mischt sich in den Kirchen in den Duft der
Rosen die K"alte de{\s} Marmor{\s}, so da"s man zusammenschauert.
E{\s} lag ein seltsamer Zauber darin, dem niemand entrann.

"`Sie ist eine Frau gro"sen Stil{\s},"' sagte der Apotheker
einmal, "`sie m"u"ste einen Minister zum Manne haben!"'

Die Spie"sb"urger r"uhmten ihre Sparsamkeit, die Patienten ihr
h"ofliche{\s} Wesen, die armen Leute ihren milden Sinn.

Innerlich aber war sie voller Begierden, voll Grimm und Ha"s.
Hinter ihrem kl"osterlichen Kleid st"urmte ein weltverlangende{\s}
Herz, und ihre keuschen Lippen verheimlichten alle Qualen der
Sinnlichkeit. Sie war in Leo verliebt. Sie suchte die Einsamkeit,
um in der Vorstellung ungest"ort zu schwelgen. Diese Wollust der
Tr"aume ward ihr durch den leibhaftigen Anblick de{\s} Geliebten
nur gest"ort. Beim H"oren seiner Tritte zitterte sie. Sobald er
aber eintrat, verflog diese Erregung, und sie f"uhlte nicht{\s}
al{\s} namenlose Verwunderung und tiefe Schwermut.

Leo ahnte nicht, da"s Emma an{\s} Fenster eilte, um ihm
nachzusehen, wenn er entmutigt von ihr gegangen war. Voller Unruhe
beobachtete sie alle seine Bewegungen und forschte in seinen
Augen. Sie erfand einen ganzen Roman, nur um einen Vorwand zu
haben, sein Zimmer einmal zu sehen. Die Apothekerin erschien ihr
beneiden{\s}wert, weil sie mit ihm unter einem Dache schlafen
durfte. Ihre Gedanken lie"sen sich immer wieder auf seinem Hause
nieder, just wie die Tauben vom Goldnen L"owen, die hingeflogen
kamen, um ihre roten Stelzen und wei"sen Fl"ugel in der Dachrinne
zu netzen.

Je klarer sich Emma ihrer Leidenschaft bewu"st ward, um so mehr
dr"angte sie sie zur"uck. Ihre Liebe sollte unsichtbar und klein
bleiben. Wohl war e{\s} ihr Sehnen, da"s Leo die Wahrheit bemerke;
sie ertr"aumte sich Zuf"alle und Katastrophen, die die{\s}
herbeif"uhrten. Aber ihre Passivit"at, die Angst vor der
Entscheidung und auch ihr Schamgef"uhl hielten sie zur"uck. Sie
bildete sich ein, sie h"atte sich ihn bereit{\s} allzusehr
entfremdet, e{\s} w"are nun zu sp"at und alle{\s} sei verloren.
Und dann sagte sie sich voll Stolz und Freude: "`Ich bin eine
anst"andige Frau geblieben!"' Sie stellte sich vor den Spiegel in
der Pose der Resignation. Da{\s} tr"ostete sie ein wenig ob de{\s}
Opfer{\s}, da{\s} sie zu bringen w"ahnte.

Ihre unbefriedigte Sinnlichkeit, ihre L"usternheit nach Reichtum
und Luxu{\s} und ihre schwerm"utige Liebe ergaben alle{\s} in
allem ein einzige{\s} Weh. Statt aber ihre Gedanken andern Dingen
zuzuwenden, verlor sie sich immer mehr in diese{\s} Leid, gefiel
sich darin und trug e{\s} in alle Einzelheiten ihre{\s} Leben{\s}.
Ein ungeschickt servierte{\s} Gericht, eine offengelassene T"ure
brachte sie in Aufregung. Ein h"ubsche{\s} Kleid, da{\s} sie nicht
haben konnte, ein Vergn"ugen, auf da{\s} sie verzichten mu"ste,
machte sie ungl"ucklich. Weil sich ihre k"uhnen Tr"aume nicht
erf"ullten, ward ihr da{\s} Hau{\s} zu eng.

Da"s Karl keine Dulderin in ihr sah, da{\s} emp"orte sie am
allermeisten. Seine felsenfeste "Uber\-zeu\-gung, da"s er seine
Frau gl"ucklich mache, d"unkte sie Beschr"anktheit, Beleidigung,
Undankbarkeit. F"ur wen war sie denn so vern"unftig? War e{\s}
nicht gerade Karl, der sie von jedwedem Gl"uck trennte? War nicht
er der Anla"s all ihre{\s} Elend{\s}, da{\s} Schlo"s an der T"ur
ihre{\s} qualvollen K"afig{\s}?

So h"aufte sie auf ihn alle Bitternisse ihre{\s} Herzen{\s}. Jeder
Versuch, diese Verstimmungen zu bek"ampfen, verschlimmerten sie
nur. Denn die vergebliche M"uhe machte sie noch mutloser und
entfernte sie noch mehr von ihrem Manne. Gerade seine
Gutm"utigkeit reizte sie zur Rebellion. Die Spie"serlichkeit ihrer
Wohnung verlockte sie zu Utopien von Pracht und Herrlichkeit, und
die ehelichen Freuden zu ehebrecherischen Gel"usten. Sie bedauerte
e{\s}, da"s Karl sie nicht schlecht behandelte; dann h"atte sie
gerechten Anla"s gehabt, sich an ihm zu r"achen. Zuweilen freilich
erschrak sie vor den Irrwegen, auf die sie in Gedanken geriet. Und
immer mu"ste sie l"acheln, wenn sie in einem fort h"orte, da"s sie
gl"ucklich sei, oder wenn sie sich gar selber noch M"uhe gab, so
zu tun und die Leute in ihrem Glauben zu lassen.

Manchmal hatte sie diese Kom"odie satt. Sie f"uhlte sich versucht,
mit dem Geliebten auf und davon zu gehen, irgendwohin, weit, weit
fort, wo ein andrer Stern ihrer harrte. Zugleich jedoch drohten
ihr in Gedanken riefe, dunkle Abgr"unde.

"`Er liebt mich ja gar nicht mehr!"' sagte sie sich. "`Wa{\s} soll
da au{\s} mir werden? Welche Zuflucht, welcher Trost, welche
Erleichterung bleibt mir noch?"'

Gebrochen, fiebernd, halbtot schluchzte sie leise vor sich hin,
unter endlosen Tr"anen.

"`Warum sagt e{\s} die gn"adige Frau nicht dem Herrn Doktor?"'
fragte da{\s} Dienstm"adchen, al{\s} e{\s} einmal w"ahrend
eine{\s} solchen Anfalle{\s} in{\s} Zimmer kam.

"`Ach wa{\s}! Ich bin nerv"o{\s}!"' erkl"arte Emma. "`Da"s du ihm
ja nicht{\s} davon erz"ahlst! Du w"urdest ihn nur beunruhigen."'

"`Ach Gott"', meinte Felicie. "`Der Tochter de{\s} alten
Fischer{\s} Gu\'erin au{\s} Pollet, einer Bekannten von mir in
Dieppe, wo ich vorher gedient habe, der ging e{\s} ganz genau so.
War die tr"ubsinnig! Schrecklich tr"ubsinnig! Und leichenbla"s sah
sie immer au{\s}. Ihr Leiden war so wa{\s} wie ein Nebel im Kopfe,
und die "Arzte und sogar der Pfarrer wu"sten kein Mittel dagegen.
Wenn{\s} ganz schlimm kam, dann lief sie immer ganz allein an{\s}
Meer. Der Zollaufseher hat sie auf seiner Patrouille oft gesehen,
platt auf dem Bauche liegen und auf den Steinen weinen. Sp"ater,
al{\s} sie einen Mann hatte, soll sich{\s} gegeben haben~..."'

"`Bei mir aber"', erwiderte Emma, "`ist e{\s} erst nach der
Hochzeit so gekommen."'


\newpage\begin{center}
{\large \so{Se{ch}{st}e{\s} Kapitel}}\bigskip\bigskip
\end{center}

Eine{\s} Abend{\s} sa"s Emma am offnen Fenster. Eben hatte sie
noch Lestiboudoi{\s}, dem Kirchendiener, zugesehen, wie er unten
im Garten den Buch{\s}baum zugestutzt hatte. Pl"otzlich drang ihr
da{\s} Ave-Maria-L"auten in{\s} Ohr.

E{\s} war Anfang April. Die Primeln bl"uhten, und ein lauer Wind
h"upfte "uber die aufgeharkten Beete. Der Garten putzte sich f"ur
die Festtage de{\s} Sommer{\s}. Durch die Latten der Laube und
weiterhin leuchtete der Bach, der sich in schn"orkeligen Windungen
in den flachen Wiesen hinwand. Der Abenddunst schwebte um die noch
kahlen Pappeln und l"oste die Linien ihrer Aste zu weichem Violett
auf, duftig und durchsichtig wie ein feiner Schleier. In der Ferne
zogen Herden heim, aber ihr Huftritt und ihr Br"ullen verklangen.
Nur die Abendglocke l"autete immerfort und f"ullte die Luft mit
wehm"utigem Frieden.

Bei diesen gleichf"ormigen T"onen verloren sich die Gedanken der
jungen Frau in alte Jugend- und Klostererinnerungen. Sie dachte an
die hohen Leuchter auf dem Hochaltar, die sich "uber die
blumenreichen Vasen und "uber da{\s} Tabernakel mit seinen
S"aulchen emporgereckt hatten. Wie einst h"atte sie wieder knien
m"ogen in der langen Reihe der wei"sen Schleier, die sich grell
abhoben von den schwarzen steifen Kapuzen der in ihren Betst"uhlen
hingesunkenen Schwestern. Sonntag{\s} w"ahrend der Messe, wenn sie
aufschaute und in da{\s} von bl"aulichem Weihrauch umwobene holde
Antlitz der Madonna blickte, dann war sie immer tief ergriffen und
ganz weich gestimmt gewesen, leicht und ohne Last wie eine
Flaumfeder, die der Sturmwind wegweht~...

Mit einem Male, ohne da"s sie sich "uber den Vorgang klar ward,
fand sie sich auf dem Wege zur Kirche. Ein Drang nach Andacht
hatte sie ergriffen: ihre Seele sehnte sich, darin aufzugehen und
alle{\s} Irdische zu vergessen.

Auf dem Marktplatze begegnete ihr Lestiboudoi{\s}, der bereit{\s}
wieder au{\s} der Kirche kam, um zu seiner unterbrochenen Arbeit
zur"uckzukehren. Die war ihm immer die Hauptsache, und da{\s}
L"auten der Glocke besorgte er, wie e{\s} ihm gerade pa"ste.
"Ubrigen{\s} war da{\s} L"auten ein Zeichen f"ur die Kinder im
Dorfe, da"s e{\s} Zeit zur Katechi{\s}mu{\s}stunde war.

Ein paar Jungen waren schon da und spielten Ball auf den
Friedhof{\s}steinen. Andre sa"sen rittling{\s} auf der Mauer,
baumelten mit den Beinen und k"opften mit ihren Schuhspitzen die
hohen Brennesseln, die zwischen der letzten Gr"aberreihe und der
niedrigen Umfassung{\s}mauer aufgeschossen waren. Da{\s} war
da{\s} einzige bi"schen Gr"un, denn die Grabm"aler standen ganz
dicht aneinander, und "uber ihnen lag best"andig feiner Staub, der
dem reinigenden Besen trotzte. Die Kinder liefen in Str"umpfen
dar"uber wie "uber einen eigen{\s} f"ur sie hingebreiteten
Teppich, und ihre aufjauchzenden Stimmen mischten sich in da{\s}
letzte Au{\s}klingen der Glocken. Da{\s} Summen verstummte, und
der Strang der gro"sen Glocke, der vom Kirchturm herabhing und mit
dem Ende auf dem Erdboden hin und her geschleift war, beruhigte
sich allm"ahlich. Schwalben schossen pfeilschnell durch die Luft,
kurze Schreie au{\s}sto"send, und flogen zur"uck in ihre gelben
Nester unter dem Turmdache. Im Chor der Kirche brannte eine Lampe
oder vielmehr ein Nachtlicht unter einer h"angenden Gla{\s}glocke.
Von weitem sah die Flamme wie ein "uber dem "Ol schwimmender
zittriger wei"ser Fleck au{\s}. Ein langer Sonnenstrahl
durchquerte da{\s} Hauptschiff; in um so tieferem Dunkel lagen die
Nebenschiffe und Nischen.

"`Wo ist der Pfarrer?"' fragte Frau Bovary einen Knaben, der sich
damit belustigte, die bereit{\s} lockere Klinke der
Friedhof{\s}pforte v"ollig abzuw"urgen.

"`Der wird gleich kommen!"' war die Antwort.

Wirklich knarrte die T"ur de{\s} Pfarrhause{\s}, und der Abb\'e
Bournisien erschien. Die Kinder rannten eiligst in die Kirche
hinein.

"`Rasselbande!"' murmelte der Priester. "`Einen wie alle Tage!"'
Er hob einen zerflederten Katechi{\s}mu{\s} auf, an den sein Fu"s
gesto"sen war. "`Nicht{\s} wird respektiert!"' Da bemerkte er Frau
Bovary.

"`Verzeihung!"' sagte er. "`Ich hatte Sie nicht erkannt."'

Er steckte den Katechi{\s}mu{\s} in die Tasche und blieb stehen,
indem er den schweren Sakristeischl"ussel auf zwei Fingern
balancierte.

Der Schein der Abendsonne fiel ihm voll in{\s} Gesicht und nahm
seiner Soutane alle Farbe. Sie gl"anzte "ubrigen{\s} an den
Ellenbogen bereit{\s}, und in den S"aumen war sie au{\s}gefasert.
Fett- und Tabakflecke begleiteten die Linie der kleinen Kn"opfe
die Brust entlang. Nach dem Kragen zu, unter dem Doppelkinn
seine{\s} Gesicht{\s}, wurden sie zahlreicher. E{\s} war von
Sommersprossen bes"at, die sich in seinen stoppeligen grauen Bart
hinein verloren. Er kam vom Essen und atmete ger"auschvoll.

"`Wie geht e{\s} Ihnen?"' erkundigte er sich.

"`Schlecht!"' antwortete Emma.

"`Ja, ja! Ganz wie mir"', erwiderte der Priester. "`Die ersten
warmen Tage machen einen unglaublich matt, nicht wahr? Aber e{\s}
ist nun einmal so! Wir sind zum Leiden geboren, wie Sankt
Paulu{\s} sagt. Und wie denkt Herr Bovary dar"uber?"'

"`Ach der!"' Sie machte eine ver"achtliche Geb"arde.

"`Wa{\s}?"' erwiderte der ehrw"urdige Mann ganz erstaunt.
"`Verordnet er Ihnen denn nicht{\s}?"'

"`Ach,"' meinte sie, "`irdische Heilmittel, die nutzen mir
nicht{\s}."'

Trotzdem sich der Geistliche unterhielt, warf er seinen Blick doch
hin und wieder in die Kirche, wo die Jungen, die niedergekniet
waren, sich gegenseitig mit den Schultern anrempelten, so da"s sie
reihenweise wie die Kegel umpurzelten.

"`Ich m"ochte gern wissen~..."', fuhr Emma fort.

"`Warte nur, Boudet, warte du nur!"' unterbrach sie der Priester
in zornigem Tone. "`Ich werde dich gleich an den Ohren kriegen, du
Schlingel, du!"' Zu Emma gewandt, f"ugte er hinzu: "`Da{\s} ist
der Junge vom Zimmermann Boudet. Seine Eltern sind schwache Leute;
sie lassen dem Jungen die gr"o"sten Narrenpossen durch. Der Bengel
k"onnte sehr wohl wa{\s} lernen, wenn er nur wollte, denn er ist
gar nicht dumm ... Na, und wie geht{\s} dem Herrn Gemahl?"'

Emma tat, al{\s} ob sie die Frage "uberh"ort h"atte. Der
Geistliche fuhr fort:

"`Immer t"uchtig besch"aftigt, nicht wahr? Ja, ja! Er und ich, wir
beiden haben im Kirchspiel zweifello{\s} am meisten zu tun~..."'
Er lachte beh"abig, "`... er al{\s} Arzt de{\s} Leibe{\s} und ich
der Seele."'

Emma schaute ihn flehentlich an.

"`Sie! Ja!"' sagte sie. "`Sie heilen alle Wunden!"'

"`Oh! Sprechen Sie nicht so, Frau Bovary! Gerade heute vormittag,
da bin ich nach Ba{\s}-Diauville gerufen worden, zu einer
wassers"uchtigen Kuh. Die Leute glaubten, da{\s} Tier sei verhext.
Merkw"urdig! Alle K"uhe da ... Verzeihen Sie mal! -- Longuemarre
und Boudet! Zum Donnerwetter! Wollt ihr stille sein!"' Mit einem
gro"sen Satze war er drinnen in der Kirche.

Da flohen die Knaben hinter da{\s} Me"spult oder kletterten auf
den Sitz de{\s} Vors"anger{\s}. Andre verkrochen sich in den
Beichtstuhl. Aber der Pfarrer teilte behend recht{\s} und link{\s}
einen Hagel von Backpfeifen au{\s}; einen der Jungen packte er am
Rockkragen, hob ihn in die Luft und duckte ihn dann in die Knie,
al{\s} ob er ihn mit aller Gewalt in die Steinfliese
hineindr"ucken wollte.

"`So!"' sagte er zu Frau Bovary, al{\s} er wieder bei ihr war,
w"ahrend er sein gro"se{\s} Kattuntaschentuch entfaltete und sich
den Schwei"s von der Stirn wischte. "`Die Landleute sind recht zu
bedauern~..."'

"`Andre Leute auch"', meinte sie.

"`Gewi"s! Die Arbeiter in den St"adten zum Beispiel."'

"`Die meine ich nicht."'

"`Erlauben Sie mir! Ich habe unter ihnen Familienm"utter kennen
lernen, ehrbare Frauen, ich sage Ihnen: wahre Heilige. Und sie
hatten nicht einmal da{\s} t"agliche Brot."'

"`Ich meine solche,"' fuhr Emma fort, und ihre Mundwinkel
zitterten, w"ahrend sie sprach, "`solche, Herr Pfarrer, die zwar
ihr t"aglich Brot haben, aber kein~..."'

"`Kein Holz im Winter~..."', erg"anzte der Priester.

"`Ach, wa{\s} liegt daran?"'

"`Wa{\s} daran liegt? Mich d"unkt, wer gut zu essen hat und eine
warme Stube ... denn schlie"slich~..."'

"`O du mein Gott!"' seufzte Emma.

"`Ist Ihnen nicht wohl?"' fragte er, indem er sich ihr besorgt
n"aherte. "`Gewi"s Magenbeschwerden? Sie m"ussen heimgehen, Frau
Bovary, und eine Tasse Tee trinken! Da{\s} wird Sie kr"aftigen.
Oder vielleicht lieber eine Limonade?"'

"`Wozu?"'

Sie sah au{\s}, al{\s} erwache sie au{\s} einem Traume.

"`Sie fa"sten mit der Hand nach Ihrer Stirn, und da glaubte ich,
e{\s} sei Ihnen schwindlig."' Er besann sich. "`Aber wollten Sie
mich nicht etwa{\s} fragen? Mir ist e{\s} so. Wa{\s} war e{\s}
denn?"'

"`Ich? Nicht{\s} ... oh, nicht{\s}!"' stammelte Emma.

Ihr Blick, der in der Ferne verweilt hatte, fiel m"ud auf den
alten Mann in der Soutane. Sie sahen sich beide in die Augen, ohne
etwa{\s} zu sagen.

"`Dann entschuldigen Sie, Frau Bovary"', sagte er nach einer
Weile. "`Die Pflicht ruft mich. Ich mu"s zu meinen Taugenichtsen
da. Die erste Kommunion r"uckt heran. Ich f"urchte, sie
"uberrumpelt un{\s}. Seit Himmelfahrt behalte ich die Kinder alle
Mittwoch eine Stunde l"anger hier. Die armen Kleinen! Man kann sie
nicht fr"uh genug auf den Weg de{\s} Herrn leiten, wie e{\s}
Gotte{\s} Sohn un{\s} ja anbefohlen hat ... Recht gute Besserung,
Frau Doktor! Empfehlen Sie mich, bitte, Ihrem Herrn Gemahl!"'

Damit trat er in die Kirche, nachdem er an der Schwelle da{\s}
Knie gebeugt hatte. Emma sah ihm nach, bi{\s} er zwischen den
B"anken verschwand. Er ging schwerf"allig, den Kopf ein wenig
eingezogen, die beiden H"ande in segnender Haltung.

Sie wandte sich um, mit einem kurzen Ruck. wie eine Figur auf
einer Drehscheibe, und schickte sich an, nach Hause zu gehen. Eine
Weile h"orte sie hinter sich noch die rauhe Stimme de{\s}
Geistlichen und die hellen Antworten der Knaben~...

"`Bist du ein Christ?"'

"`Ja, ich bin ein Christ."'

"`Wer ist ein Christ?"'

"`Wer getauft ist und~..."'

Zu Hau{\s} stieg sie die Treppe hinauf, wobei sie sich am
Gel"ander festhielt. In ihrem Zimmer angekommen, sank sie in ihren
Lehnstuhl.

Da{\s} Licht de{\s} hellen Abend{\s} drau"sen flutete weich durch
die Scheiben herein. Die M"obel schlummerten still auf ihren
Pl"atzen, halb versunken in den Schatten der D"ammerung wie in
einen schwarzen Weiher. Im Kamin war die Glut erloschen, und
eint"onig tickte die Uhr immerzu. Diese Ruhe der Dinge hier um
sich herum empfand Emma al{\s} einen wunderlichen Kontrast zu dem
wilden Sturm in ihrem Innern~...

Vom N"ahtischfenster her tappte die kleine Berta in ihren
gewirkten Schuhchen und versuchte zu ihrer Mutter zu gelangen. Sie
haschte nach den B"andern ihrer Sch"urze.

"`La"s mich!"' sagte Emma und wehrte da{\s} Kind mit der Hand ab.

Aber die Kleine kam noch n"aher und schmiegte sich an ihre Knie.
Sie umfa"ste sie mit ihren "Armchen und schaute mit ihren gro"sen
blauen Augen zur Mutter auf. Dabei liefen ein paar Tropfen
Speichel au{\s} dem Munde de{\s} Kinde{\s} auf Emma{\s} seidne
Sch"urze.

"`La"s mich!"' wiederholte die junge Mutter sehr unwillig.

Ihr Gesicht{\s}au{\s}druck erschreckte da{\s} Kind. E{\s} begann
zu schreien.

"`Aber so la"s mich doch!"' sagte Emma barsch und stie"s ihr Kind
mit dem Ellenbogen zur"uck.

Berta fiel gegen die Kommode, gerade auf den Messingbeschlag, der
ihr die Wange ritzte, so da"s sie blutete. Frau Bovary st"urzte
auf da{\s} Kind zu und hob e{\s} auf. Dann ri"s sie heftig am
Klingelzug und rief da{\s} Dienstm"adchen herbei. Sie war nahe
daran, sich Vorw"urfe zu machen, da erschien Karl. E{\s} war um
die Essen{\s}zeit. Er kam von seiner Praxi{\s} heim.

"`Sieh, mein Lieber,"' sagte sie ruhigen Tone{\s}, "`die Kleine
ist beim Spielen gefallen und hat sich ein bi"schen geschunden."'

Karl beruhigte sie; e{\s} sei nicht schlimm. Er holte Heftpflaster.

Frau Bovary ging zum Essen nicht hinunter. Sie wollte ihr Kind
allein pflegen. Al{\s} sie dann aber sah, wie e{\s} ruhig schlief,
verging ihr bi"schen Beunruhigung, und sie kam sich selber recht
t"oricht und schlapp vor, weil sie sich wegen einer
Geringf"ugigkeit gleich so aufgeregt habe. In der Tat klagte die
Kleine nicht mehr. Ihre Atemz"uge hoben und senkten die wollene
Bettdecke kaum merkbar. Ein paar dicke Tr"anen hingen ihr in den
halbgeschlossenen Wimpern, durch die zwei tiefliegende blasse
Augensterne schimmerten. Da{\s} auf die Backe geklebte Pflaster
verzog die Haut.

"`Merkw"urdig!"' dachte Emma bei sich. "`Wie h"a"slich da{\s} Kind
ist!"'

Al{\s} Karl um elf Uhr nach Hause kam -- er war nach Tisch zum
Apotheker gegangen --, fand er seine Frau an der Wiege stehen.

"`Aber ich habe dir doch gesagt, da"s e{\s} nicht{\s} ist!"'
versicherte er ihr, indem er ihr einen Ku"s auf die Stirn gab.
"`"Angstige dich nicht, arme{\s} Lieb, du wirst mir sonst krank!"'

Er war lange beim Apotheker geblieben. Er hatte sich zwar gar
nicht besonder{\s} aufgeregt gezeigt, trotzdem hatte sich
Homai{\s} f"ur verpflichtet gef"uhlt, ihn "`aufzurappeln"'. Dann
hatte man von den tausend Gefahren gesprochen, denen kleine Kinder
au{\s}gesetzt sind, und von der Unachtsamkeit der Dienstboten.
Frau Homai{\s} mu"ste ein Lied davon zu singen. Noch heute hatte
sie auf der Brust ein Brandmal: auf diese Stelle hatte die
damalige K"ochin einmal die Kohlenpfanne fallen lassen!
Infolgedessen waren die braven Homai{\s} "uber die Ma"sen
vorsichtig. Die Tischmesser wurden nicht geschliffen und der
Fu"sboden nicht gebohnt. Vor den Fenstern waren eiserne Gitter und
vor dem Kamin ein paar Querst"abe angebracht. Die
Apotheker{\s}kinder, so verwahrlost sie im "ubrigen waren, konnten
keinen Schritt tun, ohne da"s jemand dabei sein mu"ste. Bei der
geringsten Erk"altung stopfte sie der Vater mit Hustenbonbon{\s}
voll, und al{\s} sie bereit{\s} "uber vier Jahre alt waren,
mu"sten sie ohne Gnade noch dickgepolsterte Fallringe um die
K"opfe tragen. Da{\s} war lediglich eine Schrulle der Mutter; der
Apotheker war in{\s}geheim sehr betr"ubt dar"uber, weil er Angst
hatte, diese{\s} Zusammenpressen k"onne dem Gehirn sch"adlich
sein. Einmal entfuhr e{\s} ihm:

"`Willst du denn Hottentotten au{\s} deinen Kindern machen?"'

Karl hatte etliche Male den Versuch gemacht, die Unterhaltung in
eine andre Richtung zu bringen. Beim Gehen, al{\s} Leo vor ihm die
Treppe hinunterstieg, raunte er ihm leise zu:

"`Ich wollte Sie noch etwa{\s} fragen!"'

"`Sollte er etwa{\s} gemerkt haben?"' fragte sich der Adjunkt. Er
bekam Herzklopfen und verlor sich in tausend Vermutungen.

Al{\s} die T"ure hinter ihnen geschlossen war, bat Karl, er solle
sich doch einmal in Rouen danach erkundigen, wa{\s} ein
h"ubsche{\s} Lichtbild koste. Er hegte n"amlich schon lange den
sentimentalen Plan, seine Frau mit dieser zarten Aufmerksamkeit zu
"uberraschen. Er gedachte sich im schwarzen Rocke verewigen zu
lassen. Nur wollte er vorher wissen, wieviel die Geschichte so
ungef"ahr zu stehen k"ame. Dem Adjunkt mache da{\s} wohl keine
besondre M"uhe, da er doch beinahe aller acht Tage nach der Stadt
f"uhre.

Zu welchem Zwecke eigentlich? Homai{\s} vermutete
Junggesellenabenteuer oder eine Liebschaft. Aber da t"auschte er
sich. Leo hatte keine galanten Beziehungen. Mehr denn je war er in
Wertherstimmung. Die L"owenwirtin merkte e{\s} daran, da"s er
seine Portionen nicht mehr aufa"s. Um hinter die Ursache zu
kommen, fragte sie Binet; aber der Steuereinnehmer erwiderte
unwirsch, er sei kein Polizeib"uttel.

Allerding{\s} kam Leo auch seinem Tischgenossen recht sonderbar
vor. Oft lehnte er sich in seinen Stuhl zur"uck, packte sich mit
den H"anden hinten am Kopfe und lie"s sich in unbestimmten Klagen
"uber da{\s} menschliche Dasein au{\s}.

"`Sie sollten sich ein bi"schen mehr zerstreuen"', meinte der
Steuereinnehmer.

"`Womit denn?"'

"`Na, an Ihrer Stelle schaffte ich mir eine Drehbank an."'

"`Aber ich kann doch nicht drechseln"', erwiderte der Adjunkt.

"`Ach ja, freilich!"'

Binet strich sich selbst\/zufrieden-ver"achtlich da{\s} Kinn.

Leo war e{\s} m"ude, erfolglo{\s} zu lieben. Da{\s} eint"onige
Leben begann ihn abzustumpfen; er hatte keine Interessen, die ihn
erf"ullten, keine Hoffnungen, die ihn st"arkten. Yonville und die
Yonviller "odeten ihn derma"sen an, da"s er gewisse Leute und
bestimmte H"auser nicht mehr erblicken konnte, ohne in Wut zu
geraten. Besonder{\s} unau{\s}stehlich wurde ihm nachgerade der
biedere Apotheker. Gleichwohl schreckte ihn die Au{\s}sicht auf
v"ollig neue Verh"altnisse genau so sehr, wie er sich danach
sehnte. Diese{\s} bange Gef"uhl wandelte sich nach und nach in
Unruhe, und nun lockte ihn Pari{\s}, da{\s} ferne Pari{\s} mit der
rauschenden Musik seiner Ma{\s}kenfeste und dem Lachen seiner
Grisetten. Er sollte daselbst sowieso sein Studium vollenden.
Warum ging er nicht endlich dahin? Wa{\s} hielt ihn zur"uck?

In Gedanken fing er nun an, seine Vorbereitungen zu treffen. Er
machte heimliche Pl"ane. Er tr"aumte sich sein Pariser Zimmer
au{\s}. Dort wollte er da{\s} Leben eine{\s} Boh\'emien f"uhren.
Gitarre wollte er spielen lernen, einen Schlafrock tragen, dazu
ein Samtbarett und Hau{\s}schuhe au{\s} blauem Pl"usch. Und "uber
dem Kamin sollten zwei gekreuzte Florett{\s} h"angen, ein
Totensch"adel dar"uber und die Gitarre darunter. Wundervoll!

Da{\s} Schwierige war nur, die Einwilligung seiner Mutter zu
bekommen. Aber im Grunde war sein Plan doch der
allervern"unftigste! Sogar sein Chef redete ihm zu, sich in einer
andern Kanzlei weiter au{\s}zubilden. So entschied sich Leo
zun"achst zu einem Mittelding. Er bewarb sich um einen
Adjunktenposten in Rouen. Al{\s} ihm die{\s} mi"slang, schrieb er
schlie"slich seiner Mutter einen langen Brief, in dem er ihr
au{\s}f"uhrlich au{\s}einandersetzte, warum er ohne weitere{\s}
nach Pari{\s} "ubersiedeln wollte. Sie war damit einverstanden.

Trotz alledem beeilte er sich keine{\s}weg{\s}. Volle vier Wochen
lang gingen von Yonville nach Rouen und von Rouen nach Yonville
Koffer, Rucks"acke und Pakete f"ur ihn hin und her. Er
vervollst"andigte seine Garderobe, lie"s seine drei Lehnst"uhle
aufpolstern, schaffte sich einen Vorrat von seidnen
Hal{\s}t"uchern an, kurz und gut, er traf Vorbereitungen, al{\s}
wolle er eine Reise um die Welt antreten. So verstrich Woche auf
Woche, bi{\s} ein zweiter m"utterlicher Brief seine Abreise
beschleunigte. Er h"atte doch die Absicht, ein Examen nach einem
Semester zu machen.

Al{\s} der Augenblick de{\s} Abschied{\s} gekommen war, da weinte
Frau Homai{\s}, Justin heulte, und Homai{\s} verbarg seine
R"uhrung, wie sich da{\s} f"ur einen ernsten Mann schickt. Er
lie"s e{\s} sich jedoch nicht nehmen, den Mantel seine{\s}
Freunde{\s} eigenh"andig bi{\s} zur Gartenpforte de{\s} Notar{\s}
zu tragen, wo de{\s} letzteren Kutsche wartete, die den
Scheidenden nach Rouen fahren sollte.

Im letzten Viertelst"undchen machte Leo seinen Abschied{\s}besuch
im Hause de{\s} Arzte{\s}.

Al{\s} er die Treppe hinaufgestiegen war, blieb er stehen, um Atem
zu sch"opfen. Bei seinem Eintritt kam ihm Frau Bovary lebhaft
entgegen.

"`Da bin ich noch einmal!"' sagte Leo.

"`Ich hab e{\s} erwartet!"'

Emma bi"s sich auf die Unterlippe. Eine Blutwelle scho"s unter der
Haut ihre{\s} Gesicht{\s} hin und f"arbte e{\s} "uber und "uber
rot, vom Hal{\s}kragen an bi{\s} hinauf zu den Haarwurzeln. Sie
blieb stehen und lehnte die Schulter gegen die Holzt"afelung.

"`Ihr Herr Gemahl ist wohl nicht zu Hause?"'

"`Er ist fort."'

Dann trat Schweigen ein. Sie sahen sich beide an, und ihre
Gedanken, von gleichem Bangen durchwoben, schmiegten sich
aneinander wie zwei klopfende Herzen.

"`Ich m"ochte Berta gern einen Abschied{\s}ku"s geben"', sagte
Leo.

Emma ging hinau{\s}, ein paar Stufen hinunter, und rief Felicie.
Leo warf schnell einen hei"sen Blick auf die W"ande, die M"obel,
den Kamin, al{\s} wollte er alle{\s} umfassen, alle{\s} mit sich
nehmen. Aber da war sie auch schon wieder im Zimmer. Da{\s}
M"adchen brachte die kleine Berta, die einen Hampelmann an einem
Faden in der Hand hielt, verkehrt, den Kopf nach unten.

Leo k"u"ste die Kleine ein paarmal auf die Stirn.

"`Lebwohl, arme{\s} Kind! Lebwohl, liebe{\s} Bertchen! Lebwohl!"'

Er gab da{\s} Kind der Mutter zur"uck.

"`Bring sie weg!"' befahl Emma.

Sie waren wiederum allein.

Frau Bovary wandte Leo den R"ucken zu und pre"ste ihr Gesicht
gegen eine Fensterscheibe. Er hielt seine Reisem"utze in der Hand
und schlug damit leise gegen seinen Schenkel.

"`E{\s} wird wohl regnen"', bemerkte Emma.

"`Ich habe einen Mantel"', antwortete er.

"`So!"'

Sie wandte sich wieder um, da{\s} Kinn gesenkt. Da{\s} Licht glitt
"uber ihre vorgebeugte Stirn wie "uber glatten Marmor bi{\s} hinab
in die Augenbrauen. Man konnte nicht sehen, wa{\s} in ihren Augen
geschrieben stand, noch wa{\s} die Gedanken dahinter sannen.

"`Also adieu!"' seufzte Leo.

Sie hob den Kopf mit einer j"ahen Bewegung.

"`Ja, adieu! Sie m"ussen gehen!"'

Sie kamen aufeinander zu. Er reichte ihr die Hand hin. Sie
z"ogerte.

"`Sozusagen ein franz"osischer Abschied!"' meinte sie, indem sie
ihm die Hand "uberlie"s. Dabei l"achelte sie gezwungen.

Leo f"uhlte ihre Finger in den seinen. E{\s} kam ihm vor, al{\s}
str"ome ihr ganze{\s} Ich in seine Haut. Al{\s} er seine Hand
wieder "offnete, begegneten sich beider Augen noch einmal. Dann
ging er.

Al{\s} er unter den Hallen war, blieb er stehen, wobei er sich
hinter einem Pfeiler verbarg. Er wollte ein letzte{\s} Mal ihr
wei"se{\s} Hau{\s} mit seinen vier gr"unen Fensterl"aden sehen. Da
vermeinte er, ihren Schatten hinter der Gardine ihre{\s}
Zimmer{\s} zu erblicken. Aber der Vorhang hatte sich wohl von
selbst gebauscht und fiel nun wieder langsam in seine langen
senkrechten Falten zur"uck, in denen er dann regung{\s}lo{\s}
stehen blieb wie eine Mauer von Gip{\s}. Leo eilte von dannen.

Von weitem sah er schon den Wagen seine{\s} Chef{\s} auf der
Stra"se halten. Ein Mann in leinenem Kittel stand daneben und
hielt da{\s} Pferd. Der Apotheker und der Notar plauderten
miteinander. Man wartete auf ihn.

"`Lassen Sie sich noch einmal umarmen!"' sagte Homai{\s}, Tr"anen
in den Augen. "`Hier ist Ihr Mantel, mein lieber Freund! Erk"alten
Sie sich unterweg{\s} nicht! Schonen Sie sich recht und nehmen Sie
sich ordentlich in acht!"'

"`Einsteigen, Herr D"upui{\s}!"' mahnte der Notar.

Der Apotheker beugte sich "uber da{\s} Spritzleder und stammelte
mit tr"a\-nen\-er\-stick\-ter Stimme nicht{\s} al{\s} die
beiden wehm"utigen Worte:

"`Gl"uckliche Reise!"'

"`Guten Abend, Herr Apotheker!"' rief Guillaumin. "`Lo{\s}!"'

Die beiden fuhren weg, und Homai{\s} wandte sich heimw"art{\s}.

\begin{center}
\makebox[15em]{\hrulefill}\bigskip
\end{center}

Frau Bovary hatte da{\s} nach dem Garten gehende Fenster ihre{\s}
Zimmer{\s} ge"offnet und betrachtete die Wolken. In der Richtung
nach Rouen, nach Westen zu, standen sie zusammengeballt.
Leichtere{\s} finstere{\s} Gew"olk zog von daher im raschen Fluge
heran, durchleuchtet von schr"agen Sonnenstrahlen, die wie die
goldnen Strahlenb"undel einer aufgeh"angten Troph"ae
hervorschossen. Der "ubrige wolkenlose Teil de{\s}
Himmel{\s}zelte{\s} war wei"s wie Porzellan. Ruckweise Windst"o"se
beugten die H"aupter der Pappeln; pl"otzlich rauschte Regen herab
und prasselte durch da{\s} gr"unschimmernde Laubwerk. Bald kam die
Sonne wieder herau{\s}. Die Hennen gackerten. Die Spatzen
sch"uttelten ihre Fl"ugel auf dem nassen Gezweig, und in den
Wasserrinnen auf dem sandigen Boden schwammmen rote
Akazienbl"uten.

"`Wie weit mag er nun schon sein!"' dachte sie.

Halb sieben, beim Essen, erschien Homai{\s} gewohnterweise.

"`Na,"' sagte er, indem er sich an den Tisch setzte, "`unsern
jungen Freund h"atten wir gl"ucklich verfrachtet!"'

"`Wie man mir berichtet hat"', gab der Arzt zur Antwort. Sich auf
seinem Stuhle nach ihm wendend, fuhr er fort: "`Und wa{\s}
gibt{\s} bei Ihnen Neue{\s}?"'

"`Nicht{\s} weiter. Meine Frau war heute nachmittag nur ein
bi"schen aufgeregt. Sie wissen, die Frauen sind immer gleich
au{\s} dem H"au{\s}chen. Und meine ganz besonder{\s}! Aber man
soll ihnen darau{\s} keinen Vorwurf machen. Ihre Nerven sind eben
zarter besaitet al{\s} unsre."'

"`Der arme Leo,"' bemerkte Karl, "`wie wird{\s} ihm in Pari{\s}
ergehen? Wird er sich dort einleben?"'

Frau Bovary seufzte.

"`Nat"urlich!"' meinte der Apotheker und schnalzte mit der Zunge.
"`Feine Souper{\s}! Ma{\s}kenb"alle! Sekt! Daran gew"ohnt man sich
schon, versichre ich Ihnen."'

"`Ich glaube nicht, da"s er unsolid werden wird"', warf Bovary
ein.

"`Gott bewahre!"' entgegnete Homai{\s} lebhaft. "`Aber mit den
W"olfen wird er halt heulen m"ussen. Sonst wird er al{\s}
Duckm"auser verschrien. Sie haben keine Ahnung, wa{\s} diese
Kerlchen{\s} im Studentenviertel f"ur ein flotte{\s} Leben
f"uhren! Mit ihren kleinen M"adchen! "Ubrigen{\s} sind die
Studenten in Pari{\s} "uberall gern gesehen. Wenn einer nur ein
bi"schen gesellige Talente hat, stehen ihm die allerbesten Kreise
offen. Und e{\s} gibt sogar in der Vorstadt Saint-Germain feine
Damen, die sich Studenten zu Liebsten nehmen, und da{\s} gibt
ihnen dann die beste Gelegenheit, sich reich zu verheiraten."'

"`Da{\s} mag schon sein,"' sagte der Arzt, "`ich habe nur Angst,
er ... wird ... dort~..."'

"`Sehr richtig,"' unterbrach ihn der Apotheker, "`da{\s} ist die
Kehrseite der Medaille! In Pari{\s}, da mu"s man sich fortw"ahrend
die Taschen zuhalten. Zum Beispiel, Sie sitzen in einer
"offentlichen Anlage. Nimmt da jemand neben Ihnen Platz,
anst"andig angezogen, wom"oglich ein Orden{\s}b"andchen im
Knopfloch. Man k"onnte ihn f"ur einen Diplomaten halten. Er
spricht Sie an. Sie kommen in{\s} Plaudern. Er bietet Ihnen eine
Prise an oder hebt Ihnen den Hut auf. So wird man intimer. Er
nimmt Sie mit in{\s} Caf\'e, ladet Sie in sein Landhau{\s} ein,
macht Sie bei einem Gla{\s} Wein mit Tod und Teufel bekannt -- und
da{\s} Ende vom Liede: er pumpt Sie an oder verstrickt Sie in
gef"ahrliche Abenteuer."'

"`So ist e{\s}!"' gab Karl zu. "`Aber ich dachte vor allem an die
Krankheiten, die dem Studenten au{\s} der Provinz in der
Gro"sstadt drohen. Zum Beispiel ... der Typhu{\s}."'

Emma zuckte zusammen.

"`Der kommt von der g"anzlich ver"anderten Leben{\s}weise"', fuhr
der Apotheker fort, "`und der dadurch hervorgebrachten Umw"alzung
de{\s} ganzen Organi{\s}mu{\s}. Und dann denken Sie an da{\s}
Pariser Wasser! An da{\s} Essen in den Restaurant{\s}! Diese
starkgew"urzten Speisen verderben schlie"slich da{\s} Blut. Man
mag sagen, wa{\s} man will, mit einer guten Hau{\s}mann{\s}kost
sind sie nicht zu vergleichen. Ich f"ur meinen Teil, ich sch"atze
von jeher die b"urgerliche K"uche. Die ist am ges"undesten. Al{\s}
ich \begin{antiqua}stud. pharm.\end{antiqua} in Rouen war, da habe
ich de{\s}halb regelm"a"sig in einer Pension gegessen. Die Herren
Professoren a"sen auch da~..."'

In dieser Weise fuhr er fort, sich "uber seine Ansichten im
allgemeinen und seinen pers"onlichen Geschmack im besondern
au{\s}zulassen, bi{\s} Justin kam und ihn zur Bereitung einer
bestellten Arznei holte.

"`Man hat aber auch keinen Augenblick seine Ruhe!"' schimpfte er.
"`Immer liegt man an der Kette! Keine Minute kann man fort. Ein
Arbeitstier bin ich, da{\s} Blut schwitzen mu"s. Da{\s} ist ein
Hundedasein!"'

In der T"ur sagte er noch:

"`"Ubrigen{\s}, wissen Sie schon da{\s} Neueste?"'

"`Wa{\s} denn?"'

Homai{\s} zog die Brauen hoch und machte eine hochwichtige Miene.

"`E{\s} ist sehr wahrscheinlich, da"s die Versammlung der
Landwirte unser{\s} Departement{\s} heuer in Yonville stattfindet.
Man munkelt wenigsten{\s}. In der heutigen Zeitung steht auch
schon eine Andeutung. Da{\s} w"are f"ur die hiesige Gegend von
gro"ser Bedeutung! Aber dar"uber reden wir noch einmal! Danke, ich
sehe schon. Justin hat die Laterne mit~..."'


\newpage\begin{center}
{\large \so{Siebente{\s} Kapitel}}\bigskip\bigskip
\end{center}

Der n"achste Tag war f"ur Emma ein Tag der Betr"ubni{\s}. Alle{\s}
um sie herum erschien ihr wie von lichtlosem Nebel umflort,
verschwommen, zerrissen. Der Schmerz strich durch ihre Seele mit
leisen Klagen wie der Winterwind um ein einsame{\s} Schlo"s. Sie
verfiel in die Tr"aumerei, die den Menschen umspinnt, wenn er
etwa{\s} auf immerdar verloren hat. Sie empfand die M"udigkeit,
die ihn der vollendeten Tatsache gegen"uber "ubermannt, den
Schmerz, der ihn "uberkommt, wenn eine ihm zur Gewohnheit gewordne
Bewegung pl"otzlich stockt, wenn Schwingungen j"ah aufh"oren, die
lange in ihm vibriert haben.

Wie damal{\s} nach der R"uckkehr vom Schlosse Vaubyessard, al{\s}
die wirbelnden Walzermelodien ihr nicht au{\s} dem Sinne wollten,
war sie voll d"usterer Schwermut, in dumpfer Leben{\s}unlust. Leo
stand vor ihrer Phantasie immer gr"o"ser, sch"oner,
verf"uhrerischer. Wie ein Ideal. Wenn er auch fern von ihr war, so
hatte er sie doch nicht verlassen. Er war da, und an den W"anden
ihre{\s} Hause{\s} schien sein Schatten noch zu haften. Immer
wieder schaute sie auf den Teppich, "uber den er so oft gegangen,
auf die leeren St"uhle, wo er gesessen. Drau"sen kroch da{\s}
Fl"u"slein noch immer vorbei mit seinen niedlichen Wellen,
zwischen den schlammigen Ufern hin. An seinem Gestade waren sie so
oft gewandelt, bei dem Rauschen der Fluten um die moosigen Steine.
Wie warm hatte da die Sonne geschienen! Wie traulich waren die
Nachmittage gewesen, wenn sie hinten im schattigen Garten allein
gesessen hatten! Er hatte laut vorgelesen, blo"sen Kopfe{\s}, in
einem Korbstuhl sitzend. Der frische Wind, der dr"uben von den
Wiesen her wehte, hatte die Bl"atter de{\s} Buche{\s} bewegt und
die violetten Bl"uten der Glycinen an der Laube ... Ach, nun war
er fort, die einzige Freude ihre{\s} Dasein{\s}, die einzige
Hoffnung, da"s sich ihr da{\s} ertr"aumte Gl"uck noch erf"ulle!
Warum hatte sie diese{\s} Gl"uck nicht mit beiden H"anden
festgehalten, in den Scho"s genommen, e{\s} nicht in die Ferne
gelassen? Sie verw"unschte sich, Leo{\s} Geliebte nicht geworden
zu sein. Sie d"urstete nach seinen Lippen. Am liebsten w"are sie
ihm nachgelaufen, h"atte sich in seine Arme geworfen und ihm
gesagt: "`Hier bin ich! Nimm mich!"' Aber vor den Hindernissen,
die sich der Verwirklichung diese{\s} Drange{\s} entgegengestellt
h"atten, verzagte Emma von vornherein, und der Schmerz dar"uber
sch"urte ihre Sehnsucht zu noch hei"serer Glut.

Fortan war die Erinnerung an Leo der Kristallisation{\s}punkt
ihrer Bitternisse. Sie flackerte verlockender al{\s} ein
einsame{\s} Lagerfeuer, da{\s} Wanderer in einer sibirischen
Steppe inmitten de{\s} Schnee{\s} angez"undet haben. Zu diesem
Feuer fl"uchtete sie, kauerte sich daneben nieder und fachte e{\s}
sorgf"altig wieder an, wenn e{\s} zu verl"oschen drohte. Im
Umkreise um sich herum suchte sie alle{\s} m"ogliche herbei, um
diese Flammen zu n"ahren. Die fernsten Erinnerungen und die
frischesten Ereignisse, Erlebte{\s} und Ertr"aumte{\s}, die
wuchernden Phantastereien ihrer Sinnlichkeit, ihre Sehnsucht nach
Sonne, geknickt wie trockne{\s} Gezweig im Wind, ihre nutzlose
Tugend, ihre get"auschten Illusionen, die Armseligkeit ihre{\s}
Hau{\s}wesen{\s}, alle{\s} da{\s} sammelte sie, raffte e{\s}
zusammen und warf e{\s} in die Glut, um ihre Tr"ubsal daran zu
w"armen.

Mit der Zeit verglomm da{\s} Feuer aber doch, sei e{\s}, weil ihm
die Nahrung fehlte, sei e{\s}, weil die "Uberf"ulle von Brennstoff
e{\s} erstickte. In der Abwesenheit de{\s} Geliebten verkam
allm"ahlich ihre Liebe. Da{\s} Ineinemfort t"otete den Schmerz,
und am Himmel ihrer Gef"uhle verbla"ste der erst grellrote
Feuerschein und wich nach und nach schwarzem Dunkel. W"ahrend
ihre{\s} phantastischen Zustande{\s} hatte sich ihr Widerwille
gegen den Gatten in Schw"armerei f"ur den Geliebten verwandelt,
und die Glut ihre{\s} Hasse{\s} hatte ihre z"artliche Sehnsucht
gew"armt. Aber nunmehr, da ihre st"urmische unbefriedigte
Leidenschaft zu Asche gebrannt war, da{\s} keine Hilfe kam und
keine neue Sonne aufging, ward tiefe Nacht um sie herum. In
eisiger K"alte stand sie einsam da und erstarrte.

Die schrecklichen Tage von Toste{\s} wiederholten sich nun. Nur
bildete sie sich ein, noch ungl"ucklicher denn damal{\s} zu sein,
weil sie jetzt ein wirkliche{\s} Herzeleid trug und genau wu"ste,
da"s e{\s} nie ander{\s} werden k"onne.

Eine Frau, die so viel geopfert, sei -- so sagte sie sich --
wohlberechtigt, sich ein paar harmlose Liebhabereien zu g"onnen.
Sie schaffte sich einen gotischen Betstuhl an und verbrauchte in
vier Wochen f"ur vierzehn Franken Zitronen zur Pflege ihrer
H"ande. Sie schrieb nach Rouen und bestellte sich ein blaue{\s}
Kaschmirkleid. Bei Lheureux suchte sie sich den sch"onsten Schal
au{\s} und trug ihn "uber ihrem Hau{\s}kleid. Sie schlo"s die
L"aden, nahm ein Buch zur Hand und blieb so stundenlang auf dem
Sofa liegen.

H"aufig "anderte sie ihre Haartracht. Bald trug sie eine hohe
Frisur, bald lose Locken, bald einen Kranz von Z"opfen, bald einen
Scheitel.

Sie geriet auf den Einfall, Italienisch lernen zu wollen, und so
kaufte sie sich ein W"orterbuch, eine Grammatik und eine Menge
Schreibpapier. Dann versuchte sie e{\s} mit ernsthafter Lekt"ure,
la{\s} Geschicht{\s}werke und philosophische Schriften.

Nacht{\s} fuhr Karl mitunter in die H"ohe, im Glauben, man hole
ihn zu einem Kranken. Noch halb im Schlafe rief er:

"`Ich bin gleich fertig!"'

Aber e{\s} war nur da{\s} Knistern de{\s} Streichholze{\s}
gewesen, mit dem sich Emma die Lampe angez"undet hatte. Sie wollte
lesen. Aber e{\s} ging ihr wie mit ihren Stickereien, von denen
ein ganzer Sto"s angefangen im Schranke lag. Sie pflegte sie
anzufangen, dann liegen zu lassen und eine andre zu beginnen.

Sie hatte launenhafte Stimmungen, in denen man sie leicht zu dem
Unglaublichsten verleiten konnte. Einmal behauptete sie ihrem
Manne gegen\-"uber, sie k"onne ein Weingla{\s} voll Schnap{\s} mit
einem Zuge leeren, und da Karl so t"oricht war, e{\s} zu
bezweifeln, tat sie e{\s} wirklich.

Bei allen ihren "`Extravaganzen"' (die Spie"sb"urger von Yonville
nannten da{\s} so!) sah Emma keine{\s}weg{\s}
unternehmung{\s}lustig au{\s}. Im Gegenteil. Um ihre Mundwinkel
lagerten sich jene gewissen starren Falten, die alte Jungfern und
verbissene Streber zu haben pflegen. Sie war v"ollig bla"s, wei"s
wie Leinwand; die Haut ihrer Nase bildete nach den Fl"ugeln zu
F"altchen, und ihre Augen blickten wie in{\s} Leere. Seitdem sie
an den Schl"afen ein paar graue Haare entdeckt hatte, nannte sie
sich gespr"ach{\s}weise eine alte Frau.

Oft hatte sie Schwindelanf"alle, und eine{\s} Tage{\s} spuckte sie
sogar Blut. Aber al{\s} sich Karl eifrig um sie bem"uhte und seine
Besorgni{\s} verriet, meinte sie:

"`La"s mich! E{\s} ist mir alle{\s} gleich!"'

Karl zog sich in sein Sprechzimmer zur"uck. Er sank in seinen
Schreibsessel, st"utzte sich mit den Ellbogen auf den Tisch und
weinte -- unter dem phrenologischen Sch"adel.

Nach einer Weile setzte er einen Brief an seine Mutter auf und bat
sie zu kommen. E{\s} fand zwischen beiden eine lange Konferenz
Emma{\s} wegen statt. Welche Ma"snahmen sollten getroffen werden?
Wa{\s} sollte geschehen? Wo sie jedwede "arztliche Behandlung
ablehnte!

"`Wei"st du, wa{\s} deiner Frau fehlt?"' meinte Frau Bovary
schlie"slich. "`Eine ordentliche Besch"aftigung! K"orperliche
Arbeit! Wenn sie wie so manch andre ihr t"agliche{\s} Brot selber
verdienen m"u"ste, dann h"atte sie keine Nerven und Launen. Die
kommen blo"s von den "uberspannten Ideen, die sie sich au{\s}
purer Langweile in den Kopf setzt."'

"`Besch"aftigung hat sie doch aber!"' erwiderte Karl.

"`So! Sie hat Besch"aftigung? Wa{\s} f"ur welche denn? Romane
schm"okert sie, schlechte B"ucher, Schriften gegen die Religion,
in denen die Geistlichen verh"ohnt werden mit Reden{\s}arten
au{\s} dem Voltaire! Armer Junge, da{\s} f"uhrt zu nicht{\s}
Gutem, und wer kein guter Christ ist, mit dem nimmt e{\s} mal ein
schlechte{\s} Ende!"'

Also ward beschlossen, Emma am Romanlesen zu hindern. Da{\s}
schien nicht so einfach, aber Mutter Bovary nahm die Sache auf
sich. Auf ihrer Heimreise wollte sie in Rouen pers"onlich zum
Leihbibliothekar gehen und Emma{\s} Abonnement abbestellen. Wenn
der Mann trotzdem sein Vergiftung{\s}werk fortsetzte, sollte man
da nicht da{\s} Recht haben, sich an die Polizei zu wenden?

Der Abschied zwischen Schwiegermutter und Schwiegertochter war
steif. In den drei Wochen ihre{\s} Beisammensein{\s} hatten sie,
abgesehen von den h"au{\s}lichen Anordnungen und den h"oflichen
Formeln bei Tisch und abend{\s} vor dem Zubettgehen, keine drei
Worte gewechselt.

Die alte Frau Bovary reiste ab an einem Mittwoch, dem Markttage
von Yonville. Vom fr"uhen Morgen ab war an diesem Tage auf dem
Marktplatz, gleichlaufend mit den H"ausern von der Kirche bi{\s}
zum Goldnen L"owen, eine lange Reihe von Leiterwagen aufgefahren,
Fahrzeug an Fahrzeug, alle mit hochgespie"sten Deichseln. Auf der
andern Seite de{\s} Platze{\s} standen Zeltbuden, in denen
Baumwollenwaren, Decken und Str"umpfe feilgeboten wurden, daneben
Pferdegeschirre und Haufen von bunten B"andern, deren Enden im
Winde flatterten. Zwischen Eierpyramiden und K"asek"orben, au{\s}
denen klebrige{\s} Stroh herau{\s}ragte, lagen allerhand
Eisenwaren auf dem Pflaster au{\s}gebreitet. Neben Ackerger"at
gackerten H"uhner in flachen K"orben und steckten ihre H"alse
durch die Luftl"ocher. Die Menge schob sich, ohne zu weichen,
gerade nach den Stellen, wo da{\s} Gedr"ange schon am dichtesten
war. So geriet bi{\s}weilen da{\s} Schaufenster der Apotheke
wirklich in Gefahr. An den Markttagen ward diese nie leer. E{\s}
standen immer eine Menge Leute darin, weniger um Arzneien zu
kaufen al{\s} vielmehr um den Apotheker zu konsultieren. Herr
Homai{\s} war in den benachbarten Ortschaften ein ber"uhmter Mann.
Seine r"ucksicht{\s}lose Sicherheit fing die Bauern. Sie hielten
ihn f"ur einen besseren Arzt al{\s} alle Doktoren im ganzen Lande.

Emma sa"s an ihrem Fenster, wie so oft. Da{\s} Fenster ersetzt in
der Kleinstadt da{\s} Theater und den Korso. Sie belustigte sich
"uber da{\s} wimmelnde Landvolk; da bemerkte sie einen Herrn in
einem Rock von gr"unem Samt, mit gelben Handschuhen;
sonderbarerweise trug er dazu derbe Gamaschen. Ein Bauer{\s}knecht
mit gesenktem Kopf und recht tr"ubseliger Miene folgte ihm. Beide
gingen auf da{\s} Bovarysche Hau{\s} zu.

"`Ist der Herr Doktor zu sprechen?"' fragte der Herr den
Apothekergehilfen, der an der Hau{\s}t"ure mit Felicie plauderte.
Er hielt ihn f"ur den Diener de{\s} Arzte{\s}. "`Melden Sie Herrn
Rudolf Boulanger von der H"uchette."'

E{\s} war keine{\s}weg{\s} Eitelkeit, da"s der Ank"ommling sein
Gut zu seinem Namen f"ugte. Er wollte nur genau angeben, wer er
war. Die H"uchette war n"amlich ein Rittergut in der N"ahe von
Yonville, da{\s} er samt zwei Meiereien unl"angst gekauft hatte.
Er bewirtschaftete e{\s} selber, jedoch ohne sich allzusehr dabei
anzustrengen. Er war Junggeselle und hatte "`so mindesten{\s}
seine f"unfzehntausend Franken"' im Jahr zu verzehren.

Karl begab sich in sein Sprechzimmer hinunter. Boulanger
"uberwie{\s} ihm seinen Knecht, der einen Aderla"s w"unsche, weil
er am ganzen K"orper ein Kribbeln wie von Ameisen habe.

"`Da{\s} wird mich erleichtern"', wiederholte der Bursche auf alle
Einw"ande. Bovary lie"s sich nunmehr eine Leinwandbinde und eine
Sch"ussel bringen. Er bat Justin, behilflich zu sein.

Dann wandte er sich an den Knecht, der schon ganz bla"s geworden
war.

"`Nur keine Angst, mein Lieber!"'

"`Ach nee, Herr Doktor, machen Sie nur lo{\s}!"' erwiderte er.

Dabei hielt er mit prahlerischer Geb"arde seinen dicken Arm hin.
Unter dem Stich der Lanzette sprang da{\s} Blut hervor und
spritzte bi{\s} zum Spiegel hin.

"`Die Sch"ussel!"' rief Karl.

"`Donnerwetter!"' meinte der Knecht. "`Da{\s} ist ja der reine
Springbrunnen! Und wie rot da{\s} Blut ist! Da{\s} ist ein
gute{\s} Zeichen, nicht wahr?"'

Bei diesen Worten sank der Mann mit einem Ruck in den Sessel
zur"uck, da"s die Lehne krachte.

"`Da{\s} hab ich mir gleich gedacht!"' bemerkte Bovary, indem er
mit den Fingern die angestochne Ader zudr"uckte. "`Erst geht{\s}
ganz gut, dann kommt die Ohnmacht, gerade bei solchen robusten
Kerlen wie dem da!"'

Die Sch"ussel in Justin{\s} H"anden geriet in{\s} Schwanken. Die
Knie schlotterten ihm; er wurde leichenfahl.

"`Emma! Emma!"' rief der Arzt.

Mit einem Satze war sie die Treppe hinunter.

"`Essig!"' rief ihr Karl zu. "`Ach du mein Gott! Gleich zweie auf
einmal!"'

In seiner Aufregung konnte er kaum den Verband anlegen.

"`'{\s} ist weiter nicht{\s}!"' meinte Boulanger gelassen, der
Justin aufgefangen hatte. Er setzte ihn auf die Tischplatte und
lehnte ihn mit dem R"ucken gegen die Wand.

Frau Bovary machte sich daran, dem Ohnm"achtigen da{\s}
Hal{\s}tuch aufzukn"upfen. Der Knoten wollte sich nicht gleich
l"osen, und so ber"uhrte sie ein paar Minuten lang leise mit ihren
Fingern den Hal{\s} de{\s} jungen Burschen. Dann go"s sie Essig
auf ihr Batisttaschentuch, betupfte ihm ein paarmal behutsam die
Schl"afen und blie{\s} dann ein wenig darauf.

Der Knecht war bereit{\s} wieder munter, aber Justin{\s} Ohnmacht
dauerte an. Seine Aug"apfel verschwammen in ihrem bleichen Gallert
wie blaue Blumen in Milch.

"`Er darf da{\s} da nicht sehen!"' ordnete Karl an.

Frau Bovary ergriff die Sch"ussel und setzte sie unter den Tisch.
Bei diesem Sichb"ucken bauschte sich ihr Rock (ein weiter gelber
Rock mit vier Falbeln) um sie herum und stand wie steif auf der
Diele, und je nach der Bewegung Emma{\s}, die sich neigte, die
Arme au{\s}streckte und sich dabei in den H"uften ein wenig hin
und her drehte, wogte der Stoff auf und nieder. Dann nahm sie eine
Wasserflasche und l"oste ein paar St"uck Zucker in einem Glase.

In diesem Augenblicke trat der Apotheker ein. Da{\s} M"adchen
hatte ihn vor Schreck herbeigeholt. Al{\s} er seinen Gehilfen
wieder bei Bewu"stsein sah, atmete er auf. Dann ging er um ihn
herum und betrachtete sich ihn von oben bi{\s} unten.

"`Dummkopf!"' brummte er. "`Ein Dummkopf, wie er im Buche steht!
Al{\s} ob{\s} wer wei"s wa{\s} w"are! Ein bi"schen Aderla"s!
Weiter nicht{\s}! Und da{\s} will ein forscher Kerl sein! Ja, wenn
e{\s} gilt, von den h"ochsten B"aumen die N"usse herunterzuholen,
da klettert er wie ein Eichh"ornchen ... Na, tu deinen Mund auf
und zeig dich mal in deiner Gloria! Da{\s} sind ja nette
Eigenschaften f"ur einen, der mal Apotheker werden will! Ich sage
dir: al{\s} Apotheker kommt man in die schwierigsten Lagen. So zum
Beispiel vor Gericht al{\s} Sachverst"andiger. Da hei"st e{\s}
kaltbl"utig sein, h"ubsch ruhig "uberlegen und ein ganzer Mann
sein! Sonst gilt man al{\s} Schwachmatiku{\s}~..."'

Justin sagte kein Wort. Der Apotheker fuhr fort:

"`Wer hat dir denn "ubrigen{\s} gesagt, da"s du hierher gehen
sollst? In einem fort bel"astigst du Herrn und Frau Doktor! Noch
dazu an den Markttagen, wo du dr"uben so notwendig gebraucht
wirst! E{\s} warten zurzeit zwanzig Kunden im Laden. Deinetwegen
habe ich alle{\s} stehn und liegen lassen. Marsch! Hin"uber! Trab!
Gib auf die Arzneien acht! Ich komme gleich nach!"'

Al{\s} Justin seine Kleidung wieder in Ordnung gebracht hatte und
fort war, plauderte man noch ein wenig "uber Ohnmachtanf"alle.
Frau Bovary sagte, sie h"atte noch nie einen gehabt.

"`Ja, bei Damen kommt so wa{\s} sehr selten vor!"' behauptete
Boulanger. "`E{\s} gibt aber auch Leute, die allzu zimperlich
sind. Da hab ich gelegentlich eine{\s} Duell{\s} erlebt, da"s ein
Zeuge ohnm"achtig wurde, al{\s} die Pistolen beim Laden
knackten."'

"`Wa{\s} mich anbelangt,"' erkl"arte der Apotheker, "`mich st"ort
der Anblick fremden Blute{\s} ganz und gar nicht. Aber der blo"se
Gedanke, ich selber k"onne bluten, der macht mich schwindlig, wenn
ich nicht schnell an wa{\s} andre{\s} denke."'

Inzwischen hatte Boulanger seinen Knecht fortgeschickt, nachdem er
ihn ermahnt, sich nun zu beruhigen.

"`Nun ist{\s} aber alle mit der Einbildung!"' sagte er ihm. "`Die
hat mir die Ehre Ihrer Bekanntschaft verschafft"', f"ugte er
hinzu. Bei dieser Phrase blickte er Emma an. Dann legte er einen
Taler auf die Tischecke, gr"u"ste fl"uchtig und verschwand.

Bald darauf erschien er dr"uben auf dem andern Ufer de{\s}
Bache{\s}. Da{\s} war sein Weg nach der H"uchette. Emma sah ihm
von einem der Hinterfenster nach, wie er "uber die Wiesen ging,
die Pappeln entlang, langsam wie einer, der "uber etwa{\s}
nachdenkt.

"`Allerliebst!"' sagte er bei sich. "`Wirklich allerliebst, diese
Doktor{\s}frau. Sch"one Z"ahne, schwarze Augen, niedliche F"u"se
und schick wie eine Pariserin! Zum Teufel, wo mag sie her sein? Wo
mag sie dieser Schlot nur aufgegabelt haben?"'

Rudolf Boulanger war vierunddrei"sig Jahre alt von roher
Gem"ut{\s}art und scharfem Verstand. Er hatte sich viel mit
Weibern abgegeben und war Kenner auf diesem Gebiete. Die da gefiel
ihm. Somit besch"aftigte sie ihn in Gedanken, ebenso ihr Mann.

"`Ich glaube, er ist mord{\s}bl"ode. Sie hat ihn satt,
zweifelsohne. Er hat dreckige Fingern"agel und rasiert sich nur
aller drei Tage. Wenn er seine Patienten abzurennen hat, sitzt sie
daheim und stopft Str"umpfe. Und langweilt sich. Sehnt sich nach
der gro"sen Stadt und m"ochte am liebsten alle Abende auf den
Ball. Arme kleine Frau! So wa{\s} schnappt nach Liebe wie ein
Karpfen auf dem K"uchentisch nach Wasser! Drei nette Worte, und
sie ist futsch! Sicherlich! Da{\s} w"ar wa{\s} f"ur{\s} Herze!
Scharmant! Aber wie kriegt man sie hinterher wieder lo{\s}?"'

Diese Einschr"ankung de{\s} in der Ferne stehenden Genusse{\s}
erinnerte ihn -- zum Kontrast -- an seine Geliebte, eine
Schauspielerin in Rouen, die er au{\s}hielt. Er vergegenw"artigte
sich ihren K"orper, dessen er sogar in der Vorstellung
"uberdr"ussig war.

"`Ja, diese Frau Bovary,"' dachte er bei sich, "`die ist viel
h"ubscher, vor allem frischer. Virginie wird entschieden zu fett.
Sie zu haben, ist langweilig. Dazu ihre alberne Leidenschaft f"ur
Krebse!"'

Die Fluren waren menschenleer. Rudolf h"orte nicht{\s} al{\s}
da{\s} taktm"a"sige Rascheln der Halme, die er beim Gehen
streifte, und da{\s} ferne Gezirpe der Grillen im Hafer. Er
schaute Emma vor sich, in ihrer Umgebung, angezogen, wie er sie
gesehen hatte. Und in der Phantasie entkleidete er sie.

"`Oh, ich werde sie haben!"' rief er au{\s} und zerschlug mit
einem Schlage seine{\s} Spazierstocke{\s} eine Erdscholle, die im
Wege lag.

Sodann "uberlegte er sich den taktischen Teil der Unternehmung. Er
fragte sich:

"`Wie kann ich mit ihr zusammenkommen? Wie bring ich da{\s}
zustande? Sie wird egal ihr Baby im Arme haben. Und dann da{\s}
Dienstm"adel, die Nachbarn, der Mann und der unvermeidliche
Klatsch! Ach wa{\s}! Unn"utze Zeitvergeudung!"'

Nach einer Weile begann er von neuem:

"`Sie hat Augen, die einem wie Bohrer in da{\s} Herz dringen! Und
wie bla"s sie ist ... Blasse Frauen sind meine Schw"armerei!"'

Auf der H"ohe von Argueil war sein Krieg{\s}plan fertig.

"`Ich brauche blo"s noch g"unstige Gelegenheiten. Gut! Ich werde
ein paarmal gelegentlich mit hingehen, ihnen Wildbret schicken und
Gefl"ugel. N"otigenfall{\s} lasse ich mich ein bi"schen
schr"opfen. Wir m"ussen gute Freunde werden. Dann lade ich die
beiden zu mir ein ... Teufel noch mal, n"achsten{\s} ist doch der
Landwirtschaftliche Tag! Da wird sie hinkommen, da werde ich sie
sehen! Dann hei"st{\s}: Attacke! Und feste drauf! Da{\s} ist immer
da{\s} Beste."'


\newpage\begin{center}
{\large \so{A{ch}te{\s} Kapitel}}\bigskip\bigskip
\end{center}

Endlich war sie da, die ber"uhmte Jahre{\s}versammlung der
Landwirte! Vom fr"uhen Morgen an standen alle Einwohner von
Yonville an ihren Hau{\s}t"uren und sprachen von den Dingen, die
da kommen sollten. Die Stirnseite de{\s} Rathause{\s} war mit
Efeugirlanden geschm"uckt. Dr"uben auf einer Wiese war ein
gro"se{\s} Zelt f"ur da{\s} Festmahl aufgeschlagen worden, und
mitten auf dem Markte vor der Kirche stand ein B"oller, der die
Ankunft de{\s} Landrat{\s} und die Prei{\s}kr"onung donnernd
verk"unden sollte. Die B"urgergarde von B"uchy -- in Yonville gab
e{\s} keine -- war anmarschiert und hatte sich mit der heimischen
Feuerwehr, deren Hauptmann Herr Binet war, zu einem Korp{\s}
vereinigt. Selbiger trug an diesem Tage einen noch h"oheren Kragen
al{\s} gew"ohnlich. In die Litewka eingezw"angt, war sein
Oberk"orper so steif und starr, da"s e{\s} au{\s}sah, al{\s} sei
alle{\s} Leben in ihm in seine beiden Beine gerutscht, die sich
parademarschm"a"sig bewegten. Da der Oberst der B"urgergarde und
der Hauptmann der Feuerwehr eifers"uchtig aufeinander waren,
wollte jeder den andern au{\s}stechen, und so exerzierten beide
ihre Mannschaft f"ur sich. Abwechselnd sah man die roten
Epauletten und die schwarzen Schutzleder vorbeimarschieren und
wieder abschwenken. Da{\s} ging immer wieder von neuem an und nahm
schier kein Ende!

Noch nie hatte man in Yonville derartige Pracht und Herrlichkeit
gesehen. Verschiedene B"urger hatten tag{\s} zuvor ihre H"auser
abwaschen lassen. Wei"s-rot-blaue Fahnen hingen au{\s} den
halboffnen Fenstern herab, alle Kneipen waren voll; und da
sch"one{\s} Wetter war, sahen die gest"arkten H"aubchen wei"ser
wie Schnee au{\s}, die Orden und Medaillen blitzten in der Sonne
wie eitel Gold, und die bunten T"ucher leuchteten buntscheckig
au{\s} dem tristen Einerlei der schwarzen R"ocke und blauen Blusen
hervor. Die P"achter{\s}frauen kamen au{\s} den umliegenden
D"orfern geritten; beim Absitzen zogen sie die langen Nadeln
herau{\s}, mit denen sie ihre R"ocke hochgesteckt hatten, damit
sie unterweg{\s} nicht schmutzig werden sollten. Die M"anner
andrerseit{\s} hatten zum Schutze ihrer H"ute die Sackt"ucher
dar"uber gezogen, deren Zipfel sie mit den Z"ahnen festhielten.

Die Menge str"omte von beiden Enden de{\s} Ort{\s} auf der
Landstra"se heran und ergo"s sich in alle Gassen, Alleen und
H"auser. "Uberall klingelten die T"uren, um die B"urgerinnen
herau{\s}zulassen, die in Zwirnhandschuhen nach dem Festplatze
wallten.

Zwei mit Lampion{\s} beh"angte hohe Taxu{\s}b"aume, zu beiden
Seiten der vor dem Rathause errichteten Estrade f"ur die
Ehreng"aste, erregten ganz besonder{\s} die allgemeine
Bewunderung. "Ubrigen{\s} hatte man an den vier S"aulen am
Rathause so etwa{\s} wie vier Stangen aufgepflanzt; jede trug eine
Art Standarte au{\s} gr"uner Leinwand. Auf der einen la{\s} man:
\begin{antiqua}HANDEL,\end{antiqua} auf der zweiten: \begin{antiqua}ACKERBAU,\end{antiqua} der dritten: \begin{antiqua}INDUSTRIE,\end{antiqua} der
vierten: \begin{antiqua}KUNST UND WISSENSCHAFT.\end{antiqua}

Die Freudensonne, die auf allen Gesichtern zu leuchten begann,
warf auch ihren Schatten und zwar auf da{\s} Antlitz der Frau
Franz, der L"owenwirtin. Auf der kleinen Vortreppe ihre{\s}
Gasthofe{\s} stehend, r"asonierte sie vor sich hin:

"`So eine Torheit! So eine Eselei, eine Leinwandbude aufzubaun!
Glaubt diese Bagage wirklich, da"s der Herr Landrat besonder{\s}
erg"otzt sein wird, wenn er unter einem Zeltdache dinieren soll,
wie ein Seilt"anzer? Dabei soll der ganze Rummel der hiesigen
Gegend zugute kommen! War e{\s} wirklich der M"uhe wert, extra
einen Koch au{\s} Neufch\^atel herkommen zu lassen? F"ur wen
"ubrigen{\s}? F"ur Kuhjungen und Lumpenpack!"'

Der Apotheker ging vor"uber in schwarzem Rock, gelben Buxen,
Lackschuhen und -- au{\s}nahm{\s}weise (statt de{\s} gewohnten
K"appchen{\s}) -- einem Hut von niedriger Form.

"`Ihr Diener!"' sagte er. "`Ich hab{\s} eilig!"'

Al{\s} die dicke Witwe ihn fragte, wohin er ginge, erwiderte er:

"`E{\s} kommt Ihnen komisch vor, nicht wahr? Ich, der ich sonst
den ganzen Tag in meinem Laboratorium stecke wie eine Made im
K"ase~..."'

"`In wa{\s} f"ur K"ase?"' unterbrach ihn die Wirtin.

"`Nein, nein. Da{\s} ist nur bildlich gemeint"', entgegnete
Homai{\s}. "`Ich wollte damit nur sagen, Frau Franz, da"s e{\s} im
allgemeinen meine Gewohnheit ist, zu Hause zu hocken. Heute
freilich mu"s ich in Anbetracht~..."'

"`Ah! Sie gehen auch hin?"' fragte sie in geringsch"atzigem Tone.

"`Gewi"s gehe ich hin!"' sagte der Apotheker erstaunt. "`Ich
geh"ore ja zu den Prei{\s}richtern!"'

Die L"owenwirtin sah ihn ein paar Sekunden an, schlie"slich meinte
sie l"achelnd:

"`Da{\s} ist wa{\s} ander{\s}! Aber wa{\s} geht Sie eigentlich die
Landwirtschaft an? Verstehen Sie denn wa{\s} davon?"'

"`Selbstverst"andlich verstehe ich etwa{\s} davon! Ich bin doch
Pharmazeut, also Chemiker. Und die Chemie, Frau Franz,
besch"aftigt sich mit den Wechselwirkungen und den
Molekularverh"altnissen aller K"orper, die in der Natur vorkommen.
Folglich geh"ort auch die Landwirtschaft in da{\s} Gebiet meiner
Wissenschaft. In der Tat, die Zusammensetzung der D"ungemittel,
die G"arungen der S"afte, die Analyse der Gase und die Wirkung der
Mia{\s}men --, ich bitte Sie, wa{\s} ist da{\s} weiter al{\s} pure
bare Chemie?"'

Die L"owenwirtin erwiderte nicht{\s}, und Homai{\s} fuhr fort:

"`Glauben Sie denn: um Agronom zu sein, m"usse man selber in der
Erde gebuddelt oder G"anse genudelt haben? Keine Spur! Aber die
Beschaffenheit der Substanzen, mit denen der Landwirt zu tun hat,
die mu"s man unbedingt studiert haben, die geologischen
Gruppierungen, die atmo{\s}ph"arischen Vorkommnisse, die
Beschaffenheit de{\s} Erdboden{\s}, de{\s} Gestein{\s}, de{\s}
Wasser{\s}, die Dichtigkeit der verschiedenen K"orper und ihre
Kapillarit"at! Und tausend andre Dinge! Dazu mu"s man mit den
Grunds"atzen der Hygiene v"ollig vertraut sein, um den Bau von
Geb"auden, die Unterhaltung der Hau{\s}- und Arbeit{\s}tiere und
die Ern"ahrung der Dienstboten leiten und kontrollieren zu
k"onnen. Fernerhin, Frau Franz, mu"s man die Botanik intu{\s}
haben. Man mu"s die Pflanzen unterscheiden k"onnen, verstehen Sie,
die n"utzlichen von den sch"adlichen, die nutzlosen und die
nahrhaften, welche Arten man vertilgen und welche man pflegen,
welche man hier wegnehmen und dort anpflanzen mu"s. Kurz und gut,
man mu"s sich in der Wissenschaft auf dem Laufenden halten, indem
man die Brosch"uren und die "offentlichen Bekanntmachungen liest,
und immer auf dem Damme sein, um mit dem Fortschritte zu
gehen~..."'

Die Wirtin lie"s unterdessen den Eingang de{\s} Caf\'e
Fran\c{c}ai{\s} nicht au{\s} den Augen. Der Apotheker redete
weiter:

"`Wollte Gott, unsre Agrarier w"aren zugleich Chemiker, oder sie
h"orten wenigsten{\s} besser auf die Ratschl"age der Wissenschaft!
Da habe ich k"urzlich selbst eine gro"se Abhandlung verfa"st, eine
Denkschrift von mehr al{\s} 72 Seiten, betitelt: "`Der Apfelwein.
Seine Herstellung und seine Wirkung. Nebst einigen neuen
Betrachtungen hier"uber."' Ich habe sie der "`Rouener
Agronomischen Gesellschaft"' "ubersandt, die mich daraufhin unter
ihre Ehrenmitglieder (Sektion Landwirtschaft, Abteilung f"ur
Pomologie) aufgenommen hat. Ja, wenn so ein Werk gedruckt
erschiene~..."'

Der Apotheker hielt ein. Er merkte, da"s Frau Franz von etwa{\s}
ganz andrem in Anspruch genommen war.

"`Sehr richtig!"' unterbrach er sich selber. "`Eine unglaubliche
Spelunke!"'

Die L"owenwirtin zuckte so heftig die Achseln, da"s sich die
Maschen ihrer Trikottaille weit au{\s}\-ein\-an\-der\-zogen. Mit
beiden H"anden deutete sie auf da{\s} Konkurrenzlokal, au{\s} dem
w"uster Gesang her"uberhallte.

"`Na! Lange wird die Herrlichkeit da dr"uben nicht mehr dauern!"'
bemerkte sie. "`In acht Tagen ist der Rummel alle!"'

Homai{\s} trat erschrocken einen Schritt zur"uck. Die Wirtin kam
die drei Stufen herunter und fl"usterte ihm in{\s} Ohr:

"`Wa{\s}? Da{\s} wissen Sie nicht? Noch in dieser Woche wird er
au{\s}gepf"andet und festgesetzt. Lheureux hat ihm den Hal{\s}
abgeschnitten. Mit Wechseln!"'

"`Eine f"urchterliche Katastrophe!"' rief der Apotheker au{\s},
der f"ur alle m"oglichen Ereignisse immer da{\s} passende
Begleitwort zur Hand hatte.

Die L"owenwirtin begann ihm nun die ganze Geschichte zu erz"ahlen.
Sie wu"ste sie von Theodor, dem Diener de{\s} Notar{\s}. Obgleich
sie Tellier, den Besitzer de{\s} Caf\'e Fran\c{c}ai{\s}, nicht
au{\s}stehen konnte, mi"sbilligte sie doch da{\s} Vorgehen von
Lheureux. Sie nannte ihn einen Gauner, einen Hal{\s}abschneider.

"`Da! Sehen Sie!"' f"ugte sie hinzu. "`Da geht er! Unter den
Hallen! Jetzt begr"u"st er Frau Bovary. Sie hat einen gr"unen Hut
auf und geht am Arm von Herrn Boulanger."'

"`Frau Bovary!"' echote Homai{\s}. "`Ich mu"s ihr schnell guten
Tag sagen. Vielleicht ist ihr ein reservierter Platz auf der
Trib"une vor dem Rathause erw"unscht."'

Ohne auf die L"owenwirtin zu h"oren, die ihm ihre lange Geschichte
weitererz"ahlen wollte, stolzierte der Apotheker davon. Mit
l"achelnder Miene gr"u"ste er nach link{\s} und recht{\s}, wobei
ihn die langen Sch"o"se seine{\s} schwarzen Rocke{\s} im Winde
umflatterten, da"s er wer wei"s wieviel Raum einnahm.

Rudolf hatte ihn l"angst bemerkt. Er beschleunigte seine Schritte.

Da aber Emma au"ser Atem kam, ging er wieder langsamer. Lachend
und in brutalem Tone sagte er zu ihr:

"`Ich wollte nur dem Dicken entgehen, wissen Sie, dem Apotheker!"'

Sie versetzte ihm ein{\s} mit dem Ellbogen.

"`Wa{\s} soll da{\s} hei"sen?"' fragte er sie. Dabei blinzelte er
sie im Weitergehen von der Seite an.

Ihr Gesicht blieb unbeweglich; nicht{\s} darin verriet ihre
Gedanken. Die Linie ihre{\s} Profil{\s} schnitt sich scharf in die
lichte Luft, unter der Rundung ihre{\s} Kapotthute{\s}, dessen
bla"sfarbene Bindeb"ander wie Schilfbl"atter au{\s}sahen. Ihre
Augen blickten geradeau{\s} unter ihren etwa{\s} nach oben
gebogenen langen Wimpern. Obgleich sie v"ollig ge"offnet waren,
erschienen sie doch ein wenig zugedr"uckt durch den oberen Teil
der Wangen, weil da{\s} Blut die feine Haut straffte. Durch die
Nasenwand schimmerte Rosenrot, und zwischen den Lippen gl"anzte
da{\s} Perlmutter ihrer spitzen Z"ahne. Den Kopf neigte sie zur
einen Schulter.

"`Mokiert sie sich "uber mich?"' fragte sich Rudolf.

In Wirklichkeit hatte der Ruck, den ihm Emma versetzt hatte, nur
ein Zeichen sein sollen, da"s Lheureux neben ihnen herlief. Von
Zeit zu Zeit redete der H"andler die beiden an, um mit ihnen
in{\s} Gespr"ach zu kommen.

"`Ein herrlicher Tag heute! -- Alle Welt ist auf den Beinen! --
Wir haben Ostwind!"'

Frau Bovary wie Rudolf gaben kaum eine Antwort, w"ahrend Lheureux
bei der geringsten Bewegung, die ein{\s} der beiden machte, mit
einem ewigen "`Wie meinen?"' dazwischenfuhr, wobei er jede{\s}mal
den Hut l"uftete.

Vor der Schmiede bog Rudolf mit einem Male von der Hauptstra"se ab
in einen Fu"sweg ein. Er zog Frau Bovary mit sich und rief laut:

"`Leben Sie wohl, Herr Lheureux! Viel Vergn"ugen!"'

"`Den haben Sie aber fein abgesch"uttelt!"' lachte Emma.

"`Warum sollen wir un{\s} von fremden Leuten bel"astigen lassen?"'
meinte Rudolf. "`Noch dazu heute, wo ich da{\s} Gl"uck habe, mit
Ihnen~..."'

Sie wurde rot. Er vollendete seine Phrase nicht und sprach vom
sch"onen Wetter und wie h"ubsch e{\s} sei, so durch die Fluren
spazieren zu gehen.

Ein paar G"ansebl"umchen standen am Raine.

"`Die niedlichen Dinger da!"' sagte er. "`Und so viele! Genug
Orakel f"ur die verliebten M"adel{\s} de{\s} ganzen Lande{\s}!"'
Ein paar Augenblicke sp"ater setzte er hinzu: "`Soll ich welche
pfl"ucken? Wa{\s} denken Sie dar"uber?"'

"`Sind Sie denn verliebt?"' fragte Emma und hustete ein wenig.

"`Wer wei"s?"' meinte Rudolf.

Sie kamen auf die Festwiese, auf der da{\s} Gedr"ange immer mehr
zunahm. Bauer{\s}frauen mit Riesenregenschirmen, einen Korb am
einen und einen S"augling im andern Arme, rempelten sie an.
H"aufig mu"sten sie Platz machen, wenn eine lange Reihe nach Milch
riechender Dorfsch"onen in blauen Str"umpfen, derben Schuhen und
silbernen Ohrringen vorbeizog, alle Hand an Hand.

Die Prei{\s}verteilung fand statt. Die Z"uchter traten, einer nach
dem andern, in eine Art Arena, die durch ein lange{\s} Seil an
Pf"ahlen gebildet wurde. Innerhalb de{\s} so abgegrenzten
Raume{\s} standen die Tiere, mit den Schnauzen nach au"sen, die
ungleich hohen Kruppen in einer unordentlichen Richtung{\s}linie.
Schl"afrige Schweine w"uhlten mit ihren R"usseln in der Erde.
K"alber br"ullten, Schafe bl"okten. K"uhe lagen hingestreckt, die
B"auche im Grase, die Beine eingezogen, kauten gem"achlich wieder
und zuckten mit ihren schwerf"alligen Lidern, wenn die sie
umschw"armenden Bremsen stachen. Pferdeknechte, die Arme
entbl"o"st, hielten an Trensenz"ugeln steigende Zuchthengste, die
mit gebl"ahten N"ustern nach der Seite hin wieherten, wo die
Stuten standen. Diese verhielten sich friedlich und lie"sen die
K"opfe und M"ahnen h"angen, w"ahrend ihre F"ullen in ihrem
Schatten ruhten und ab und zu an ihnen saugten. "Uber der wogenden
Masse aller dieser Leiber sah man von weitem hie und da da{\s}
Wei"s einer M"ahne wie eine Springflut im Winde aufwehen oder ein
spitze{\s} Horn hervorspringen, und "uberall dazwischen die
H"aupter wimmelnder Menschen. Au"serhalb der Umseilung, etwa
hundert Schritte davon entfernt, stand -- unbeweglich wie au{\s}
Bronze gegossen -- ein gro"ser schwarzer Stier mit verbundenen
Augen und einem Eisenring durch die Nase. Ein zerlumpte{\s} Kind
hielt ihn an einem Stricke.

Ein paar Herren schritten langsam zwischen den beiden Reihen hin,
besichtigten jede{\s} Tier einzeln und eingehend und berieten sich
jede{\s}mal hinterher in fl"usternder Weise. Einer von ihnen,
offenbar der Einflu"sreichste, schrieb im Gehen Bemerkungen in ein
Buch. Da{\s} war der Vorsitzende der Prei{\s}richter, Herr
Derozeray{\s}, Besitzer de{\s} Rittergute{\s} La Panville. Al{\s}
er Rudolf bemerkte, ging er lebhaft auf ihn zu und sagte
verbindlich-freundlich zu ihm:

"`Herr Boulanger, Sie lassen un{\s} ja im Stich?"'

Rudolf versicherte, er werde gleich zur Stelle sein. Al{\s} er
jedoch au"ser H"orweite de{\s} Vorsitzenden war, meinte er:

"`Der Fuch{\s} soll mich holen, wenn ich hinginge! Ich bleibe
lieber bei Ihnen!"'

Er machte seine Witze "uber da{\s} Prei{\s}richterkollegium,
wa{\s} ihn aber nicht abhielt, seinen eignen Au{\s}wei{\s} al{\s}
Mitglied de{\s} Festau{\s}schusse{\s} mit Grandezza zu zeigen,
wenn er irgendwo durchwollte, wo ein Schutzmann stand. Mehrfach
blieb er auch vor dem oder jenem "`Prachtst"uck"' stehen. Frau
Bovary bewunderte nicht{\s} mit. Da{\s} beobachtete er, und nun
begann er sp"ottische Bemerkungen "uber die Toiletten der Damen
von Yonville lo{\s}zulassen. Dabei entschuldigte er sich, da"s er
selber auch nicht elegant gehe. Seine Kleidung war ein
Nebeneinander von Allt"aglichkeit und Au{\s}gesuchtheit. Der
oberfl"achliche Menschenkenner h"alt derlei meist f"ur da{\s}
"au"sere Kennzeichen einer exzentrischen Natur, die bizarr in
ihrem Gef"uhl{\s}leben, k"unstlerisch beanlagt und allem
Herk"ommlichen abhold ist, und empfindet "Argerni{\s} oder
Bewunderung davor. Rudolf{\s} wei"se{\s} Batisthemd mit
gef"alteten Manschetten bauschte sich im Au{\s}schnitt seiner
grauen Flanellweste, wie e{\s} dem Winde gerade gefiel; seine
breitgestreiften Hosen reichten nur bi{\s} an die Kn"ochel und
lie"sen die gelben Halbschuhe ganz frei, auf deren spiegelblanke
Lackspitzen da{\s} Gra{\s} Reflexe warf. Er trat unbek"ummert in
die Pferde"apfel. Eine Hand hatte er in der Rocktasche, und der
Hut sa"s ihm schief auf dem Kopfe.

"`Ein Bauer wie ich~..."', meinte er.

"`Bei dem ist Hopfen und Malz verloren"', scherzte Emma.

"`Sehr richtig! "Ubrigen{\s} ist kein einziger von all diesen
Biederm"annern imstande, den Schnitt eine{\s} Rocke{\s} zu
beurteilen."'

Dann sprachen sie von dem Leben in der Provinz, wo die Eigenart
de{\s} einzelnen erstickt und da{\s} Leben keinen Schwung hat.

"`Darum verfalle ich der Melancholie~..."', sagte er.

"`Sie?"' erwiderte Emma erstaunt. "`Ich halte Sie gerade f"ur sehr
leben{\s}lustig."'

"`Ach, da{\s} sieht nur so au{\s}! Weil ich vor den Leuten die
Ma{\s}ke de{\s} Sp"otter{\s} trage. Aber wie oft habe ich mich
beim Anblick eine{\s} Friedhofe{\s} im Mondenscheine gefragt, ob
einem nicht am wohlsten w"are, wenn man schliefe, wo die Toten
schlafen~..."'

"`Sie haben doch Freunde. Vergessen Sie die nicht!"'

"`Ich? Freunde? Welche denn? Ich habe keine. Um mich k"ummert sich
niemand."'

Dabei gab er einen pfeifenden Ton von sich.

Sie mu"sten sich einen Augenblick voneinander trennen, weil sich
ein Mann zwischen sie dr"angte, der einen Turm von St"uhlen
schleppte. Er war derartig "uberladen, da"s man nicht{\s} von ihm
sah al{\s} seine Holzpantoffeln und seine Ellbogen. E{\s} war
Lestiboudoi{\s}, der Totengr"aber, der ein Dutzend Kirchenst"uhle
herbeischaffte. Findig, wie er immer war, wo e{\s} etwa{\s} zu
verdienen gab, war er auf den Einfall gekommen, au{\s} dem
Bunde{\s}tage seinen Vorteil zu schlagen. Und damit hatte er sich
nicht verrechnet; er wu"ste gar nicht, wen er zuerst befriedigen
sollte. Die Bauern, denen e{\s} hei"s war, rissen sich f"ormlich
um diese St"uhle, deren Strohsitze nach Weihrauch dufteten. Sie
lehnten sich mit wahrer Kirchenstimmung gegen die hohen
wach{\s}beklecksten Stuhlr"ucken.

Frau Bovary nahm Rudolf{\s} Arm von neuem. Er fuhr fort, al{\s}
spr"ache er mit sich selbst.

"`Ja, ja! Ich habe viele{\s} entbehren m"ussen! Immer einsam! Ach,
wenn mein Dasein einen Zweck gehabt h"atte, wenn ich einer gro"sen
Leidenschaft begegnet w"are, wenn ich ein Herz gefunden h"atte ...
Oh, alle meine Leben{\s}kraft h"atte ich daran gesetzt, ich w"are
"uber alle Hindernisse hinweggest"urmt, h"atte alle{\s}
"uberwunden~..."'

"`Mich d"unkt, Sie seien gar nicht besonder{\s} beklagen{\s}wert"',
wandte Emma ein.

"`So, finden Sie?"'

"`Zum mindesten sind Sie frei~..."' Sie z"ogerte. "`... und
reich!"'

"`Spotten Sie doch nicht "uber mich!"' bat er.

Sie beteuerte, e{\s} sei ihr Ernst. Da donnerte ein B"ollerschu"s.
Al{\s}bald w"alzte und dr"angte sich alle{\s} der Ortschaft zu.
Aber e{\s} war ein falscher Alarm gewesen. Der Landrat war noch
gar nicht da. Der Festau{\s}schu"s war nun in der gr"o"sten
Verlegenheit. Sollte der feierliche Akt beginnen, oder sollte man
noch warten?

Endlich tauchte an der Ecke de{\s} Markte{\s} eine riesige
Mietkutsche auf, von zwei mageren G"aulen gezogen, auf die ein
Kutscher im Zylinderhut au{\s} Leibe{\s}kr"aften mit der Peitsche
lo{\s}hieb.

Binet, der Feuerwehrhauptmann, kommandierte in aller Hast:

"`An die Gewehre!"'

Und der Oberst der B"urgergarde br"ullte da{\s} Echo dazu.

Hal{\s} "uber Kopf st"urzte man an die Gewehrpyramiden. Etliche
der B"urgergardisten verga"sen in der Eile, sich den Kragen
zuzukn"opfen. Aber der Landauer de{\s} Herrn Landrat{\s} schien
die Verwirrung zum Gl"uck zu ahnen. Die beiden Pferde kamen im
langsamsten Zotteltrabe gerade in dem Moment vor der Vorhalle
de{\s} Rathause{\s} an, al{\s} sich Feuerwehr und B"urgergarde in
Reih und Glied unter Trommelschlag davor aufgestellt hatten.

"`Stillgestanden! Pr"asentiert da{\s} Gewehr!"' kommandierte
Binet.

"`Stillgestanden! Pr"asentiert da{\s} Gewehr!"' der Oberst auf der
andern Seite.

Die Trageringe rasselten in den Reihen, al{\s} ob ein Kupferkessel
eine Treppe hinunterkollerte. Die Gewehre flogen nur so.

Nun sah man einen Herrn au{\s} der Karosse steigen, in einer
silberbestickten Hofuniform. Er hatte eine gro"se Glatze, ein
Toupet auf dem Hinterhaupte, sah bla"s im Gesicht au{\s} und war
offenbar sehr leutselig. Um die Menschenmenge besser zu sehen,
kniff er seine Augen, die zwischen dicken Lidern hervorquollen,
halb zusammen, wobei er gleichzeitig seine spitzige Nase hob und
seinen eingefallenen Mund zum L"acheln verschob. Er erkannte den
B"urgermeister an seiner Sch"arpe und teilte ihm mit, da"s der
Landrat verhindert sei, pers"onlich zu kommen. Er selber sei
Regierung{\s}rat. E{\s} folgten noch ein paar verbindliche
Reden{\s}arten.

T"uvache, der B"urgermeister, begr"u"ste ihn ehrerbietig. Der Rat
erkl"arte, er f"uhle sich besch"amt. Die beiden standen sich dicht
gegen"uber, Angesicht zu Angesicht; um sie herum der
Festau{\s}schu"s, der Gemeinderat, die Honoratioren, die
B"urgergarde und da{\s} Publikum. Der Regierung{\s}rat schwenkte
seinen kleinen schwarzen Dreimaster gegen die Brust und sagte ein
paar Begr"u"sung{\s}worte. W"ahrenddem klappte T"uvache in einem
fort wie ein Taschenmesser zusammen, l"achelnd, stotternd, nach
Worten suchend. Darauf beteuerte er die K"onig{\s}treue der
Yonviller und dankte f"ur die ihnen widerfahrene gro"se Ehre.

Hippolyt, der Hau{\s}knecht au{\s} dem Goldnen L"owen, nahm die
Pferde der Kutsche an den Kandaren und zog da{\s} Gef"ahrt
humpelnd nach dem Gasthofe, an dessen Hoftor ein Schwarm von
gaffenden Landleuten stand. Die Trommeln wirbelten, der B"oller
krachte.

Die Herren vom Festau{\s}schu"s begaben sich nun auf die vor dem
Rathause errichtete Estrade und setzten sich in die roten
Pl"uschsessel, die von der Frau B"urgermeisterin zur Verf"ugung
gestellt worden waren.

Alle die M"anner glichen einander. Alle hatten sie
au{\s}druck{\s}lose blonde, apfelweinfarbene Gesichter, die von
der Sonne etwa{\s} gebr"aunt waren, buschige Backenb"arte, die
sich unter hohen steifen Hal{\s}kragen verloren, und wei"se,
sorglich gebundene Krawatten. Die Samtweste fehlte keinem,
ebensowenig an den Uhrketten da{\s} ovale Petschaft au{\s}
Karneol. Alle stemmten sie die Arme auf die Schenkel, nachdem sie
die Falten de{\s} Beinkleide{\s} sorgsam zurechtgestrichen hatten.
Da{\s} nicht dekatierte Hosentuch gl"anzte mehr al{\s} da{\s}
Leder ihrer derben Stiefel.

Die Damen der Gesellschaft hielten sich hinter der Estrade auf,
unter der Vorhalle zwischen den S"aulen, w"ahrend die gro"se Menge
dem Rathause gegen"uber stand oder teilweise auf St"uhlen sa"s.
Der Kirchendiener hatte die erst nach der Wiese getragenen St"uhle
rasch wieder hierhergeschleppt und brachte immer noch mehr au{\s}
der Kirche herzu. Durch seinen Handel entstand ein derartige{\s}
Gedr"ange, da"s man nur mit M"uhe und Not zu der kleinen Treppe
der Estrade dringen konnte.

"`Ich finde,"' sagte Lheureux zu dem Apotheker, der sich nach der
Estrade durchdr"angelte und gerade an ihm vor"uberkam, "`man
h"atte zwei venezianische Maste aufpflanzen und sie mit
irgendeinem schweren kostbaren Stoff drapieren sollen, mit einer
Nouveaut\'e. Da{\s} w"urde sehr h"ubsch au{\s}gesehen haben!"'

"`Gewi"s!"' meinte Homai{\s}. "`Aber Sie wissen ja! Der
B"urgermeister macht alle{\s} blo"s nach seinem eignen Kopfe. Er
hat nicht viel Geschmack, der gute T"uvache, und k"unstlerischen
Sinn nun gleich gar nicht!"'

Mittlerweile waren Rudolf und Emma in den ersten Stock de{\s}
Rathause{\s} gestiegen, in den Sitzung{\s}saal. Da dieser leer
war, erkl"arte Boulanger, da{\s} w"are so recht der Ort, da{\s}
Schauspiel bequem zu genie"sen. Er nahm zwei St"uhle von dem
ovalen Tisch, der unter der B"uste von Majest"at stand, und trug
sie an ein{\s} der Fenster.

Die beiden setzten sich nebeneinander hin.

Unten auf der Estrade ging e{\s} lebhaft her. Alle{\s} plauderte
und tuschelte. Da erhob sich der Regierung{\s}rat von seinem
Sitze. Man hatte inzwischen erfahren, da"s er Lieuvain hie"s, und
nun lief sein Name von Mund zu Mund durch die Menge. Nachdem er
ein paar Zettel geordnet und sich dicht vor die Augen gehalten
hatte, begann er:

"`Meine Herren!

Ehe ich auf den eigentlichen Zweck der heutigen Versammlung
eingehe, sei e{\s} mir zun"achst gestattet, -- und ich bin
"uberzeugt, Sie sind in{\s}gesamt damit einverstanden! -- sei
e{\s} mir gestattet, sage ich, der Beh"orden und der Regierung zu
gedenken, vor allem, meine Herren, Seiner Majest"at, unser{\s}
allergn"adigsten und allverehrten Lande{\s}herrn, dem jede{\s}
Gebiet der "offentlichen und privaten Wohlfahrt am Herzen liegt,
der mit sicherer und kluger Hand da{\s} Staat{\s}schiff durch die
unaufh"orlichen Gefahren eine{\s} st"urmischen Ozean{\s} lenkt und
dabei jedem sein Recht l"a"st, dem Frieden wie dem Kriege, der
Industrie, dem Handel, der Landwirtschaft, den K"unsten und
Wissenschaften~..."'

"`Vielleicht setze ich mich ein wenig weiter zur"uck"', sagte
Rudolf.

"`Warum?"' fragte Emma.

In diesem Augenblicke bekam die Stimme de{\s}
Regierung{\s}rate{\s} besonderen Schwung. Er deklamierte:

"`Die Zeiten sind vor"uber, meine Herren, wo die Zwietracht der
B"urger unsre "offentlichen Pl"atze mit Blut besudelte, wo der
Grundbesitzer, der Kaufmann, ja selbst der Arbeiter, wenn er
abend{\s} friedlich schlafen ging, bef"urchten mu"ste, durch
da{\s} St"urmen der Brandglocken j"ah wieder aufgeschreckt zu
werden, wo Umsturzideen frech an den Grundfesten r"uttelten~..."'

"`Nur weil man mich von unten bemerken k"onnte"', gab Rudolf zur
Antwort. "`Dann m"u"ste ich mich vierzehn Tage lang entschuldigen.
Und bei meinem schlechten Rufe~..."'

"`Sie verleumden sich"', warf Emma ein.

"`I wo! Der ist unter aller Kritik! Da{\s} schw"or ich Ihnen."'

"`Meine Herren!"' fuhr der Redner fort. "`Wenn wir unsre Blicke
von diesen d"ustern Bildern der Vergangenheit abwenden und auf den
gegenw"artigen Zustand unser{\s} sch"onen Vaterlande{\s} richten:
wa{\s} sehen wir da? "Uberall stehen Handel, Wissenschaften und
K"unste in Bl"ute, "uberall erwachsen neue Verkehr{\s}wege und
-mittel, gleichsam wie neue Adern im Leibe de{\s} Staate{\s}, und
schaffen neue Beziehungen, neue{\s} Leben. Unsre gro"sen
Industriezentren sind von neuem in vollster T"atigkeit. Die
Religion ist gekr"aftigt und w"armt wieder aller Herzen. Unsre
H"afen strotzen, der Staat{\s}kredit ist fest. Frankreich atmet
endlich wieder auf~..."'

"`Da{\s} hei"st,"' sagte Rudolf, "`vom gesellschaftlichen
Standpunkt hat man vielleicht recht."'

"`Wie meinen Sie da{\s}?"' fragte sie.

"`Wissen Sie denn nicht,"' erl"auterte er, "`da"s e{\s}
problematische Naturen gibt? Halb Tr"aumer, halb Tatenmenschen?
Heute leben sie den hehrsten Idealen und morgen den wildesten
Gen"ussen. Nicht{\s} ist ihnen zu toll, zu phantastisch~..."'

Sie blickte ihn an, wie man einen Polarfahrer anschaut. Dann sagte
sie:

"`Un{\s} armen Frauen dagegen, un{\s} sind die Freuden solcher
Kontraste verboten!"'

"`Sch"one Freuden!"' entgegnete er bitter. "`Da{\s} Gl"uck liegt
wo ganz ander{\s}!"'

"`Ach, so findet man{\s} nirgend{\s}?"'

"`Doch! Eine{\s} Tage{\s} begegnet man dem Gl"uck!"' fl"usterte er.

"`Und da{\s} wissen Sie alle gerade am besten,"' fuhr der
Regierung{\s}rat fort, "`Sie, die Sie Landwirte und Landarbeiter
sind, friedliche Vork"ampfer eine{\s} Kulturideal{\s}, M"anner
de{\s} Fortschritte{\s} und der Ordnung! Sie wissen da{\s}, sage
ich, da"s politische St"urme weit furchtbarer sind denn St"urme in
der Natur~..."'

"`Ja, eine{\s} Tage{\s} begegnet man ihm!"' wiederholte Rudolf,
"`ganz unerwartet, gerade wenn man alle Hoffnung verloren hat!
Dann "offnet sich der Himmel, und e{\s} ist einem, al{\s} riefe
eine Stimme: {\glq}Hier ist da{\s} Gl"uck!{\grq} Und dem Menschen,
den Sie da gefunden haben, dem m"ussen Sie au{\s} innerm Drange
herau{\s} ihr Leben anvertrauen, ihm alle{\s} geben, alle{\s}
opfern! E{\s} werden keine Worte gewechselt. Alle{\s} ist nur
Ahnung, Gef"uhl! Man hat sich ja l"angst im Traumland gesehen~..."'

Er blickte Emma an.

"`Endlich ist er da, der Schatz, den man so lange gesucht hat,
leibhaftig da! Er gl"anzt und strahlt! Noch immer h"alt man ihn
f"ur ein Traumbild. Man wagt nicht, an ihn zu glauben. Man ist
geblendet, al{\s} k"ame man pl"otzlich au{\s} der Nacht in die
Sonne~..."'

Rudolf begleitete seine Worte mit Geb"arden. Er pre"ste die Rechte
auf sein Gesicht wie jemand, dem e{\s} schwindelt. Dann lie"s er
sie auf Emma{\s} Hand sinken. Sie zog sie weg.

Der Rat sprach immer weiter:

"`Wen k"onnte da{\s} auch verwundern, meine Herren? H"ochsten{\s}
Leute, die so blind w"aren, so verbohrt (ich scheue mich nicht,
diese{\s} Wort zu gebrauchen!), so verbohrt in die Vorurteile
abgetaner Zeiten, da"s sie die Gesinnung der Landwirte noch immer
verkennen. Wo findet man, frage ich, mehr Patrioti{\s}mu{\s}
al{\s} auf dem Lande? Wo mehr Opferfreudigkeit in Dingen de{\s}
Gemeinwohl{\s}? Mit einem Worte: wo mehr Intelligenz? Meine
Herren, ich meine nat"urlich nicht jene oberfl"achliche
Intelligenz, mit der sich m"u"sige Geister br"usten, nein, ich
meine die gr"undliche und ma"svolle Intelligenz, die sich nur mit
ersprie"slichen Absichten bet"atigt und damit dem Vorteile de{\s}
Einzelnen wie der F"orderung der Allgemeinheit dient und eine
St"utze de{\s} Staate{\s} ist, durchdrungen von der Achtung vor
den Gesetzen und dem Gef"uhle der Pflichterf"ullung~..."'

"`Pflichterf"ullung!"' wiederholte Rudolf. "`Immer und "uberall
die Pflicht! Wie mich diese{\s} Wort anwidert! Ein Chor von alten
Schaf{\s}\-k"opfen in Schlaf\-r"ocken und von Betschwestern mit
W"armbullen und Gesangb"uchern kr"achzt un{\s} ewig die alte
Litanei vor: {\glq}Die Pflicht, die Pflicht!{\grq} Der Teufel soll
sie holen! Unsre Pflicht ist e{\s}, alle{\s} Gro"se in der Welt
mit\/zuf"uhlen, da{\s} Sch"one anzubeten und sich nicht immer gleich
unter alle m"oglichen gesellschaftlichen Konvenienzen zu ducken,
sich nicht zu Sklaven herabw"urdigen zu lassen~..."'

"`Indessen ... indessen~..."', wandte Emma ein.

"`Nein, nein! Warum immer gegen die Leidenschaften k"ampfen? Sind
sie nicht vielmehr da{\s} Allersch"onste, wa{\s} e{\s} auf Erden
gibt, der Quell de{\s} Heldensinn{\s}, der Begeisterung, der
Dichtung, der Musik, aller K"unste, alle{\s} Leben{\s} im wahren
Sinne?"'

"`Aber man mu"s sich doch ein wenig nach den Leuten richten und
sich ihrer Moral f"ugen"', meinte Emma.

"`So! Da{\s} ist dann eben die doppelte Moral,"' eiferte er. "`Die
eine: die kleinliche, herk"ommliche, die der Leute, die in einem
fort ein andre{\s} Gesicht zieht, immer Ach und Weh schreit, im
tr"uben fischt und auf dem Erdboden kriecht. Da{\s} ist die all
der versammelten Troddel da unten. Und die andre: die g"ottliche,
die um un{\s} ist und "uber un{\s} wie die Landschaft, die un{\s}
umprangt, und der blaue Himmel, der "uber un{\s} leuchtet~..."'

Lieuvain wischte sich den Mund mit dem Taschentuche, dann sprach
er weiter:

"`Soll ich Ihnen, meine Herren, den Nutzen der Landwirtschaft hier
noch im einzelnen darlegen? Wer sorgt f"ur unser t"aglich Brot?
Wer schafft un{\s} die Unterhaltung{\s}mittel? Tut e{\s} nicht der
Landmann? Er und kein anderer? Meine Herren, dem Landmann, der mit
seiner schwieligen Hand da{\s} Saatkorn in die fruchtbringenden
Furchen s"at, verdanken wir da{\s} Getreide, da{\s} dann, von
sinnreichen Maschinen zu Mehl gemahlen, in die St"adte zu den
B"ackern kommt, die Brot darau{\s} backen f"ur arm und reich! Ist
e{\s} nicht der Landmann, der auf den Weiden die Schafherden
h"utet, damit wir Kleider haben? Wie sollten wir un{\s} anziehen,
wie un{\s} n"ahren, ohne die Landwirtschaft? Aber, meine Herren,
wir brauchen gar nicht so weit zu gehen. Hat nicht jeder von
un{\s} schon manchmal "uber die Bedeutung jene{\s} bescheidenen
Tierchen{\s} nachgedacht, da{\s} die Zierde unserer Bauernh"ofe
ist und un{\s} gleichzeitig ein weiche{\s} Kopfkissen, einen
saftigen Braten f"ur unsern Tisch und die Eier schenkt? Ich k"ame
nicht zu Ende, wenn ich alle die andern verschiedenen Erzeugnisse
l"uckenlo{\s} aufz"ahlen m"u"ste, mit denen die wohlbebaute Erde
wie eine gro"sm"utige Mutter ihre Kinder "ubersch"uttet. Ich nenne
nur den Weinstock, den Baum, der un{\s} den Apfelwein spendet, und
den Rap{\s}. Dann haben wir den K"ase und den Flach{\s}. Meine
Herren, vergessen wir den Flach{\s} nicht! Der Flach{\s}bau hat in
den letzten Jahren einen bedeutenden Aufschwung genommen, auf den
ich Ihre Aufmerksamkeit ganz besonder{\s} hinlenken m"ochte~..."'

Dieser Appell war eigentlich unn"otig, denn die Menge lauschte
offenen Munde{\s} und lie"s sich kein W"ortchen entgehen. Der
B"urgermeister, der zur Seite de{\s} Redner{\s} sa"s, horchte mit
aufgerissenen Augen. Derozeray{\s} schlo"s die seinen hin und
wieder voller Andacht. Und der Apotheker, der seinen Platz
etwa{\s} weiter weg hatte, hielt sich eine Hand an{\s} Ohr, um
Silbe f"ur Silbe ordentlich zu verstehen. Die "ubrigen
Prei{\s}richter nickten bed"achtig mit den gesenkten H"auptern, um
ihre Zustimmung zu erkennen zu geben. Die Feuerwehr st"utzte sich
auf ihre Gewehre, und Binet stand immer noch stramm da im
Stillgestanden und mit vorschrift{\s}m"a"siger S"abelhaltung.
H"oren konnte er vielleicht, aber sehen nicht, weil ihm die Blende
seine{\s} Helm{\s} bi{\s} "uber die Nase reichte. Sein Leutnant,
der j"ungste Sohn de{\s} B"urgermeister{\s}, hatte einen noch
gr"o"seren auf. Diese{\s} Unget"um wackelte ihm fortw"ahrend auf
dem Kopfe hin und her. "Uberdie{\s} sah der Zipfel eine{\s}
seidnen Tuche{\s} hervor, da{\s} er untergestopft hatte. Er
l"achelte wie ein artige{\s} Kind unter dem Helme hervor, und sein
schmale{\s} blasse{\s} Gesicht, "uber da{\s} Schwei"stropfen
rannen, verriet zugleich helle Freude und m"ude Abspannung.

Der Marktplatz war bi{\s} an die H"auser heran voller Menschen. In
allen Fenstern erblickte man Leute, ebenso auf allen
T"urschwellen. Vor dem Schaufenster der Apotheke stand Justin,
ganz versunken in da{\s} Schauspiel vor seinen Augen. Trotzdem um
den Redner herum Stille herrschte, verlor sich seine Stimme doch
bereit{\s} in einiger Entfernung im Winde. Nur einzelne
abgerissene Worte drangen weiter, von denen da{\s} Ger"ausch hin-
und herger"uckter St"uhle auch noch einen Teil verschlang. Noch
weiter weg vernahm man dicht hinter sich langgedehnte{\s}
Rindergebr"ull oder da{\s} Bl"oken der Schafe, die sich einander
antworteten. Die Kuhjungen und Hirten hatten n"amlich ihre Tiere
inzwischen bi{\s} auf den Markt getrieben, wo sie sich nun von
Zeit zu Zeit laut bemerkbar machten.

Rudolf war dicht an Emma heranger"uckt und fl"usterte ihr hastig
zu:

"`Mu"s einen diese Tyrannei der Gesellschaft denn nicht zum
Rebellen machen? Gibt e{\s} ein einzige{\s} Gef"uhl, da{\s} sie
nicht verdammt? Die edelsten Triebe, die reinsten Neigungen werden
von ihr verfolgt und verleumdet, und wenn sich zwei arme Herzen
trotz alledem finden, so verb"undet sich alle{\s}, damit sie
einander nicht geh"oren k"onnen. Aber sie werden e{\s} dennoch
versuchen, sie regen ihre Fl"ugel, und sie rufen sich. Fr"uher
oder sp"ater, in sieben Monaten oder in sieben Jahren, sind sie
doch vereint in ihrer Liebe, weil e{\s} da{\s} Schicksal so will
und weil sie f"ureinander geschaffen sind~..."'

Er hatte die Arme verschr"ankt und st"utzte sie auf seine Knie,
und so schaute er Emma an, ganz au{\s} der N"ahe, mit starrem
Blicke. Sie konnte in seinen Augen die kleinen goldnen
Krei{\s}linien sehen, um die schwarzen Pupillen herum, und sie
roch sogar da{\s} leise Parf"um in seinem Haar. Woll"ustige
M"udigkeit "uberfiel sie. Der Vicomte, mit dem sie im Schlosse
Vaubyessard getanzt hatte, kam ihr in den Sinn. Sein Bart hatte
genau so geduftet wie diese{\s} Haar, nach Vanille und Zitronen.
Unwillk"urlich schlo"s sie die Augenlider, um den Geruch st"arker
zu sp"uren. Aber al{\s} sie sich in ihren Stuhl zur"ucklehnte,
fiel ihr Blick gerade auf die alte Postkutsche, fern am Horizonte,
die langsam die H"ohe von Leux herabfuhr und eine lange Staubwolke
nach sich zog. In derselben gelben Kutsche war Leo so oft zu ihr
zur"uckgekommen, und auf dieser Stra"se da war er von ihr
weggefahren auf immerdar! Sie glaubte sein Antlitz zu sehen, im
Rahmen seine{\s} Fenster{\s}. Dann verschwamm alle{\s}, und Nebel
zogen vor"uber. E{\s} kam ihr vor, al{\s} wirble sie wie damal{\s}
im Walzer, in der Lichtflut de{\s} Ballsaale{\s}, im Arme de{\s}
Vicomte. Und Leo w"are nicht weit weg, sondern k"ame wieder ...
Dabei sp"urte sie in einem fort Rudolf{\s} Haar dicht neben sich.
Die s"u"se Empfindung seiner N"ahe verm"ahlte sich mit den alten
Gel"usten; und wie Staubk"orner, die der Wind aufjagt, umtanzten
sie diese Gef"uhle zusammen mit dem leisen Dufte und bet"aubten
ihr die Seele. Ein paarmal "offnete sie weit die Nasenfl"ugel, um
-- sto"sweise -- den frischen Geruch der Girlanden einzuatmen, die
um die S"aulen geschlungen waren.

Sie streifte sich die Handschuhe ab und trocknete sich die
feuchtgewordnen H"ande; dann f"achelte sie ihren Wangen mit dem
Taschentuche K"uhlung zu, wobei sie mitten durch da{\s} H"ammern
de{\s} Blute{\s} in ihren Schl"afen da{\s} Gesumme der Menge und
die immer noch Phrasen dreschende Stimme de{\s}
Regierung{\s}rate{\s} verworren vernahm.

Er predigte:

"`Fahren Sie fort! Bleiben Sie auf Ihrem Wege! Lassen Sie sich
nicht beirren, weder durch H"angenbleiben an veralteten
"Uberlieferungen noch durch allzu hastige Annahme von k"uhnen
Neuerungen! Richten Sie Ihren Eifer vor allem auf die Verbesserung
de{\s} Boden{\s}, auf eine gute D"ungung, auf die Veredelung der
Pferde-, Rinder-, Schafe- und Schweinezucht! M"oge diese
Versammlung f"ur Sie eine Art friedlicher Kampfplatz sein, auf dem
der Sieger beim Verlassen der Arena dem Besiegten die Hand dr"uckt
wie einem Bruder und ihm den gleichen Erfolg f"ur die Zukunft
w"unscht! Und Ihr, Ihr w"urdigen Dienstboten, bescheidene{\s}
Hofgesinde, um deren m"uhevolle Arbeit sich bi{\s}her noch keine
Regierung gek"ummert hat, kommt her und empfangt den Lohn f"ur
Eure stille T"uchtigkeit und seid "uberzeugt, da"s die F"ursorge
de{\s} Staate{\s} fortan auch Euch gelten wird, da"s er Euch
ermutigt und besch"utzt, da"s er Euch auf begr"undete Beschwerden
hin recht geben wird und Euch, soweit e{\s} in seiner Macht steht,
die B"urde Eurer opferfreudigen Arbeit erleichtern wird!"'

Darnach setzte sich der Regierung{\s}rat. Jetzt erhob sich Herr
Derozeray{\s} und begann eine zweite Rede. Sie war nicht so
schwungvoll wie die Lieuvain{\s}, daf"ur war sie sachlicher,
da{\s} hei"st: sie verriet Fachkenntnisse und gab tiefergehenden
Betrachtungen Raum. Da{\s} Lob auf die Regierung war k"urzer
gefa"st; die Rede besch"aftigte sich mehr mit der Landwirtschaft
und der Religion. Die Wechselbeziehungen zwischen beiden wurden
beleuchtet. Beide h"atten zu allen Zeiten die Zivilisation
gef"ordert. Rudolf plauderte mit Frau Bovary "uber Tr"aume,
Vorahnungen und Suggestion. Der Redner ging auf die Anf"ange der
menschlichen Gesellschaft zur"uck und schilderte die barbarischen
Zeiten, da sich der Mensch im Urwalde von Eicheln gen"ahrt hatte.
Sp"ater h"atte man die Tierfelle abgelegt und sich mit Tuch
bekleidet, h"atte Feldwirtschaft und Weinbau begonnen. War die{\s}
nun ein Vorteil oder brachten nicht die neuen Besch"aftigungen
ungleich mehr M"uhen denn Nutzen? "Uber diese{\s} Problem stellte
Derozeray{\s} allerhand Betrachtungen an.

Von der Suggestion war Rudolf unterdessen allm"ahlich auf die
Wahlverwandtschaft gekommen, und w"ahrend der Redner unten vom
Pfluge de{\s} Cincinnatu{\s} sprach, von Diocletian und seinen
Kohlplantagen und von den chinesischen Kaisern, die zu Neujahr
eigenh"andig s"aen, setzte der junge Mann der jungen Frau
au{\s}einander, da"s die Ursache einer solchen unwiderstehlichen
gegenseitigen Anziehung in einer fr"uheren Existenz zu suchen sei.

"`Nehmen Sie beispiel{\s}weise un{\s} beide!"' sagte er. "`Warum
haben wir un{\s} kennen gelernt? Hat die{\s} allein der Zufall
gef"ugt? War e{\s} nicht vielmehr in beiden ein geheimer Drang,
der un{\s} gegenseitig einander zuf"uhrte, wie zwei Str"ome
ineinander flie"sen, jeder von weiter Ferne her?"'

Er ergriff wiederum ihre Hand. Sie ent\/zog sie ihm nicht.

"`Prei{\s} f"ur gute Bewirtschaftung~..."', rief unten der Redner.

"`Denken Sie doch daran, wie ich zum ersten Male in Ihr Hau{\s}
kam~..."'

"`Herrn Bizet au{\s} Quincampoix!"'

"`Wu"ste ich damal{\s}, da"s wir so bald gute Freunde werden
sollten?"'

"`Siebzig Franken~..."'

"`Hundertmal habe ich reisen wollen, aber ich bin immer wieder zu
Ihnen gekommen und hier geblieben~..."'

"`F"ur Erfolge im D"ungen."'

"`... heute und morgen, alle Tage, mein ganze{\s} Leben~..."'

"`Herrn Caron au{\s} Argueil eine goldene Medaille!"'

"`... denn noch keine{\s} Menschen Gesellschaft hat mich so
v"ollig bezaubert~..."'

"`Herrn Bain au{\s} Givry-Saint-Martin~..."'

"`... und so werde ich Ihr Bild in mir tragen~..."'

"`... f"ur einen Merino-Schafbock~..."'

"`Sie aber werden mich vergessen! Ich bin an Ihnen
vor"ubergewandelt wie ein Schatten!"'

"`Herrn Belot au{\s} Notre-Dame~..."'

"`Aber nein, nicht wahr? Manchmal werden Sie sich doch meiner
erinnern?"'

"`F"ur Schweinezucht ein Prei{\s} geteilt, je achtzig Franken, den
Herren Leh\'eriss\'e und C"ullembourg!"'

Rudolf dr"uckte Emma{\s} Hand. Sie f"uhlte sich ganz hei"s an und
zitterte wie eine gefangene Taube, die fortfliegen m"ochte. Sei
e{\s} nun, da"s Emma versuchte, ihre Hand zu befreien, oder da"s
sie Rudolf{\s} Druck wirklich erwidern wollte: sie machte mit
ihren Fingern eine Bewegung. Da rief er au{\s}:

"`Ach, ich danke Ihnen! Sie sto"sen mich nicht zur"uck! Sie sind
so gut! Sie f"uhlen, da"s ich Ihnen geh"ore! Ich will Sie ja nur
sehen, nur anschauen!"'

Ein Windsto"s, der durch die Fenster fuhr, bauschte die Tischdecke
de{\s} Tische{\s} im Saal, und unten auf dem Markte flatterten die
m"achtigen Haubenschleifen der B"auerinnen wie wei"se
Schmetterling{\s}fl"ugel auf.

"`F"ur die Herstellung von "Olkuchen~..."'

Der Vorsitzende fing an sich zu beeilen.

"`F"ur Mastversuche nach flandrischer Art ... Weinbau ...
Feldbew"asserung ... langj"ahrigen Pacht ... treue Dienste~..."'

Rudolf sprach nicht mehr. Sie sahen sich beide an. Emma{\s}
trockne Lippen bebten in hei"sestem Begehren. Weich und ganz von
selbst verschlangen sich ihre H"ande.

"`Katharine Nikasia Elisabeth Leroux au{\s} Sassetot-la-Guerri\`ere
f"ur vier\-und\-f"unfzig\-j"ahrigen Dienst auf ein und demselben
Gute eine silberne Medaille im Werte von f"unfundzwanzig Franken!"'

Nach einer Weile h"ort man: "`Wo ist Katharine Leroux?"'

Sie erschien nicht, aber man vernahm fl"usternde Stimmen.

"`Geh doch!"'

"`Ach nein!"'

"`Brauchst keine Angst zu haben!"'

"`Nee, ist die dumm!"'

"`Hier! Hier steckt sie!"'

"`So mag sie doch vorkommen!"' rief der B"urgermeister dazwischen.

Da begann eine kleine alte Frau mit "angstlicher Geb"arde zur
Estrade hinzulaufen. In ihren Lumpen sah sie selber wie zerfallen
au{\s}. Sie hatte die F"u"se in derben Holzschuhen und um die
H"uften eine gro"se blaue Sch"urze. Ihr magere{\s} Gesicht, von
einer schlichten Haube umrahmt, war runzeliger al{\s} ein
verschrumpfelter Apfel, und au{\s} den "Armeln ihrer roten Jacke
langten zwei d"urre H"ande mit knochigen Gelenken herau{\s}. Vom
Staub der Scheunen, der Lauge der W"asche und dem Fett der
Schafwolle waren sie so hornig, hart und rissig, da"s sie wie
schmutzig au{\s}sahen, und doch waren sie in reinem Wasser
t"uchtig gewaschen worden. Da"s sie unz"ahlige Strapazen hinter
sich hatten, da{\s} verrieten sie von selbst an ihrer dem"utigen
Haltung: sie standen halboffen, wie bereit, ewig Dienste zu
empfangen. Etwa{\s} wie kl"osterliche Strenge sprach au{\s} den
Z"ugen der alten Frau und verlieh ihnen eine Spur von Vornehmheit.
E{\s} lebte nicht{\s} Weiche{\s} in ihrem bleichen Gesicht,
nicht{\s} Traurige{\s} oder R"uhrselige{\s}. Im steten Umgang mit
Tieren war ihr stumme Geduld zur Natur geworden. Heute befand sie
sich zum ersten Male inmitten einer solchen Masse von Menschen.
Die Fahnen, der Trommelwirbel, die vielen Herren in schwarzen
R"ocken, da{\s} Kreuz der Ehrenlegion auf der Brust de{\s}
Rate{\s}, alle{\s} da{\s} ersch"uttertere bi{\s} in{\s} Herz. Sie
stand ganz erstarrt da, sie wu"ste nicht, ob sie zur Estrade
vorlaufen oder enteilen sollte, und sie begriff nicht, warum man
sie nach vorn dr"angte und warum ihr die Prei{\s}richter
freundlich zul"achelten. Sie stand vor diesen beh"abigen B"urgern
al{\s} ein verk"orperte{\s} halbe{\s} S"akulum der Knechtschaft.

"`Treten Sie n"aher, verehrung{\s}w"urdige Katharine Nikasia
Elisabeth Leroux!"' sagte der Regierung{\s}rat, der die Liste der
Prei{\s}gekr"onten au{\s} den H"anden de{\s} Vorsetzenden
entgegengenommen hatte. Indem er abwechselnd auf den Bogen und auf
die Greisin blickte, wiederholte er in v"aterlichem Tone:

"`N"aher, immer n"aher!"'

"`Sind Sie denn taub?"' rief T"uvache heftig und sprang von seinem
Sitze auf.

"`F"ur vierundf"unfzigj"ahrige Dienst\/zeit eine silberne Medaille
im Werte von f"unfundzwanzig Franken! Die ist f"ur Sie!"' wurde
ihr laut gesagt.

Die alte Frau nahm sie und sah sie sich lange an, und ein L"acheln
de{\s} Gl"ucke{\s} sonnte ihr Gesicht. Al{\s} sie wegging, h"orte
man sie vor sich hinmurmeln:

"`Ich werde sie dem Herrn Pfarrer bei un{\s} zu Hause geben, damit
er mir dermaleinst eine Messe liest."'

"`Selig die Geiste{\s}armen!"' meinte der Apotheker, zum Notar
gewandt.

Der feierliche Akt war zu Ende. Die Menge verlief sich. Und
nachdem nun die Prei{\s}verteilung vor"uber war, nahm jeder wieder
seinen Rang ein, und alle{\s} lief im alten Gleise. Die Herren
schnauzten ihre Knechte an, und die Knechte pr"ugelten da{\s}
Vieh, da{\s} mit gr"unen Kr"anzen um die H"orner in seine St"alle
zur"ucktrottete. Ahnung{\s}lose Triumphatoren.

Die B"urgergarde und die Feuerwehr traten weg und begaben sich in
den ersten Stock de{\s} Rathause{\s}. Der Bataillon{\s}tambour
schleppte einen Korb Weinflaschen, und die Mannschaft spie"ste
sich die spendierten Butterbrote auf die Bajonette.

Frau Bovary ging an Rudolf{\s} Arm nach Hau{\s}. An der T"ure
nahmen sie Abschied. Sodann ging er bi{\s} zur Stunde de{\s}
Festmahle{\s} allein durch die Wiesen spazieren.

Der Schmau{\s} dauerte lange. E{\s} war l"armig, die Bedienung
schlecht. Man sa"s so eng aneinander, da"s man f"ur die Ellenbogen
gar keine Freiheit hatte, und die schmalen Bretter, die al{\s}
B"anke dienten, drohten unter der Last der G"aste zusammenzubrechen.
Man a"s unmenschlich viel. Jeder wollte auf seine Kosten kommen.
Allen perlte der Schwei"s von der Stirne. Zwischen der Tafel und
den H"angelampen schwebte wei"slicher Dunst, wie der Nebel "uber
dem Flusse an einem Herbstmorgen.

Rudolf, der seinen Platz an der Zeltwand hatte, verlor sich
v"ollig in Tr"aumereien an Emma, so da"s er nicht{\s} sah und
h"orte. Hinter ihm, drau"sen auf dem Rasen, schichteten die
Kellner die gebrauchten Teller. Wenn ihn einer seiner Nachbarn
anredete, gab er ihm keine Antwort. Man f"ullte ihm da{\s}
Gla{\s}, ohne da"s er e{\s} wahrnahm. Trotz de{\s} allgemeinen
immer st"arker werdenden L"arme{\s} war e{\s} in ihm ganz still.
Er sann "uber da{\s} nach, wa{\s} Emma gesagt hatte, und "uber die
Linien ihrer Lippen dabei. Ihr Bild schimmerte ihm wie au{\s}
Zauberspiegeln au{\s} allem entgegen, wa{\s} gl"anzte, sogar
au{\s} dem Messingbeschlag der Feuerwehrhelme. Die Zeltwand hatte
Falten, die ihn an die ihre{\s} Kleide{\s} erinnerten. Und vor
ihm, in der Ferne der Zukunft, winkte eine endlo{\s} lange Reihe
verliebter Tage.

Am Abend sah er Emma wieder, beim Feuerwerk. Aber sie war in der
Gesellschaft ihre{\s} Manne{\s}, der Frau Homai{\s} und de{\s}
Apotheker{\s}. Der letztere beunruhigte sich sehr "uber die
M"oglichkeit, da"s einmal eine Rakete versehentlich in da{\s}
Publikum gehen k"onnte. Aller Augenblicke verlie"s er seine
Freunde, um Binet zur gr"o"sten Vorsicht zu vermahnen. Die
Feuerwerk{\s}k"orper waren vorher au{\s} "ubertriebener
"Angstlichkeit im Hause de{\s} B"urgermeister{\s} aufbewahrt
worden, in dessen Keller. Da{\s} feucht gewordene Pulver
ent\/z"undete sich nun schwer, und da{\s} Hauptst"uck, eine
Schlange, die sich in den Schwanz bei"st, versagte vollst"andig.
Ab und zu zischte ein d"urftige{\s} Feuerrad. Dann schrie die
gaffende Menge vor Vergn"ugen laut auf, und in diese{\s} Geschrei
mischte sich da{\s} Kreischen der Weiber, die im Dunkeln von
dreisten H"anden angefa"st wurden.

Emma schmiegte sich schweigsam an Karl{\s} Arm. Den Kopf gehoben,
verfolgte sie die Feuerlinien der Raketen auf dem schwarzen
Himmel. Rudolf betrachtete sie im Scheine der Lampion{\s}. Nach
und nach verl"oschten diese, und nun leuchteten nur die Gestirne.
Ein paar Regentropfen fielen. Frau Bovary legte sich ihr Tuch
"uber da{\s} unbedeckte Haar.

In diesem Augenblicke fuhr der Landauer de{\s} Regierung{\s}rate{\s}
vom Gasthofe weg. Der Kutscher war bezecht und hockte verschlafen
auf seinem Bocke. Man sah von weitem, wie die schwere Masse
seine{\s} K"orper{\s} zwischen den Wagenlichtern hin und her
pendelte, je nach den Bewegungen de{\s} Wagen{\s} auf dem
holperigen Pflaster.

"`Man sollte wirklich strenger gegen die Trunksucht vorgehen"',
bemerkte der Apotheker. "`Mein Vorschlag geht dahin, allw"ochentlich
am Rathause die Namen derer au{\s}zuh"angen, die sich in der Woche
vorher sinnlo{\s} betrunken haben. Da{\s} erg"abe nebenbei eine
Statistik, die man in gewissen F"allen ... Aber entschuldigen
Sie!"'

Er eilte wiederum zum Feuerwehrhauptmann, der sich gerade
anschickte, nach Hause zu gehen. Ihn trieb die Sehnsucht nach
seiner Drehbank.

"`Vielleicht t"aten Sie gut,"' mahnte ihn Homai{\s}, "`wenn Sie
einen von Ihren Leuten schickten, oder noch besser, wenn Sie
selber gingen~..."'

"`Lassen Sie mich doch in Ruhe!"' murrte der Steuereinnehmer.
"`Da{\s} h"atte ja gar keinen Sinn!"'

Der Apotheker gesellte sich wieder zu seinen Freunden.

"`Wir k"onnen v"ollig beruhigt sein"', sagte er zu ihnen. "`Herr
Binet hat mir soeben versichert, da"s alle Vorsicht{\s}ma"sregeln
getroffen sind. E{\s} ist keine Feuergefahr mehr vorhanden. Und
die Spritzen stehen voller Wasser bereit. Gehen wir schlafen!"'

"`Ach ja! Ich hab{\s} sehr n"otig!"' erwiderte Frau Homai{\s}, die
schon immer t"uchtig geg"ahnt hatte. "`Aber sch"on war{\s} doch!"'

Rudolf wiederholte leise mit einem z"artlichen Blicke:

"`Wundersch"on!"'

Dann verabschiedete man sich und ging voneinander.

Zwei Tage darauf stand im "`Leuchtturm von Rouen"' ein langer
Bericht "uber die Landwirtschaftliche Versammlung. Der Apotheker
hatte ihn am Morgen darauf schwungvoll verfa"st.

"`Wa{\s} k"unden diese Girlanden, diese Blumen und Kr"anze? Wohin
w"alzt sich die Menge, gleichwie die Wogen de{\s} st"urmischen
Weltmeere{\s} unter den Strahlenb"uscheln der tropischen Sonne,
die unsere Fluren sengt?"'

Sodann sprach er von der Lage der Landbev"olkerung. "`Gewi"s, die
Regierung hat hier viel getan, aber noch nicht genug. Mut! Tausend
Reformen sind unerl"a"slich. Man gehe an sie heran!"' Bei der
Schilderung der Ankunft de{\s} Regierung{\s}vertreter{\s} feierte
er "`da{\s} martialische Au{\s}sehen unsrer Miliz"', die
"`behenden Dorfsch"onen,"' die "`kahlk"opfigen Greise, diese
Patriarchen, die Letzten der unsterblichen Legionen, deren
Soldatenherzen beim Wirbeln der Trommeln h"oher schlagen."' Seinen
eigenen Namen z"ahlte er unter den Prei{\s}richtern al{\s} ersten
auf und erw"ahnte in einer Anmerkung sogar, da"s Herr Homai{\s},
der Apotheker von Yonville, unl"angst eine Denkschrift "uber den
Apfelwein an die Rouener Agronomische Gesellschaft eingereicht
habe. Bei der Prei{\s}verteilung angelangt, schilderte er die
Freude der Au{\s}gezeichneten mit dithyrambischer Begeisterung.
"`V"ater fielen ihren S"ohnen um den Hal{\s}, Br"uder ihren
Br"udern, Gatten ihren Gattinnen. Mehr denn einer zeigte voll
Stolz seine schlichte Medaille, und heimgekehrt in sein stille{\s}
K"ammerlein, mag sie so mancher, Tr"anen in den Augen, an die Wand
geh"angt haben ... Gegen sech{\s} Uhr abend{\s} vereinigte ein
Festmahl in dem auf der Herrn Li\'egeard geh"orenden Wiese
errichteten gro"sen Zelte die hervorragendsten Festteilnehmer. Von
Anfang bi{\s} Ende herrschte die gr"o"ste Gem"utlichkeit. Mehrere
Toaste wurden au{\s}gebracht. Herr Regierung{\s}rat Lieuvain trank
auf Seine Majest"at, Herr B"urgermeister T"uvache auf den Herrn
Landrat, sodann Herr Rittergut{\s}besitzer Derozeray{\s} auf
da{\s} Gedeihen der Landwirtschaft, Herr Apotheker Homai{\s} auf
die Industrie und ihre Schwestern, die K"unste und Wissenschaften,
so zuletzt Herr Leplichey auf den Fortschritt. Am Abend
erleuchtete ein pr"achtige{\s} Feuerwerk pl"otzlich alle
Gesichter. Man kann wohl sagen, e{\s} war ein wahre{\s}
Kaleidoskop, eine herrliche Operndekoration, und im Moment durfte
sich unser kleiner Ort in die Wunderwelt von Tausendundeiner Nacht
entr"uckt w"ahnen. Zum Schlusse stellen wir mit Freuden fest, da"s
auch nicht ein einiger unliebsamer Vorfall da{\s} Volk{\s}fest
gest"ort hat. Zu bemerken w"are nur noch da{\s} Fernbleiben der
Geistlichkeit. Offenbar hat man unter ihr andre Ansichten von
Allgemeinwohl und Fortschritt. Haltet e{\s}, wie ihr wollt, ihr
J"unger Loyola{\s}!"'


\newpage\begin{center}
{\large \so{Neunte{\s} Kapitel}}\bigskip\bigskip
\end{center}

Sech{\s} Wochen flossen hin. Rudolf kam nicht. Endlich, eine{\s}
Sp"atnachmittag{\s}, erschien er.

"`Man darf sich nicht so schnell wieder sehen lassen. Da{\s} w"are
ein Fehler!"'

Nach dem Feste war er auf die Jagd gegangen. Und nach der Jagd
hatte er sich gesagt, nun sei e{\s} zu sp"at zu einem Besuche.
Sein Gedankengang war folgender:

"`Wenn sie mich vom ersten Tage an geliebt hat, wird sie mich nach
dem Hangen und Bangen de{\s} Warten{\s} nur um so mehr lieben.
Warten wir also noch eine Weile!"'

Al{\s} er Emma in der Gro"sen Stube entgegentrat, sah er, wie sie
bla"s wurde. Da wu"ste er, da"s er sich nicht verrechnet hatte.

Sie war allein. E{\s} d"ammerte. Die kleinen Mullgardinen an den
Scheiben der Fenster vermehrten da{\s} Halbdunkel. Da{\s} blanke
Metall de{\s} Barometer{\s}, auf da{\s} ein Sonnenstrahl fiel,
glitzerte auf der Fl"ache de{\s} Spiegel{\s} "uber dem Kamin wider
wie flammende{\s} Feuer.

Rudolf stand noch immer. Emma antwortete nur mit M"uhe auf seine
ersten H"oflichkeit{\s}worte.

"`Ich war stark besch"aftigt. Und dann bin ich auch krank
gewesen."'

"`Ernstlich?"' fragte sie erregt.

"`Na,"' erwiderte Rudolf, indem er sich ihr zur Seite auf einen
niedrigen Sessel setzte, "`eigentlich wollte ich nicht
wiederkommen."'

"`Warum?"'

"`Erraten Sie e{\s} nicht?"'

Wiederum sah er sie an, die{\s}mal so leidenschaftlich, da"s sie
rot wurde und die Augen senkte.

Er begann von neuem:

"`Emma!"'

"`Herr Boulanger!"' rief sie und r"uckte ein wenig von ihm ab.

"`Ah!"' sagte er in wehm"utigem Tone. "`Sehen Sie, wie recht ich
hatte, wenn ich nicht wiederkommen wollte! Ihr Name~..., dieser
Name, der mein ganze{\s} Herz erf"ullt~..., er ist mir
entschl"upft, und Sie verbieten mir, ihn au{\s}zusprechen! Frau
Bovary! Alle Welt nennt Sie so! So hei"sen Sie! Und doch ist
da{\s} der Name -- eine{\s} andern!"' Nach einer Weile wiederholte
er: "`Eine{\s} andern!"' Er hielt sich die H"ande vor sein
Gesicht. "`Ach, ich denke fortw"ahrend an Sie ... Die Erinnerung
bringt mich in Verzweiflung ... Verzeihen Sie mir ... Ich gehe ...
Leben Sie wohl! Ich will weit, weit weg ... so weit gehen, da"s
Sie nicht{\s} mehr von mir h"oren werden! Aber heute ... heute ...
ach, ich wei"s nicht, wa{\s} mich mit aller Gewalt hierher zu
Ihnen getrieben hat! Gegen sein Schicksal kann keiner k"ampfen!
Und wo Engel l"acheln, wer k"onnte da widerstehen? Man l"a"st sich
hinrei"sen von der, die so sch"on, so s"u"s, so anbeten{\s}wert
ist!"'

E{\s} war da{\s} erstemal, da"s Emma solche Dinge h"orte, und
al{\s} ob sie sich im Bade woll"ustig dehnte, so f"uhlte sie sich
in ihrem Selbstbewu"stsein von der warmen Flut dieser Sprache
umkost.

"`Aber wenn ich mich auch nicht habe sehen lassen,"' fuhr er fort,
"`wenn ich nicht mit Ihnen reden durfte, so habe ich doch
wenigsten{\s} da{\s} gesehen, wa{\s} Sie umgibt. Ach, nacht{\s},
Nacht f"ur Nacht habe ich mich erhoben und bin hierher geeilt, um
Ihr Hau{\s} zu schauen, Ihr Dach im Scheine de{\s} Monde{\s}, die
B"aume in Ihrem Garten, die ihre Wipfel vor Ihrem Fenster wiegen,
und da{\s} Lampenlicht, den hellen Schimmer, der durch die
Scheiben hinau{\s}leuchtete in da{\s} Dunkel! Ach, Sie haben e{\s}
nicht geahnt, da"s da unten, Ihnen so nahe und doch so fern, ein
Armer, ein Ungl"ucklicher stand~..."'

Sie schluchzte auf und sah ihn an.

"`Sie sind ein guter Mensch!"' fl"usterte sie.

"`Nein! Ich liebe Sie! Weiter nicht{\s}! Glauben Sie mir da{\s}?
Sagen Sie mir{\s}! Ein Wort! Ein einzige{\s} Wort!"'

Leise glitt Rudolf von seinem Sitze zur Erde. Aber von der K"uche
her drang da{\s} Klappern von Holzpantoffeln. Auch war die T"ure
nicht geschlossen. Er erinnerte sich daran.

"`E{\s} w"are barmherzig von Ihnen,"' sagte er, sich wieder
erhebend, "`wenn Sie mir einen Wunsch erf"ullten."'

Er bat darum, ihm da{\s} Hau{\s} zu zeigen. Er wolle e{\s} kennen
lernen. Frau Bovary hatte nicht{\s} dagegen. Sie gingen beide zur
T"ure, da trat Karl ein.

"`Guten Tag, Doktor!"' begr"u"ste ihn Rudolf.

Der Arzt, den der ihm nicht zukommende akademische Titel
schmeichelte, stotterte ein paar verbindliche Worte.
W"ahrenddessen wurde der andre wieder v"ollig Herr der Situation.

"`Die gn"adige Frau hat mir soeben von ihrem Befinden erz"ahlt~..."',
begann er.

Karl unterbrach ihn. Er sei in der Tat "au"serst besorgt. Seine
Frau habe bereit{\s} einmal an "ahnlichen Zust"anden gelitten.

Rudolf fragte, ob da nicht Reiten gut w"are.

"`Gewi"s! Ganz au{\s}gezeichnet! Vortrefflich! Da{\s} ist wirklich
ein guter Rat! Den solltest du tats"achlich befolgen, Emma!"'

Sie wandte ein, da"s sie kein Pferd habe, aber Rudolf bot ihr
ein{\s} an. Sie lehnte sein Anerbieten ab, und er drang nicht
weiter in sie. Dann erz"ahlte er -- um seinen Besuch zu motivieren
--, sein Knecht, der Mann, dem Karl neulich zur Ader gelassen
habe, leide immer noch an Schwindelanf"allen.

"`Ich werde mal bei Ihnen auf dem Gute vorsprechen"', sagte
Bovary.

"`Nein, nein! Ich schicke ihn lieber her. Wir kommen wieder
zusammen. Da{\s} ist bequemer f"ur Sie!"'

"`Sehr g"utig! Ganz wie Sie w"unschen!"'

Al{\s} da{\s} Ehepaar dann allein war, fragte Karl:

"`Warum hast du eigentlich da{\s} Angebot de{\s} Herrn Boulanger
abgelehnt? E{\s} war doch sehr lieben{\s}w"urdig!"'

Emma tat, al{\s} ob sie schmollte; sie wu"ste nicht gleich, wa{\s}
sie sagen sollte, und schlie"slich erkl"arte sie, die Leute
k"onnten e{\s} "`komisch"' finden.

"`Ich pfeif auf die Leute!"' sagte Karl und machte eine
ver"achtliche Ge\-b"arde. "`Die Gesundheit ist tausendmal mehr
wert! Da{\s} war nicht richtig von dir!"'

"`Aber ich habe doch auch kein Reitkleid!"'

"`Dann mu"st du dir ein{\s} bestellen!"'

Da{\s} Reitkleid gab den Au{\s}schlag.

Al{\s} e{\s} fertig war, schrieb Bovary an Boulanger, seine Frau
stehe ihm zur Verf"ugung. Sie n"ahme sein g"utige{\s} Anerbieten
an.

Andern Tag{\s} um zw"olf Uhr hielt Rudolf mit zwei Reitpferden vor
dem Hause de{\s} Arzte{\s}. Da{\s} eine trug einen Damensattel
au{\s} Wildleder und einen roten Stirnriemen. Er selbst hatte hohe
Reitstiefel au{\s} feinstem weichen Leder an. Er nahm an, da"s
Emma solche gewi"s noch nie gesehen hatte; und in der Tat war sie
"uber sein Au{\s}sehen ent\/z"uckt, al{\s} sie ihn in seinem langen
dunkelbraunen Samtrock und den wei"sen Breeche{\s} an der T"ure
erblickte. Sie hatte auf ihn gewartet und war bereit.

Justin stahl sich au{\s} der Apotheke. Er mu"ste sie sehen. Auch
den Apotheker litt e{\s} nicht in seinem Laden. Er gab Rudolf
allerlei gute Ratschl"age.

"`E{\s} passiert so leicht ein Malheur!"' sagte er. "`Reiten Sie
vorsichtig! Sind die Tiere fromm?"'

Emma vernahm "uber sich ein Ger"ausch. E{\s} war Felicie, die mit
der Hand gegen eine Fensterscheibe trommelte, um der kleinen Berta
einen Spa"s zu bereiten. Da{\s} Kind warf der Mutter ein
Ku"sh"andchen zu. Die Reiterin winkte mit der Gerte.

"`Viel Vergn"ugen!"' rief Homai{\s}. "`Ja recht vorsichtig! Recht
vorsichtig!"'

Er sah den Wegreitenden noch lange nach und schwenkte gr"u"send
mit seiner Zeitung.

Sobald Emma{\s} Pferd weichen Boden unter sich f"uhlte, fing e{\s}
von selbst an zu galoppieren. Da sprengte auch Rudolf sein Pferd
an. Hin und wieder wechselten sie ein Wort. Da{\s} Kinn ein wenig
eingezogen, die hochgenommene linke Hand mit den Z"ugeln nach dem
Widerrist zu vorhaltend, so "uberlie"s sie sich der wiegenden
Galoppade.

E{\s} ging die Anh"ohe hinauf, immer im Galopp. Oben parierten die
G"aule pl"otzlich. Emma{\s} langer blauer Schleier flatterte
weiter.

E{\s} war einer der ersten Oktobertage. Nebel lag "uber den
Fluren. In langen Schwaden beengten sie den Gesicht{\s}krei{\s}
und lie"sen die H"ugel nur in Umri"slinien erkennen. Hin und
wieder rissen die Nebel au{\s}einander, flogen wie in Fetzen auf
und zerstoben. Dann erblickte man durch die L"ucken in der Ferne
die D"acher von Yonville im Sonnenscheine, die G"arten am
Bachufer, die Geh"ofte und Hecken und den Kirchturm. Emma gab sich
M"uhe, ihr Hau{\s} herau{\s}zufinden, und noch nie war ihr der
armselige Ort, in dem sie da lebte, so klein vorgekommen. Von der
H"ohe, auf der sie hielten, glich die ganze Niederung einem
ungeheuer gro"sen, fahlen, verdunstenden See. Die buschigen
B"aume, die hie und da au{\s} ihm herau{\s}ragten, sahen wie
schwarze Riffe au{\s}, und die Reihen der hohen Pappeln wie lange
Wellenz"uge, die der Wind kr"auselt.

"Uber dem Rasen unter den Tannen sickerte braune{\s} Licht durch
die laue Luft. Der Boden, r"otlich wie zerbl"atterter Tabak,
d"ampfte die Tritte. Abgefallene Tannenzapfen rollten "uber den
Weg, von den Hufen ber"uhrt.

Rudolf und Emma ritten den Waldsaum entlang. Ab und zu sah sie zur
Seite, um seinem Blicke zu entgehen; dann glitten die St"amme der
B"aume, einer nach dem andern, so rasch an ihr vor"uber, da"s die
unaufh"orliche Wiederholung sie halb schwindlig machte. Die Pferde
keuchten.

Gerade, al{\s} sie in den Wald kamen, trat die Sonne hervor.

"`Gott ist mit un{\s}!"' sagte Rudolf.

"`Glauben Sie denn an ihn?"' fragte sie.

"`Galopp! Galopp!"' rief er von neuem und schnalzte mit der Zunge.
Beide Tiere gehorchten.

Hohe Farne, wie sie zu beiden Seiten de{\s} Pfade{\s} standen,
verfingen sich in Emma{\s} Steigb"ugel. Rudolf, der zur Linken
Emma{\s} ritt, b"uckte sich jede{\s}mal im Weiterreiten und
befreite sie wieder. Ein paarmal galoppierte er ganz dicht neben
ihr hin, um "uberh"angende Zweige von ihr abzuwehren; dann f"uhlte
sie, wie sein rechte{\s} Knie ihr linke{\s} Bein ber"uhrte.

Inzwischen war der Himmel ganz blau geworden. Kein Blatt r"uhrte
sich. Sie kamen "uber weite Felder, ganz voll bl"uhenden
Heidekraut{\s}, und hie und da leuchteten unter dem grauen und
gelben und goldbraunen Bl"atterwerk der B"aume Flecke von wilden
Veilchen auf. Im Geb"usch regte sich "ofter{\s} leiser
Fl"ugelschlag. Leise kr"achzend flogen Raben um die Eichen.

Sie sa"sen ab. Rudolf band die Pferde an. Emma schritt ihm
vorau{\s}, den Weg weiter, "uber Moo{\s} in alten Wagenspuren. Ihr
lange{\s} Reitkleid erschwerte ihr da{\s} Gehen, obwohl sie e{\s}
mit der einen Hand aufgerafft hatte. Rudolf ging hinter ihr. Er
sah zwischen dem schwarzen Tuch und den schwarzen Stiefeln da{\s}
lockende Wei"s ihre{\s} Strumpfe{\s}, da{\s} er wie ein St"uck
Nacktheit empfand.

Emma blieb stehen.

"`Ich bin m"ude!"' sagte sie.

"`Gehen wir weiter! Versuchen Sie e{\s}!"' bat er. "`Mut!"'

Hundert Schritte weiter blieb sie abermal{\s} stehen. Der blaue
Schleier, der ihr von ihrem Herrenhute bi{\s} zu den H"uften
herabwallte, "ubergo"s ihr Gesicht mit bl"aulichem Licht. E{\s}
sah au{\s}, wie in da{\s} Blau de{\s} Himmel{\s} getaucht.

"`Wohin gehen wir denn?"'

Er gab keine Antwort. Sie atmete heftig. Rudolf hielt Umschau und
bi"s sich in den Schnurrbart. Sie standen in einer Lichtung, in
der gef"allte Baumst"amme dalagen. Sie setzten sich beide auf
einen.

Von neuem begann Rudolf, von seiner Liebe zu reden. Um Emma nicht
durch "Uberschwenglichkeit zu verprellen, blieb er ruhig, ernst,
schwerm"utig. Sie h"orte ihm gesenkten Haupte{\s} zu, w"ahrend sie
mit der Spitze ihre{\s} Stiefel{\s} den Waldboden aufscharrte.
Aber bei dem Satze:

"`Sind unsre beiden Leben{\s}pfade nunmehr nicht in einen
zusammengelaufen?"' unterbrach sie ihn:

"`Nein! Da{\s} wissen Sie doch! E{\s} ist unm"oglich!"'

Sie stand auf und wollte gehen. Er umfa"ste ihr Handgelenk, und so
blieb sie. Sie sah ihn eine kleine Weile liebevoll und mit feucht
schimmernden Augen an, dann sagte sie hastig:

"`Genug! Reden wir nicht mehr davon! Gehen wir zur"uck zu unsern
Pferden!"'

Rudolf machte eine Bewegung zornigen "Arger{\s}. Sie wiederholte:

"`Gehen wir zu unsern Pferden!"'

Da l"achelte er seltsam und n"aherte sich ihr mit vorgestreckten
H"anden, zusammengebissenen Z"ahnen und starrem Blicke. Sie wich
zitternd zur"uck und stammelte:

"`Ich f"urchte mich vor Ihnen! Sie tun mir weh! Gehen wir zur"uck!"'

"`Wenn e{\s} sein mu"s!"' gab er zur Antwort. Sein
Gesicht{\s}au{\s}druck wandelte sich. Er sah wieder ehrerbietig,
z"artlich, sch"uchtern au{\s}.

Emma reichte ihm den Arm. Sie traten den R"uckweg an.

"`Wa{\s} hatten Sie denn vorhin?"' fragte er. "`Wa{\s} war e{\s}?
Ich habe Sie nicht begriffen. Gewi"s haben Sie mich mi"sverstanden.
Sie thronen in meinem Herzen wie eine Madonna, hoch und hehr und
unerreichbar! Aber ich kann ohne Sie nicht leben! Ich mu"s Ihre
Augen sehen, Ihre Stimme h"oren, Ihre Gedanken wissen! Seien Sie
meine Freundin, meine Schwester, mein Schutzengel!"'

Er schlang seinen Arm um ihre Taille. Sie versuchte, sich ihm
sanft zu entwinden, aber er lie"s sie nicht lo{\s}. So gingen sie
nebeneinander hin. Da h"orten sie ihre Pferde, die Bl"atter von
den B"aumen rupften.

"`Noch nicht!"' bat Rudolf. "`Reiten wir noch nicht zur"uck!
Bleiben Sie!"'

Er zog sie mit sich vom Wege ab in die N"ahe eine{\s} kleinen
Weiher{\s}, dessen Spiegel mit Wasserlinsen bedeckt war. Zwischen
Schilf tr"aumten verwelkte Wasserrosen. Vor dem Ger"ausch ihrer
Schritte im Gra{\s} h"upften die Fr"osche davon und verschwanden.

"`E{\s} ist nicht recht von mir ... e{\s} ist nicht recht von mir!
Ich bin toll, da"s ich auf Sie h"ore!"'

"`Warum? Emma! Emma!"'

"`Ach, Rudolf!"' fl"usterte die junge Frau, indem sie sich an ihn
anschmiegte.

Da{\s} Tuch ihre{\s} Jackett{\s} lag dicht am Samt seine{\s}
Rocke{\s}. Sie bog ihren wei"sen Hal{\s} zur"uck, den ein Seufzer
schwellte. Halb ohnm"achtig und tr"anen"uberstr"omt, die H"ande
auf ihr Gesicht pressend und am ganzen Leib zitternd, gab sie sich
ihm hin~...

Die D"ammerung sank herab. Die Sonne stand blendend am Horizont
und flammte in den Zweigen. Hier und da, um die beiden herum, im
Laub und auf dem Boden, tanzten lichte Flecke, al{\s} h"atten
Kolibri{\s} im Vorbeifliegen ihre schimmernden Federn verloren.
Ring{\s} tiefe{\s} Schweigen. Die B"aume atmeten s"u"se
Melancholie.

Emma f"uhlte, wie ihr Herz wieder klopfte, wie ihr da{\s} Blut
durch den K"orper kreiste.

In der Ferne, hinter dem Walde, "uber der H"ohe ert"onte ein
langgezogener seltsamer Schrei, unaufh"orlich. Dem lauschte sie
schweigend. Er mischte sich in die verklingenden Schwingungen
ihrer zuckenden Nerven und ward zu Musik~...

Rudolf rauchte eine Zigarette und stellte mit Hilfe seine{\s}
Taschenmesser{\s} einen zerrissenen Z"ugel wieder her.

Auf demselben Wege ritten sie nach Yonville zur"uck. Sie sahen im
weichen Boden die Spuren ihre{\s} Hinritte{\s}, die Huftritte
beider Pferde dicht beieinander, sie erkannten die B"usche wieder
und einzelne Steine am Rain. Nicht{\s} um sie herum hatte sich
ver"andert, und doch kam e{\s} Emma vor, al{\s} sei etwa{\s}
h"ochst Bedeutsame{\s} geschehen, al{\s} seien die Berge von ihrem
Platze geschoben. Von Zeit zu Zeit beugte sich Rudolf zu ihr
her"uber, um ihre rechte Hand zu erfassen und zu k"ussen. Er fand
Emma im Sattel ent\/z"uckend au{\s}sehend, bei ihrem geraden Sitz,
ihrer schlanken Figur, der schicken Haltung ihre{\s} rechten
Knie{\s}, ihren von der scharfen Luft ger"oteten Wangen, --
alle{\s} im Abendrot.

Al{\s} sie Yonville erreichten, wurde ihr Pferd unruhig. Einmal
machte e{\s} sogar kehrt. Au{\s} allen Fenstern sah man ihr zu.

Beim Essen machte Karl die Bemerkung, Emma s"ahe vorz"uglich
au{\s}. Al{\s} er sich aber darnach erkundigte, wie der
Spazierritt gewesen sei, tat sie, al{\s} h"atte sie die Frage
"uberh"ort. Sie st"utzte sich auf die Ellenbogen und starrte "uber
ihren Teller weg in die flackernden Kerzen.

"`Emma!"'

"`Wa{\s} denn?"'

"`Wei"st du, ich bin heute nachmittag beim Pferdeh"andler gewesen.
Er hat eine recht gut au{\s}sehende alte Mutterstute zu verkaufen.
Die Knie sind nur ein bi"schen durch. Ich bin "uberzeugt, f"ur
hundert Taler~..."' Da sie nicht{\s} dazu sagte, fuhr er nach ein
paar Augenblicken fort: "`Ich habe gedacht, e{\s} sei dir
erw"unscht, und da habe ich mir den Gaul zur"uckstellen lassen ...
nein, gleich gekauft ... Ist{\s} dir recht? Sag mal!"'

Sie nickte bejahend mit dem Kopfe.

Eine Viertelstunde sp"ater fragte sie:

"`Gehst du heute abend au{\s}?"'

"`Ja. Warum denn?"'

"`Ach, ich wollt e{\s} blo"s wissen, Bester!"'

Sobald sie von Karl befreit war, ging sie in ihr Zimmer hinauf und
schlo"s sich ein.

Sie war zun"achst noch wie unter einem Banne. Sie sah im Geist die
B"aume, die Wege, die Gr"aben, den Geliebten und f"uhlte seine
Umarmung. Da{\s} Laub wisperte um sie herum, und da{\s} Schilf
rauschte. Dann aber erblickte sie sich im Spiegel. Sie staunte
"uber ihr Au{\s}sehen. So gro"se schwarze Augen hatte sie noch nie
gehabt! Und wie tief sie lagen! Etwa{\s} Unsagbare{\s} umflo"s
ihre Gestalt. Sie kam sich wie verkl"art vor.

Immer wieder sagte sie sich: "`Ich habe einen Geliebten! Einen
Geliebten!"'

Der Gedanke ent\/z"uckte sie. E{\s} war ihr, al{\s} sei sie jetzt
erst Weib geworden. Endlich waren die Liebe{\s}freuden auch f"ur
sie da, die fiebernde Gl"uckseligkeit, auf die sie bereit{\s}
keine Hoffnung mehr gehabt hatte! Sie war in eine Wunderwelt
eingetreten, in der alle{\s} Leidenschaft, Verz"uckung und Rausch
war. Blaue Unerme"slichkeit breitete sich ring{\s} um sie her, vor
ihrer Phantasie gl"anzte da{\s} Hochland der Gef"uhle, und fern,
tief unten, im Dunkel, weit weg von diesen H"ohen, lag der Alltag.

Sie erinnerte sich an allerlei Romanheldinnen, und diese Schar
empfindsamer Ehebrecherinnen sangen in ihrem Ged"achtnisse mit den
Stimmen der Klosterschwestern. Ent\/z"uckende Kl"ange! Jene
Phantasiegesch"opfe gewannen Leben in ihr; der lange Traum ihrer
M"adchenzeit ward zur Wirklichkeit. Nun war sie selber eine der
amoureusen Frauen, die sie so sehr beneidet hatte! Dazu da{\s}
Gef"uhl befriedigter Rache! Hatte sie nicht genug gelitten? Jetzt
triumphierte sie, und ihre so lange unterdr"uckte Sinnlichkeit
wallte nun auf und sch"aumte leben{\s}freudig "uber. Sie geno"s
ihre Liebe ohne Gewissen{\s}k"ampfe, ohne Nervosit"at, ohne
Wirrungen.

Der Tag darauf verging in neuem s"u"sen Gl"uck. Sie schworen sich
ewige Treue. Emma erz"ahlte ihm von ihren Leiden und Tr"ubsalen.
Er unterbrach sie mit K"ussen. Sie sah ihn mit halbgeschlossenen
Augen an und bat ihn immer wieder, sie bei ihrem Vornamen zu
nennen und ihr noch einmal zu sagen, da"s er sie liebe. E{\s} war
wiederum im Walde, in einer verlassenen Holzschuhmacherh"utte. Die
W"ande waren von Strohmatten und da{\s} Dach so niedrig, da"s man
drin nicht aufrecht stehen konnte. Sie sa"sen dicht beieinander
auf einer Streu von trocknem Laub.

Von diesem Tag an schrieben sie sich beide regelm"a"sig alle
Abende. Emma trug ihren Brief hinter in den Garten, wo sie ihn
unter einen lockeren Stein der kleinen Treppe, die zum Bach
f"uhrte, verbarg. Dort holte ihn Rudolf ab und legte einen von
sich hin. Seine Briefe waren sehr kurz, wor"uber sie sich alle
Tage beklagte.

Eine{\s} Morgen{\s}, da Karl bereit{\s} vor Sonnenaufgang
fortgegangen war, geriet sie pl"otzlich auf den Einfall,
unverweilt Rudolf sehen zu wollen. Ehe die Yonviller aufst"anden,
konnte sie nach der H"uchette gehen, eine Stunde dort verweilen
und wieder zur"uckkommen. Dieser Plan lie"s sie gar nicht recht
zur Besinnung kommen. Ein paar Augenblicke sp"ater war sie schon
mitten in den Wiesen. Ohne sich umzublicken, schritt sie eilig
ihre{\s} Weg{\s}.

Der Tag begann zu grauen. Schon von weitem erkannte sie da{\s} Gut
de{\s} Geliebten. Der Schwalbenschwanz der Wetterfahne auf dem
h"ochsten Giebel zeichnete sich schwarz vom fahlen Himmel ab.

"Uber den Hof weg stand ein gro"se{\s} Geb"aude. Da{\s} mu"ste
da{\s} Herrenhau{\s} sein. Dort trat sie ein. E{\s} war ihr,
al{\s} "offnete sich ihr alle{\s} von selbst. Eine breite Treppe
f"uhrte auf einen Gang. Emma dr"uckte auf die Klinke einer T"ur,
und da erblickte sie im Hintergrunde diese{\s} Zimmer{\s} einen
Mann im Bett. E{\s} war Rudolf. Sie frohlockte laut.

"`Du? Du!"' rief er au{\s}. "`Wie hast du da{\s} fertig gebracht?
Dein Kleid ist feucht~..."'

"`Ich liebe dich!"' war ihre Antwort, indem sie ihm die Arme um
den Hal{\s} schlang.

Nachdem ihr diese{\s} Wagni{\s} beim ersten Male gegl"uckt war,
kleidete sich Emma jede{\s}mal, wenn Karl fr"uhzeitig fort mu"ste,
rasch an und schlich sich wie ein Wiesel durch die hintere
Gartenpforte, auf dem Treppchen, da{\s} hinunter nach dem Bache
f"uhrte, au{\s} dem Hause. Aber wenn die Planke, die al{\s} Steg
"uber da{\s} Wasser diente, zuf"allig weggenommen war, mu"ste sie
ein St"uck bi{\s} zum n"achsten Steg an den Gartenmauern l"ang{\s}
de{\s} Bache{\s} hingehen. Die bewachsene B"oschung war steil und
glitschig, und so mu"ste sie sich mit der einen Hand an B"uscheln
der vertrockneten Mauerblumen festhalten, um nicht zu fallen. Dann
aber eilte sie querfeldein "uber die "Acker, ungeachtet, da"s ihre
zierlichen Schuhe einsanken, da"s sie oft stolperte oder stecken
blieb. Da{\s} Chiffontuch, da{\s} sie sich um Kopf und Hal{\s}
gewunden hatte, flatterte im Winde. Au{\s} Angst vor den weidenden
Ochsen begann sie zu laufen. Atemlo{\s}, mit gl"uhenden Wangen,
ganz vom frischen Duft der Natur, ihrer S"afte, ihre{\s} Gr"un{\s}
und der freien Luft durchtr"ankt, kam sie an. Rudolf schlief dann
meist noch. Sie kam zu ihm in sein Gemach wie der
leibhaftgewordene Fr"uhling{\s}morgen.

Die gelben Gardinen vor den Fenstern machten da{\s} eindringende
goldene Morgenlicht traulich und d"ammerig. Mit blinzelnden Augen
fand sich Emma zurecht. Die Tautropfen an ihren Gew"andern
leuchteten wie Topase und verliehen ihr etwa{\s} Feenhafte{\s}.
Rudolf zog sie lachend zu sich und dr"uckte sie an sein Herz.

Darnach sah sie sich im Zimmer alle{\s} an, zog alle F"acher auf,
k"ammte sich mit seinem Kamm und betrachtete sich in seinem
Rasierspiegel. Mitunter nahm sie seine gro"se Tabak{\s}pfeife in
den Mund, die auf dem Nachttisch lag, zwischen Zitronen und
Zuckerst"ucken, neben der Wasserflasche.

Zum Abschiednehmen brauchten sie immer eine Viertelstunde. Emma
vergo"s Tr"anen. Am liebsten w"are sie gar nicht wieder von ihm
weggegangen. Eine unwiderstehliche Gewalt trieb sie immer von
neuem in seine Arme.

Da eine{\s} Tage{\s}, al{\s} er sie unerwartet eintreten sah,
machte er ein bedenkliche{\s} Gesicht, al{\s} ob e{\s} ihm nicht
recht w"are.

"`Wa{\s} hast du denn?"' fragte sie. "`Hast du Schmerzen?
Sprich!"'

Schlie"slich erkl"arte er ihr in ernstem Tone, ihre Besuche
beg"onnen unvorsichtig zu werden. Sie kompromittiere sich.


\newpage\begin{center}
{\large \so{Zehnte{\s} Kapitel}}\bigskip\bigskip
\end{center}

Allm"ahlich machten Rudolf{\s} Bef"urchtungen auf Emma Eindruck.
Zuerst hatte die Liebe sie berauscht, und so hatte sie an
nicht{\s} andre{\s} gedacht. Jetzt aber, da ihr diese Liebe zu
einer Leben{\s}bedingung geworden war, erwachte die Furcht in ihr,
e{\s} k"onne ihr etwa{\s} davon verloren gehen oder man k"onne sie
ihr gar st"oren. Wenn sie von dem Geliebten wieder heimging, hielt
sie mit rastlosen Blicken Umschau; sie sp"ahte nach allem, wa{\s}
sich im Gesicht{\s}kreise regte, sie suchte die H"auser de{\s}
Orte{\s} bi{\s} hinauf in die Dachluken ab, ob jemand sie
beobachte. Sie lauschte auf jede{\s} Ger"ausch, jeden Tritt,
jede{\s} R"adergeknarr. Manchmal blieb sie stehen, blasser und
zittriger al{\s} da{\s} Laub der Pappeln, die sich "uber ihrem
Haupte wiegten.

Eine{\s} Morgen{\s}, auf dem Heimwege, erblickte sie mit einem
Male den Lauf eine{\s} Gewehr{\s} auf sich gerichtet. E{\s} ragte
schr"ag "uber den oberen Rand einer Tonne hervor, die zur H"alfte
in einem Graben stand und vom Geb"usch verdeckt wurde. Vor Schreck
halb ohnm"achtig ging Emma dennoch weiter. Da tauchte ein Mann
au{\s} der Tonne wie ein Springteufel au{\s} seinem Kasten. Er
trug Wickelgamaschen bi{\s} an die Knie, und die M"utze hatte er
tief in{\s} Gesicht hereingezogen, so da"s man nur eine rote Nase
und bebende Lippen sah. E{\s} war der Feuerwehrhauptmann Binet,
der auf dem Anstand lag, um Wildenten zu schie"sen.

"`Sie h"atten schon von weitem rufen sollen!"' schrie er ihr zu.
"`Wenn man ein Gewehr sieht, mu"s man sich bemerkbar machen!"'

Der Steuereinnehmer suchte durch seine Grobheit seine eigene Angst
zu bem"anteln. E{\s} bestand n"amlich eine landr"atliche
Verordnung, nach der man die Jagd auf Wildenten nur vom Kahne
au{\s} betreiben durfte. Bei allem Respekt vor den Gesetzen machte
sich also Binet einer "Ubertretung schuldig. De{\s}halb schwebte
er in steter Furcht, der Landgendarm k"onne ihn erwischen, und
doch f"ugte die Aufregung seinem Vergn"ugen einen Reiz mehr zu.
Wenn er so einsam in seiner Tonne sa"s, war er stolz auf sein
Jagdgl"uck und seine Schlauheit.

Al{\s} er erkannte, da"s e{\s} Frau Bovary war, fiel ihm ein
gro"ser Stein vom Herzen. Er begann sofort ein Gespr"ach mit ihr.

"`E{\s} ist kalt heute! Ordentlich kalt!"'

Emma gab keine Antwort. Er fuhr fort:

"`Sie sind heute schon zeitig auf den Beinen?"'

"`Jawohl!"' stotterte sie. "`Ich war bei den Leuten, wo mein Kind
ist..."'

"`So so! Na ja! Und ich! So wie Sie mich sehen, sitze ich schon
seit Morgengrauen hier. Aber da{\s} Wetter ist so ruppig, da"s man
auch nicht einen Schwanz vor die Flinte kriegt~..."'

"`Adieu, Herr Binet!"' unterbrach sie ihn und wandte sich kurz von
ihm ab.

"`Ihr Diener, Frau Bovary!"' sagte er trocken und kroch wieder in
seine Tonne.

Emma bereute e{\s}, den Steuereinnehmer so unfreundlich stehen
gelassen zu haben. Zweifello{\s} hegte er allerlei ihr nachteilige
Vermutungen. Auf eine d"ummere Au{\s}rede h"atte sie auch wirklich
nicht verfallen k"onnen, denn in ganz Yonville wu"ste man, da"s
da{\s} Kind schon seit einem Jahre wieder bei den Eltern war. Und
sonst wohnte in dieser Richtung kein Mensch. Der Weg f"uhrte
einzig und allein nach der H"uchette. Somit mu"ste Binet erraten,
wo Emma gewesen war. Sicherlich w"urde er nicht schweigen, sondern
e{\s} au{\s}klatschen! Bi{\s} zum Abend marterte sie sich ab, alle
m"oglichen L"ugen zu ersinnen. Immer stand ihr dieser Idiot mit
seiner Jagdtasche vor Augen.

Al{\s} Karl nach dem Essen merkte, da"s Emma bek"ummert war,
schlug er ihr vor, zur Zerstreuung mit zu "`Apotheker{\s}"' zu
gehen.

Die erste Person, die sie schon von drau"sen in der Apotheke im
roten Lichte erblickte, war -- au{\s}gerechnet -- der
Steuereinnehmer. Er stand an der Ladentafel und sagte gerade:

"`Ich m"ochte ein Lot Vitriol."'

"`Justin,"' schrie der Apotheker, "`bring mir mal die Schwefels"aure
her!"' Dann wandte er sich zu Frau Bovary, die die Treppe zum
Zimmer von Frau Homai{\s} hinaufgehen wollte.

"`Ach, bleiben Sie nur gleich unten! Meine Frau kommt jeden
Augenblick herunter. W"armen Sie sich inzwischen am Ofen ...
Entschuldigen Sie!"' Und zu Bovary sagte er: "`Guten Abend,
Doktor!"' Der Apotheker pflegte n"amlich diesen Titel mit einer
gewissen Vorliebe in den Mund zu nehmen, al{\s} ob der Glanz, der
darauf ruhte, auch auf ihn ein paar Strahlen w"urfe. "`Justin,
nimm dich aber in acht und wirf mir die M"orser nicht um! So! Und
nun holst du ein paar St"uhle au{\s} dem kleinen Zimmer! Aber
nicht etwa die Fauteuil{\s} au{\s} dem Salon! Verstanden?"'

Homai{\s} wollte selber zu seinen Fauteuil{\s} st"urzen, aber
Binet bat noch um ein Lot Zuckers"aure.

"`Zuckers"aure?"' fragte der Apotheker eingebildet. "`Kenne ich
nicht! Gibt e{\s} nicht! Sie meinen wahrscheinlich Oxals"aure?
Also Oxals"aure, nicht wahr?"'

Der Steuereinnehmer setzte ihm au{\s}einander, da"s er nach einem
selbsterfundenen Rezepte ein Putzwasser herstellen wollte, zur
Reinigung von verrostetem Jagdger"at.

Bei dem Wort "`Jagd"' schrak Emma zusammen.

Der Apotheker versetzte:

"`Gewi"s! Bei solch schlechtem Wetter braucht man da{\s}!"'

"`E{\s} gibt aber doch Leute, die e{\s} nicht anficht!"' meinte
Binet bissig.

Emma bekam keine Luft.

"`Und dann m"ocht ich noch~..."'

"`Will er denn ewig hier bleiben!"' seufzte sie bei sich.

"`... je ein Lot Kolophonium und Terpentin, acht Lot gelbe{\s}
Wach{\s} und sieben Lot Knochenkohle, bitte! Zum Polieren
meine{\s} Lederzeug{\s}."'

Der Apotheker wollte gerade da{\s} Wach{\s} abschneiden, al{\s}
seine Frau erschien, die kleine Irma im Arme, Napoleon zur Seite,
und Athalia hinterdrein. Sie setzte sich auf die mit Pl"usch
"uberzogene Fensterbank. Der Junge l"ummelte sich auf einen
niedrigen Sessel, w"ahrend sich seine "altere Schwester am Kasten
mit den Malzbonbon{\s} zu schaffen machte, in n"achster N"ahe von
"`Papachen"', der mit dem Trichter hantierte, die Fl"aschchen
verkorkte, Etiketten darauf klebte und dann alle{\s} zu einem
Paket verpackte. Um ihn herrschte Schweigen. Man h"orte nicht{\s},
al{\s} von Zeit zu Zeit da{\s} Klappern der Gewichte auf der Wage
und ein paar leise anordnende Worte, die der Apotheker dem
Lehrling erteilte.

"`Wie geht{\s} Ihrem T"ochterchen?"' fragte pl"otzlich Frau
Homai{\s}.

"`Ruhe!"' rief ihr Gatte, der den Betrag in da{\s}
Gesch"aft{\s}buch eintrug.

"`Warum haben Sie{\s} nicht mitgebracht?"' fragte sie weiter.

"`Sst! Sst!"' machte Emma und wie{\s} mit dem Daumen nach dem
Apotheker.

Binet, der in die erhaltene Nota ganz vertieft war, schien nicht
darauf geh"ort zu haben. Endlich ging er. Erleichtert stie"s Emma
einen lauten Seufzer au{\s}.

"`Bi"schen asthmatisch?"' bemerkte Frau Homai{\s}.

"`Ach nein, e{\s} ist nur recht hei"s hier!"' entgegnete Frau
Bovary.

Alle{\s} da{\s} hatte zur Folge, da"s die Liebenden tag{\s} darauf
beschlossen, ihre Zusammenk"unfte ander{\s} einzurichten. Emma
schlug vor, ihr Hau{\s}m"adchen in{\s} Vertrauen zu ziehen und
durch ein Geschenk mundtot zu machen. Rudolf aber hielt e{\s} f"ur
besser, in Yonville irgendein stille{\s} Winkelchen au{\s}findig
zu machen. Er versprach, sich darnach umzusehen.

Den ganzen Winter "uber kam er drei- oder viermal in der Woche bei
Anbruch der Nacht in den Garten. Emma hatte ihm den Schl"ussel zur
Hinterpforte gegeben, w"ahrend Karl glaubte, er sei verloren
gegangen. Zum Zeichen, da"s er da war, warf Rudolf jede{\s}mal
eine Handvoll Sand gegen die Jalousien. Emma erhob sich daraufhin,
aber oft mu"ste sie noch warten, denn Karl hatte die Angewohnheit,
am Kamine zu sitzen und in{\s} Endlose hinein zu plaudern. Emma
verging beinahe vor Ungeduld und w"unschte ihren Mann wer wei"s
wohin. Schlie"slich begann sie ihre Nachttoilette zu machen; dann
nahm sie ein Buch zur Hand und tat so, al{\s} sei da{\s} Buch
"uber alle Ma"sen fesselnd. Karl ging indessen zu Bett und rief
ihr zu, sie solle auch schlafen gehn.

"`Komm doch, Emma!"' rief er. "`E{\s} ist schon sp"at!"'

"`Gleich! Gleich!"' erwiderte sie.

Da{\s} Kerzenlicht blendete ihn. Er drehte sich gegen die Wand und
schlief ein. Sie schl"upfte hinau{\s}, mit verhaltenem Atem,
l"achelnd, zitternd, halbnackt.

Rudolf h"ullte sie ganz mit hinein in seinen weiten Mantel,
schlang die Arme um sie und zog sie wortlo{\s} hinter in den
Garten, in die Laube, auf die morsche Holzbank, auf der sie
dereinst so oft mit Leo gesessen hatte. Da{\s} war an
Sommerabenden gewesen. Wie verliebt hatten seine Augen
geschimmert! Aber jetzt dachte Emma nicht mehr an ihn.

Durch die kahlen Zweige der Ja{\s}minb"usche funkelten die Sterne.
Hinter dem Paare rauschte der Bach, und hin und wieder knackte am
Ufer da{\s} vertrocknete hohe Schilf. Manchmal formte e{\s} sich
im Dunkel zu einem massigen Schatten, der mit einem Male Leben
bekam, sich emporrichtete und wieder neigte und wie ein
schwarze{\s} Unget"um auf die beiden zuzukommen schien, um sie zu
erdr"ucken.

In der K"alte der Nacht wurden ihre Umarmungen um so inniger und
ihr Liebe{\s}gestammel um so inbr"unstiger. Ihre Augen, die sie
gegenseitig kaum erkennen konnten, erschienen ihnen gr"o"ser, und
in der Stille ring{\s}um bekamen ihre ganz leise gefl"usterten
Worte einen kristallenen Klang, drangen tief in die Seelen und
zitterten in ihnen tausendfach wider.

Wenn die Nacht regnerisch war, fl"uchteten sie in Karl{\s}
Sprechzimmer, da{\s} zwischen dem Wagenschuppen und dem
Pferdestall gelegen war. Emma z"undete eine K"uchenlampe an, die
sie hinter den B"uchern bereitgestellt hatte. Rudolf machte
sich{\s} bequem, al{\s} sei er zu Hause. Der Anblick der
"`Bibliothek"', de{\s} Schreibtische{\s}, der ganzen Einrichtung
erregte seine Heiterkeit. Er konnte nicht umhin, "uber Karl
allerhand Witze zu machen, wa{\s} Emma ungern h"orte. Sie h"atte
ihn viel lieber ernst sehen m"ogen, ihretwegen theatralischer, wie
er e{\s} einmal gewesen war, al{\s} sie in der Pappelallee da{\s}
Ger"ausch von n"aherkommenden Tritten hinter sich zu vernehmen
w"ahnten.

"`E{\s} kommt jemand!"' sagte sie einmal.

Er blie{\s} da{\s} Licht au{\s}.

"`Hast du eine Pistole bei dir?"'

"`Wozu?"'

"`Damit du ... dich ... verteidigen kannst!"'

"`Gegen deinen Mann? Der arme Junge!"' Dazu machte er eine
Geb"arde, die etwa sagen sollte: "`Der mag mir nur kommen!"'

Dieser Mut ent\/z"uckte sie, wenngleich sie die Unzartheit und
urw"uchsige Roheit herau{\s}h"orte und dar"uber entsetzt war.

Rudolf dachte viel "uber diese kleine Szene nach.

"`Wenn da{\s} ihr Ernst war,"' sagte er sich, "`so war da{\s}
recht l"acherlich, sogar h"a"slich."' Er hatte doch wahrlich
keinen Anla"s, ihren gutm"utigen Mann zu hassen. Sozusagen "`von
Eifersucht verzehrt"', da{\s} war er nicht. "Uberdie{\s} hatte ihm
Emma ihre k"orperliche Treue mit einem feierlichen Eid beteuert,
der ihm ziemlich abgeschmackt erschienen war. "Uberhaupt fing sie
an, recht sentimental zu werden. Er hatte Miniaturbildnisse mit
ihr tauschen m"ussen, und sie hatten sich alle beide eine ganze
Handvoll Haare f"ur einander abgeschnitten, und jetzt w"unschte
sie sich sogar einen wirklichen Ehering von ihm, zum Zeichen
ewiger Zusammengeh"origkeit. H"aufig schw"armte sie ihm von den
Abendglocken vor oder von den Stimmen der Natur. Oder sie
erz"ahlte von ihrer seligen Mutter und wollte von der seinigen
etwa{\s} wissen. Rudolf{\s} Mutter war schon zwanzig Jahre tot.
Trotzdem tr"ostete ihn Emma mit allerlei Koseworten der
Klein-Kindersprache, al{\s} ob e{\s} g"olte, ein Wickelkind zu
beruhigen. Mehr al{\s} einmal hatte sie, zu den Sternen
aufblickend, au{\s}gerufen:

"`Ich glaube fest, da droben, unsre beiden M"utter segnen unsre
Liebe!"'

Aber sie war so h"ubsch! Und eine so unverdorbene Frau hatte er
noch nie besessen. Solch eine Liebschaft ohne Unz"uchtigkeiten war
ihm, der da{\s} Verdorbenste kannte, etwa{\s} ganz Neue{\s},
da{\s} seinen Manne{\s}stolz und seine Sinnlichkeit verf"uhrerisch
umschmeichelte. Selbst Emma{\s} "Uberschwenglichkeiten, so zuwider
sie einem Naturmenschen wie ihm waren, fand er bei n"aherer
Betrachtung reizend, da sie doch ihm galten. Aber weil er so
sicher war, da"s er geliebt wurde, lie"s er sich gehen, und
allm"ahlich "anderte sich sein Benehmen.

Nicht mehr wie einst hatte er f"ur sie jene s"u"sen Worte, die
Emma zu Tr"anen r"uhrten, nicht mehr die st"urmischen
Liebkosungen, die sie toll gemacht hatten. Und so kam e{\s} ihr
vor, al{\s} ob der Strom ihrer eignen gro"sen Liebe, in der sie
v"ollig untergetaucht war, niedriger w"urde; sie sah gleichsam auf
den schlammigen Grund. Vor dieser Erkenntni{\s} schauderte sie,
und darum verdoppelte sie ihre Z"artlichkeiten. Rudolf indessen
verriet seine Gleichg"ultigkeit immer mehr.

Emma war sich selber nicht klar dar"uber, ob sie e{\s} bereuen
m"usse, sich ihm geschenkt zu haben, oder ob e{\s} nicht besser
f"ur sie sei, wenn sie ihn noch viel mehr liebte. Dann aber begann
sie ihre Schwachheit al{\s} Schmach zu empfinden, und der Groll
dar"uber beeintr"achtigte ihr den sinnlichen Genu"s. Sie gab sich
ihm nicht mehr hin, sie lie"s sich jede{\s}mal von neuem
verf"uhren. Aber er meisterte sie, und sie f"urchtete sich beinahe
vor ihm.

Ihre Beziehungen zueinander gewannen nach au"sen ein harmlose{\s}
Gepr"age wie nie zuvor. Da{\s} war so recht nach Rudolf{\s}
Wunsch. So war ihm der Ehebruch recht. Nach einem halben Jahre,
al{\s} der Fr"uhling in{\s} Land kam, waren sie fast wie zwei
Eheleute zueinander, die ihre Liebe{\s}opfer an der gem"utlichen
Flamme de{\s} h"au{\s}lichen Herde{\s} bringen.

Um diese Zeit schickte Vater Rouault wie allj"ahrlich eine
Truthenne zur Erinnerung an da{\s} geheilte Bein. Mit der Gabe
kam, wie immer, ein Brief. Emma zerschnitt den Bindfaden, mit dem
er an den Korb gebunden war, und la{\s} die folgenden Zeilen:

"`Meine liben Kinder, hofentlig trift euch di hir gesund und wol
und i{\s} si so gut wi di fr"ueren. Mir komt sie n"amlig ein
bissel zarter vor sozusagen nich so kombakt, da{\s} n"achste mal
schik ich euch zur abwek{\s}lung mal einen Han oder wolt "ur liber
ein par junge un schikt mir den Korb zer"uk, bite un auch di
vorgen, ich hab Ungl"uk mit der r"omise gehabt der ihr Dach ist
mir neulig nacht{\s} bei dem grosen Sturm in die B"aume geflogen,
die ernte ist die{\s}mal nich besonder{\s} ber"umt. Kurz und gut
ich wei{\s} nicht wan ich zu euch zu besuch kome, da{\s} ist jez
so ne Sache, ich kan schwer vom Hofe weg seit ich allein bin meine
arme Emma."'

Hier war ein gro"ser Absatz, al{\s} ob der gute Mann seine Feder
hingelegt hatte, um dazwischen eine Weile zu tr"aumen.

"`Wa{\s} mich anbelangt so geht{\s} mir leidlig bi{\s} auf den
Schnuppen den ich mir neulig auf der messe in Yvetot geholt hab wo
ich war, einen neuen Sch"afer zu mieten. Den alten hab ich n"amlig
nau{\s}geschmisen wegen seiner Grosen klape. E{\s} i{\s} wirklig
schrecklig mit diesen Gesindel, mausen tat er "ubrigen{\s} auch.

"`Von nem Hausierer der vergangnen Winter durch eure Gegend
gekomen i{\s} und sich bei euch nen Zan hat zihn lasen, hab ich
vernomen da{\s} Karl imer feste ze tun hat. Da{\s} wundert mich
kar nich und den Zan hat er mir gezeigt. Ich hab in zu ner tase
Kafee dabehalten. Ich fragt in ob er dich auch gesehen hat, da
sagte er Nein aber im Stale h"ate er zwei G"aule stehn sehn
worau{\s} ich schlise da{\s} der kurkenhandel bei euch gut geht.
Da{\s} freut mich sehr meine liben Kinder der libe got m"og euch
ale{\s} m"oglige Gl"uk schenken. E{\s} tut mir s"or leid da{\s}
ich mein libe{\s} Enkelkind Berta Bovary noch imer nich kene. Ich
habe f"ur si unter deiner Stube ein Flaumenb"aumgen geflanzt.
Da{\s} sol nich anger"urt werden auser sp"ater um die Flaumen f"ur
Berta einzumagen. Di werde ich dan im schrank aufheben und wen si
komt krigt si imer welge. Adi"o libe Kinder. Ig k"use dich libe
Emma un auch dich liber Schwigerson und di kleine auf ale beide
Baken un verbleibe mit tausen Gr"usen euer euch \nopagebreak

\hfill libender vater \hspace{7em}\nopagebreak

\hfill Theodor Rouault."' \hspace{5em}

Ein paar Minuten hielt sie da{\s} St"uck grobe{\s} Papier noch
nach dem Lesen in den H"anden. Die Verst"o"se gegen die
Rechtschreibung jagten sich in den v"aterlichen Zeilen nur so,
aber Emma ging einzig und allein dem lieben Geist darin nach, der
wie eine Henne au{\s} einer dicken Dornenhecke allenthalben
hervorgackerte. Rouault hatte die noch nassen Schrift\/z"uge
offenbar mit Herdasche getrocknet, denn au{\s} dem Briefe rieselte
eine Menge grauen Staube{\s} auf da{\s} Kleid der Leserin. Sie
glaubte, den Vater geradezu leibhaftig vor sich zu sehen, wie er
sich nach dem Aschekasten b"uckte. Ach, wie lange war e{\s} schon
her, da"s sie nicht mehr bei ihm war! Im Geiste sah sie sich
wieder auf der Bank am Herde sitzen, wie sie da{\s} Ende eine{\s}
Stecken{\s} an der gro"sen Flamme de{\s} Funken spr"uhenden
Ginsterreisig{\s} anbrennen lie"s. Und dann dachte sie zur"uck an
gewisse sonnendurchgl"uhte Sommerabende, wo die F"ullen so hell
aufwieherten, wenn man in ihre N"ahe kam, und dann
weggaloppierten. Diese drolligen Galoppspr"unge! Im Vaterhause,
unter ihrem Fenster, da stand ein Bienenkorb, und manchmal waren
die Bienen, wenn sie in der Sonne au{\s}schw"armten, gegen die
Scheiben geflogen wie fliegende Goldkugeln. Da{\s} war doch
eigentlich eine gl"uckliche Zeit gewesen! Voller Freiheit! Voller
Erwartung und voller Illusionen! Nun waren sie alle zerronnen! Bei
dem, wa{\s} sie erlebt, hatte sie ihre Seele verbraucht, in allen
den verschiedenen Abschnitten ihre{\s} Dasein{\s}, al{\s}
junge{\s} M"adchen, dann al{\s} Gattin, zuletzt al{\s} Geliebte.
Sie hatte von ihrer Seele verloren in einem fort, wie jemand, der
auf einer Reise in jedem Gasthause immer ein St"uck von seinen
Habseligkeiten liegen l"a"st.

Aber warum war sie denn so ungl"ucklich? Wa{\s} war Bedeutsame{\s}
geschehen, da"s sie mit einem Male au{\s} allen Himmeln gest"urzt
war? Sie erhob sich und blickte um sich, gleichsam al{\s} suche
sie den Anla"s ihre{\s} Herzeleid{\s}.

Ein Strahl der Aprilsonne glitzerte auf dem Porzellan de{\s}
Wandbrette{\s}. Im Kamin war Feuer. Durch ihre Hau{\s}schuhe
hindurch sp"urte sie den weichen Teppich. E{\s} war ein heller
Fr"uhling{\s}tag, und die Luft war lau.

Da h"orte sie, wie ihr Kind drau"sen laut aufjauchzte.

Die kleine Berta rutschte im Grase herum. Da{\s} Kinderm"adchen
wollte sie am Kleide wieder in die H"ohe ziehen. Lestiboudoi{\s}
war dabei, den Rasen zu scheren. Jede{\s}mal, wenn er in die N"ahe
de{\s} Kinde{\s} kam, streckte e{\s} ihm beide "Armchen entgegen.

"`Bring sie mir mal herein!"' rief sie dem M"adchen zu und ri"s
ihr T"ochterchen hastig an sich, um e{\s} zu k"ussen. "`Wie ich
dich liebe, mein arme{\s} Kind! Wie ich dich liebe!"'

Al{\s} sie bemerkte, da"s e{\s} am Ohre etwa{\s} schmutzig war,
klingelte sie rasch und lie"s sich warme{\s} Wasser bringen. Sie
wusch die Kleine, zog ihr frische W"asche und reine Str"umpfe an.
Dabei tat sie tausend Fragen, wie e{\s} mit der Gesundheit der
Kleinen stehe, just al{\s} sei sie von einer Reise zur"uckgekehrt.
Schlie"slich k"u"ste sie sie noch einmal und gab sie tr"anenden
Auge{\s} dem M"adchen wieder. Felicie war ganz verdutzt "uber
diesen Z"artlichkeit{\s}anfall der Mutter.

Am Abend fand Rudolf, Emma sei nachdenklicher denn sonst.

"`Eine vor"ubergehende Laune!"' tr"ostete er sich.

Dreimal hintereinander vers"aumte er da{\s} Stelldichein. Al{\s}
er wieder erschien, behandelte sie ihn k"uhl, fast geringsch"atzig.

"`Schade um die Zeit, mein Liebchen!"' meinte er. Und er tat so,
al{\s} merke er weder ihre sentimentalen Seufzer noch da{\s}
Taschentuch, da{\s} sie herau{\s}zog.

Jetzt kam wirklich die Reue "uber sie. Sie fragte sich, au{\s}
welchem Grunde sie eigentlich ihren Mann hasse und ob e{\s} nicht
besser gewesen w"are, wenn sie ihm treu h"atte bleiben k"onnen.
Aber Karl bot ihr keine besondere Gelegenheit, ihm ihren
Gef"uhl{\s}wandel zu offenbaren. Wenn der Apotheker nicht
zuf"allig eine solche heraufbeschworen h"atte, w"are alle ihre
hingebung{\s}volle Anwandlung tatenlo{\s} geblieben.


\newpage\begin{center}
{\large \so{Elfte{\s} Kapitel}}\bigskip\bigskip
\end{center}

Homai{\s} hatte letzthin die Lobpreisung einer neuen Methode,
Klumpf"u"se zu heilen, gelesen, und al{\s} Fortschrittler, der er
war, verfiel er sofort auf die partikularistische Idee, auch in
Yonville m"usse e{\s} strephopodische Operationen geben, damit
e{\s} auf der H"ohe der Kultur bleibe.

"`Wa{\s} ist denn dabei zu ri{\s}kieren?"' fragte er Frau Bovary.
Er z"ahlte ihr die Vorteile eine{\s} solchen Versuche{\s} an den
Fingern auf. Erfolg so gut wie sicher. Wiederherstellung de{\s}
Kranken. Befreiung von einem Sch"onheit{\s}fehler. Bedeutende
Reklame f"ur den Operateur. "`Warum soll Ihr Herr Gemahl nicht
beispiel{\s}weise den armen Hippolyt vom Goldnen L"owen kurieren?
Bedenken Sie, da"s er seine Heilung allen Reisenden erz"ahlen
w"urde. Und dann~..."' Der Apotheker begann zu fl"ustern und
blickte scheu um sich, "`... wa{\s} sollte mich daran hindern,
eine kleine Notiz dar"uber in die Zeitung zu bringen? Du mein
Gott! So ein Artikel wird "uberall gelesen ... man spricht davon
... schlie"slich wei"s e{\s} die ganze Welt. Au{\s} Schneeflocken
werden am Ende Lawinen! Und wer wei"s? Wer wei"s?"'

Warum nicht? Bovary konnte in der Tat Erfolg haben. Emma hatte gar
keinen Anla"s, Karl{\s} chirurgische Geschicklichkeit zu
bezweifeln, und wa{\s} f"ur eine Befriedigung w"are e{\s} f"ur
sie, die geistige Urheberin eine{\s} Entschlusse{\s} zu sein, der
sein Ansehen und seine Einnahmen steigern mu"ste. Sie verlangte
mehr al{\s} blo"s die Liebe diese{\s} Manne{\s}.

Vom Apotheker und von seiner Frau best"urmt, lie"s sich Karl
"uberreden. Er bestellte sich in Rouen da{\s} Werk de{\s}
Doktor{\s} D"uval, und nun vertiefte er sich jeden Abend, den Kopf
zwischen den H"anden, in diese Lekt"ure. W"ahrend er sich "uber
Pferdefu"sbildungen, Varu{\s} und Valgu{\s}, Strephocatopodie,
Strephendopodie, Strephexopodie (d.h. "uber die verschiedenartigen
inneren und "au"serlichen Verkr"uppelungen de{\s} menschlichen
Fu"se{\s}), Strephypopodie und Strephanopodie (da{\s} sind
Fu"sleiden, die oberhalb oder unterhalb der Verkr"uppelung um sich
greifen) unterrichtete, suchte Homai{\s} den Hau{\s}knecht vom
Goldnen L"owen mit allen Mitteln der "Uberredung{\s}kunst zur
Operation zu bewegen.

"`Du wirst h"ochsten{\s} einen ganz leichten Schmerz sp"uren"',
sagte er zu ihm. "`E{\s} ist nicht{\s} weiter al{\s} ein Einstich
wie beim Aderlassen, nicht schlimmer, al{\s} wenn du dir ein
H"uhnerauge schneiden l"a"st."'

Hippolyt{\s} bl"ode Augen blickten unschl"ussig um sich.

"`Im "ubrigen"', fuhr der Apotheker fort, "`kann mir{\s}
nat"urlich ganz egal sein. Dein Nutzen ist e{\s}. Ich rate dir{\s}
nur au{\s} purer N"achstenliebe. Mein lieber Freund, ich m"ochte
dich gar zu gern von deinem scheu"slichen Hinkfu"s befreit sehen,
von diesem ewigen Hin- und Herwackeln mit den H"uften. Du kannst
dagegen sagen, wa{\s} du willst: e{\s} st"ort dich in der
Au{\s}"ubung deine{\s} Beruf{\s} doch erheblich!"'

Nun schilderte ihm Homai{\s}, wie frei und flott er sich nach
einer Operation werde bewegen k"onnen. Auch gab er ihm zu
verstehen, da"s er dann mehr Gl"uck bei den Weibern haben w"urde,
wor"uber der Bursche albern grinste.

"`Schockschwerebrett! Du bist doch auch ein Mann! Du h"attest doch
auch nicht kneifen k"onnen, wenn man dich zu den Soldaten
au{\s}gehoben und in den Krieg geschickt h"atte! Also Hippolyt!"'

Homai{\s} wandte sich von ihm ab und meinte, so ein Dickkopf sei
ihm noch nicht vorgekommen. Er begreife nicht, wie man sich den
Wohltaten der Wissenschaft derartig st"orrisch ent\/ziehen k"onne.

Endlich gab der arme Schlucker nach. Da{\s} war ja die reine
Verschw"orung gegen ihn! Binet, der sich sonst niemal{\s} um die
Angelegenheiten anderer k"ummerte, die L"owenwirtin, Artemisia,
die Nachbarn und selbst der B"urgermeister, alle drangen sie in
ihn, redeten ihm zu und machten ihn l"acherlich. Und wa{\s}
vollend{\s} den Au{\s}schlag gab: die Operation sollte ihm keinen
roten Heller kosten. Bovary versprach sogar, Material und
Medikamente umsonst zu liefern. Emma war die Anstifterin dieser
Generosit"at. Karl pflichtete ihr bei und sagte sich im stillen:
"`Meine Frau ist doch wirklich ein Engel!"'

Beraten vom Apotheker, lie"s Karl nach drei fehlgeschlagenen
Versuchen durch den Tischler unter Beihilfe de{\s} Schlosser{\s}
eine Art Geh"ause anfertigen. E{\s} wog beinahe acht Pfund, und an
Holz, Eisen, Blech, Leder, Schrauben usw. war nicht gespart
worden.

Um nun zu bestimmen, welche Sehne zu durchschneiden sei, mu"ste
zu\-n"achst festgestellt werden, welche besondere Art von Klumpfu"s
hier vorlag. Hippolyt{\s} Fu"s setzte sich an sein Schienbein
nahezu geradlinig an. Dazu war er noch nach innen zu verdreht.
E{\s} war also Pferdefu"s, verbunden mit etwa{\s} Varu{\s} oder,
ander{\s} au{\s}gedr"uckt, ein Fall leichten Varu{\s} mit starker
Neigung zu einem Pferdefu"s.

Trotz diese{\s} Klumpfu"se{\s}, der in der Tat plump wie ein
Pferdehuf war und runzelige Haut, au{\s}ged"orrte Sehnen und dicke
Zehen mit schwarzen wie eisern au{\s}sehenden N"ageln hatte, war
der Kr"uppel von fr"uh bi{\s} abend munter wie ein Wiesel. Man sah
ihn unaufh"orlich im Hofe um die Wagen herumhumpeln. E{\s} hatte
sogar den Anschein, al{\s} sei sein mi"sratene{\s} Bein kr"aftiger
denn da{\s} gesunde. Offenbar hatte sich Hippolyt, von Jugend auf
im schweren Dienst, sehr viel Geduld und Au{\s}dauer zu eigen
gemacht.

An einem Pferdefu"s mu"s zun"achst die Achille{\s}sehne
durchschnitten werden, dann die vordere Schienbeinmu{\s}kel. Eher
kann der Varu{\s} nicht beseitigt werden. Karl wagte e{\s} kaum,
beide Schnitte auf einmal zu machen. Auch hatte er gro"se Angst,
einen wichtigen Teil zu verletzen. Seine anatomischen Kenntnisse
waren mangelhaft.

Ambrosiu{\s} Par\'e, der f"unfzehn Jahrhunderte nach Celsu{\s} die
erste unmittelbare Unterbindung einer Arterie wagte, D"upuytren,
der e{\s} unternahm, einen Abs\/ze"s am Gehirn zu "offnen,
Gensoul, der al{\s} erster eine Oberkiefer-Abtragung
au{\s}f"uhrte, -- allen diesen hat sicherlich nicht so da{\s} Herz
geklopft und die Hand gezittert, und sie waren gewi"s nicht so
aufgeregt wie Bovary, al{\s} er Hippolyt unter sein Messer nahm.

Im St"ubchen de{\s} Hau{\s}knecht{\s} sah e{\s} au{\s} wie in
einem Lazarett. Auf dem Tische lagen Haufen von Scharpie,
gewichste F"aden, Binden, alle{\s} wa{\s} in der Apotheke an
Verband{\s}zeug vorr"atig gewesen war. Homai{\s} hatte da{\s}
alle{\s} eigenh"andig vorbereitet, sowohl um die Leute zu
verbl"uffen al{\s} auch um sich selbst etwa{\s} vorzumachen.

Karl f"uhrte den Einschnitt au{\s}. Ein platzende{\s} Ger"ausch.
Die Sehne war zerschnitten, die Operation beendet.

Hippolyt war vor Erstaunen au"ser aller Fassung. Er nahm
Bovary{\s} H"ande und bedeckte sie mit K"ussen.

"`Erst mal Ruhe!"' gebot der Apotheker. "`Die Dankbarkeit f"ur
deinen Wohlt"ater kannst du ja sp"ater bezeigen!"'

Er ging hinunter, um da{\s} Ereigni{\s} den f"unf oder sech{\s}
Neugierigen mit\/zuteilen, die im Hofe herumstanden und sich
eingebildet hatten, Hippolyt werde erscheinen und mit einem Male
laufen wie jeder andere. Karl schnallte seinem Patienten da{\s}
Geh"ause an und begab sich sodann nach Hau{\s}, wo ihn Emma
angstvoll an der T"ure erwartete. Sie fiel ihm um den Hal{\s}.

Sie setzten sich zu Tisch. Er a"s viel und verlangte zum Nachtisch
sogar eine Tasse Kaffee; diesen Luxu{\s} erlaubte er sich sonst
nur Sonntag{\s}, wenn ein Gast da war.

Der Abend verlief in heiterer Stimmung unter Gespr"achen und
gemeinsamem Pl"aneschmieden. Sie plauderten vom kommenden Gl"ucke,
von der Hebung ihre{\s} Hau{\s}stande{\s}. Er sah seinen
"arztlichen Ruf wachsen, seinen Wohlstand gedeihen und die Liebe
seiner Frau immerdar w"ahren. Und sie, sie f"uhlte sich begl"uckt
und verj"ungt, ges"under und besser in ihrer wiedererstandenen
leisen Zuneigung f"ur diesen armen Mann, der sie so sehr liebte.
Fl"uchtig scho"s ihr der Gedanke an Rudolf durch den Kopf, aber
ihre Augen ruhten al{\s}bald wieder auf Karl, und dabei bemerkte
sie erstaunt, da"s seine Z"ahne eigentlich gar nicht h"a"slich
waren.

Sie waren bereit{\s} zu Bett, al{\s} Homai{\s} trotz der Abwehr
de{\s} M"adchen{\s} pl"otzlich in{\s} Zimmer trat, in der Hand ein
frisch beschriebene{\s} St"uck Papier. E{\s} war der
Reklame-Aufsatz, den er f"ur den "`Leuchtturm von Rouen"' verfa"st
hatte. Er brachte ihn, um ihn dem Arzte zum Lesen zu geben.

"`Lesen Sie ihn vor!"' bat Bovary.

Der Apotheker tat e{\s}:

\begin{quotation}\noindent
"`Ungeachtet der Vorurteile, in die ein Teil der Europ"aer noch
immer verstrickt ist wie in ein Netz, beginnt e{\s} in unserer
Gegend doch zu tagen. Am Dienstag war unser St"adtchen Yonville
der Schauplatz einer chirurgischen Tat, die zugleich ein Beispiel
edelster Menschenliebe ist. Herr Karl Bovary, einer unserer
angesehensten praktischen "Arzte,~..."'
\end{quotation}

"`Ach, da{\s} ist zu viel! Da{\s} ist zu viel!"' unterbrach ihn
Karl, vor Erregung tief atmend.

"`Aber durchau{\s} nicht! Wieso denn?"'

Er la{\s} weiter:

\begin{quotation}\noindent
"`... hat den verkr"uppelten Fu"s~..."'
\end{quotation}

Er unterbrach sich selbst:

"`Ich habe hier absichtlich den \begin{antiqua}terminus
technicus\end{antiqua} vermieden, wissen Sie! In einer
Tage{\s}zeitung mu"s alle{\s} gemeinverst"andlich sein ... die
gro"se Masse~..."'

"`Sehr richtig!"' meinte Bovary. "`Bitte fahren Sie fort!"'

"`Ich wiederhole:

\begin{quotation}\noindent
Herr Karl Bovary, einer unserer angesehensten praktischen "Arzte,
hat den verkr"uppelten Fu"s eine{\s} gewissen Hippolyt Tautain
operiert, de{\s} langj"ahrigen Hau{\s}knecht{\s} im Hotel zum
Goldnen L"owen der verwitweten Frau Franz am Markt. Da{\s}
aktuelle Ereigni{\s} und da{\s} allgemeine Interesse an der
Operation hatten eine derartig gro"se Volk{\s}menge angezogen,
da"s der Zugang zu dem Etablissement gesperrt werden mu"ste. Die
Operation selbst vollzog sich wunderbar schnell. Blutergu"s trat
so gut wie nicht ein. Kaum ein paar Blut{\s}tropfen verrieten,
da"s ein hartn"ackige{\s} Leiden endlich der Macht der
Wissenschaft wich. Der Kranke versp"urte dabei erstaunlicherweise
-- wie der Berichterstatter al{\s} Augenzeuge versichern darf --
nicht den geringsten Schmerz, und sein Zustand l"a"st bi{\s} jetzt
nicht{\s} zu w"unschen "ubrig. Allem Daf"urhalten nach wird die
vollst"andige Heilung rasch erfolgen, und wer wei"s, ob der brave
Hippolyt nicht bei der kommenden Kirme{\s} mit den flotten
Urlaubern um die Wette tanzen und seine Wiederherstellung durch
muntere Spr"unge feiern wird? Ehre aber den hochherzigen
Gelehrten, Ehre den unerm"udlichen Geistern, die ihre N"achte der
Menschheit zum Heile opfern! Ehre, dreimal Ehre ihnen!

Der Tag wird noch kommen, wo verk"undet werden wird, da"s die
Blinden sehen, die Tauben h"oren und die Lahmen gehen! Wa{\s} der
kirchliche Aberglaube ehedem nur den Au{\s}erw"ahlten versprach,
schenkt die Wissenschaft mehr und mehr allen Menschen. Wir werden
unsere verehrten Leser "uber den weiteren Verlauf dieser so
ungemein merkw"urdigen Kur auf dem laufenden erhalten."'
\end{quotation}

Trotz alledem kam f"unf Tage darauf die L"owenwirtin ganz
verst"ort gelaufen und rief:

"`Zu Hilfe! Er stirbt! Ich wei"s nicht, wa{\s} ich machen soll!"'

Karl rannte Hal{\s} "uber Kopf nach dem Goldnen L"owen, und der
Apotheker, der den Arzt so "uber den Markt st"urmen sah, verlie"s
sofort im blo"sen Kopfe seinen Laden. Atemlo{\s}, aufgeregt und
mit rotem Gesichte erreichte er den Gasthof und fragte jeden, dem
er auf der Treppe begegnete:

"`Na, wa{\s} macht denn unser interessanter Strephopode?"'

Der Strephopode wand sich in schrecklichen Zuckungen, so da"s
da{\s} Ge\-h"ause, in da{\s} sein Bein eingezw"angt war, gegen
die Wand geschlagen ward und ent\/zwei zu gehen drohte.

Mit vieler Vorsicht, um ja dabei die Lage de{\s} Fu"se{\s} nicht
zu verschieben, entfernte man da{\s} Holzgeh"ause. Und nun bot
sich ein gr"a"slicher Anblick dar. Die Form de{\s} Fu"se{\s} war
unter einer derartigen Schwellung verschwunden, da"s e{\s}
au{\s}sah, al{\s} platze demn"achst die ganze Haut. Diese war
blutunterlaufen und von Druckflecken bedeckt, die da{\s} famose
Geh"ause verursacht hatte. Hippolyt hatte von Anfang an "uber
Schmerzen geklagt, aber man hatte ihn nicht angeh"ort. Nachdem man
nunmehr einsah, da"s er im Rechte gewesen war, g"onnte man ihm ein
paar Stunden Befreiung. Aber sowie die Schwellung ein wenig
zur"uckgegangen war, hielten e{\s} die beiden Heilk"unstler f"ur
angebracht, da{\s} Bein wieder einzuschienen und e{\s} noch fester
einzupressen, um dadurch die Wiederherstellung zu beschleunigen.

Aber nach drei Tagen vermochte e{\s} Hippolyt nicht mehr
au{\s}zuhalten. Man nahm ihm den Apparat abermal{\s} ab und war
h"ochst "uber da{\s} verwundert, wa{\s} sich nunmehr
herau{\s}stellte. Die schw"arzlichblau gewordene Schwellung
erstreckte sich "uber da{\s} ganze Bein, da{\s} ganz voller Blasen
war; eine dunkle Fl"ussigkeit sonderte sich ab. Man wurde
bedenklich.

Hippolyt begann sich zu langweilen, und Frau Franz lie"s ihn in
die kleine Gaststube bringen neben der K"uche, damit er
wenigsten{\s} etwa{\s} Zerstreuung h"atte. Aber der Steuereinnehmer,
der dort seinen Stammplatz hatte, beschwerte sich "uber diese
Nachbarschaft. Nunmehr schaffte man den Kranken in da{\s}
Billardzimmer. Dort lag er wimmernd unter seinen schweren Decken,
bla"s, unrasiert, mit eingesunkenen Augen. Von Zeit zu Zeit wandte
er seinen in Schwei"s gebadeten Kopf auf dem schmutzigen Kissen
hin und her, wenn ihn die Fliegen qu"alten.

Frau Bovary besuchte ihn. Sie brachte ihm Leinwand zu den
Umschl"agen, tr"ostete ihn und sprach ihm Mut ein. Auch sonst
fehlte e{\s} ihm nicht an Gesellschaft, zumal an den Markttagen,
wenn die Bauern drin bei ihm Billard spielten, mit den Queuen
herumfuchtelten, rauchten, zechten, sangen und Spektakel machten.

"`Wie geht dir{\s} denn?"' fragten sie ihn und klopften ihm auf
die Schulter. "`So recht auf dem Damme bist du wohl nicht? Bist
aber selber schuld daran!"' Er h"atte die{\s} oder jene{\s} machen
sollen. Sie erz"ahlten ihm von Leuten, die durch ganz andere
Heilmittel wiederhergestellt worden seien. Und zum sonderbaren
Trost meinten sie:

"`Du bist viel zu zimperlich! Steh doch auf! Du l"a"st dich wie
ein F"urst verh"atscheln! Da{\s} ist Unsinn, alter Schlaumeier!
Und besonder{\s} gut riechst du auch nicht!"'

Inzwischen griff der Brand immer weiter um sich. Bovary ward fast
selber krank davon. Er kam aller Stunden, aller Augenblicke.
Hippolyt sah ihn mit angsterf"ullten Augen an. Schluchzend
stammelte er:

"`Lieber Herr Doktor, wann werd ich denn wieder gesund? Ach,
helfen Sie mir! Ich bin so ungl"ucklich, so ungl"ucklich!"'

Bovary schrieb ihm alle Tage vor, wa{\s} er essen solle. Dann
verlie"s er ihn.

"`H"or nur gar nicht auf ihn, mein Junge!"' meinte die
L"owenwirtin. "`Sie haben dich schon gerade genug geschunden!
Da{\s} macht dich blo"s immer noch schw"acher! Da, trink!"'

Sie gab ihm hin und wieder Fleischbr"uhe, ein St"uck Hammelkeule,
Speck und manchmal ein Gl"aschen Schnap{\s}, den er kaum an seine
Lippen zu bringen wagte.

Abb\'e Bournisien, der geh"ort hatte, da"s e{\s} Hippolyt
schlechter ging, kam ihn zu besuchen. Er bedauerte ihn, dann aber
erkl"arte er, in gewisser Beziehung m"usse sich der Kranke freuen,
denn e{\s} sei de{\s} Herrn Wille, der ihm Gelegenheit g"abe, sich
mit dem Himmel zu vers"ohnen.

"`Siehst du,"' sagte der Priester in v"aterlichem Tone, "`du hast
deine Pflichten recht vernachl"assigt! Man hat dich selten in der
Kirche gesehen. Wieviel Jahre lang hast du da{\s} heilige
Abendmahl nicht genommen? Ich gebe zu, da"s deine Besch"aftigung
und der Trubel der Welt dich abgehalten haben, f"ur dein
Seelenheil zu sorgen. Aber jetzt ist e{\s} an der Zeit, da"s du
dich darum k"ummerst. Verzweifle indessen nicht! Ich habe gro"se
S"under gekannt, die, kurz ehe sie vor Gotte{\s} Thron traten, (du
bist noch nicht so weit, da{\s} wei"s ich wohl!) seine Gnade
erfleht haben; sie sind ohne Verdammni{\s} gestorben! Hoffen wir,
da"s auch du un{\s} gleich ihnen ein gute{\s} Beispiel gibst!
Darum: sei vorsichtig! Niemand verwehrt dir, morgen{\s} ein
Ave-Maria und abend{\s} ein Paternoster zu beten! Ja, tue da{\s}!
Mir zuliebe! Wa{\s} kostet dich da{\s}? Willst du mir da{\s}
versprechen?"'

Der arme Teufel gelobte e{\s}. Tag f"ur Tag kam der Seelsorger
wieder. Er plauderte mit ihm und der Wirtin, und bi{\s}weilen
erz"ahlte er den beiden sogar Anekdoten, Sp"a"se und faule Witze,
die Hippolyt allerding{\s} nicht verstand. Aber bei jeder
Gelegenheit kam er auf religi"ose Dinge zu sprechen, wobei er
jede{\s}mal eine salbung{\s}volle Miene annahm.

Dieser Eifer verfehlte seine Wirkung nicht. E{\s} dauerte nicht
lange, da bekundete der Strephopode die Absicht, eine Wallfahrt
nach Bon-Secour{\s} zu unternehmen, wenn er wieder gesund w"urde,
worauf der Priester entgegnete, da{\s} sei nicht "ubel. Doppelt
gen"aht halte besser. Er ri{\s}kiere ja dabei nicht{\s}.

Der Apotheker war emp"ort "uber "`diese Pfaffenschliche"', wie er
sich au{\s}dr"uckte. Er behauptete, da{\s} verz"ogre die Genesung
de{\s} Hau{\s}knecht{\s} nur.

"`La"st ihn doch nur in Ruhe!"' sagte er zur L"owenwirtin. "`Mit
euren Salbadereien macht ihr den Mann nur verdreht!"'

Aber die gute Frau wollte davon nicht{\s} h"oren. Er und kein
anderer sei ja an der ganzen Geschichte schuld! Und auch rein
au{\s} Widerspruch{\s}geist hing sie dem Kranken zu H"aupten einen
Weihwasserkessel und einen Buch{\s}baumzweig auf.

Allerding{\s} n"utzten offenbar weder der kirchliche noch der
chirurgische Segen. Unaufhaltsam schritt die Blutvergiftung vom
Beine weiter in den K"orper hinauf. Man versuchte immer neue
Salben und Pflaster, aber der Fu"s wurde immer brandiger, und
schlie"slich antwortete Bovary mit einem zustimmenden Kopfnicken,
al{\s} Mutter Franz ihn fragte, ob man angesicht{\s} dieser
hoffnung{\s}losen Lage nicht den Doktor Canivet au{\s}
Neufch\^atel kommen lassen solle, der doch weitber"uhmt sei.

Canivet war Doktor der Medizin, f"unfzig Jahre alt, ebenso
wohlhabend wie selbstbewu"st. Er kam und entbl"odete sich nicht,
"uber den Kollegen geringsch"atzig zu l"acheln, al{\s} er da{\s}
bi{\s} an da{\s} Knie brandig gewordene Bein untersuchte. Sodann
erkl"arte er, da{\s} Glied m"usse amputiert werden.

Er suchte den Apotheker auf und wetterte gegen "`die Esel, die
da{\s} arme Luder so zugerichtet"' h"atten. Er fa"ste Homai{\s} am
Rockknopf und hielt ihm in seiner Apotheke eine Standpauke:

"`Da habt Ihr so 'ne Pariser Erfindung! Solchen Unsinn hecken die
Herren Gelehrten der Weltstadt nun au{\s}! Genau so steht e{\s}
mit ihren Schieloperationen, Chloroform-Bet"aubungen,
Blaseneingriffen! Da{\s} ist alle{\s} Kapitalunfug gegen den sich
der Staat in{\s} Zeug legen sollte! Diese Scharlatane wollen blo"s
immer wa{\s} zu tun haben. Sie erfinden die unglaublichsten
Verfahren, aber an die Folgen denken sie nicht. Wir andern aber,
wir sind r"uckst"andig. Wir sind keine Gelehrten, keine
Zauberk"unstler, keine Salonhelden. Wir haben unsre Praxi{\s}, wir
heilen lumpige Krankheiten, aber e{\s} f"allt un{\s} nicht ein,
Leute zu operieren, die kerngesund herumlaufen! Klumpf"u"se gerade
zu hacken! Du lieber Gott! Ebenso k"onnte man auch einem Buckligen
seinen H"ocker abhobeln wollen!"'

Homai{\s} war bei diesem Ergu"s gar nicht besonder{\s} wohl
zumute, aber er verbarg sein Mi"sbehagen hinter einem
verbindlichen L"acheln. Er mu"ste mit Canivet auf gutem Fu"se
bleiben, dieweil dieser in der Yonviller Gegend "ofter{\s}
konsultiert wurde und ihm dabei durch Rezepte zu verdienen gab.
Au{\s} diesem Grunde h"utete er sich, f"ur Bovary einzutreten. Er
vermuckste sich nicht, lie"s Grunds"atze Grunds"atze sein und
opferte seine W"urde den ihm wichtigeren Interessen seine{\s}
Gesch"aft{\s}.

Die Amputation de{\s} Beine{\s}, die der Doktor Canivet
au{\s}f"uhrte, war f"ur den ganzen Ort ein wichtige{\s}
Ereigni{\s}. Fr"uhzeitig waren die Leute schon auf den Beinen, und
die Hauptstra"se war voller Menschen, die allesamt etwa{\s}
Tr"ubselige{\s} an sich hatten, al{\s} solle eine Hinrichtung
stattfinden. Im Laden de{\s} Kr"amer{\s} stritt man sich "uber
Hippolyt{\s} Krankheit. An{\s} Kaufen dachte niemand. Und Frau
T"uvache, die Gattin de{\s} B"urgermeister{\s}, lag vom fr"uhen
Morgen in ihrem Fenster, um ja nicht zu verpassen, wenn der
Operateur ank"ame.

Er kam in seinem W"agelchen angefahren, da{\s} er selber
kutschierte. Durch die Last seine{\s} K"orper{\s} war die rechte
Feder de{\s} Gef"ahrt{\s} derartig niedergedr"uckt, da"s der
Wagenkasten schief stand. Neben dem Insassen auf dem Sitzpolster
stand eine rotlederne Reisetasche, deren Messingschl"osser
pr"achtig funkelten. In starkem Trabe fuhr Canivet bi{\s} vor die
kleine Freitreppe de{\s} Goldnen L"owen. Mit lauter Stimme befahl
er, da{\s} Pferd au{\s}zuspannen. Er ging mit in den Stall und
"uberzeugte sich, da"s der Gaul ordentlich Hafer gesch"uttet
bekam. E{\s} war seine Gewohnheit, da"s er sich immer zuerst
seinem Tier und seinem Fuhrwerk widmete. Er galt de{\s}halb im
Munde der Leute f"ur einen "`Pferdejockel"'. Aber gerade weil er
sich darin unabbringbar gleichblieb, sch"atzte man ihn um so mehr.
Und wenn der letzte Mensch auf Gotte{\s} ganzem Erdboden in den
letzten Z"ugen gelegen h"atte: Doktor Canivet w"are zun"achst
seiner kavalleristischen Pflicht nachgekommen.

Homai{\s} stellte sich ein.

"`Ich rechne auf Ihre Unterst"utzung!"' sagte der Chirurg. "`Ist
alle{\s} bereit? Na, dann kann{\s} lo{\s}gehen!"'

Der Apotheker gestand err"otend ein, da"s er zu empfindlich sei,
um einer solchen Operation assistieren zu k"onnen. "`Al{\s}
passiver Zuschauer"', sagte er, "`greift einen so wa{\s} doppelt
an. Meine Nerven sind so herunter~..."'

"`Quatsch!"' unterbrach ihn Canivet. "`Mir machen Sie vielmehr den
Eindruck, al{\s} solle Sie demn"achst der Schlag r"uhren.
"Ubrigen{\s} kein Wunder! Ihr Herren Apotheker hockt ja von fr"uh
bi{\s} abend{\s} in Eurer Giftbude. Da{\s} mu"s sich ja
schlie"slich auf die Nerven legen! Gucken Sie mich mal an! Tag
f"ur Tag stehe ich vier Uhr morgen{\s} auf, wasche mich mit
ei{\s}kaltem Wasser ... Frieren kenne ich nicht, Flanellhemden
gibt{\s} f"ur mich nicht, da{\s} Zipperlein kriege ich nicht, und
mein Magen ist mord{\s}gesund. Dabei lebe ich heute so und morgen
so, wie mir{\s} gerade einf"allt, aber immer al{\s}
Leben{\s}k"unstler! Und de{\s}halb bin ich auch nicht so
zimperlich wie Sie. E{\s} ist mir total Wurst, ob ich einem Rebhuhn
oder einem christlichen Individuum da{\s} Bein abschneide. Sie
haben mir neulich mal gesagt, der Mensch sei ein Gewohnheit{\s}tier.
Sehr richtig! E{\s} ist alle{\s} blo"s Gewohnheit~..."'

Ohne irgendwelche R"ucksicht auf Hippolyt, der nebenan auf seinem
Lager vor Angst schwitzte, f"uhrten die beiden ihre Unterhaltung
in diesem Stile weiter. Der Apotheker verglich die Kaltbl"utigkeit
eine{\s} Chirurgen mit der eine{\s} Feldherrn. Durch diesen
Vergleich geschmeichelt, lie"s sich Canivet de{\s} l"angeren "uber
die Erfordernisse seiner Kunst au{\s}. Der Beruf de{\s} Arzte{\s}
sei ein Priesteramt, und wer e{\s} nicht al{\s} da{\s}, sondern
al{\s} gemeine{\s} Handwerk au{\s}"ube, der sei ein
Heiligtumsch"ander.

Endlich erinnerte er sich de{\s} Patienten und begann da{\s} von
Homai{\s} gelieferte Verband{\s}zeug zu pr"ufen. E{\s} war
dasselbe, da{\s} bereit{\s} bei der ersten Operation zur Stelle
gewesen war. Sodann erbat er sich jemanden, der da{\s} Bein
festhalten k"onne. Lestiboudoi{\s} ward geholt.

Der Doktor zog den Rock au{\s}, streifte sich die Hemd{\s}"armel
hoch und begab sich in da{\s} Billardzimmer, w"ahrend der
Apotheker in die K"uche ging, wo die Wirtin sowie Artemisia
neugierig und "angstlich warteten. Die Gesichter der beiden Frauen
waren wei"ser al{\s} ihre Sch"urzen.

W"ahrenddessen wagte sich Bovary nicht au{\s} seinem Hause
herau{\s}. Er sa"s unten in der Gro"sen Stube, zusammengeduckt und
die H"ande gefaltet, im Winkel neben dem Kamin, in dem kein Feuer
brannte, und starrte vor sich hin. "`Welch ein Mi"sgeschick!"'
seufzte er. "`Wa{\s} f"ur eine gro"se Entt"auschung!"' Er hatte
doch alle denkbaren Vorsicht{\s}ma"sregeln getroffen, und doch war
der Teufel mit seiner Hand dazwischengekommen! Nicht zu "andern!
Wenn Hippolyt noch st"urbe, dann w"are er schuld daran! Und wa{\s}
sollte er antworten, wenn ihn seine Patienten darnach fragten?
Sollte er sagen, er habe einen Fehler begangen? Aber welchen? Er
wu"ste doch selber keinen, so sehr er auch dar"uber nachsann. Die
ber"uhmtesten Chirurgen versehen sich einmal. Aber da{\s} wird
kein Mensch bedenken. Sie werden ihn alle nur au{\s}lachen und in
Verruf bringen. Die Sache wird bi{\s} Forge{\s} ruchbar werden,
bi{\s} Neufch\^atel, bi{\s} Rouen und noch weiter! Vielleicht
w"urde irgendein Kollege einen Bericht gegen ihn ver"offentlichen,
dem dann eine Polemik folgte, die ihn zw"ange, in den Zeitungen
eine Entgegnung zu bringen. Hippolyt k"onnte auf Schadenersatz
klagen.

Karl sah sich entehrt, zugrunde gerichtet, verloren! Seine von
tausend Bef"urchtungen best"urmte Phantasie schwankte hin und her
wie eine leere Tonne auf den Wogen de{\s} Meere{\s}.

Emma sa"s ihm gegen"uber und beobachtete ihn. An seine Dem"utigung
dachte sie nicht. Ihre Gedanken arbeiteten in andrer Richtung. Wie
hatte sie sich nur einbilden k"onnen, da"s sich ein Mann seine{\s}
Schlage{\s} zu einer Leistung aufschw"ange, wo sich seine
Unf"ahigkeit doch schon mehr al{\s} ein dutzendmal erwiesen hatte!

Er lief im Zimmer auf und ab. Seine Stiefel knarrten.

"`Setz dich doch!"' sagte sie. "`Du machst mich noch ganz
verr"uckt!"'

Er tat e{\s}.

Wie hatte sie e{\s} nur fertig gebracht -- wo sie doch so klug
war! --, da"s sie sich abermal{\s} so get"auscht hatte? Aber ja,
ihr ganzer Leben{\s}pfad war doch fortw"ahrend durch da{\s}
traurige Tal der Entbehrungen gegangen. Wie vom Wahnwitz geleitet!
Sie rief sich alle{\s} einzeln in{\s} Ged"achtni{\s} zur"uck:
ihren unbefriedigten Hang zum Leben{\s}genu"s, die Einsamkeit
ihrer Seele, die Armseligkeit ihrer Ehe, ihre{\s}
Hau{\s}stande{\s}, ihre Tr"aume und Illusionen, die in den Sumpf
hinabgefallen waren wie verwundete Schwalben. Sie dachte an
alle{\s} da{\s}, wa{\s} sie sich ersehnt, an alle{\s}, wa{\s} sie
von sich gewiesen, an alle{\s}, wa{\s} sie h"atte haben k"onnen!
Sie begriff den geheimen Zusammenhang nicht. Warum war denn
alle{\s} so? Warum?

Da{\s} St"adtchen lag in tiefer Ruhe. Pl"otzlich erscholl ein
herzzerrei"sender Schrei. Bovary ward bla"s und beinahe ohnm"achtig.
Emma zuckte nerv"o{\s} mit den Augenbrauen. Dann aber war ihr
nicht{\s} mehr anzusehen.

Der da, der war der Schuldige! Dieser Mensch ohne Intelligenz und
ohne Feingef"uhl! Da sa"s er, stumpfsinnig und ohne Verst"andni{\s}
daf"ur, da"s er nicht nur seinen Namen l"acherlich und ehrlo{\s}
gemacht hatte, sondern den gemeinsamen Namen, also auch ihren
Namen! Und sie, sie hatte sich solche M"uhe gegeben, ihn zu lieben!
Hatte unter Tr"anen bereut, da"s sie ihm untreu geworden war!

"`Vielleicht war e{\s} ein Valgu{\s}?"' rief Karl pl"otzlich laut
au{\s}. Da{\s} war da{\s} Ergebni{\s} seine{\s} Nachsinnen{\s}.

Bei dem unerwarteten Schlag, den dieser Au{\s}ruf den Gedanken
Emma{\s} versetzte -- er fiel wie eine Bleikugel auf eine silberne
Platte --, hob sie erschrocken ihr Haupt. Wa{\s} wollte er damit
sagen, fragte sie sich. Sie sahen einander stumm an, gleichsam
erstaunt, sich gegenseitig zu erblicken. Alle beide waren sie sich
seelisch himmelweit fern. Karl starrte sie an mit dem wirren Blick
eine{\s} Trunkenen und lauschte dabei, ohne sich zu regen, den
verhallenden Schreien de{\s} Amputierten. Der heulte in
langgedehnten T"onen, die ab und zu von grellem Gebr"ull
unterbrochen wurden. Alle{\s} da{\s} klang wie da{\s} ferne
Gejammer eine{\s} Tiere{\s}, da{\s} man schlachtet. Emma bi"s sich
auf die blassen Lippen. Ihre Finger spielten mit dem Blatt einer
Blume, die sie zerpfl"uckt hatte, und ihre hei"sen Blicke trafen
ihn wie Brandpfeile. Jetzt reizte sie alle{\s} an ihm; sein
Gesicht, sein Anzug, sein Schweigen, seine ganze Erscheinung, ja
seine Existenz. Wie "uber ein Verbrechen empfand sie darob Reue,
da"s sie ihm so lange treu geblieben, und wa{\s} noch von
Anh"anglichkeit "ubrig war, ging jetzt in den lodernden Flammen
ihre{\s} Ingrimm{\s} auf. Mit wilder Schadenfreude geno"s sie den
Siege{\s}jubel "uber ihre gebrochene Ehe. Von neuem gedachte sie
de{\s} Geliebten und f"uhlte sich taumelnd zu ihm gezogen. Sein
Bild ent\/z"uckte und verf"uhrte sie in Gedanken abermal{\s}. Sie
gab ihm ihre ganze Seele. E{\s} war ihr, al{\s} sei Karl au{\s}
ihrem Leben herau{\s}gerissen, f"ur immer entfremdet, unm"oglich
geworden, au{\s}getilgt. Al{\s} sei er gestorben, nachdem er vor
ihren Augen den Tode{\s}kampf gek"ampft hatte. Vom Trottoir her
drang da{\s} Ger"ausch von Tritten herauf. Karl ging an da{\s}
Fenster und sah durch die niedergelassenen Jalousien den Doktor
Canivet an den Hallen in der vollen Sonne hingehen. Er wischte
sich gerade die Stirn mit seinem Taschentuche. Hinter ihm schritt
Homai{\s}, die gro"se rote Reisetasche in der Hand. Beide
steuerten auf die Apotheke zu.

In einem Anfall von Mutlosigkeit und Liebe{\s}bed"urfni{\s}
n"aherte sich Karl seiner Frau:

"`Gib mir einen Ku"s, Geliebte!"'

"`La"s mich!"' wehrte sie ab, ganz rot vor Zorn.

"`Wa{\s} hast du denn? Wa{\s} ist dir?"' fragte er betroffen.
"`Sei doch ruhig! "Argere dich nicht! Du wei"st ja, wie sehr ich
dich liebe! Komm!"'

"`Weg!"' rief sie mit verzerrtem Gesicht. Sie st"urzte au{\s} dem
Zimmer, wobei sie die T"ur so heftig hinter sich zuschlug, da"s
da{\s} Barometer von der Wand fiel und in St"ucke ging.

Karl sank in seinen Lehnstuhl. Erschrocken sann er dar"uber nach,
wa{\s} sie wohl habe. Er bildete sich ein, sie leide an einer
Nervenkrankheit. Er fing an zu weinen im ahnenden Vorgef"uhl von
etwa{\s} Unheilvollem, Unfa"sbarem.

Al{\s} Rudolf an diesem Abend hinten in den Garten kam, fand er
seine Geliebte auf der obersten Stufe der kleinen Gartentreppe
sitzen und auf ihn warten. Sie k"u"sten sich, und all ihr "Arger
schmolz in der Glut der Umarmung wie der Schnee vor der Sonne.


\newpage\begin{center}
{\large \so{Zw"olfte{\s} Kapitel}}\bigskip\bigskip
\end{center}

Ihre Liebe begann von neuem. Oft schrieb ihm Emma mitten am Tage.
Sie winkte sich Justin durch da{\s} Fenster her. Der legte schnell
seine Arbeit{\s}sch"urze ab und trabte nach der H"uchette. Rudolf
kam al{\s}bald. Sie hatte ihm nicht{\s} zu sagen, al{\s} da"s sie
sich langweile, da"s ihr Mann gr"a"slich sei und ihr Dasein
schrecklich.

"`Kann ich da{\s} "andern?"' rief er einmal ungeduldig au{\s}.

"`Ja, wenn du wolltest!"'

Sie sa"s auf dem Fu"sboden zwischen seinen Knien, mit aufgel"ostem
Haar und traumverlorenem Blick.

"`Wieso?"' fragte er.

Sie seufzte.

"`Wir m"ussen irgendwo ander{\s} ein neue{\s} Leben beginnen ...
weit weg von hier~..."'

"`Ein toller Einfall!"' lachte er. "`Unm"oglich!"'

Sie kam immer wieder darauf zur"uck. Er tat so, al{\s} sei ihm
da{\s} unverst"andlich, und begann von etwa{\s} anderm zu
sprechen.

Wa{\s} Rudolf in der Tat nicht begriff, da{\s} war ihr ganze{\s}
aufgeregte{\s} Wesen bei einer so einfachen Sache wie der Liebe.
Sie m"usse dazu doch Anla"s haben, Motive. Sie klammere sich doch
an ihn, al{\s} ob sie bei ihm Hilfe suche.

Wirklich wuch{\s} ihre Z"artlichkeit zu dem Geliebten von Tag zu
Tag im gleichen Ma"se, wie sich ihre Abneigung gegen ihren Mann
verschlimmerte. Je mehr sie sich jenem hingab, um so mehr
verabscheute sie diesen. Karl kam ihr nie so unertr"aglich vor,
seine H"ande nie so vierschr"otig, sein Geist nie so
schwerf"allig, seine Manieren nie so gew"ohnlich, al{\s} wenn sie
nach einem Stelldichein mit Rudolf wieder mit ihm zusammen war.
Sie bildete sich ein, sie sei Rudolf{\s} Frau, seine treue Gattin.
Immerw"ahrend tr"aumte sie von seinem dunklen welligen Haar,
seiner braunen Stirn, seiner kr"aftigen und doch eleganten
Gestalt, von dem ganzen so klugen und in seinem Begehren doch so
leidenschaftlichen Menschen. Nur f"ur ihn pflegte sie ihre N"agel
mit der Sorgfalt eine{\s} Ziseleur{\s}, f"ur ihn verschwendete sie
eine Unmenge von Coldcream f"ur ihre Haut und von Peau d'E{\s}pagne
f"ur ihre W"asche. Sie "uberlud sich mit Armb"andern, Ringen und
Hal{\s}ketten. Wenn sie ihn erwartete, f"ullte sie ihre gro"sen
blauen Gla{\s}vasen mit Rosen und schm"uckte ihr Zimmer und sich
selber wie eine Kurtisane, die einen F"ursten erwartet. Felicie
wurde gar nicht mehr fertig mit Waschen; den ganzen Tag steckte
sie in ihrer K"uche.

Justin leistete ihr h"aufig Gesellschaft und sah ihr bei ihrer
Arbeit zu. Die Ellenbogen auf da{\s} lange B"ugelbrett gest"utzt,
auf dem sie pl"attete, betrachtete er l"ustern alle die um ihn
herum aufgeschichtete Damenw"asche, die Pikee-Unterr"ocke, die
Spitzent"ucher, die Hal{\s}kragen, die breith"uftigen Unterhosen.

"`Wozu hat man da{\s} alle{\s}?"' fragte der Bursche, indem er mit
der Hand "uber einen der Reifr"ocke strich.

"`Hast du sowa{\s} noch niegesehen?"' Felicie lachte. "`Deine
Herrin, Frau Homai{\s}, hat da{\s} doch auch!"'

"`So? Die Frau Homai{\s}!"' Er sann nach. "`Ist sie denn eine Dame
wie die Frau Doktor?"'

Felicie liebte e{\s} gar nicht, wenn er sie so umschn"uffelte. Sie
war drei Jahre "alter al{\s} er, und "ubrigen{\s} machte ihr
Theodor, der Diener de{\s} Notar{\s}, neuerding{\s} den Hof.

"`La"s mich in Ruhe!"' sagte sie und stellte den St"arketopf
beiseite. "`Scher dich lieber an \so{deine} Arbeit! Sto"s deine
Mandeln! Immer mu"st du an irgendeiner Sch"urze h"angen! Eh du
dich damit befa"st, la"s dir mal erst die Stoppeln unter der Nase
wachsen, du Knirp{\s}, du nicht{\s}n"utziger!"'

"`Ach, seien Sie doch nicht gleich b"o{\s}! Ich putze Ihnen auch
die Schuhe f"ur die Frau Doktor!"'

Alsobald machte er sich "uber ein Paar von Frau Bovary{\s} Schuhen
her, die in der K"uche standen. Sie waren "uber und "uber mit
eingetrocknetem Stra"senschmutz bedeckt -- vom letzten
Stelldichein her --, der beim Anfassen in Staub zerfiel und, wo
gerade die Sonne schien, eine leichte Wolke bildete. Justin
betrachtete sie sich.

"`Hab nur keine Angst! Die gehen nicht ent\/zwei!"' sagte Felicie,
die, wenn sie die Schuhe selber reinigte, keine besondere Sorgfalt
anwandte, weil die Herrin sie ihr "uberlie"s, sobald sie nicht
mehr tadello{\s} au{\s}sahen. Emma hatte eine Menge Schuhzeug in
ihrem Schranke, sie trieb damit eine wahre Verschwendung, aber
Karl wagte nicht den geringsten Einwand dagegen.

So gab er auch dreihundert Franken f"ur ein h"olzerne{\s} Bein
au{\s}, da{\s} Hippolyt ihrer Ansicht nach geschenkt bekommen
m"usse. Die Fl"ache, mit der e{\s} anlag, war mit Kork "uberzogen.
E{\s} hatte Kugelgelenke und eine komplizierte Mechanik. Hose und
Schuh verdeckten e{\s} vollkommen. Hippolyt wagte e{\s} indessen
nicht in den Alltag{\s}gebrauch zu nehmen und bat Frau Bovary, ihm
noch ein andere{\s}, einfachere{\s} zu besorgen. Wohl oder "ubel
mu"ste der Arzt auch diese Au{\s}gabe tragen. Nun konnte der
Hau{\s}knecht von neuem seinem Berufe nachgehen. Wie ehedem sah
man ihn wieder durch den Ort humpeln. Wenn Karl von weitem den
harten Anschlag de{\s} Stelzfu"se{\s} auf dem Pflaster vernahm,
schlug er schnell einen anderen Weg ein.

Lheureux, der Modewarenh"andler, hatte da{\s} Holzbein besorgt.
Da{\s} gab ihm Gelegenheit, Emma h"aufig aufzusuchen. Er plauderte
mit ihr "uber die neuesten Pariser Moden und "uber tausend Dinge,
die Frauen interessieren. Dabei war er immer "au"serst gef"allig
und forderte niemal{\s} bare Bezahlung. Alle Launen und Einf"alle
Emma{\s} wurden im Handumdrehen befriedigt. Einmal wollte sie
Rudolf einen sehr sch"onen Reitstock schenken, den sie in Rouen in
einem Schirmgesch"aft gesehen hatte. Eine Woche sp"ater legte
Lheureux ihn ihr auf den Tisch. Am folgenden Tage aber
"uberreichte er ihr eine Rechnung im Gesamtbetrage von
zweihundertundsiebzig Franken und so und soviel Centime{\s}. Emma
war in der gr"obsten Verlegenheit. Die Kasse war leer.
Lestiboudoi{\s} hatte noch Lohn f"ur vierzehn Tage zu bekommen,
Felicie f"ur acht Monate. Dazu kam noch eine Menge andrer
Schulden. Bovary wartete schon mit Schmerzen auf den Eingang
de{\s} Honorar{\s} von Herrn Derozeray{\s}, da{\s} allj"ahrlich
gegen Ende Oktober einzugehen pflegte.

Ein paar Tage gelang e{\s} ihr, Lheureux zu vertr"osten. Dann
verlor er aber die Geduld. Man dr"ange auch ihn, er brauche Geld,
und wenn er nicht al{\s}bald welche{\s} von ihr bek"ame, m"usse er
ihr alle{\s} wieder abnehmen, wa{\s} er ihr geliefert habe.

"`Gut!"' meinte Emma. "`Holen Sie sich{\s}!"'

"`Ach wa{\s}! Da{\s} hab ich nur so gesagt!"' entgegnete er.
"`Indessen um den Reitstock tut{\s} mir wirklich leid! Bei Gott,
den werd ich mir vom Herrn Doktor zur"uckgeben lassen!"'

"`Um Gotte{\s} willen!"' rief sie au{\s}.

"`Warte nur! Dich hab ich!"' dachte Lheureux bei sich.

Jetzt war er seiner Vermutung sicher. Indem er sich entfernte,
lispelte er in seinem gewohnten Fl"ustertone vor sich hin:

"`Na, wir werden ja sehen! Wir werden ja sehen!"'

Frau Bovary gr"ubelte gerade dar"uber nach, wie sie diese Geschichte
in Ordnung bringen k"onne, da kam da{\s} M"adchen und legte eine
kleine in blaue{\s} Papier verpackte Geldrolle auf den Kamin. Eine
Empfehlung von Herrn Derozeray{\s}. Emma sprang auf und brach die
Rolle auf. E{\s} waren dreihundert Franken in Napoleon{\s}, da{\s}
schuldige Honorar. Karl{\s} Tritte wurden drau"sen auf der Treppe
h"orbar. Sie legte da{\s} Gold rasch in die Schublade und steckte
den Schl"ussel ein.

Drei Tage darauf erschien Lheureux abermal{\s}.

"`Ich m"ochte Ihnen einen Vergleich vorschlagen"', sagte er.
"`Wollen Sie mir nicht statt de{\s} baren Gelde{\s} lieber~..."'

"`Hier haben Sie Ihr Geld!"' unterbrach sie ihn und z"ahlte ihm
vierzehn Goldst"ucke in die Hand.

Der Kaufmann war verbl"ufft. Um seine Entt"auschung zu verbergen,
brachte er endlose Entschuldigungen vor und bot Emma alle
m"oglichen Dienste an, die sie allesamt ablehnte.

Eine Weile stand sie dann noch nachdenklich da und klimperte mit
dem Kleingeld, da{\s} sie wieder herau{\s}bekommen und in die
Tasche ihrer Sch"urze gesteckt hatte. Sie nahm sich vor, t"uchtig
zu sparen, damit sie recht bald~...

"`Wa{\s} ist da weiter dabei?"' beruhigte sie sich. "`Er wird
nicht gleich dran denken!"'

Au"ser dem Reitstocke mit dem vergoldeten Silbergriffe hatte
Rudolf auch noch ein Petschaft von ihr geschenkt bekommen, mit dem
Wahlspruch: \begin{antiqua}Amor nel Cor!\end{antiqua} (Liebe im
Herzen!), fernerhin ein seidene{\s} Hal{\s}tuch und eine
Zigarrentasche, zu der sie al{\s} Muster die Tasche genommen
hatte, die Karl damal{\s} auf der Landstra"se gefunden hatte,
al{\s} sie vom Schlosse Vaubyessard heimfuhren. Emma hatte sie
sorglich aufbewahrt. Rudolf nahm diese Geschenke erst nach langem
Str"auben. Sie waren ihm peinlich. Aber Emma drang in ihn, und so
mu"ste er sich schlie"slich f"ugen. Er fand da{\s} aufdringlich
und h"ochst r"ucksicht{\s}lo{\s}.

Sie hatte wunderliche Einf"alle.

"`Wenn e{\s} Mitternacht schl"agt,"' bat sie ihn einmal, "`mu"st
du an mich denken!"'

Al{\s} er hinterher gestand, er habe e{\s} vergessen, bekam er
endlose Vorw"urfe zu h"oren, die alle in die Worte au{\s}klangen:

"`Du liebst mich nicht mehr!"'

"`Ich dich nicht mehr lieben?"'

"`"Uber alle{\s}?"'

"`Nat"urlich!"'

"`Hast du auch vor mir nie eine andre geliebt, sag?"'

"`Glaubst du, ich h"atte meine Unschuld bei dir verloren?"' brach
er lachend au{\s}.

Sie fing an zu weinen, und Rudolf vermochte sie nur mit viel M"uhe
zu beruhigen, indem er seine Worte durch allerlei Scherze zu
mildern suchte.

"`Ach, du wei"st gar nicht, wie ich dich liebe!"' begann sie von
neuem. "`Ich liebe dich so sehr, da"s ich nicht von dir lassen
kann! Verstehst du da{\s}? Manchmal habe ich solche Sehnsucht,
dich zu sehen, und dann springt mir beinahe da{\s} Herz vor lauter
Liebe! Ich frage mich: wo ist er? Vielleicht spricht er mit andern
Frauen? Sie l"acheln ihm zu. Er macht ihnen den Hof ... Ach nein;
nicht wahr, e{\s} gef"allt dir keine? E{\s} gibt ja sch"onere
al{\s} ich, aber keine kann dich so lieben wie ich! Ich bin deine
Magd, deine Liebste! Und du bist mein Herr, mein Gott! Du bist so
gut! So sch"on! So klug und stark!"'

Dergleichen hatte er in seinem Leben schon so oft geh"ort, da"s
e{\s} ihm ganz und gar nicht{\s} Neue{\s} mehr war. Emma war darin
nicht ander{\s} al{\s} alle seine fr"uheren Geliebten, und der
Reiz der Neuheit fiel St"uck um St"uck von ihr ab wie ein Gewand,
und da{\s} ewige Einerlei der sinnlichen Leidenschaft trat nackt
zutage, die immer dieselbe Gestalt, immer dieselbe Sprache hat. Er
war ein vielerfahrener Mann, aber er ahnte nicht, da"s unter den
n"amlichen Au{\s}druck{\s}formen himmelweit voneinander
verschiedene Gef"uhl{\s}arten existieren k"onnen. Weil ihm die
Lippen liederlicher oder k"auflicher Frauenzimmer schon die
gleichen Phrasen zugefl"ustert hatten, war sein Glaube an die
Aufrichtigkeit einer Frau wie dieser nur schwach.

"`Man darf die "uberschwenglichen Worte nicht gelten lassen,"'
sagte er sich, "`sie sind nur ein M"antelchen f"ur
Alltag{\s}empfindungen."'

Aber ist e{\s} nicht oft so, da"s ein "ubervolle{\s} Herz mit den
banalsten Worten nach Au{\s}druck sucht? Und vermag denn jemand
genau zu sagen, wie gro"s sein W"unschen und Wollen, seine
Innenwelt, seine Schmerzen sind? De{\s} Menschen Wort ist wie eine
gesprungene Pauke, auf der wir eine Melodie herau{\s}trommeln,
nach der kaum ein B"ar tanzt, w"ahrend wir die Sterne bewegen
m"ochten.

Aber mit der "Uberlegenheit, die kritischen Naturen eigent"umlich
ist, die immer Herren ihrer selbst bleiben, entlockte Rudolf auch
dieser Liebschaft neue Gen"usse. Er nahm keine ihm unbequeme
R"ucksicht auf Emma{\s} Schamhaftigkeit mehr. Er behandelte sie
bar jede{\s} Zwange{\s}. Er machte sie zu allem f"ugsam und
verdarb sie gr"undlich. Sie hegte eine geradezu h"undische
Anh"anglichkeit zu ihm. An ihm bewunderte sie alle{\s}. Woll"ustig
empfand sie Gl"uckseligkeiten, die sie von Sinnen machten. Ihre
Seele ertrank in diesem Rausche.

Der Wandel in erotischen Dingen bei ihr begann sich in ihrem
"au"serlichen Wesen zu verraten. Ihre Blicke wurden k"uhner, ihre
Rede freim"utiger. Sie hatte sogar den Mut, in Begleitung
Rudolf{\s}, eine Zigarette im Munde, spazieren zu gehen, "`um die
Spie"ser zu "argern"', wie sie sagte. Und um ihren guten Ruf war
e{\s} g"anzlich geschehen, al{\s} man sie eine{\s} sch"onen
Tage{\s} in einem regelrechten Herrenjackett der Rouener
Postkutsche entsteigen sah. Die alte Frau Bovary, die nach einem
heftigen Zank mit ihrem Manne wieder einmal bei ihrem Sohne
Zuflucht gesucht hatte, entsetzte sich nicht weniger al{\s} die
Yonviller Philister. Und noch viele{\s} andre mi"sfiel ihr.
Zun"achst hatte Karl ihrem Rate entgegen da{\s} Roman-Lesen doch
wieder zugelassen. Und dann war "uberhaupt die "`ganze
Wirtschaft"' nicht nach ihrem Sinne. Al{\s} sie sich Bemerkungen
dar"uber gestattete, kam e{\s} zu einem "argerlichen Auftritt.
Felicie war die n"ahere Veranlassung dazu.

Die alte Frau Bovary hatte da{\s} M"adchen eine{\s} Abend{\s},
al{\s} sie durch den Flur ging, in der Gesellschaft eine{\s} nicht
mehr besonder{\s} jungen Manne{\s} "uberrascht. Der Betreffende
trug ein braune{\s} Hal{\s}tuch und verschwand bei der Ann"aherung
der alten Dame. Emma lachte, al{\s} ihr der Vorfall berichtet
ward, aber die Schwiegermutter ereiferte sich und erkl"arte, wer
bei seinen Dienstboten nicht auf Anstand hielte, lege selber wenig
Wert darauf.

"`Sie sind wohl au{\s} Hinterpommern?"' fragte die junge Frau so
impertinent, da"s sich die alte Frau die Frage nicht verkneifen
konnte, ob sie sich damit selber verteidigen wolle.

"`Verlassen Sie mein Hau{\s}!"' schrie Emma und sprang auf.

"`Emma! Mutter!"' rief Karl beschwichtigend.

In ihrer Erregung waren beide Frauen au{\s} dem Zimmer gest"urzt.
Emma stampfte mit dem Fu"se auf, al{\s} er ihr zuredete.

"`So eine ungebildete Person! So ein Bauernweib!"' rief sie.

Er eilte zur Mutter. Sie war ganz au"ser sich und stammelte:

"`So eine Unversch"amtheit! Eine leichtsinnige Trine.
Schlimmere{\s} vielleicht noch!"'

Sie wollte unverweilt abreisen, wenn sie nicht sofort um
Verzeihung gebeten w"urde.

Karl ging abermal{\s} zu seiner Frau und beschwor sie auf den
Knien, doch nachzugeben. Schlie"slich sagte sie:

"`Meinetwegen!"'

In der Tat streckte sie ihrer Schwiegermutter die Hand hin, mit
der W"urde einer F"urstin.

"`Verzeihen Sie mir, Frau Bovary!"'

Dann eilte sie in ihr Zimmer hinauf, warf sich in ihr Bett, auf
den Bauch, und weinte wie ein Kind, den Kopf in da{\s} Kissen
vergraben.

F"ur den Fall, da"s sich irgend etwa{\s} Besondere{\s} ereignen
sollte, hatte sie mit Rudolf vereinbart, an die Jalousie einen
wei"sen Zettel zu stecken. Wenn er zuf"allig in Yonville w"are,
solle er daraufhin sofort durch da{\s} G"a"schen an die hintere
Gartenpforte eilen.

Diese{\s} Signal gab Emma. Dreiviertel Stunden sa"s sie wartend am
Fenster, da bemerkte sie mit einem Male den Geliebten an der Ecke
der Hallen. Beinahe h"atte sie da{\s} Fenster aufgerissen und ihn
hergerufen. Aber schon war er wieder verschwunden; Verzweiflung
"uberkam sie.

Bald darauf vernahm sie unten auf dem B"urgersteige Tritte. Da{\s}
war er. Zweifello{\s}! Sie eilte die Treppe hinunter und "uber den
Hof. Rudolf war hinten im Garten. Sie fiel in seine Arme.

"`Sei doch ein bi"schen vorsichtiger!"' mahnte er.

"`Ach, wenn du w"u"stest!"' Und sie begann ihm den ganzen Vorfall
zu erz"ahlen, in aller Eile und ohne rechten Zusammenhang. Dabei
"ubertrieb sie manche{\s}, dichtete etliche{\s} hinzu und machte
eine solche Unmenge von Bemerkungen dazwischen, da"s er nicht
da{\s} mindeste von der ganzen Geschichte begriff.

"`So beruhige dich nur, mein Schatz! Mut und Geduld!"'

"`Geduld? Seit vier Jahren hab ich die. Wie ich leide!"' erwiderte
sie. "`Eine Liebe wie die unsrige braucht da{\s} Tage{\s}licht
nicht zu scheuen! Man martert mich! Ich halte e{\s} nicht mehr
au{\s}! Rette mich!"'

Sie schmiegte sich eng an ihn an. Ihre Augen, voll von Tr"anen,
gl"anzten wie Lichter unter Wasser. Ihr Busen wogte ungest"um.

Rudolf war verliebter denn je. Einen Augenblick war er nicht der
k"uhle Gedankenmensch, der er sonst immer war. Und so sagte er:

"`Wa{\s} soll ich tun? Wa{\s} willst du?"'

"`Flieh mit mir!"' rief sie. "`Weit weg von hier! Ach, ich bitte
dich um alle{\s} in der Welt!"'

Sie pre"ste sich an seinen Mund, al{\s} wolle sie ihm mit einem
Kusse da{\s} Ja einhauchen und wieder herau{\s}saugen.

"`Aber~..."'

"`Kein Aber, Rudolf!"'

"`... und dein Kind?"'

Sie dachte ein paar Sekunden nach. Dann sagte sie:

"`Da{\s} nehmen wir mit! Da{\s} ist ihm schon recht!"'

"`Ein Teufel{\s}weib!"' dachte er bei sich, wie er ihr nachsah.
Sie mu"ste in{\s} Hau{\s}. Man hatte nach ihr gerufen.

W"ahrend der folgenden Tage war die alte Frau Bovary "uber da{\s}
ver"anderte Wesen ihrer Schwiegertochter h"ochst verwundert.
Wirklich, sie zeigte sich au"serordentlich f"ugsam, ja ehrerbietig,
und da{\s} ging so weit, da"s Emma sie um ihr Rezept, Gurken
einzulegen, bat.

Verstellte sie sich, um Mann und Schwiegermutter um so sicherer zu
t"auschen? Oder fand sie eine schmerzliche Wollust darin, noch
einmal die volle Bitterni{\s} alle{\s} dessen durchzukosten,
wa{\s} sie im Stiche lassen wollte? Nein, da{\s} lag ihr
durchau{\s} nicht im Sinne. Der Gegenwart entr"uckt, lebte sie im
Vorgeschmacke de{\s} kommenden Gl"ucke{\s}. Davon schw"armte sie
dem Geliebten immer und immer wieder vor. An seine Schulter
gelehnt, fl"usterte sie:

"`Sag, wann werden wir endlich zusammen in der Postkutsche sitzen?
Kannst du dir au{\s}denken, wie da{\s} dann sein wird? Mir ist
e{\s} wie ein Traum! Ich glaube, in dem Augenblick, wo ich sp"ure,
da"s sich der Wagen in Bewegung setzt, werde ich da{\s} Gef"uhl
haben, in einem Luftschiffe aufzusteigen, zur Reise in die Wolken
hinein! Wei"st du, ich z"ahle die Tage ... Und du?"'

Frau Bovary hatte nie so sch"on au{\s}gesehen wie jetzt. Sie
besa"s eine unbeschreibliche Art von Sch"onheit, die au{\s}
Leben{\s}freude, Schw"armerei und Siege{\s}gef"uhl zusammenstr"omt
und da{\s} Symbol seelischer und k"orperlicher Harmonie ist. Ihre
heimlichen L"uste, ihre Tr"ubsal, ihre erweiterten
Liebe{\s}k"unste und ihre ewig jungen Tr"aume hatten sich stetig
entwickelt, just wie D"unger, Regen, Wind und Sonne eine Blume zur
Entfaltung bringen, und nun erst erbl"uhte ihre volle Eigenart.
Ihre Lider waren wie ganz besonder{\s} dazu geschnitten,
schmachtende Liebe{\s}blicke zu werfen; sie verschleierten ihre
Aug"apfel, w"ahrend ihr Atem die feinlinigen Nasenfl"ugel weitete
und e{\s} leise um die H"ugel der Mundwinkel zuckte, die im
Sonnenlichte ein leichter schwarzer Flaum beschattete. Man war
versucht zu sagen: ein Verf"uhrer und K"unstler habe den Knoten
ihre{\s} Haare{\s} "uber dem Nacken geordnet. Er sah au{\s} wie
eine schwere Welle, und doch war er nur lose und l"assig
geschlungen, weil er im Spiel de{\s} Ehebruch{\s} Tag f"ur Tag
aufgenestelt ward. Emma{\s} Stimme war weicher und grazi"oser
geworden, "ahnlich wie ihre Gestalt. Etwa{\s} unsagbar Zarte{\s},
Bezaubernde{\s} str"omte au{\s} jeder Falte ihrer Kleider und
au{\s} dem Rhythmu{\s} ihre{\s} Gange{\s}. Wie in den
Flitterwochen erschien sie ihrem Manne ent\/z"uckend und ganz
unwiderstehlich.

Wenn er nacht{\s} sp"at nach Hause kam, wagte er sie nicht zu
wecken. Da{\s} in seiner Porzellanschale schwimmende Nachtlicht
warf tanzende Kringel an die Decke. Am Bett leuchtete im
Halbdunkel wie ein wei"se{\s} Zelt die Wiege mit ihren zugezogenen
bauschigen Vorh"angen. Karl betrachtete sie und glaubte die leisen
Atemz"uge seine{\s} Kinde{\s} zu h"oren. E{\s} wuch{\s} sichtlich
heran, jeder Monat brachte e{\s} vorw"art{\s}. Im Geiste sah er
e{\s} bereit{\s} abend{\s} au{\s} der Schule heimkehren, froh und
munter, Tintenflecke am Kleid, die Schultasche am Arm. Dann mu"ste
da{\s} M"adel in eine Pension kommen. Da{\s} w"urde viel Geld
kosten. Wie sollte da{\s} geschafft werden? Er sann nach. Wie
w"are e{\s}, wenn man in der Umgegend ein kleine{\s} Gut pachtete?
Alle Morgen, ehe er seine Kranken besuchte, w"urde er hinreiten
und da{\s} N"otige anordnen. Der Ertrag k"ame auf die Sparkasse,
sp"ater k"onnten ja irgendwelche Papiere daf"ur gekauft werden.
Inzwischen erweiterte sich auch seine Praxi{\s}. Damit rechnete
er, denn sein T"ochterchen sollte gut erzogen werden, sie sollte
etwa{\s} Ordentliche{\s} lernen, auch Klavier spielen. Und h"ubsch
w"urde sie sein, die dann F"unfzehnj"ahrige! Ein Ebenbild ihrer
Mutter! Ganz wie sie m"u"ste sie im Sommer einen gro"sen runden
Strohhut tragen. Dann w"urden die beiden von weitem f"ur zwei
Schwestern gehalten. Er stellte sich sein T"ochterchen in Gedanken
vor: abend{\s}, beim Lampenlicht, am Tisch arbeitend, bei Vater
und Mutter, Pantoffeln f"ur ihn stickend. Und in der Wirtschaft
w"urde sie helfen und da{\s} ganze Hau{\s} mit Lachen und Frohsinn
erf"ullen. Und weiter dachte er an ihre Versorgung. E{\s} w"urde
sich schon irgendein braver junger Mann in guten Verh"altnissen
finden und sie gl"ucklich machen. Und so bliebe e{\s} dann
immerdar~...

Emma schlief gar nicht. Sie stellte sich nur schlafend, und
w"ahrend ihr Gatte ihr zur Seite zur Ruhe ging, hing sie fernen
Tr"aumereien nach.

Seit acht Tagen sah sie sich, von vier flotten Rossen entf"uhrt,
auf der Reise nach einem andern Lande, au{\s} dem sie nie wieder
zur"uckzukehren brauchte. Sie und der Geliebte fuhren und fuhren
dahin, Hand in Hand, still und schweigsam. Zuweilen schauten sie
pl"otzlich von Berge{\s}h"oh auf irgendwelche m"achtige Stadt
hinab, mit ihrem Dom, ihren Br"ucken, Schiffen, Limonenhainen und
wei"sen Marmorkirchen mit spitzen T"urmen. Zu Fu"s wanderten sie
dann durch die Stra"sen. Frauen in roten Miedern boten ihnen
Blumenstr"au"se an. Glocken l"auteten, Maulesel schrien, und
dazwischen girrten Gitarren und rauschten Font"anen, deren k"uhler
Wasserstaub auf Haufen von Fr"uchten herabspr"uhte. Sie lagen zu
Pyramiden aufgeschichtet da, zu F"u"sen bleicher Bilds"aulen, die
unter dem Spr"uhregen l"achelten. Und eine{\s} Abend{\s}
erreichten sie ein Fischerdorf, wo braune Netze im Winde
trockneten, am Strand und zwischen den H"utten. Dort wollte sie
bleiben und immerdar wohnen, in einem kleinen Hause mit flachem
Dache, im Schatten hoher Zypressen, an einer Bucht de{\s}
Meere{\s}. Sie fuhren in Gondeln und tr"aumten in H"angematten.
Da{\s} Leben war ihnen so leicht und weit wie ihre seidenen
Gew"ander, und so warm und sternbes"at wie die s"u"sen N"achte,
die sie schauernd genossen ... Da{\s} war ein unerme"slicher
Zukunft{\s}traum; aber bi{\s} in die Einzelheiten dachte sie ihn
nicht au{\s}. Ein Tag glich dem andern, wie im Meer eine Woge der
andern gleicht, an Pracht und Herrlichkeit. Und diese Wogen
fluteten fernhin bi{\s} in den Horizont, endlo{\s}, in leiser
Bewegung, stahlblau und sonnenbegl"anzt~...

Da{\s} Kind in der Wiege begann zu husten, und Bovary schnarchte
laut. Emma schlief erst gegen Morgen ein, al{\s} da{\s} wei"se
D"ammerlicht an den Scheiben stand und Justin dr"uben die L"aden
der Apotheke "offnete.

Emma hatte Lheureux kommen lassen und ihm gesagt:

"`Ich brauche einen Mantel, einen gro"sen gef"utterten Reisemantel
mit einem breiten Kragen."'

"`Sie wollen verreisen?"' fragte der H"andler.

"`Nein, aber ... da{\s} ist ja gleichg"ultig! Ich kann mich auf
Sie verlassen? Nicht wahr? Und recht bald!"'

Lheureux machte einen Kratzfu"s.

"`Und dann brauche ich noch einen Koffer ... keinen zu schweren
... einen handlichen~..."'

"`Sch"on! Sch"on! Ich wei"s schon: zweiundneunzig zu f"unfzig! Wie
man sie jetzt meist hat!"'

"`Und eine Handtasche f"ur da{\s} Nachtzeug!"'

"`Aha,"' dachte der H"andler, "`sie hat sicher Krakeel gehabt!"'

"`Da!"' sagte Frau Bovary, indem sie ihre Taschenuhr au{\s} dem
G"urtel nestelte. "`Nehmen Sie da{\s}! Machen Sie sich damit
bezahlt!"'

Aber Lheureux str"aubte sich dagegen. Da{\s} ginge nicht. Sie
w"are doch eine so gute Kundin. Ob sie kein Vertrauen zu ihm habe?
Wa{\s} solle denn da{\s}? Doch sie bestand darauf, da"s er
wenigsten{\s} die Kette n"ahme.

Er hatte sie bereit{\s} eingesackt und war schon drau"sen, da rief
ihn Emma zur"uck.

"`Behalten Sie da{\s} Bestellte vorl"aufig bei sich! Und den
Mantel~...,"' sie tat so, al{\s} ob sie sich{\s} "uberlegte "`...
den bringen Sie auch nicht erst ... oder noch besser: geben Sie
mir die Adresse de{\s} Schneider{\s} und sagen Sie ihm, der Mantel
soll bei ihm zum Abholen bereitliegen."'

Die Flucht sollte im kommenden Monat erfolgen. Emma sollte
Yonville unter dem Vorwande verlassen, in Rouen Besorgungen zu
machen. Rudolf sollte dort schon vorher die Pl"atze in der Post
bestellen, P"asse besorgen und nach Pari{\s} schreiben, damit
da{\s} Gep"ack gleich direkt bi{\s} Marseille bef"ordert w"urde.
In Marseille wollten sie sich eine Kalesche kaufen, und dann
sollte die Reise ohne Aufenthalt weiter nach Genua gehen. Emma{\s}
Gep"ack sollte Lheureux mit der Post wegbringen, ohne da"s
irgendwer Verdacht sch"opfte. Bei allen diesen Vorbereitungen war
von ihrem Kinde niemal{\s} die Rede. Rudolf vermied e{\s}, davon
zu sprechen. "`Sie denkt vielleicht nicht mehr daran"', sagte er
sich.

Er erbat sich zun"achst zwei Wochen Frist, um seine Angelegenheiten
zu ordnen; nach weiteren acht Tagen forderte er nochmal{\s} zwei
Wochen Zeit. Hernach wurde er angeblich krank, sodann mu"ste er
eine Reise machen. So verging der August, bi{\s} sie sich nach
allen diesen Verz"ogerungen schlie"slich "`unwiderruflich"' auf
Montag den 4. September einigten.

Am Sonnabend vorher stellte sich Rudolf zeitiger denn gew"ohnlich
ein.

"`Ist alle{\s} bereit?"' fragte sie ihn.

"`Ja."'

Sie machten einen Rundgang um die Beete und setzten sich dann auf
den Rand der Gartenmauer.

"`Du bist verstimmt?"' fragte Emma.

"`Nein. Warum auch?"'

Dabei sah er sie mit einem sonderbaren z"artlichen Blick an.

"`Vielleicht weil e{\s} nun fortgeht?"' fragte sie. "`Weil du
Dinge, die dir lieb sind, verlassen sollst, dein ganze{\s}
jetzige{\s} Leben? Ich verstehe da{\s} wohl, wenn ich selber auch
nicht{\s} derlei auf der Welt habe. Du bist mein alle{\s}! Und
ebenso m"ochte ich dir alle{\s} sein, Familie und Vaterland. Ich
will dich hegen und pflegen. Und dich lieben!"'

"`Wie lieb du bist!"' sagte er und zog sie an sein Herz.

"`Wirklich?"' fragte sie in lachender Wollust. "`Du liebst mich?
Schw"ore mir{\s}!"'

"`Ob ich dich liebe! Ob ich dich liebe! Ich bete dich an,
Liebste!"'

Der Vollmond ging purpurrot auf, dr"uben "uber der Linie de{\s}
flachen Horizont{\s}, wie mitten in den Wiesen. Rasch stieg er
hoch, und schon stand er hinter den Pappeln und schimmerte durch
ihre Zweige, versteckt wie hinter einem l"ochrigen, schwarzen
Vorhang. Und bald erschien er gl"anzend-wei"s im klaren Raume
de{\s} weiten Himmel{\s}. Er ward immer silberner, und nun
rieselte seine Lichtflut auch unten im Bache "uber den Wellen in
zahllosen funkelnden Sternen, wie ein Strom geschmolzener
Diamanten. Ring{\s}um leuchtete die laue lichte Sommernacht. Nur
in den Wipfeln hingen dunkle Schatten.

Mit halbgeschlossenen Augen atmete Emma in tiefen Z"ugen den
k"uhlen Nachtwind ein. Sie sprachen beide nicht, ganz versunken
und verloren in ihre Gedanken. Die Z"artlichkeit vergangener Tage
ergriff von neuem ihre Herzen, unersch"opflich und schweigsam wie
der dahinflie"sende Bach, lind und leise wie der Fliederduft. Die
Erinnerung an da{\s} Einst war von Schatten durchwirkt, die
verschwommener und wehm"utiger waren al{\s} die der unbeweglichen
Weiden, deren Umrisse au{\s} den Gr"asern wuchsen. Zuweilen
raschelte auf seiner n"achtlichen Jagd ein Tier durch{\s}
Gestr"auch, ein Igel oder ein Wiesel, oder man h"orte, wie ein
reifer Pfirsich von selber zur Erde fiel.

"`Wa{\s} f"ur eine wunderbare Nacht!"' sagte Rudolf.

"`Wir werden noch sch"onere erleben!"' erwiderte Emma. Und wie zu
sich selbst fuhr sie fort: "`Ach, wie herrlich wird unsere Reise
werden ... Aber warum ist mir da{\s} Herz so schwer? Warum wohl?
Ist e{\s} die Angst vor dem Unbekannten ... oder die Scheu, da{\s}
Gewohnte zu verlassen ... oder wa{\s} ist{\s}? Ach, e{\s} ist
da{\s} "Uberma"s von Gl"uck! Ich bin zaghaft, nicht? Verzeih
mir!"'

"`Noch ist e{\s} Zeit!"' rief er au{\s}. "`"Uberleg dir{\s}! Wird
e{\s} dich auch niemal{\s} reuen?"'

"`Niemal{\s}!"' beteuerte sie leidenschaftlich.

Sie schmiegte sich an ihn.

"`Wa{\s} k"onnte mir denn Schlimme{\s} bevorstehen! E{\s} gibt
keine W"uste, kein Weltmeer, die ich mit dir zusammen nicht
durchqueren w"urde! Je l"anger wir zusammen leben werden, um so
inniger und vollkommener werden wir un{\s} lieben! Keine Sorge,
kein Hinderni{\s} wird un{\s} mehr qu"alen! Wir werden allein sein
und ein{\s} immerdar ... Sprich doch! Antworte mir!"'

Er antwortete wie ein Uhrwerk in gleichen Zwischenr"aumen:

"`Ja ... ja ... ja!"'

Sie strich mit den H"anden durch sein Haar und fl"usterte wie ein
kleine{\s} Kind unter gro"sen rollenden Tr"anen immer wieder:

"`Rudolf ... Rudolf ... ach, Rudolf ... mein lieber guter
Rudolf~..."'

E{\s} schlug Mitternacht.

"`Mitternacht!"' sagte sie. "`Nun hei"st e{\s}: morgen! Nur noch
ein Tag!"'

Er stand auf und schickte sich an zu gehen. Und al{\s} ob diese
Geb"arde ein Symbol ihrer Flucht sei, wurde Emma mit einem Male
fr"ohlich.

"`Hast du die P"asse?"' fragte sie.

"`Ja."'

"`Hast du nicht{\s} vergessen?"'

"`Nein."'

"`Wei"st du da{\s} genau?"'

"`Ganz genau!"'

"`Nicht wahr, du erwartest mich im Provencer Hof? Mittag{\s}?"'

Er nickte.

"`Also morgen auf Wiedersehen!"' sagte Emma mit einem letzten
Kusse.

Er ging, und sie sah ihm nach.

Er blickte sich nicht um. Da lief sie ihm nach bi{\s} an den
Bachrand und rief durch die Weiden hindurch:

"`Auf morgen!"'

Er war schon dr"uben auf dem andern Ufer und eilte den Pfad durch
die Wiesen hin. Nach einer Weile blieb er stehen. Al{\s} er sah,
wie ihr wei"se{\s} Kleid allm"ahlich im Schatten verschwand wie
eine Vision, da bekam er so heftige{\s} Herzklopfen, da"s er sich
gegen einen Baum lehnen mu"ste, um nicht umzusinken.

"`Ich bin kein Mann!"' rief er au{\s}. "`Hol mich der Teufel! Ein
h"ubsche{\s} Weib war{\s} doch!"'

Emma{\s} Reize und all die Freuden der Liebschaft mit ihr lockten
ihn noch einmal. Er ward weich. Dann aber emp"orte er sich gegen
diese R"uhrung.

"`Nein, nein! Ich kann Hau{\s} und Hof nicht verlassen!"'

Er gestikulierte heftig.

"`Und dann da{\s} l"astige Kind ... die Scherereien ... die
Kosten!"'

Er z"ahlte sich da{\s} alle{\s} auf, um sich stark zu machen.

"`Nein, nein! Tausendmal nein! E{\s} w"are eine Riesentorheit!"'


\newpage\begin{center}
{\large \so{Dreizehnte{\s} Kapitel}}\bigskip\bigskip
\end{center}

Kaum auf seinem Gute angekommen, setzte sich Rudolf eiligst an den
Schreibtisch, "uber dem an der Wand ein Hirschgeweih, eine
Jagdtroph"ae, hing. Aber sowie er die Feder in der Hand hatte,
wu"ste er nicht, wa{\s} er schreiben sollte. Den Kopf zwischen
beide H"ande gest"utzt, begann er nachzudenken. Emma war ihm in
weite Ferne entr"uckt. Der blo"se Entschlu"s, mit ihr zu brechen,
hatte sie ihm mit einem Male ungeheuerlich entfremdet.

Um sie greifbarer vor sich zu haben, suchte er au{\s} dem
Schranke, der am Kopfende seine{\s} Bette{\s} stand, eine alte
Blechschachtel hervor, in der urspr"unglich einmal Kake{\s} drin
gewesen waren und in der er seine "`Weiberbriefe"' aufbewahrte.
Geruch von Moder und vertrockneten Rosen drang ihm entgegen. Zu
oberst lag ein Taschentuch, verbla"ste Blutflecken darauf. E{\s}
war von Emma; auf einem ihrer gemeinsamen Spazierg"ange hatte sie
einmal Nasenbluten bekommen. Jetzt fiel e{\s} ihm wieder ein.
Daneben lag ein Bild von ihr, da{\s} sie ihm geschenkt hatte. Alle
vier Ecken daran waren abgesto"sen. Da{\s} Kleid, da{\s} sie auf
diesem Bilde anhatte, kam ihm theatralisch vor und ihr himmelnder
Blick j"ammerlich. Wie er sich ihr Konterfei so betrachtete und
sich da{\s} Urbild in die Phantasie zur"uckzurufen suchte,
verschwammen Emma{\s} Z"uge in seinem Ged"achtnisse, gleichsam
al{\s} ob sich die noch lebende Erinnerung und da{\s} gemalte
Bildchen gegenseitig befehdeten und ein{\s} da{\s} andre
vernichtete.

Nun fing er an, in ihren Briefen zu lesen. Die au{\s} der letzten
Zeit wimmelten von Anspielungen auf die Reise; sie waren kurz,
sachlich und in Eile hingeschrieben, wie Gesch"aft{\s}briefe. Er
suchte nach den langen Briefen von einst. Da sie zu unterst lagen,
mu"ste er den ganzen Kasten durchw"uhlen. Au{\s} dem Wust von
Papieren und kleinen Gegenst"anden zog er mechanisch welke Blumen,
ein Strumpfband, eine schwarze Ma{\s}ke, Haarnadeln und Locken
herau{\s}. Braune und blonde Locken. Ein paar Haare davon hatten
sich in{\s} Scharnier gezw"angt und rissen nun beim
Herau{\s}nehmen~...

Mit allen diesen Andenken vertr"odelte er eine Weile. Er stellte
seine Betrachtungen "uber die verschiedenen Handschriften an,
"uber den Stil in den einzelnen Briefb"undeln, "uber die nicht
minder variierende Rechtschreibung darin. Die einen hatten
z"artlich geschrieben, andre lustig, witzig oder r"uhrselig. Die
wollten Liebe, jene Geld. Zuweilen erinnerte sich Rudolf bei einem
bestimmten Worte an Gesichter, an gewisse Gesten, an den Klang
einer Stimme. Manche wiederum beschworen nicht die geringste
Erinnerung herauf.

Alle diese Frauen kamen ihm jetzt alle auf einmal in den Sinn.
Jede war eine Feindin der andern. Alle zogen sie sich gegenseitig
in den Schmutz. Etwa{\s} Gemeinsame{\s} -- die Liebe -- stellte
sie allesamt auf ein und da{\s}selbe Niveau.

Wahllo{\s} nahm er einen Sto"s Briefe in die Finger, bildete eine
Art F"acher darau{\s} und spielte damit. Schlie"slich aber warf er
sie, halb gelangweilt, halb vertr"aumt, wieder in den Kasten und
stellte diesen in den Schrank zur"uck.

"`Lauter Bl"odsinn!"'

Da{\s} war der Extrakt seiner Leben{\s}wei{\s}heit. Sein Herz war
wie ein Schulhof, auf dem die Kinder so erbarmung{\s}lo{\s}
herumgetrampelt waren, da"s kein gr"uner Halm mehr spro"s. Die
Freuden de{\s} Dasein{\s} hatten noch gr"undlicher gewirtschaftet.
Die Sch"uler kritzeln ihre Namen an die Mauern. In Rudolf{\s} Herz
war keiner zu lesen.

"`Nun aber lo{\s}!"' rief er sich zu.

Er begann zu schreiben:

"`Liebe Emma!

Sei tapfer! Ich will Dir Deine Existenz nicht zertr"ummern~..."'

"`Eigentlich sehr richtig!"' dachte er bei sich. "`Da{\s} ist nur
in ihrem Interesse. Also durchau{\s} anst"andig von mir~..."'

"`... Hast Du Dir Deinen Entschlu"s wirklich reiflich "uberlegt?
Hast Du aber auch den Abgrund bemerkt, arme{\s} Lieb, in den ich
Dich beinahe schon gef"uhrt h"atte? Wohl nicht! Du folgst mir
tollk"uhn und zuversichtlich, im festen Glauben an da{\s} Gl"uck,
an die Zukunft! Ach, wie ungl"ucklich sind wir! Und wie verblendet
waren wir!"'

Rudolf h"orte zu schreiben auf. Er suchte nach guten
Au{\s}fl"uchten. "`Wenn ich ihr nun sagte, ich h"atte mein
Verm"ogen verloren? Ach, nein, lieber nicht! "Ubrigen{\s} n"utzte
da{\s} nicht{\s}. Die Geschichte ging dann doch wieder von neuem
lo{\s}. E{\s} ist, wei"s Gott, verdammt schwer, so eine Frau
wieder vern"unftig zu machen!"'

Er sann nach, dann schrieb er weiter:

"`Ich werde Dich niemal{\s} vergessen. Glaube mir da{\s}! Mein
ganze{\s} Leben lang werde ich in inniger Verehrung Deiner
gedenken. So aber h"atte sich unsre Leidenschaft (da{\s} ist nun
einmal da{\s} Schicksal alle{\s} Menschlichen!) eine{\s} Tage{\s},
fr"uher oder sp"ater, doch verfl"uchtet. Zweifello{\s}! Wir w"aren
ihrer m"ude geworden, und wer wei"s, ob mir nicht der gr"a"sliche
Schmerz beschieden gewesen w"are, Deine Reue zu erleben und selber
welche zu empfinden al{\s} Veranlasser der Deinigen? Die blo"se
Vorstellung, Dir diese{\s} Leid verursachen zu k"onnen, martert
mich. Liebste Emma, vergi"s mich! Wir h"atten un{\s} nie kennen
lernen sollen! Warum bist Du so sch"on! Bin ich der Schuldige? Bei
Gott, nein, nein! Wir m"ussen da{\s} Schicksal anklagen~..."'

"`Diese{\s} Wort machte immer Eindruck"', sagte er zu sich.

"`Ja, wenn Du eine leichtsinnige Frau w"arst, wie e{\s} ihrer so
viele gibt, ja dann h"atte ich den Versuch wagen k"onnen, au{\s}
Egoi{\s}mu{\s}, ohne Gefahr f"ur Dich. Aber bei Deiner k"ostlichen
schw"armerischen Art, dem Quell Deine{\s} Reize{\s} und zugleich
Deine{\s} vielen Kummer{\s}, bist Du nicht imstande, Du Beste
aller Frauen, die Kehrseite unsrer zuk"unftigen Stellung in der
Welt vorau{\s}zusehen. Auch ich habe zun"achst gar nicht daran
gedacht, habe mich in unserm H"ohengl"ucke behaglich gesonnt, mich
in ein M"archenland getr"aumt und mich um keine Folgen
gek"ummert~..."'

"`Vielleicht glaubt sie, ich z"oge mich au{\s} Geiz zur"uck ...
Auch egal! Desto besser! Wenn{\s} nur Schlu"s wird!"'

"`... Die Welt ist grausam, geliebte Emma. Man h"atte un{\s}
"uberall, wohin wir gekommen w"aren, Schwierigkeiten bereitet. Du
h"attest unversch"amte Fragen, Verleumdungen, Schm"ahungen und
vielleicht Beleidigungen "uber Dich ergehen lassen m"ussen.
Beleidigungen, Du! Und ich wollte Dich zu meiner K"onigin erheben.
Du solltest mein Heiligste{\s} sein. Nun bestrafe ich mich mit der
Verbannung, weil ich Dir so viel Schlimme{\s} angetan habe. Ich
gehe fort. Wohin? Ach, ich wei"s e{\s} nicht, ich bin wahnsinnig!

Lebwohl! Bleib immer gut! Und vergi"s den Ungl"ucklichen nicht
ganz, der Dich verloren hat! Lehre Deine Kleine meinen Namen,
damit sie mich in ihre Gebete einschlie"st!"'

Die Lichter der beiden Kerzen flackerten unruhig. Rudolf stand vom
Schreibtisch auf und schlo"s da{\s} Fenster.

"`So! Ich denke, da{\s} gen"ugt! Halt! Noch etwa{\s}! Auf keinen
Fall eine Au{\s}sprache!"'

Er setzte sich wieder hin und schrieb weiter:

"`Wenn Du diese betr"ubten Zeilen lesen wirst, bin ich schon weit
weg, denn ich mu"s eilend{\s} fliehen, um der Versuchung zu
entrinnen, Dich wiedersehen zu wollen. Ich darf nicht schwach
werden! Wenn ich wiederkomme, dann werden wir vielleicht
miteinander von unsrer verlorenen Liebe reden, k"uhl und
vern"unftig. Adieu!"'

Er setzte noch ein "`A dieu!"' darunter, in zwei Worten
geschrieben. Da{\s} hielt er f"ur sehr geschmackvoll.

"`Wie soll ich nun unterzeichnen?"' fragte er sich. "`Dein
ergebenster? Nein! Dein treuer Freund? Ja, ja! Machen wir!"'

Und er schrieb:

\hfill "`Dein treuer Freund

\hfill R."' \hspace{3em}

Er la{\s} den ganzen Brief noch einmal durch. Er gefiel ihm.

"`Arme{\s} Frauchen!"' dachte er in einem Anflug von R"uhrseligkeit.
"`Sie wird denken, ich sei gef"uhllo{\s} wie Stein. Eigentlich
fehlen ein paar Tr"anenspuren. Aber heulen kann ich nicht. Da{\s}
ist mein Fehler."'

Er go"s etwa{\s} Wasser au{\s} der Flasche in ein Gla{\s}, tauchte
einen Finger hinein, hielt die Hand hoch und lie"s einen gro"sen
Tropfen auf den Briefbogen herabfallen. Die Tinte der Schrift
f"arbte ihn bla"sblau. Um den Brief zu versiegeln, suchte er nun
nach einem Petschaft. Da{\s} mit dem Wahlspruch
\begin{antiqua}Amor nel Cor\end{antiqua} geriet ihm in die Hand.

"`Pa"st eigentlich nicht gerade!"' dachte er. "`Ach wa{\s}! Tut
nicht{\s}!"'

Er rauchte noch drei Pfeifen und ging dann schlafen.

E{\s} war sp"at geworden. Am andern Tage stand er mittag{\s} gegen
zwei Uhr auf. Al{\s}bald lie"s er ein K"orbchen Aprikosen
pfl"ucken, legte den Brief unter die Weinbl"atter am Boden und
befahl Gerhard, seinem Kutscher, den Korb unverz"uglich Frau
Bovary zu bringen. Auf diese Art hatte er Emma h"aufig Nachrichten
zukommen lassen, je nach der Jahre{\s}zeit, zusammen mit Fr"uchten
oder Wild.

"`Wenn sie sich nach mir erkundigt,"' instruierte er, "`dann
antwortest du, ich sei verreist! Den Korb gibst du ihr pers"onlich
in die H"ande! Verstanden? So! Ab!"'

Gerhard zog seine neue Bluse an, kn"upfte sein Taschentuch "uber
die Aprikosen und marschierte in seinen Nagelschuhen mit
schwerf"alligen Schritten voller Gem"ut{\s}ruhe gen Yonville.

Al{\s} der Kutscher dort ankam, war Frau Bovary gerade damit
besch"aftigt, auf dem K"uchentische zusammen mit Felicie W"asche
zu falten.

"`Eine sch"one Empfehlung von meinem Herrn,"' vermeldete er, "`und
da{\s} schickt er hier!"'

Emma "uberkam eine bange Ahnung, und w"ahrend sie in ihrer
Sch"urzentasche nach einem Geldst"ucke zum Trinkgeld suchte, sah
sie den Mann mit verst"ortem Blick an. Der betrachtete sie
verwundert; er begriff nicht, da"s ein solche{\s} Geschenk
jemanden so sehr aufregen k"onne. Dann ging er.

Felicie war noch da. Emma hielt e{\s} nicht l"anger au{\s}, sie
eilte in da{\s} E"szimmer, indem sie sagte, sie wolle die
Aprikosen dahin tragen. Dort sch"uttete sie den Korb au{\s}, nahm
die Weinbl"atter herau{\s} und fand den Brief. Sie "offnete ihn
und floh hinauf nach ihrem Zimmer, al{\s} brenne e{\s} hinter ihr.
Sie war fassung{\s}lo{\s} vor Angst.

Karl war auf dem Flur. Sie sah ihn. Er sagte etwa{\s} zu ihr. Sie
verstand e{\s} nicht. Nun lief sie hastig noch eine Treppe h"oher,
au"ser Atem, wie vor den Kopf geschlagen, halbverr"uckt, immer den
unseligen Brief fest in der Hand, der ihr zwischen den Fingern
knisterte. Im zweiten Stock blieb sie vor der geschlossenen
Bodent"ure stehen.

Sie wollte sich beruhigen. Der Brief kam ihr nicht au{\s} dem
Sinn. Sie wollte ihn ordentlich lesen, aber sie wagte e{\s} nicht.
Nirgend{\s} war sie ungest"ort.

"`Ja, hier geht{\s}!"' sagte sie sich. Sie klinkte die T"ur auf
und trat in die Bodenkammer.

Unter den Schieferplatten de{\s} Dache{\s} br"utete dumpfe
Schw"ule, die ihr auf die Schl"afen dr"uckte und den Atem benahm.
Sie schleppte sich bi{\s} zu dem gro"sen Bodenfenster und stie"s
den Holzladen auf. Grelle{\s} Licht flutete ihr entgegen.

Vor ihr, "uber den D"achern, breitete sich da{\s} Land bi{\s} in
die Fernen. Unter ihr der Markt war menschenleer. Die Steine
de{\s} Fu"ssteig{\s} gl"anzten. Die Wetterfahnen der H"auser
standen unbeweglich. Au{\s} dem Eckhause schr"ag gegen"uber,
au{\s} einem der Dachfenster drang ein schnarrende{\s},
kreischende{\s} Ger"ausch herauf. Binet sa"s an seiner Drehbank.

Emma lehnte sich an da{\s} Fensterkreuz und la{\s} den Brief mit
zornverzerrtem Gesicht immer wieder von neuem. Aber je
gr"undlicher sie ihn studierte, um so wirrer wurden ihre Gedanken.
Im Geist sah sie den Geliebten, h"orte ihn reden, zog ihn
leidenschaftlich an sich. Da{\s} Herz schlug ihr in der Brust wie
mit wuchtigen Hammerschl"agen, die immer rascher und
unregelm"a"siger wurden. Ihre Augen irrten im Kreise. Sie f"uhlte
den Wunsch in sich, da"s die ganze Welt zusammenst"urze. Wozu
weiterleben? Wer hinderte sie, ein Ende zu machen, sie, die
Vogelfreie?

Sie bog sich weit au{\s} dem Fenster herau{\s} und starrte hinab
auf da{\s} Stra"senpflaster.

"`Mut! Mut!"' rief sie sich zu.

Da{\s} leuchtende Pflaster da unten zog die Last ihre{\s}
K"orper{\s} f"ormlich in die Tiefe. Sie hatte die Empfindung,
al{\s} bewege sich die Fl"ache de{\s} Marktplatze{\s} und hebe
sich an den H"ausermauern empor zu ihr. Und die Diele, auf der sie
stand, begann zu schwanken wie da{\s} Deck eine{\s} Seeschiffe{\s}
... Sie lehnte sich noch weiter zum Fenster hinau{\s}. Schon hing
sie beinahe im freien Raume. Der weite blaue Himmel umgab sie, und
die Luft strich ihr um den wie hohlen Kopf. Sie brauchte nur noch
sich nicht mehr fest\/zuhalten, nur noch die H"ande lo{\s}zulassen
... Ohne Unterla"s summte unten die Drehbank wie die rufende
Stimme eine{\s} b"osen Geiste{\s}~...

In diesem Moment rief Karl:

"`Emma! Emma!"'

Da kam sie wieder zur Besinnung.

"`Wo steckst du denn? Komm doch!"'

Der Gedanke, da"s sie soeben dem Tode entronnen war, erf"ullte sie
mit Schrecken und Grauen. Sie schlo"s die Augen. Zusammenfahrend
f"uhlte sie sich von jemandem am Arm gefa"st: e{\s} war Felicie.

"`Gn"adige Frau, die Suppe ist angerichtet. Herr Bovary wartet."'

Sie mu"ste hinunter, mu"ste sich mit zu Tisch setzen.

Sie versuchte zu essen, aber sie brachte nicht einen Bissen
hinunter. Sie faltete ihre Serviette au{\s}einander, al{\s} ob sie
sich die au{\s}gebesserten Stellen genau ansehen wollte, und
wirklich tat sie da{\s} und begann die F"aden de{\s} Gewebe{\s} zu
z"ahlen ... Pl"otzlich fiel ihr der Brief wieder ein. Hatte sie
ihn oben fallen lassen? Wohin war er? Aber ihr Geist war zu matt,
al{\s} da"s sie imstande gewesen w"are, einen Vorwand zu ersinnen,
um bei Tisch aufstehen zu k"onnen. Sie war feig geworden. Sie
hatte Furcht vor Karl. Sicherlich wu"ste er nun alle{\s},
sicherlich! Und wahrhaftig, da sagte er mit eigent"umlicher
Betonung:

"`Rudolf werden wir wohl nicht sobald wieder zu sehen kriegen?"'

"`Wer hat dir da{\s} gesagt?"' fragte sie zitternd.

"`Wer mir da{\s} gesagt hat?"' wiederholte er, ein wenig betroffen
von dem harten Klang ihrer Frage. "`Na, sein Kutscher, dem ich
vorhin vor dem Cafe Fran\c{c}ai{\s} begegnet bin. Boulanger ist
verreist, oder er steht im Begriff zu verreisen~..."'

Emma schluchzte laut auf.

"`Wundert dich da{\s}?"' fuhr er fort. "`Er verdr"uckt sich doch
immer mal von Zeit zu Zeit so. Um sich zu zerstreuen. Kann{\s} ihm
nicht verdenken. Wenn man da{\s} n"otige Geld dazu hat und
Junggeselle ist~... "Ubrigen{\s} ist unser Freund ein
Leben{\s}k"unstler! Ein alter Sch"aker! Langloi{\s} hat mir
erz"ahlt~..."'

Er verstummte, au{\s} Anstand, weil da{\s} Dienstm"adchen gerade
hereinkam. Sie legte die Aprikosen wieder ordentlich in da{\s}
K"orbchen, da{\s} auf der Kredenz stand. Karl lie"s e{\s} sich auf
den Tisch bringen, ohne zu bemerken, da"s seine Frau rot wurde. Er
nahm eine der Fr"uchte und bi"s hinein.

"`Ah!"' machte er. "`Vorz"uglich! Koste mal!"'

Er schob ihr da{\s} K"orbchen zu. Sie wehrte leicht ab.

"`So riech doch wenigsten{\s}! Da{\s} ist ein Duft!"'

Er hielt ihr eine Aprikose link{\s} und recht{\s} an die Nase.

"`Ich bekomm keine Luft!"' rief sie und sprang auf. Aber schnell
beherrschte sie sich wieder, mit Aufgebot aller ihrer Kraft.
"`E{\s} war nicht{\s}! Gar nicht{\s}! Wieder meine Nerven! Setz
dich nur wieder hin und i"s!"'

Sie f"urchtete, er k"onne sie au{\s}fragen, um sie besorgt sein
und sie dann nicht allein lassen. Karl gehorchte ihr und setzte
sich wieder. Er spuckte die Aprikosenkerne immer erst in die Hand
und legte sie dann auf seinen Teller.

Da fuhr drau"sen ein blauer Dogcart im flotten Trabe "uber den
Markt. Emma stie"s einen Schrei au{\s} und fiel r"uckling{\s}
langhin zu Boden.

Rudolf hatte sich nach langer "Uberlegung entschlossen, nach Rouen
zu fahren. Da nun aber von der H"uchette nach dorthin kein anderer
Weg al{\s} der "uber Yonville f"uhrte, mu"ste er diesen Ort wohl
oder "ubel ber"uhren. Emma hatte ihn im Scheine der Wagenlaternen,
die drau"sen die Dunkelheit wie Sterne durchhuschten, erkannt.

Der Apotheker, der sofort gemerkt hatte, da"s im Hause de{\s}
Arzte{\s} "`wa{\s} lo{\s} sei"', st"urzte herbei. Der E"stisch war
mit allem, wa{\s} darauf gestanden, umgest"urzt. Die Teller,
da{\s} Fleisch, die Sauce, die Bestecke, Salz und "Ol, alle{\s}
lag auf dem Fu"sboden umher. Karl hatte den Kopf verloren, die
erschrockene kleine Berta schrie, und Felicie nestelte ihrer in
Zuckungen daliegenden Herrin mit bebenden H"anden die Kleider auf.

"`Ich werde schnell Kr"auteressig au{\s} meinem Laboratorium
holen!"' sagte Homai{\s}.

Al{\s} man Emma da{\s} Fl"aschchen an{\s} Gesicht hielt, schlug
sie seufzend die Augen wieder auf.

"`Nat"urlich!"' meinte der Apotheker. "`Damit kann man Tote
erwecken!"'

"`Sprich!"' bat Karl. "`Rede! Erhole dich! Ich bin ja da, dein
Karl, der dich liebt! Erkennst du mich? Hier ist auch Berta! Gib
ihr einen Ku"s!"'

Da{\s} Kind streckte die "Armchen nach der Mutter au{\s} und
wollte sie um den Hal{\s} fassen. Aber Emma wandte den Kopf weg
und stammelte:

"`Nicht doch! Niemanden!"'

Sie wurde abermal{\s} ohnm"achtig. Man trug sie in ihr Bett.

Lang au{\s}gestreckt lag sie da, mit offnem Munde, die Lider
geschlossen, die H"ande schlaff herabh"angend, regung{\s}lo{\s}
und bla"s wie ein Wach{\s}bild. Ihren Augen entquollen Tr"anen,
die in zwei Ketten langsam auf da{\s} Kissen rannen.

Karl stand an ihrem Bett; neben ihm der Apotheker, stumm und
nachdenklich, wie da{\s} bei ernsten Vorf"allen so herk"ommlich
ist.

"`Beruhigen Sie sich!"' sagte Homai{\s} und zupfte den Arzt. "`Ich
glaube, der Paroxy{\s}mu{\s} ist vor"uber."'

"`Ja,"' erwiderte Karl, die Schlummernde betrachtend. "`Jetzt
scheint sie ein wenig zu schlafen, die "Armste! Ein R"uckfall in
da{\s} alte Leiden!"'

Nun erkundigte sich Homai{\s}, wie da{\s} gekommen sei. Karl gab
zur Antwort:

"`Ganz pl"otzlich! W"ahrend sie eine Aprikose a"s."'

"`H"ochst merkw"urdig!"' meinte der Apotheker. "`E{\s} ist
indessen m"oglich, da"s die Aprikosen die Ohnmacht verursacht
haben. E{\s} gibt gewisse Naturen, die f"ur bestimmte Ger"uche
stark empf"anglich sind. E{\s} w"are eine sehr interessante
Arbeit, diese Erscheinungen wissenschaftlich zu untersuchen,
sowohl nach physiologischen wie nach pathologischen
Gesicht{\s}punkten. Die Pfaffen haben von jeher gewu"st, wie
wertvoll da{\s} f"ur sie ist. Die Verwendung von Weihrauch beim
Gotte{\s}dienst ist uralt. Damit schl"afert man den Verstand ein
und versetzt And"achtige in Ekstase, am leichtesten "ubrigen{\s}
weibliche Wesen. Die sind feinnerviger al{\s} wir M"anner. Ich
habe von F"allen gelesen, wo Frauen ohnm"achtig geworden sind beim
Geruch von verbranntem Horn, frischem Brot~..."'

"`Geben Sie acht, da"s sie nicht aufgeweckt wird!"' mahnte Bovary
mit fl"usternder Stimme.

"`Diese Anomalien kommen aber nicht allein bei Menschen vor,"'
fuhr der Apotheker fort, "`sondern sogar bei Tieren. Zweifello{\s}
ist Ihnen nicht unbekannt, da"s \begin{antiqua}Nepeta
cataria\end{antiqua}, vulg"ar Katzenminze, sonderbarerweise auf
da{\s} gesamte Katzengeschlecht al{\s} Aphrodisiakum wirkt. Einen
weiteren Beleg kann ich au{\s} meiner eigenen Erfahrung anf"uhren.
Bridoux, ein Studienfreund von mir -- er wohnt jetzt in der
Malpalu-Stra"se -- besitzt einen Foxterrier, der jede{\s}mal
Kr"ampfe bekommt, wenn man ihm eine Schnupftabak{\s}dose vor die
Nase h"alt. Ich habe diese{\s} Experiment selber ein paarmal mit
angesehen, im Landhause meine{\s} Freunde{\s} am Wilhelm{\s}walde.
Sollte man{\s} f"ur m"oglich halten, da"s ein so harmlose{\s}
Niesemittel in den Organi{\s}mu{\s} eine{\s} Vierf"u"sler{\s}
derartig eingreifen kann? Da{\s} ist h"ochst merkw"urdig, nicht
wahr?"'

"`Gewi"s!"' sagte Karl, der gar nicht darauf geh"ort hatte.

"`Da{\s} beweist un{\s},"' fuhr der andre fort,
gutm"utig-selbstgef"allig l"achelnd, "`da"s im Nervensystem
zahllose Unregelm"a"sigkeiten m"oglich sind. Ich mu"s gestehen,
da"s mir Ihre Frau Gemahlin immer au"serordentlich reizsam
vorgekommen ist. Darum m"ochte ich Ihnen, verehrter Freund, auf
keinen Fall raten, ihr eine jener Arzneien zu verordnen, die
angeblich die Symptome so einer Krankheit beseitigen sollen, in
Wirklichkeit aber nur der Gesundheit schaden. Nein, nein, hier
sind Medikamente unn"utz! Di"at! Weiter nicht{\s}! Beruhigende,
milde, kr"aftigende Kost! Und dann, k"onnte man bei ihr nicht auch
irgendwie auf die Einbildung{\s}kraft einzuwirken versuchen?"'

"`Wieso? Womit?"'

"`Ja, da{\s} ist eben die Frage! Da{\s} ist wirklich die Frage!
\begin{antiqua}That is the question\end{antiqua}! -- wie ich
neulich in der Zeitung gelesen habe."'

Emma erwachte und rief:

"`Der Brief? Der Brief?"'

Die beiden M"anner glaubten, sie rede im Delirium. In der Tat trat
da{\s} mitternacht{\s} ein. Emma hatte Gehirnent\/z"undung.

In den n"achsten sech{\s} Wochen wich Karl nicht von ihrem Lager.
Er vernachl"assigte alle seine Patienten. Er schlief kaum mehr,
unerm"udlich ma"s er ihren Pul{\s}, legte ihr Senfpflaster auf und
erneute die Kaltwasser-Umschl"age. Er schickte Justin nach
Neufch\^atel, um Ei{\s} zu holen. E{\s} schmolz unterweg{\s}.
Justin mu"ste nochmal{\s} hin. Doktor Canivet wurde konsultiert.
Professor Larivi\`ere, sein ehemaliger Lehrer, ward au{\s} Rouen
hergeholt. Karl war der v"olligen Verzweiflung nahe. Am meisten
"angstigte ihn Emma{\s} Apathie. Sie sprach nicht, interessierte
sich f"ur nicht{\s}, ja, sie schien selbst die Schmerzen nicht zu
empfinden. E{\s} war, al{\s} h"atten K"orper wie Geist bei ihr
alle ihre Funktionen eingestellt.

Gegen Mitte Oktober konnte sie, von Kissen gest"utzt, wieder
aufrecht in ihrem Bette sitzen. Al{\s} sie da{\s} erste Br"otchen
mit eingemachten Fr"uchten verzehrte, da weinte Karl. Allm"ahlich
kehrten ihre Kr"afte zur"uck. Sie durfte nachmittag{\s} ein paar
Stunden aufstehen, und eine{\s} Tage{\s} f"uhlte sie sich soweit
wohl, da"s sie an Karl{\s} Arm einen kleinen Spaziergang durch den
Garten versuchte.

Auf den sandigen Wegen lag gefallene{\s} Laub. Sie ging ganz
langsam, in Hau{\s}schuhen, ohne die F"u"se zu heben. An Karl
angeschmiegt, l"achelte sie in einem fort vor sich hin.

So schritten sie bi{\s} hinter an die Gartenmauer. Dort blieb sie
stehen und richtete sich auf. Um besser zu sehen, hob sie die Hand
"uber die Augen. Lange schaute sie hinau{\s} in die Weite. Aber
e{\s} gab in der Ferne nicht{\s} zu sehen al{\s} auf den H"ugeln
gro"se Feuer, in denen man landwirtschaftliche "Uberbleibsel
verbrannte.

"`Da{\s} Stehen wird dich zu sehr anstrengen, Beste!"' warnte Karl
und geleitete sie behutsam zur Laube hin. "`Setz dich hier ein
wenig auf die Bank! Da{\s} wird dir gut tun!"'

"`Nein, nein! Nicht hier! Hier nicht!"' stie"s sie mit
ersterbender Stimme hervor.

Sie wurde ohnm"achtig, und abend{\s} war die Krankheit von neuem
da, und zwar in erh"ohtem Grade und mit allerlei Komplikationen.
Bald hatte sie in der Herzgegend, bald in der Brust, bald im
Kopfe, bald in den Gliedern Schmerzen. Dazu gesellte sich ein
Au{\s}wurf, an dem Bovary die ersten Anzeichen der
Lungenschwindsucht zu erkennen w"ahnte.

Zu alledem hatte der arme Schelm auch noch Geldsorgen.


\newpage\begin{center}
{\large \so{Vierzehnte{\s} Kapitel}}\bigskip\bigskip
\end{center}

Zun"achst wu"ste er nicht, wie er dem Apotheker die vielen
Arzneien ver\-g"uten sollte, die er von ihm bezogen hatte. Al{\s}
Arzt brauchte er sie nicht zu bezahlen, aber da{\s} w"are ihm
peinlich gewesen. Dann war der Hau{\s}halt, jetzt wo ihn da{\s}
M"adchen f"uhrte, schrecklich teuer geworden. Die Rechnungen
regneten nur so in{\s} Hau{\s}. Die Lieferanten begannen
ungeduldig zu werden. In{\s}besondre mahnte Lheureux in l"astiger
Weise. Er hatte den H"ohepunkt von Emma{\s} Krankheit dazu
benutzt, ihre Rechnung h"oher au{\s}zuschreiben, al{\s} sie
wirklich war. Flug{\s} brachte er auch den Mantel, die Handtasche
und zwei Koffer statt de{\s} einen und noch eine Menge andrer
Gegenst"ande, die bestellt worden seien, wie er behauptete. E{\s}
n"utzte Bovary gar nicht{\s}, da"s er erkl"arte, er brauche die
Sachen nicht; der H"andler erwiderte ihm in ungezogenem Tone, alle
diese Waren seien bei ihm bestellt und er n"ahme sie nicht
zur"uck. Herr Bovary m"oge sich{\s} "uberlegen; er werde ihn eher
verklagen al{\s} sich selber benachteiligen. Karl befahl daraufhin
dem M"adchen, die Gegenst"ande im Gesch"aft abzugeben, aber
Felicie verga"s e{\s}. Er selbst hatte sich um andre Dinge zu
k"ummern und dachte nicht mehr daran. Nach einer gewissen Zeit
unternahm Lheureux einen neuen Versuch. Bald drohend, bald
jammernd, brachte er e{\s} so weit, da"s ihm Bovary schlie"slich
einen Wechsel au{\s}stellte, der in sech{\s} Monaten f"allig war.
Al{\s} er da{\s} Papier unterschrieb, kam ihm der k"uhne Gedanke,
tausend Franken von Lheureux zu leihen. Verlegen fragte er, ob er
ihm diese Summe auf ein Jahr zu beliebigem Zin{\s}fu"s verschaffen
k"onne. Der Handel{\s}mann eilte sofort in seinen Laden, brachte
da{\s} Geld und zugleich einen zweiten Wechsel, durch den sich
Bovary verpflichtete, am 1. September kommenden Jahre{\s}
eintausendundsiebzig Franken zu zahlen. Mit den bereit{\s}
anerkannten hundertundachtzig Franken ergab da{\s} eine
Gesamtschuld von zw"olfhundertundf"unfzig Franken. Lheureux machte
hierbei ein ganz h"ubsche{\s} Gesch"aft; im "ubrigen wu"ste er im
vorau{\s} genau, da"s e{\s} hierbei nicht bliebe. Er rechnete
darauf, da"s der Arzt die Wechsel am F"alligkeit{\s}tage nicht
einl"osen k"onne und sie prolongieren m"usse. Auf diese Weise
sollte da{\s} erst armselige S"ummchen im Hause de{\s} Arzte{\s}
wie in einem Sanatorium eine ordentliche Mastkur durchmachen und
eine{\s} Tage{\s} dick und rund zu ihm zur"uckkehren.

Lheureux hatte allenthalben Erfolge. Er erlangte die
regelm"a"sigen Apfelweinlieferungen f"ur da{\s} Neufch\^ateler
Krankenhau{\s}. Der Notar Guillaumin schanzte ihm Aktien der
Torfgruben zu Gr"ume{\s}nil zu. Dazu trug er sich mit dem Plane,
zwischen Argueil und Rouen eine neue Postverbindung zu er"offnen,
die den alten Rumpelkasten de{\s} Goldnen L"owen unbedingt au"ser
Konkurrenz stellen sollte, indem sie schneller f"uhre, billiger
w"are und Eilgut bestelle. Damit wollte er den ganzen Handel von
Yonville in seine H"ande bringen.

Karl gr"ubelte oftmal{\s} dar"uber nach, wie er die betr"achtliche
Wechselschuld in einem Jahre wohl tilgen k"onne. Er kam dabei auf
allerhand M"oglichkeiten. Sollte er sich an seinen Vater wenden
oder irgend etwa{\s} verkaufen? Aber erstere{\s} hatte vermutlich
keinen Erfolg, und zu verkaufen gab e{\s} nicht{\s}. Er mochte
sich sonst noch au{\s}denken, wa{\s} er wollte: "uberall drohten
die gr"o"sten Schwierigkeiten. Und so schenkte er sich nur allzu
gern weitere unerfreuliche "Uberlegungen. Er redete sich ein, er
vernachl"assige seine Frau, wenn er ihr nicht all sein Dichten und
Trachten widme. Er wollte an nicht{\s} andre{\s} denken, selbst
wenn ihr dadurch kein Abbruch gesch"ahe.

Der Winter war streng. Emma{\s} Genesung schritt nur langsam
vorw"art{\s}. Al{\s} da{\s} Wetter w"armer wurde, schob man sie in
ihrem Lehnstuhl an da{\s} Fenster, und zwar an da{\s} nach dem
Marktplatze zu gelegene. Da{\s} andre mit dem Blick in den Garten
war ihr jetzt verleidet; de{\s}halb mu"ste seine Jalousie
best"andig heruntergelassen bleiben. Sie bestimmte, da"s ihr
Reitpferd verkauft werden solle. Alle{\s}, wa{\s} ihr fr"uher lieb
gewesen, war ihr nunmehr zuwider. Sie k"ummerte sich um nicht{\s}
mehr al{\s} um ihre eigene Person. Die kleinen Mahlzeiten nahm sie
in ihrem Bett ein. Manchmal klingelte sie dem M"adchen, um sich
die Arznei reichen zu lassen oder um mit ihm zu plaudern. Der
Schnee auf dem Dache der Hallen warf seinen hellen, immer gleichen
Widerschein in da{\s} Zimmer. Dann kamen Regentage. Sie empfand
eine Art Angst vor den sich alle Tage wiederholenden
unau{\s}bleiblichen kleinen und kleinsten Ereignissen, die sie
eigentlich gar nicht{\s} angingen, am meisten vor der
allabendlichen Ankunft der Post im Goldnen L"owen. Dann redete die
Wirtin laut, allerlei andre Stimmen l"armten dazwischen, und die
Laterne Hippolyt{\s}, der unter den Koffern auf dem Wagenverdeck
herumsuchte, leuchtete wie ein Stern durch die Dunkelheit. Um die
Mittag{\s}zeit kam Karl nach Hause, dann ging er wieder. Sie trank
ihre Bouillon. Um f"unf Uhr, wenn e{\s} zu d"ammern begann, kamen
die Kinder au{\s} der Schule; sie klapperten mit ihren Holzschuhen
"uber da{\s} Trottoir, und im Vor"ubergehen schlug ein{\s} wie
da{\s} andere mit dem Lineal gegen die eisernen Riegel der
Fensterl"aden.

Um diese Zeit pflegte sich der Pfarrer einzustellen. Er erkundigte
sich nach ihrem Befinden, erz"ahlte ihr Neuigkeiten und ermahnte
sie zur Fr"ommigkeit in gef"alligem Plaudertone. Schon der Anblick
der Soutane hatte f"ur Emma etwa{\s} Beruhigende{\s}.

Eine{\s} Tage{\s}, al{\s} ihre Krankheit am schlimmsten war, hatte
sie nach dem Abendmahl verlangt, im Glauben, ihr letzte{\s}
St"undlein sei gekommen. W"ahrend man im Gemach die n"otigen
Vorbereitungen zu dieser Zeremonie traf, die mit Arzneiflaschen
bedeckte Kommode in einen Altar wandelte und den Fu"sboden mit
Blumen bestreute, da war e{\s} ihr, al{\s} "uberk"ame sie eine
geheimni{\s}volle Kraft, die ihr ihre Schmerzen, alle Empfindungen
und Wahrnehmungen nahm. Sie war wie k"orperlo{\s} geworden, sie
hegte keine Gedanken mehr, und ein neue{\s} Leben begann ihr. Sie
hatte da{\s} Gef"uhl, al{\s} schwebe ihre Seele gen Himmel, al{\s}
verl"osche sie in der Sehnsucht nach dem ewigen Frieden wie eine
Opferflamme "uber verglimmendem R"aucherwerk. Man besprengte ihr
Bett mit Weihwasser. Der Priester nahm die wei"se Hostie au{\s}
dem heiligen Ciborium. Halb ohnm"achtig vor "uberirdischer Lust,
"offnete Emma die Lippen, um den Leib de{\s} Heiland{\s} zu
empfangen, der sich ihr bot. Die Bettvorh"ange um sie herum
bauschten sich weich wie Wolken, und die beiden brennenden Kerzen
auf der Kommode leuchteten ihr mit ihrem Strahlenkranze wie
Gloriolen her"uber. Al{\s} sie mit dem Kopfe in da{\s} Kissen
zur"ucksank, glaubte sie au{\s} himmlischen H"ohen seraphische
Harfenkl"ange zu h"oren und im Azur auf goldnem Throne, umringt
von Heiligen mit gr"unen Palmen, Gott den Vater in aller seiner
erhabenen Herrlichkeit zu schaun. Er winkte, und Engel mit
Flammenfl"ugeln wallten zur Erde hernieder, um sie emporzutragen~...

Diese wundervolle Vision bewahrte Emma in ihrem Ged"achtnisse.
E{\s} war der allersch"onste Traum, den sie je getr"aumt. Sie gab
sich M"uhe, da{\s} Bild immer wieder zu empfinden. E{\s} wich ihr
nicht au{\s} der Phantasie, aber e{\s} erschien ihr nur manchmal
und in s"u"ser Verkl"arung. Ihr einst so stolzer Sinn beugte sich
in christlicher Demut. Da{\s} Gef"uhl der menschlichen Ohnmacht
ward ihr ein k"ostlicher Genu"s. Sie sah f"ormlich, wie au{\s}
ihrem Herzen der eigene Wille wich und der hereindringenden
g"ottlichen Gnade T"ur und Tor weit "offnete. E{\s} gab also
au"ser dem Erdengl"uck eine h"ohere Gl"uckseligkeit und "uber
aller Liebe hienieden eine andre erhabenere, ohne Schwankungen und
ohne Ende, eine Br"ucke in da{\s} Ewige! In neuen Illusionen
ertr"aumte sie sich "uber der Erde ein Reich der Reinheit, einen
Vorhimmel. Dort zu weilen, ward ihre Sehnsucht. Sie wollte eine
Heilige werden. Sie kaufte sich Rosenkr"anze und trug Amulette.
Ihr gr"o"ster Wunsch war, in ihrem Zimmer, zu H"aupten ihre{\s}
Bette{\s}, einen Reliquienschrein mit Smaragden zu besitzen. Den
wollte sie dann alle Abende k"ussen.

Der Pfarrer wunderte sich "uber Emma{\s} Wandlung, verhehlte sich
jedoch nicht, da"s diese allzu inbr"unstige Fr"ommigkeit sehr
leicht in "Uberschwenglichkeit und Ketzerei au{\s}arten k"onne.
Aber er war kein Seelenkenner, zumal au"sergew"ohnlichen
Erscheinungen gegen"uber. De{\s}halb wandte er sich an den
Buchh"andler de{\s} Erzbischof{\s} und bat ihn, ihm "`ein
passende{\s} Erbauung{\s}buch f"ur eine gebildete
Frauen{\s}person"' zu schicken. Mit der gr"o"sten
Gleichg"ultigkeit, al{\s} handle e{\s} sich darum, irgendwelchen
Krim{\s}kram an einen Kamerunneger zu versenden, packte der
Buchh"andler alle m"oglichen gerade vorr"atigen frommen Schriften
in ein Paket: Katechi{\s}men in Form von Frage und Antwort,
Streitschriften aufgeblasener Dogmatiker und fr"ommelnde Romane in
rosa Einb"andchen und s"u"slichem Stil, verbrochen von dichtenden
Schulmeistern oder blaustr"umpfigen Betschwestern, mit Titeln wie:
"`Die Herzpostille"', "`Der Weltmann zu F"u"sen Mari"a. Von Herrn
von ***, Ritter mehrerer Orden"', "`Voltaire{\s} Ketzereien zum
Gebrauch f"ur die Jugend"', usw. usw.

Emma war seelisch noch viel zu schwach, um sich mit geistigen
Dingen ernstlich befassen zu k"onnen. "Uberdie{\s} st"urzte sie
sich auf diese B"ucher mit allzu gro"sem Bed"urfni{\s} nach
wirklicher Erbauung. Die Starrheit der kirchlichen Lehren emp"orte
sie, die Anma"sungen der Polemik stie"sen sie ab, und die
Intoleranz, mit der ihr unbekannte Menschen verfolgt wurden,
mi"sfiel ihr. Die Romane, in denen profane Dinge durch religi"ose
Ideen aufgeputzt waren, entbehrten ihr zu sehr auch nur der
geringsten Weltkenntni{\s}. Sie verschleierten die Realit"aten
de{\s} Leben{\s}, f"ur deren Brutalit"at sie viel lieber
literarische Beweise gefunden h"atte. Trotzdem la{\s} sie weiter,
und wenn ihr ein{\s} der B"ucher au{\s} den H"anden glitt, dann
w"ahnte sie den zartesten Weltschmerz der katholischen Mystik zu
empfinden, wie ihn nur die "ubersinnlichsten Seelen zu versp"uren
imstande sind.

Da{\s} Andenken an Rudolf hatte sie in die Tiefen ihre{\s}
Herzen{\s} begraben; darin ruhte e{\s} unber"uhrter und stiller
denn eine "agyptische K"onig{\s}mumie in ihrer Kammer. Au{\s}
dieser gro"sen eingesargten Liebe drang ein leiser, alle{\s}
durchstr"omender Duft von Z"artlichkeit in da{\s} neue reine
Dasein, da{\s} Emma f"uhren wollte. Wenn sie in ihrem gotischen
Betstuhl kniete, richtete sie an ihren Gott genau die verliebten
Worte, die sie einst ihrem Geliebten zugefl"ustert hatte in den
Ekstasen de{\s} Ehebruch{\s}. Damit wollte sie der g"ottlichen
Gnade teilhaftig werden. Aber vom Himmel her kam ihr keine
Tr"ostung, und sie erhob sich mit m"uden Gliedern und dem leeren
Gef"uhl, namenlo{\s} betrogen worden zu sein. Diese{\s} Suchen,
dachte sie bei sich, sei wiederum ein Verdienst, und im Hochmut
ihrer Selbsterniedrigung verglich sich Emma mit den gro"sen Damen
der Vergangenheit, deren Ruhm ihr damal{\s}, al{\s} sie "uber den
Szenen au{\s} dem Leben de{\s} Fr"aulein{\s} von Lavalli\`ere
tr"aumte, aufgegangen war, jenen Damen in ihren mit k"oniglicher
Anmut getragenen langen kostbaren Schleppkleidern, die in einsamen
Stunden zu F"u"sen Christi ihre vom Leben verwundeten Herzen
au{\s}geweint hatten.

Nun wurde sie "uber die Ma"sen mildt"atig. Sie n"ahte Kleider f"ur
die Armen, schickte W"ochnerinnen Brennholz, und al{\s} Karl
eine{\s} Tage{\s} heimkam, fand er in der K"uche drei
Gassenjungen, die Suppe a"sen. Die kleine Berta wurde wieder
in{\s} Hau{\s} genommen; Karl hatte sie w"ahrend der Krankheit
seiner Frau von neuem zu der Amme gegeben. Nun wollte ihr Emma
da{\s} Lesen beibringen. Wenn da{\s} Kind weinte, regte sie sich
nicht mehr auf. E{\s} war eine Art Resignation "uber sie gekommen,
eine duldsame Nachsicht gegen alle{\s}. Ihre Sprache ward voll
gew"ahlter Au{\s}dr"ucke, selbst Allt"aglichkeiten gegen"uber.

Die alte Frau Bovary hatte nicht{\s} mehr an Emma au{\s}zusetzen,
abgesehen von ihrer Manie, f"ur Waisenkinder Jacken zu stricken
und ihre eigenen Wischt"ucher unau{\s}gebessert zu lassen. Aber
die gute Frau war der Zwiste in ihre{\s} Manne{\s} Hause derma"sen
m"ude, da"s ihr der Frieden am Herde ihre{\s} Sohne{\s} so
wohltat, da"s sie bi{\s} nach Ostern dablieb, um den
B"arbei"sigkeiten de{\s} alten Bovary zu entgehen, der alle
Freitage, an den Fastentagen, unbedingt eine Bratwurst auf dem
Tische sehen wollte.

Au"ser der Gesellschaft ihrer Schwiegermutter, die ihr durch ihre
Rechtlichkeit und ihr w"urdige{\s} Wesen einen gewissen Halt gab,
hatte Emma jetzt fast alle Tage Besuch bei sich. E{\s} verkehrten
mit ihr: Frau Langloi{\s}, Frau Caron, Frau D"ubreuil, Frau
T"uvache, sowie die treffliche Frau Homai{\s}, die sich
regelm"a"sig zwischen drei und f"unf Uhr einstellte. Sie hatte dem
Klatsch, der "uber ihre Nachbarin im Umlauf gewesen war,
niemal{\s} Glauben schenken wollen. Auch die Apotheker{\s}kinder
kamen mitunter in Justin{\s} Begleitung. Er brachte sie in
Emma{\s} Zimmer und blieb in der N"ahe der T"ure stehen, ohne sich
zu r"uhren und ohne ein Wort zu sagen. Oft gewahrte ihn Frau
Bovary gar nicht und lie"s sich in ihrem Toilettemachen nicht
st"oren. Sie k"ammte sich da{\s} Haar, wobei sie den Kopf nach dem
Durchziehen de{\s} Kamme{\s} jede{\s}mal mit einer eigent"umlichen
heftigen Bewegung zur"uckwarf. Al{\s} der arme Junge zum ersten
Male diese volle Haarflut sah, die in langen schwarzen Ringeln
bi{\s} zu den Knien herabwallte, war e{\s} ihm zumute, al{\s}
schaue er pl"otzlich ganz Neue{\s}, Au"sergew"ohnliche{\s}, und er
starrte wie geblendet hin.

Sicherlich bemerke Emma weder sein stumme{\s} Ent\/z"ucken noch
seine sch"uchterne Verehrung. Sie hatte keine Ahnung, da"s die
au{\s} ihrem Leben entschwundene Liebe dort, ihr ganz nahe, in
neuer Gestalt wieder auftauchte, unter einem groben Leinwandhemd,
in einem jungen Herzen, da{\s} sich der Offenbarung ihrer
Frauensch"onheit weit "offnete. Im "ubrigen war sie jetzt in jeder
Hinsicht grenzenlo{\s} gleichg"ultig. Mit dem stolzesten Gesichte
sagte sie die z"artlichsten Worte. Ihr ganze{\s} Benehmen war so
widerspruch{\s}voll, da"s man Selbstsucht nicht mehr von Mitleid
an ihr unterscheiden konnte. Man wu"ste nicht mehr, war sie
verdorben oder unnahbar.

Zum Beispiel war sie eine{\s} Abend{\s} sehr ungehalten "uber ihr
Dienst\-m"ad\-chen. E{\s} bat, au{\s}gehen zu d"urfen, und
stotterte irgendeinen Vorwand her. Un\-vermittelt fragte Emma:

"`Du liebst ihn also?"' und, ohne Felicie{\s} Antwort abzuwarten,
f"ugte sie in traurigem Tone hinzu: "`Geh! Lauf! Vergn"uge dich!"'

In den ersten Fr"uhling{\s}tagen lie"s sie den Garten vollst"andig
um"andern. Karl war anfang{\s} dagegen, dann jedoch freute er sich
dar"uber, da"s sie endlich wieder einmal einen bestimmten Wunsch
"au"serte. Nach und nach bewie{\s} sie auch anderweitig, da"s sie
sich wieder erholt hatte. Zun"achst brachte sie e{\s} zuwege, da"s
Frau Rollet, die Amme, die sich{\s} angew"ohnt hatte, Tag f"ur Tag
mit ihren S"auglingen und Ziehkindern und einem kannibalischen
Appetit in der K"uche zu erscheinen, von dannen gejagt wurde.
Sodann sch"uttelte sie sich die Familie Homai{\s} vom Halse, nach
und nach auch die andern regelm"a"sigen Besucherinnen. Sogar in
die Kirche ging sie seltener, zur gro"sen Freude de{\s}
Apotheker{\s}, der ihr daraufhin freundschaftlichst erkl"arte:

"`Ich dachte schon, Sie seien eine Betschwester geworden!"'

Bournisien kam nach wie vor alle Tage nach der
Katechi{\s}mu{\s}stunde. Am liebsten blieb er im Freien, im
"`Hain"', wie er die Laube scherzhaft zu nennen pflegte. Um
dieselbe Zeit kehrte auch Karl meist heim. Beiden war warm, und so
bekamen die beiden M"anner eine Flasche Apfelsekt vorgesetzt, den
sie "`auf die v"ollige Genesung der gn"adigen Frau"' tranken.

"Ofter{\s} fand sich auch Binet ein, da{\s} hei"st: er sa"s
etwa{\s} tiefer, vor dem Garten, am Bache, um zu krebsen. Bovary
lud ihn zu einer kleinen Erfrischung ein. Binet war ein Meister im
Aufbrechen von Sektflaschen.

"`Zun"achst mu"s man die Bulle senkrecht auf den Tisch stellen,"'
dozierte er, indem er selbstbewu"st um sich blickte, "`dann
zerschneidet man die Bindf"aden, und dann l"a"st man dem Pfropfen
ganz, ganz sachte, nach und nach Luft. Sooo!"'

Aber bei dieser Vorf"uhrung spritzte der Sekt "ofter{\s} der
ganzen Gesellschaft in die Gesichter, und der Priester unterlie"s
e{\s} niemal{\s}, behaglich schmunzelnd den Witz zu machen:

"`Seine Vortrefflichkeit springt einem buchst"ablich in die Augen!"'

Er war wirklich ein guter Mensch. Er hatte nicht einmal etwa{\s}
dagegen, al{\s} der Apotheker dem Arzte empfahl, er solle mit
seiner Frau zu ihrer Zerstreuung nach Rouen fahren und sich dort
im Theater den ber"uhmten Tenor Lagardy anh"oren. Homai{\s}
wunderte sich "uber diese Duldsamkeit und f"uhlte ihm de{\s}halb
etwa{\s} auf den Zahn. Der Priester erkl"arte, er halte die Musik
f"ur weniger sittenverderbend al{\s} die Literatur. Aber Homai{\s}
verteidigte die letztere. Er behauptete, da{\s} Theater k"ampfe
unter dem leichten Gewande de{\s} Spiel{\s} gegen veraltete Ideen
und f"ur die wahre Moral.

"`\begin{antiqua}Castigat ridendo mores\end{antiqua}, verehrter
Herr Pfarrer!"' zitierte er. "`Sehen Sie sich daraufhin mal die
Trag"odien Voltaire{\s} an! Die meisten von ihnen sind mit
philosophischen Aphori{\s}men durchsetzt, die eine wahre Schule
der Moral und Leben{\s}klugheit f"ur da{\s} Volk sind."'

"`Ich habe einmal ein St"uck gesehen,"' sagte Binet, "`e{\s}
hie"s: {\glq}Der Pariser Taugenicht{\s}.{\grq} Darin kommt ein
alter General vor, wirklich ein hahneb"uchner Kerl. Er verst"o"st
seinen Sohn, der eine Arbeiterin verf"uhrt hat; zu guter Letzt
aber~..."'

"`Gewi"s"', unterbrach ihn Homai{\s}, "`gibt e{\s} schlechte
Literatur, genau so wie e{\s} schlechte Arzneien gibt. Aber die
wichtigste aller K"unste de{\s}halb gleich in Bausch und Bogen zu
verurteilen, da{\s} d"unkt mich eine kolossale Dummheit, eine
grote{\s}ke Idee, w"urdig der abscheulichen Zeiten, die einen
Galilei im Kerker schmachten lie"sen."'

Der Pfarrer ergriff da{\s} Wort:

"`Ich wei"s sehr wohl: e{\s} gibt gute Dramen und gute
Theaterschriftsteller. Aber diese modernen St"ucke, in denen
Personen zweierlei Geschlecht{\s} in Prunkgem"achern,
vollgepfropft von weltlichem Tand, zusammengesteckt werden, diese
schamlosen B"uhnenm"atzchen, dieser Kost"umluxu{\s}, diese
Lichtvergeudung, dieser Femini{\s}mu{\s}, alle{\s} da{\s} hat
keine andre Wirkung, al{\s} da"s e{\s} leichtfertige Ideen in die
Welt setzt, sch"andliche Gedanken und unz"uchtige Anwandlungen.
Wenigsten{\s} ist da{\s} zu allen Zeiten die Ansicht der
kirchlichen Autorit"aten."'

Er nahm einen salbung{\s}vollen Ton an, w"ahrend er zwischen
seinen Fingern eine Prise Tabak hin und her rieb. "`Und wenn die
Kirche da{\s} Theater zuweilen in Acht und Bann getan hat, war sie
in ihrem vollen Rechte. Wir m"ussen un{\s} ihrem Gebote f"ugen."'

"`Jawohl,"' eiferte der Apotheker, "`man exkommuniziert die
Schauspieler. In fr"uheren Jahrhunderten nahmen sie an den
kirchlichen Feiern teil. Man spielte sogar in der Kirche
possenhafte St"ucke, die sogenannten Mysterien, in denen e{\s}
h"aufig nicht{\s} weniger al{\s} dezent zuging~..."'

Der Geistliche begn"ugte sich, einen Seufzer au{\s}zusto"sen. Der
Apotheker redete immer weiter:

"`Und wie steht{\s} mit der Bibel? E{\s} wimmelt darin -- Sie
wissen{\s} ja am besten -- von Unanst"andigkeiten und -- man kann
nicht ander{\s} sagen -- groben Schweinereien~..."' Bournisien
machte eine unwillige Geb"arde. "`Aber Sie m"ussen mir doch
zugeben, da"s da{\s} kein Buch ist, da{\s} man jungen Leuten in
die Hand geben kann. Ich werde e{\s} nie zulassen, da"s meine
Athalie~..."'

"`Da{\s} sind ja die Protestanten, nicht wir,"' rief der Pfarrer
ungeduldig, "`die den Leuten die Bibel "uberlassen!"'

"`Da{\s} kommt hier nicht in Frage"', erkl"arte Homai{\s}. "`Ich
wundre mich nur, da"s man noch in unsrer Zeit, im Jahrhundert der
wissenschaftlichen Aufkl"arung, eine geistige Erholung zu
verdammen sucht, die in gesellschaftlicher, in moralischer, ja
sogar in hygienischer Beziehung die Menschheit f"ordert! Da{\s}
ist doch so, nicht, Doktor?"'

"`Zweifello{\s}!"' erwiderte der Arzt nachl"assig. Entweder wollte
er niemandem zu nahetreten, obgleich er dieselbe Ansicht hegte,
oder er hatte hier"uber "uberhaupt keine Meinung.

Die Unterhaltung war eigentlich zu Ende, aber der Apotheker hielt
e{\s} f"ur angebracht, eine letzte Attacke zu reiten.

"`Ich habe Geistliche gekannt,"' behauptete er, "`die in Zivil
in{\s} Theater gingen, um die Balletteusen mit den Beinen
strampeln zu sehen."'

"`Ach wa{\s}!"' wehrte der Pfarrer ab.

"`Doch! Ich kenne welche!"' Und nochmal{\s} sagte er, Silbe f"ur
Silbe einzeln betonend: "`Ich -- ken -- ne -- wel -- che!"'

"`Na ja,"' meinte Bournisien nachgiebig, "`die Betreffenden haben
da aber etwa{\s} Unrechte{\s} getan."'

"`Wa{\s} Unrechte{\s}? Der Teufel soll mich holen! Sie taten noch
ganz andre Dinge!"'

"`Herr -- Apo -- the -- ker!"' rief der Geistliche mit einem so
zornigen Blicke, da"s Homai{\s} eingesch"uchtert wurde und
einlenkte:

"`Ich wollte damit ja nur sagen, da"s die Toleranz die beste
F"ursprecherin der Kirche ist."'

"`Sehr wahr! Sehr wahr!"' gab der gutm"utige Pfarrer zu, indem er
sich wieder in seinen Stuhl zur"ucklehnte. Er blieb aber nur noch
ein paar Minuten.

Al{\s} er fort war, sagte Homai{\s} zu Bovary:

"`Da{\s} war eine ordentliche Abfuhr! Dem hab ich{\s} mal
gesteckt! Sie haben{\s} ja mit angeh"ort! Um darauf
zur"uckzukommen: tun Sie da{\s} ja, f"uhren Sie Ihre Frau in
da{\s} Theater, und wenn{\s} blo"s de{\s}halb w"are, um diesen
schwarzen Raben damit zu "argern. Sapperlot! Wenn ich einen
Vertreter h"atte, begleitete ich Sie selber! Aber halten Sie sich
dazu! Lagardy singt nur einen einzigen Abend. Er hat ein
Engagement nach England f"ur ein Riesenhonorar! "Ubrigen{\s} soll
er ein toller Schweren"oter sein! Er schwimmt im Gold! Drei
Geliebte bringt er mit und seinen Leibkoch! Alle diese gro"sen
K"unstler k"onnen nicht rechnen. Sie brauchen ein
verschwenderische{\s} Dasein, e{\s} regt ihre Phantasie an.
Freilich enden sie im Spittel, weil sie in jungen Jahren nicht zu
sparen verstehen ... Na, gesegnete Mahlzeit! Auf Wiedersehn!"'

Der Gedanke, da{\s} Theater zu besuchen, schlug in Bovary{\s}
Kopfe schnell Wurzel. Er redete Emma in einem fort zu. Anfang{\s}
wollte sie nicht{\s} davon wissen und meinte, sie f"uhle sich zu
schwach, e{\s} sei zu beschwerlich und zu kostspielig.
Au{\s}nahm{\s}weise gab Karl nicht nach, zumal er sich einbildete,
da"s ihr diese Zerstreuung sehr dienlich w"are. Irgendwelche
Schwierigkeit lag nicht vor. Seine Mutter hatte ihm j"ungst ganz
unvermutet dreihundert Franken geschickt. Die laufenden
Au{\s}gaben waren nicht gro"s, und die Wechselschuld bei Lheureux
war noch lange nicht f"allig, so da"s er daran nicht zu denken
brauchte. Er dachte, Emma str"aube sich nur au{\s} R"ucksicht auf
ihn. De{\s}halb best"urmte er sie immer mehr, bi{\s} sie seinen
Bitten schlie"slich nachgab. Am andern Morgen um acht Uhr fuhren
sie mit der Post ab.

Den Apotheker hielt nicht{\s} Dringliche{\s} in Yonville zur"uck,
aber er hielt sich f"ur unabk"ommlich. Al{\s} er die beiden
einsteigen sah, jammerte er.

"`Gl"uckliche Reise!"' sagte er. "`Habt ihr{\s} gut!"' Und zu Emma
gewandt, f"ugte er hinzu: "`Sie sehen zum Anbei"sen h"ubsch
au{\s}! Sie werden in Rouen Furore machen!"'

Die Post spannte in Rouen im "`Roten Kreuz"' am Beauvoisine-Platz
au{\s}. Da{\s} war ein regelrechter Vorstadtgasthof mit
ger"aumigen St"allen und winzigen Fremdenzimmern. Mitten im Hofe
lief eine Schar H"uhner herum, die unter den verschmutzten
Einsp"annern der Gesch"aft{\s}reisenden ihre Haferk"orner
aufpickten. E{\s} war eine der Herbergen au{\s} der guten alten
Zeit. Sie haben morsche Holzbalkone, die in den Wintern"achten im
Winde knarren; die G"aste, der L"arm und die Esserei werden in
ihnen nie alle; die schwarzen Tischplatten sind voller gro"ser
Kaffeeflecke, die tr"uben dicken Fensterscheiben voller
Fliegenschmutz und die feuchten Servietten voller Rotweinspuren.
Auf der Stra"senseite gibt e{\s} ein Caf\'e und hinten nach dem
Freien zu einen Gem"usegarten. Alle{\s} tr"agt einen l"andlichen
Anstrich.

Karl machte sofort einen Besorgung{\s}gang. An der Theaterkasse
wu"ste er nicht, wa{\s} Parkett, Pros\/zenium{\s}loge, erster Rang
und Galerie war; er bat um Au{\s}kunft, wurde dadurch aber auch
nicht kl"uger. Der Kassierer wie{\s} ihn in die Direktion.
Schlie"slich rannte er noch einmal in den Gasthof zur"uck, dann
wieder an die Kasse. Auf diese Weise lief er mehrmal{\s} durch die
halbe Stadt.

Frau Bovary kaufte sich einen neuen Hut, Handschuhe und Blumen.
Karl war fortw"ahrend in Angst, den Beginn der Oper zu vers"aumen.
Und so nahmen sie sich beide keine Zeit, einen Bissen zu sich zu
nehmen. Al{\s} sie aber vor dem Theater ankamen, waren die T"uren
noch geschlossen.


\newpage\begin{center}
{\large \so{F"unfzehnte{\s} Kapitel}}\bigskip\bigskip
\end{center}

Eine Menge Menschen umlagerte die Eing"ange. "Uberall an den Ecken
der in der N"ahe gelegenen Stra"sen prangten riesige Plakate, die
in auff"alligen Lettern au{\s}schrien:

\begin{center}
\begin{antiqua}
LUCIA VON LAMMERMOOR ... OPER ... \\
DONIZETTI ... GASTSPIEL ... LAGARDY ...
\end{antiqua}
\end{center}

E{\s} war ein sch"oner, aber hei"ser Tag. Der Schwei"s rann den
Leuten "uber die Stirn, und sie f"achelten ihren erhitzten
Gesichtern mit den Taschent"uchern K"uhlung zu. Hin und wieder
wehte lauer Wind vom Strome her und bl"ahte ein wenig die
Leinwandmarkisen der Restaurant{\s}. Weiter unten, an den Kai{\s},
wurde man durch einen eisigen Luft\/zug abgek"uhlt, in den sich
Ger"uche von Talg, Leder und "Ol au{\s} den zahlreichen dunklen,
vom Rollen der gro"sen F"asser l"armigen Gew"olben der
Karren-Gasse mischten.

Au{\s} Furcht, sich l"acherlich zu machen, schlug Frau Bovary vor,
noch nicht in da{\s} Theater hineinzugehen und erst einen
Spaziergang durch die Hafenpromenaden zu machen. Dabei hielt Karl
die Eintritt{\s}karten, die er in der Hosentasche trug, vorsichtig
mit seinen Fingern fest und dr"uckte sie gegen die Bauchwand, so
da"s er sie in einem fort f"uhlte.

In der Vorhalle bekam Emma Herzklopfen. Al{\s} sie wahrnahm, da"s
sich der Menschenschwall die Nebentreppen nach den Galerien
hinaufschob, w"ahrend sie selbst die breite Treppe zum ersten
Range emporschreiten durfte, l"achelte sie unwillk"urlich vor
Eitelkeit. E{\s} gew"ahrte ihr ein kindliche{\s} Vergn"ugen, die
breiten vergoldeten T"uren mit der Hand aufzusto"sen. In vollen
Z"ugen atmete sie den Staubgeruch der G"ange ein, und al{\s} sie
in ihrer Loge sa"s, machte sie sich{\s} mit einer Ungezwungenheit
einer Principessa bequem.

Da{\s} Hau{\s} f"ullte sich allm"ahlich. Die Operngl"aser kamen
au{\s} ihren Futteralen. Die Stammsitzinhaber nickten sich au{\s}
der Entfernung zu. Sie wollten sich hier im Reiche der Kunst von
der Unrast ihre{\s} Kr"amerleben{\s} erholen, doch sie verga"sen
die Gesch"afte nicht, sondern redeten noch immer von Baumwolle,
Fusel und Indigo. Da{\s} waren Grauk"opfe mit friedfertigen
Alltag{\s}gesichtern; wei"s in der Farbe von Haar und Haut,
glichen sie einander wie abgegriffene Silberm"unzen. Im Parkett
paradierten die jungen Modenarren mit knallroten und
gra{\s}gr"unen Krawatten. Frau Bovary bewunderte sie von oben, wie
sie sich mit gelbbehandschuhten H"anden auf die goldenen Kn"aufe
ihrer St"ocke st"utzten. Jetzt wurden die Orchesterlampen
angez"undet, und der Kronleuchter ward von der Decke
herabgelassen. Sein in den Gla{\s}pri{\s}men widerglitzernde{\s}
Lichtmeer brachte frohe Stimmung in die Menschen. Dann erschienen
die Musiker, einer nach dem andern, und nun hub ein wirre{\s}
Get"ose an von brummenden Kontrab"assen, kratzenden Violinen,
fauchenden Klarinetten und winselnden Fl"oten. Endlich drei kurze
Schl"age mit dem Taktstocke de{\s} Kapellmeister{\s}.
Paukenwirbel, H"ornerklang. Der Vorhang hob sich.

Auf der B"uhne ward eine Landschaft sichtbar: ein Kreuzweg im
Walde, zur Linken eine Quelle, von einer Eiche beschattet. Bauern,
M"antel um die Schultern, sangen im Chor ein Lied. Dann tritt ein
Edelmann auf, der die Geister der H"olle mit gen Himmel gereckten
Armen um Rache anfleht. Noch einer erscheint. Beide gehen zusammen
ab. Der Chor singt von neuem.

Emma sah sich in die Atmosph"are ihrer M"adchenlekt"ure
zur"uckversetzt, in die Welt Walter Scott{\s}. E{\s} war ihr,
al{\s} h"ore sie den Klang schottischer Dudels"acke "uber die
nebelige Heide hallen. Die Erinnerung an den Roman de{\s} Briten
erleichterte ihr da{\s} Verst"andni{\s} der Oper. Aufmerksam
folgte sie der intriganten Handlung, w"ahrend eine Flut von
Gedanken in ihr aufwallte, um al{\s}bald unter den Wogen der Musik
wieder zu verflie"sen. Sie gab sich diesen schmeichelnden Melodien
hin. Sie f"uhlte, wie ihr die Seele in der Brust mit in
Schwingungen geriet, al{\s} strichen die Violinenbogen "uber ihre
Nerven. Sie h"atte hundert Augen haben m"ogen, um sich satt sehen
zu k"onnen an den Dekorationen, Kost"umen, Gestalten, an den
gemalten und doch zitternden B"aumen, an den Samtbaretten,
Ritterm"anteln und Degen, an allen diesen Trugbildern, in denen
eine so seltsame Harmonie wie um Dinge einer ganz andern Welt
lebte ... Eine junge Dame trat auf, die einem Reitknecht in
gr"unem Rocke eine B"orse zuwarf. Dann blieb sie allein, und nun
kam ein Fl"otensolo, zart wie Quellengefl"uster und
Vogelgezwitscher. Lucia begann ihre Kavatine in G-Dur. Sie sang
von ungl"ucklicher Liebe und w"unschte sich Fl"ugel. Ach, auch
Emma h"atte au{\s} diesem Leben fliehen m"ogen, weit weg in
Liebe{\s}armen!

Da erschien auf der Szene Lagardy al{\s} Edgard. Er hatte jenen
schimmernden blassen Teint, der dem S"udl"ander etwa{\s} von der
grandiosen Wirkung de{\s} Marmor{\s} verleiht. Seine m"annliche
Gestalt war in ein braune{\s} Wam{\s} gezw"angt. Ein kleiner Dolch
mit zierlichem Geh"ange schlug ihm die linke Lende. Er warf lange
schmachtende Blicke und zeigte seine blendend wei"sen Z"ahne. Man
hatte Emma erz"ahlt, eine polnische F"urstin habe ihn am Strand
von Biarritz singen h"oren, wo er Schiff{\s}zimmermann gewesen
sei, und sich in ihn verliebt. Seinetwegen habe sie sich ruiniert.
Er habe sie dann einer andern zuliebe sitzen lassen.

Derartige galante Abenteuer mit sentimentalem Finale dienten dem
be\-r"uhm\-ten K"unstler al{\s} Reklame. Der schlaue Mime brachte
e{\s} sogar fertig, in die Rezensionen der Zeitungen poetische
Flo{\s}keln "uber den bezaubernden Eindruck seiner Pers"onlichkeit
und die leichte Empf"anglichkeit seine{\s} Herzen{\s} zu
lancieren. Er besa"s eine sch"one Stimme, unfehlbare Sicherheit,
mehr Temperament al{\s} Intelligenz, mehr Patho{\s} al{\s}
Empfindung. Er war Genie und Scharlatan zugleich, und in seinem
Wesen lag ebensoviel von einem Friseur wie von einem Toreador.

Sobald er nur auf der B"uhne erschien, begeisterte er Emma. Er
schlo"s Lucia in seine Arme, wandte sich weg und kam wieder,
sichtlich verzweifelt. Bald loderte sein Ha"s wild auf, bald
klagte er in den zartesten Elegien, und die T"one perlten ihm
au{\s} der Kehle, zwischen Tr"anen und K"ussen. Emma beugte sich
weit vor, um ihn voll zu sehen, wobei sich ihre Fingern"agel in
den Pl"usch der Logenbr"ustung eingruben. Ihr Herz ward voll von
diesen wehm"utigen Melodien, die, von den Kontrab"assen dumpf
begleitet, nicht aufh"orten, gleich wie die Notschreie von
Schiffbr"uchigen im Sturmgebrau{\s}. Die junge Frau kannte alle
diese Verz"ucktheiten und Herzen{\s}"angste, die sie unl"angst
dem Tode so nahe gebracht hatten. Die Stimme der Primadonna
ersch"utterte sie wie eine laute Verk"undung ihrer heimlichsten
Beichte. Da{\s} Scheinbild der Kunst beleuchtete ihr die eigenen
Erlebnisse. Aber ach, so wie Lucia war sie doch von niemanden in
der Welt geliebt worden! Rudolf hatte nicht um sie geweint, so wie
Edgard, am letzten Abend im Mondenschein, al{\s} sie sich Lebewohl
sagten~...

Beifall durchst"urmte da{\s} Hau{\s}. Die ganze Stretta mu"ste
wiederholt werden. Noch einmal sangen die Liebenden von den Blumen
auf ihren Gr"abern, von Treue, Trennung, Verh"angni{\s} und
Hoffnungen; und al{\s} sie sich den letzten Scheidegru"s zuriefen,
stie"s Emma einen lauten Schrei au{\s}, der in der Orchestermusik
de{\s} Finale verhallte.

"`Warum l"a"st sie denn eigentlich dieser Edelmann nicht in
Ruhe?"' fragte Bovary.

"`Aber nein!"' antwortete sie. "`Da{\s} ist doch ihr Geliebter!"'

"`Er schw"ort doch, er wolle sich an ihrer Familie r"achen. Und
der andre, der dann kam, hat doch gesagt:
\begin{verse}
{\glq}Nimm, Teure, meine Schw"ure an \\
Der reinsten, w"armsten Liebe!{\grq}
\end{verse}
Und sie sagt:
\begin{verse}
{\glq}So sei e{\s} denn!{\grq}
\end{verse}
"Ubrigen{\s} der, mit dem sie fortging, Arm in Arm, der kleine
H"a"sliche mit der Hahnenfeder auf dem Hut, da{\s} war doch ihr
Vater, nicht wahr?"'

Trotz Emma{\s} Berichtigungen blieb Karl, der da{\s} Rezitativ im
zweiten Akte zwischen Lord Ashton und Gilbert mi"sverstanden
hatte, bei dem Glauben, Edgard habe Lucia ein Liebe{\s}zeichen
gesandt. Er gestand ein, von der ganzen Handlung nicht{\s}
begriffen zu haben. Die Musik st"ore, sie beeintr"achtige den
Text.

"`Wa{\s} schadet da{\s}?"' wandte Emma ein. "`Nun sei aber
still!"'

Er lehnte sich an ihren Arm. "`Ich m"ochte gern im Bilde sein.
Wei"st du?"'

"`Sei doch endlich still!"' sagte sie unwillig. "`Schweig!"'

Lucia nahte, von ihren Dienerinnen gest"utzt, einen Myrtenkranz im
Haar, bleicher al{\s} der wei"se Atla{\s} ihre{\s} Kleide{\s} ...
Emma gedachte ihre{\s} eigenen Hochzeit{\s}tage{\s}, sie sah sich
zwischen den Kornfeldern, auf dem schmalen Fu"sweg auf dem Gange
zur Kirche. Warum hatte sie sich da nicht so widersetzt wie Lucia,
unter leidenschaftlichem Flehen? Sie war vielmehr so fr"ohlich
gewesen, ohne im geringsten zu ahnen, welcher Niederung sie
zuschritt ... Ach, h"atte sie, jung und frisch und sch"on, noch
nicht besudelt durch die Ehe, noch nicht entt"auscht in ihrem
Ehebruch, auf ein feste{\s} edle{\s} Herz bauen und Tugend,
Z"artlichkeit, Sinnenlust und Pflichttreue zusammen f"uhlen
d"urfen! Niemal{\s} w"are sie von der H"ohe solcher
Gl"uckseligkeit herabgesunken! "`Nein, nein!"' rief sie
schmerzlich bei sich au{\s}. "`All da{\s} gro"se Gl"uck da unten
ist doch nur Lug und Trug, erdichtet von sehns"uchtigen oder
verzweifelten Phantasten!"' Jetzt erkannte sie, da"s die
Leidenschaften in der Wirklichkeit armselig sind und nur in der
"Uberschwenglichkeit der Kunst etwa{\s} Gro"se{\s}. Sie versuchte
sich zur n"uchternen Anschauung zu zwingen. Sie wollte in dieser
Wiedergabe ihrer eigenen Schmerzen nicht{\s} mehr sehen al{\s} ein
plastische{\s} Phantasiegebilde, nicht{\s} mehr und nicht{\s}
weniger al{\s} eine am"usante Augenweide. Und so l"achelte sie in
Gedanken "uberlegen-nachsichtig, al{\s} im Hintergrunde der B"uhne
hinter einer Samtportiere ein Mann in einem schwarzen Mantel
erschien, dem sein breitkrempiger gro"ser Hut bei einer
K"orperbewegung vom Kopfe fiel.

Da{\s} Sextett begann. S"anger und Orchester entfalten sich.
Edgard rast vor Wut; sein glockenklarer Tenor dominiert, Ashton
schleudert ihm in wuchtigen T"onen seine Tode{\s}drohungen
entgegen, Lucia klagt in schrillen Schreien, Arthur bleibt im
Ma"se der Nebenrolle, und Raimund{\s} Ba"s brummt wie
Orgelgebrau{\s}. Die Frauen de{\s} Chor{\s} wiederholen die Worte,
ein k"ostliche{\s} Echo. Gestikulierend stehen sie alle in einer
Reihe. Zorn, Rachgier, Eifersucht, Angst, Mitleid und Erstaunen
entstr"omen gleichzeitig ihren aufgerissenen M"undern. Der
w"utende Liebhaber schwingt seinen blanken Degen. Der
Spitzenkragen wogt ihm auf der schwer atmenden Brust auf und
nieder, w"ahrend er m"achtigen Schritt{\s} in seinen
sporenklirrenden Stulpenstiefeln "uber die B"uhne schreitet.

"`Er mu"s eine unersch"opfliche Liebe in sich tragen,"' dachte
Emma, "`da"s er sie an die Menge so verschwenden kann."' Ihre
Anwandlung von Gering\-sch"atzigkeit schwand vor dem Zauber seiner
Rolle. Sie f"uhlte sich zu dem Menschen hingezogen, der sie unter
dieser Gestalt berauschte. Sie versuchte, sich sein Leben
vorzustellen, sein bewegte{\s}, ungew"ohnliche{\s}, gl"anzende{\s}
Leben, an dem sie h"atte teilnehmen k"onnen, wenn e{\s} der Zufall
gef"ugt h"atte. Warum hatten sie sich nicht kennen gelernt und
sich ineinander verliebt! Sie w"are mit ihm durch alle L"ander
Europa{\s} gereist, von Hauptstadt zu Hauptstadt, h"atte mit ihm
M"uhen und Erfolge geteilt, die Blumen aufgelesen, die man ihm
streute, und seine B"uhnenkost"ume eigenh"andig gestickt. Alle
Abende h"atte sie, im Dunkel einer Loge, hinter vergoldetem Gitter
aufmerksam den S"angen seiner Seele gelauscht, die einzig und
allein ihr gewidmet w"aren. Von der Szene, beim Singen, h"atte er
zu ihr geschaut~...

Sie erschrak und ward verwirrt. Der S"anger sah zu ihr hinauf.
Kein Zweifel! Sie h"atte zu ihm hinst"urzen m"ogen, in seine Arme,
in seine Umarmung fliehen, al{\s} sei er die Verk"orperung der
Liebe, und ihm laut zurufen:

"`Nimm mich, entf"uhre mich! Komm! Ich geh"ore dir, nur dir! Dir
gelten alle meine Tr"aume, mein ganze{\s} hei"se{\s} Herz!"'

Der Vorhang fiel.

Ga{\s}geruch erschwerte da{\s} Atmen, und da{\s} F"acheln der
F"acher machte die Luft noch unertr"aglicher. Emma wollte die Loge
verlassen, aber die G"ange waren durch die vielen Menschen
versperrt. Sie sank in ihren Sessel zur"uck. Sie bekam Herzklopfen
und Atemnot. Da Karl f"urchtete, sie k"onne ohnm"achtig werden,
eilte er nach dem B"ufett, um ihr ein Gla{\s} Mandelmilch zu
holen.

Er hatte gro"se M"uhe, wieder nach der Loge zu gelangen. Da{\s}
Gla{\s} in beiden H"anden, rannte er bei jedem Schritte, den er
tat, jemanden mit den Ellenbogen an. Schlie"slich go"s er
dreiviertel de{\s} Inhalt{\s} einer Dame in au{\s}geschnittener
Toilette "uber die Schulter. Al{\s} sie da{\s} k"uhle Na"s, da{\s}
ihr den R"ucken hinabrann, sp"urte, schrie sie laut auf, al{\s} ob
man ihr an{\s} Leben wolle. Ihr Gatte, ein Rouener
Seifenfabrikant, ereiferte sich "uber diese Ungeschicktheit.
W"ahrend seine Frau mit dem Taschentuche die Flecke von ihrem
sch"onen roten Taftkleide abtupfte, knurrte er w"utend etwa{\s}
von Schadenersatz, Wert und Bezahlen. Endlich kam Karl gl"ucklich
bei Emma wieder an. G"anzlich au"ser Atem berichtete er ihr:

"`Wei"s Gott, beinahe h"att ich mich nicht durchgew"urgt! Nein,
diese Mensch\-heit! Diese Mensch\-heit!"' Nach einigem
Verschnaufen f"ugte er hinzu: "`Und ahnst du, wer mir da oben
begegnet ist? Leo!"'

"`Leo?"'

"`Jawohl! Er wird gleich kommen, dir guten Tag zu sagen!"'

Er hatte diese Worte kaum au{\s}gesprochen, al{\s} der Adjunkt
auch schon in der Loge erschien. Mit weltm"annischer
Ungezwungenheit reichte er ihr die Hand. Mechanisch streckte Frau
Bovary die ihrige au{\s}, wie im Banne eine{\s} st"arkeren
Willen{\s}. Diesen fremden Einflu"s hatte sie lange nicht
empfunden, seit jenem Fr"uhling{\s}nachmittage nicht, an dem sie
voneinander Abschied genommen. Sie hatte am Fenster gestanden, und
drau"sen war leiser Regen auf die Bl"atter gefallen. Aber rasch
besann sie sich auf da{\s}, wa{\s} die jetzige Situation und die
Konvenienz erheischten. Mit aller Kraft sch"uttelte sie den alten
Bann und die alten Erinnerungen von sich ab und begann ein paar
hastige Reden{\s}arten zu stammeln:

"`Ach, guten Tag! Wie? Sie hier?"'

"`Ruhe!"' ert"onte eine Stimme im Parkett. Inzwischen hatte
n"amlich der dritte Akt begonnen.

"`So sind Sie also in Rouen?"'

"`Ja, gn"adige Frau!"'

"`Und seit wann?"'

"`Hinau{\s}! Hinau{\s}!"'

Alle{\s} drehte sich nach ihnen um. Sie verstummten.

Von diesem Augenblick war e{\s} mit Emma{\s} Aufmerksamkeit
vorbei. Der Chor der Hochzeit{\s}g"aste, die Szene zwischen Ashton
und seinem Diener, da{\s} gro"se Duett in D-Dur, alle{\s} da{\s}
spielte sich f"ur sie wie in gro"ser Entfernung ab. E{\s} war ihr,
al{\s} kl"ange da{\s} Orchester nur noch ged"ampft, al{\s} s"angen
die Personen ihr weit entr"uckt. Sie dachte zur"uck an die
Spielabende im Hause de{\s} Apotheker{\s}, an den Gang zu der Amme
ihre{\s} Kinde{\s}, an da{\s} Vorlesen in der Laube, an die
Plauderstunden zu zweit am Kamin, an alle Einzelheiten dieser
armen Liebe, die so friedsam, so traulich und so zart gewesen war
und die sie l"angst vergessen hatte. Warum war er wieder da?
Welche{\s} Zusammentreffen von besonderen Umst"anden lie"s ihn von
neuem ihren Leben{\s}pfad kreuzen?

Er stand hinter ihr, die Schulter an die Logenwand gelehnt. Von
Zeit zu Zeit schauerte Emma zusammen, wenn sie den warmen Hauch
seiner Atemz"uge auf ihrem Haar sp"urte.

"`Macht Ihnen denn da{\s} Spa"s?"' fragte er sie, indem er sich
"uber sie beugte, so da"s die Spitze seine{\s} Schnurrbart{\s}
ihre Wange streifte.

"`Nein, nicht besonder{\s}!"' entgegnete sie leichthin.

Daraufhin machte er den Vorschlag, da{\s} Theater zu verlassen und
irgendwo eine Portion Ei{\s} zu essen.

"`Ach nein! Noch nicht! Bleiben wir!"' sagte Bovary. "`Sie hat
aufgel"oste{\s} Haar! E{\s} scheint also tragisch zu werden!"'

Aber die Wahnsinn{\s}s\/zene interessierte Emma gar nicht. Da{\s}
Spiel der S"angerin schien ihr "ubertrieben.

"`Sie schreit zu sehr!"' meinte sie, zu Karl gewandt, der
aufmerksam zu\-h"orte.

"`M"oglich! Jawohl! Ein wenig!"' gab er zur Antwort. Eigentlich
gefiel ihm die S"angerin, aber die Meinung seiner Frau, die er
immer zu respektieren pflegte, machte ihn unschl"ussig.

Leo st"ohnte:

"`Ist da{\s} eine Hitze!"'

"`Tats"achlich! Nicht zum Au{\s}halten!"' sagte Emma.

"`Vertr"agst du{\s} nicht mehr?"' fragte Bovary.

"`Ich ersticke! Wir wollen gehen!"'

Leo legte ihr behutsam den langen Spitzenschal um. Dann
schlenderten sie alle drei nach dem Hafen, wo sie vor einem
Kaffeehause im Freien Platz nahmen.

Anfang{\s} unterhielten sie sich von Emma{\s} Krankheit. Sie
versuchte mehrfach, dem Gespr"ach eine andere Wendung zu geben,
indem sie die Bemerkung machte, sie f"urchte, Herrn Leo k"onne
da{\s} langweilen. Darauf erz"ahlte dieser, er m"usse sich in
Rouen zwei Jahre t"uchtig auf die Hosen setzen, um sich in die
hiesige Recht{\s}pflege einzuarbeiten. In der Normandie mache man
alle{\s} ander{\s} al{\s} in Pari{\s}. Dann erkundigte er sich
nach der kleinen Berta, nach der Familie Homai{\s}, nach der
L"owenwirtin. Mehr konnten sie sich in Karl{\s} Gegenwart nicht
sagen, und so stockte die Unterhaltung.

Au{\s} der Oper kommende Leute gingen vor"uber, laut pfeifend und
tr"al\-lernd:
\begin{verse}
{\glq}O Engel reiner Liebe!{\grq}
\end{verse}

Leo kehrte den Kunstkenner herau{\s} und begann "uber Musik zu
sprechen. Er habe Tamburini, Rubini, Persiani, Crisi geh"ort. Im
Vergleich mit denen sei Lagardy trotz seiner gro"sen Erfolge gar
nicht{\s}.

Karl, der sein Sorbett mit Rum in ganz kleinen Dosen vertilgte,
unterbrach ihn:

"`Aber im letzten Akt, da soll er ganz wunderbar sein! Ich
bedaure, da"s ich nicht bi{\s} zu Ende drin geblieben bin. E{\s}
fing mir grade an zu gefallen!"'

"`Demn"achst gibt{\s} ja eine Wiederholung!"' tr"ostete ihn Leo.

Karl erwiderte, da"s sie am n"achsten Tage wieder nach Hause
m"u"sten. "`E{\s} sei denn,"' meinte er, zu Emma gewandt, "`du
bliebst allein hier, mein Herzchen?"'

Bei dieser unerwarteten Au{\s}sicht, die sich seiner
Begehrlichkeit bot, "anderte der junge Mann seine Taktik. Nun
lobte er da{\s} Finale de{\s} S"anger{\s}. Er sei da k"ostlich,
gro"sartig!

Von neuem redete Karl seiner Frau zu:

"`Du kannst ja am Sonntag zur"uckfahren. Entschlie"se dich nur!
E{\s} w"are unrecht von dir, wenn du e{\s} nicht t"atest, sofern
du dir auch nur ein wenig Vergn"ugen davon versprichst!"'

Inzwischen waren die Nachbartische leer geworden. Der Kellner
stand fortw"ahrend in ihrer n"achsten N"ahe herum. Karl begriff
und zog seine B"orse. Leo kam ihm zuvor und gab obendrein zwei
Silberst"ucke Trinkgeld, die er auf der Marmorplatte klirren
lie"s.

"`E{\s} ist mir wirklich nicht recht,"' murmelte Bovary, "`da"s
Sie f"ur un{\s} Geld~..."'

Der andere machte die aufrichtig gemeinte Geste der
Nebens"achlichkeit und ergriff seinen Hut.

"`E{\s} bleibt dabei! Morgen um sech{\s} Uhr!"'

Karl beteuerte nochmal{\s}, da"s er unm"oglich so lange bleiben
k"onne. Emma indessen sei durch nicht{\s} gehindert.

"`E{\s} ist nur~..."', stotterte sie, verlegen l"achelnd, "`... ich
wei"s nicht recht~..."'

"`Na, "uberleg dir{\s} noch! Wir k"onnen ja noch mal dar"uber
reden, wenn du{\s} beschlafen hast!"' Und zu Leo gewandt, der sie
begleitete, sagte er: "`Wo Sie jetzt wieder in unserer Gegend
sind, hoffe ich, da"s Sie sich ab und zu bei un{\s} zu Tisch
ansagen!"'

Der Adjunkt versicherte, er werde nicht verfehlen, da er ohnehin
dem\-n"achst in Yonville beruflich zu tun habe.

Al{\s} man sich vor dem Durchgang Saint-Herbland voneinander
verabschiedete, schlug die Uhr der Kathedrale halb zw"olf.


\newpage
\thispagestyle{empty}
\begin{center}
\vspace{5cm}
{\Huge \so{Dritte{\s} Bu{ch}}}
\end{center}


\newpage\begin{center}
{\large \so{Er{st}e{\s} Kapitel}}\bigskip\bigskip

\end{center}

Leo hatte w"ahrend seiner Pariser Studienzeit die Balls"ale
flei"sig besucht und daselbst recht h"ubsche Erfolge bei den
Grisetten gehabt. Sie hatten gefunden, er s"ahe sehr schick
au{\s}. "Ubrigen{\s} war er der m"a"sigste Student. Er trug da{\s}
Haar weder zu kurz noch zu lang, verjuchheite nicht gleich am
Ersten de{\s} Monat{\s} sein ganze{\s} Geld und stand sich mit
seinen Professoren vortrefflich. Von wirklichen Au{\s}schweifungen
hatte er sich allezeit fern gehalten, au{\s} "Angstlichkeit und
weil ihm da{\s} w"uste Leben zu grob war.

Oft, wenn er de{\s} Abend{\s} in seinem Zimmer la{\s} oder unter
den Linden de{\s} Luxemburggarten{\s} sa"s, glitt ihm sein
Code-Napol\'eon au{\s} den H"anden. Dann kam ihm Emma in den Sinn.
Aber allm"ahlich verbla"ste diese Erinnerung, und allerlei
Liebeleien "uberwucherten sie, ohne sie freilich ganz zu
ersticken. Denn er hatte noch nicht alle Hoffnung verloren, und
ein vage{\s} Versprechen winkte ihm in der Zukunft wie eine goldne
Frucht an einem Wunderbaume.

Al{\s} er sie jetzt nach dreij"ahriger Trennung wiedersah,
erwachte seine alte Leidenschaft wieder. Er sagte sich, jetzt
g"alte e{\s}, sich fest zu entschlie"sen, wenn er sie besitzen
wollte. Seine ehemalige Sch"uchternheit hatte er "ubrigen{\s} im
Verkehr mit leichtfertiger Gesellschaft abgelegt. Er war in die
Provinz zur"uckgekehrt mit einer gewissen Verachtung aller derer,
die nicht schon ein paar Lackschuhe auf dem Asphalt der Gro"sstadt
abgetreten hatten. Vor einer Pariserin in Spitzen, im Salon
eine{\s} ber"uhmten Professor{\s} mit Orden und Equipage, h"atte
der arme Adjunkt sicherlich gezittert wie ein Kind, hier aber, in
Rouen, am Hafen, vor der Frau diese{\s} kleinen Landarzte{\s}, da
f"uhlte er sich "uberlegen und eine{\s} leichten Siege{\s} gewi"s.
Sichere{\s} Auftreten h"angt von der Umgebung ab. Im ersten Stock
spricht man ander{\s} al{\s} im vierten, und e{\s} ist beinahe,
al{\s} seien die Banknoten einer reichen Frau ihr Tugendw"achter.
Sie tr"agt sie alle mit sich wie ein Panzerhemd unter ihrem
Korsett.

Nachdem sich Leo von Herrn und Frau Bovary verabschiedet hatte,
war er au{\s} einiger Entfernung den beiden durch die Stra"sen
gefolgt, bi{\s} er sie im "`Roten Kreuz"' verschwinden sah. Dann
machte er kehrt und gr"ubelte die ganze Nacht hindurch "uber einen
Krieg{\s}plan.

Am andern Tag nachmittag{\s} gegen f"unf Uhr betrat er den Gasthof
mit beklommener Kehle, blassen Wangen und dem festen Entschlu"s,
vor nicht{\s} zur"uckzuscheuen.

"`Der Herr Doktor ist schon wieder abgereist!"' vermeldete ihm ein
Kellner.

Leo fa"ste da{\s} al{\s} gute{\s} Vorzeichen auf. Er stieg hinauf.

Emma war offenbar gar nicht aufgeregt, al{\s} er eintrat. Sie bat
ihn k"uhl um Entschuldigung, da"s sie gestern vergessen habe, ihm
mit\/zuteilen, in welchem Gasthofe sie abgestiegen seien.

"`O, da{\s} habe ich erraten"', sagte Leo.

"`Wieso?"'

Er behauptete, da{\s} gute Gl"uck, eine innere Stimme habe ihn
hierher geleitet.

Sie l"achelte; und um seine Albernheit wieder gut\/zumachen, log
er nunmehr, er habe den ganzen Morgen damit zugebracht, in allen
Gasth"ofen nach ihnen zu fragen.

"`Sie haben sich also entschlossen zu bleiben?"' f"ugte er hinzu.

"`Ja,"' gab sie zur Antwort, "`aber ich h"atte e{\s} lieber nicht
tun sollen. Man darf sich nicht an unpraktische Vergn"ugungen
gew"ohnen, wenn man zu Hause tausend Pflichten hat~..."'

"`Ja, da{\s} kann ich mir denken~..."'

"`Nein, da{\s} k"onnen Sie nicht. Da{\s} kann nur eine Frau."'

Er meinte, die M"anner h"atten auch ihr Kreuz, und nach einer
philosophischen Einleitung begann die eigentliche Unterhaltung.
Emma beklagte die Armseligkeit der irdischen Freuden und die ewige
Einsamkeit, in die da{\s} Menschenherz verbannt sei.

Um sich Ansehen zu geben, oder vielleicht auch in unwillk"urlicher
Nachahmung ihrer Melancholie, die ihn angesteckt hatte, behauptete
der junge Mann, er h"atte sich w"ahrend seiner ganzen Studienzeit
ungeheuerlich gelangweilt. Die Juristerei sei ihm gr"a"slich
zuwider. Andere Beruf{\s}arten lockten ihn stark, aber seine
Mutter qu"ale ihn in jedem ihrer Briefe. Mehr und mehr schilderten
sie sich die Gr"unde ihre{\s} Leid{\s}, und je eifriger sie
sprachen, um so st"arker packte sie die wachsende Vertraulichkeit.
Aber ganz offen waren sie alle beide nicht; sie suchten nach
Worten, mit denen sie die nackte Wahrheit umschreiben k"onnten.
Emma verheimlichte e{\s}, da"s sie inzwischen einen andern
geliebt, und er gestand nicht, da"s er sie vergessen hatte.
Vielleicht dachte er auch wirklich nicht mehr an die Souper{\s}
nach den Ma{\s}kenb"allen, und sie erinnerte sich nicht ihrer
Morgeng"ange, wie sie durch die Wiesen nach dem Rittergute zu dem
Geliebten gegangen war. Der Stra"senl"arm hallte nur schwach zu
ihnen herauf, und die Enge de{\s} Zimmer{\s} schien ihr Alleinsein
noch traulicher zu machen. Emma trug ein Morgenkleid au{\s}
leichtem Stoff; sie lehnte ihren Kopf gegen den R"ucken de{\s}
alten Lehnstuhl{\s}, in dem sie sa"s. Hinter ihr die gelbe Tapete
umgab sie wie mit Goldgrund, und ihr blo"ser Kopf mit dem
schimmernden Scheitel, der ihre Ohren beinahe ganz verdeckte,
wiederholte sich wie ein Gem"alde im Spiegel.

"`Ach, verzeihen Sie!"' sagte sie. "`E{\s} ist unrecht von mir,
Sie mit meinen ewigen Klagen zu langweilen."'

"`Keine{\s}weg{\s}!"'

"`Wenn Sie w"u"sten,"' fuhr sie fort und schlug ihre sch"onen
Augen, au{\s} denen Tr"anen rollten, zur Decke empor, "`wa{\s} ich
mir alle{\s} ertr"aumt habe!"'

"`Und ich erst! Ach, ich habe so sehr gelitten! Oft bin ich
au{\s}gegangen, still f"ur mich hin, und hab mich die Kai{\s}
entlang geschleppt, nur um mich im Getriebe der Menge zu
zerstreuen und die tr"uben Gedanken lo{\s}zubekommen, die mich in
einem fort verfolgten. In einem Schaufenster eine{\s}
Kunsth"andler{\s} auf dem Boulevard habe ich einmal einen
italienischen Kupferstich gesehen, der eine Muse darstellt. Sie
tr"agt eine Tunika, einen Vergi"smeinnichtkranz im offnen Haar und
blickt zum Mond empor. Irgend etwa{\s} trieb mich immer wieder
dorthin. Oft hab ich stundenlang davor gestanden~..."' Und mit
zitternder Stimme f"ugte er hinzu: "`Sie sah Ihnen ein wenig
"ahnlich."'

Frau Bovary wandte sich ab, damit er da{\s} L"acheln um ihre
Lippen nicht bemerke, da{\s} sie nicht unterdr"ucken konnte.

"`Und wie oft"', fuhr er fort, "`habe ich an Sie Briefe
geschrieben und hinterher wieder zerrissen."'

Sie antwortete nicht.

"`Manchmal bildete ich mir ein, irgendein Zufall m"usse Sie mir
wieder in den Weg f"uhren. Oft war e{\s} mir, al{\s} ob ich Sie an
der n"achsten Stra"senecke treffen sollte. Ich bin hinter
Droschken hergelaufen, au{\s} denen ein Schal oder ein Schleier
flatterte, wie Sie welche zu tragen pflegen~..."'

Sie schien sich vorgenommen zu haben, ihn ohne Unterbrechung reden
zu lassen. Sie hatte die Arme gekreuzt und betrachtete gesenkten
Haupte{\s} die Rosetten ihrer Hau{\s}schuhe, auf deren Atla{\s}
die kleinen Bewegungen sichtbar wurden, die sie ab und zu mit den
Zehen machte.

Endlich sagte sie mit einem Seufzer:

"`Ist e{\s} nicht da{\s} Allertraurigste, ein unn"utze{\s} Leben
so wie ich f"uhren zu m"ussen? Wenn unsere Schmerzen wenigsten{\s}
jemandem n"utzlich w"aren, dann k"onnte man sich doch in dem
Bewu"stsein tr"osten, sich f"ur etwa{\s} zu opfern."'

Er prie{\s} die Tugend, die Pflicht und da{\s} stumme Sichaufopfern.
Er selbst versp"ure eine unglaubliche Sehnsucht, ganz in etwa{\s}
aufzugehen, die er nicht befriedigen k"onne.

"`Ich m"ochte am liebsten Krankenschwester sein"', behauptete sie.

"`Ach ja!"' erwiderte er. "`Aber f"ur un{\s} M"anner gibt e{\s}
keinen solchen barmherzigen Beruf. Ich w"u"ste keine Besch"aftigung
... e{\s} sei denn vielleicht die de{\s} Arzte{\s}~..."'

Emma unterbrach ihn mit einem leichten Achselzucken und begann von
ihrer Krankheit zu sprechen, an der sie beinah gestorben w"are.
Wie schade! meinte sie, dann brauche sie jetzt nicht mehr zu
leiden. Sofort schw"armte Leo f"ur die "`Ruhe im Grabe"'. Ja, er
h"atte sogar eine{\s} Abend{\s} sein Testament niedergeschrieben
und darin bestimmt, da"s man ihm in den Sarg die sch"one Decke mit
der Seidenstickerei legen solle, die er von ihr geschenkt bekommen
hatte. Nach dem, wie alle{\s} h"atte sein k"onnen, also nach einem
imagin"aren Zustand, "anderten sie jetzt in der Erz"ahlung ihre
Vergangenheit. Ist doch die Sprache immer ein Walzwerk, da{\s} die
Gef"uhle breitdr"uckt.

Bei dem M"archen von der Reisedecke fragte sie:

"`Warum denn?"'

"`Warum?"' Er z"ogerte. "`Weil ich Sie so z"artlich geliebt habe!"'

Froh, die gr"o"ste Schwierigkeit "uberwunden zu haben, beobachtete
Leo Emma{\s} Gesicht von der Seite. E{\s} leuchtete wie der
Himmel, wenn der Wind pl"otzlich eine Wolkenschicht, die dar"uber
war, zerrei"st. Die vielen traurigen Gedanken, die e{\s}
verdunkelt hatten, waren au{\s} ihren Augen wie weggeweht.

Er wartete. Endlich sagte sie:

"`Ich hab e{\s} immer geahnt~..."'

Nun begannen sie von den kleinen Begebnissen jener fernen Tage
einander zu erz"ahlen, von allem Freud und Leid, da{\s} sie soeben
in ein einzige{\s} Wort zusammengefa"st hatten. Er erinnerte sich
der Wiege au{\s} Tannenholz, ihrer Kleider, der M"obel in ihrem
Zimmer, ihre{\s} ganzen Hause{\s}.

"`Und unsere armen Kakteen, wa{\s} machen die?"'

"`Sie sind letzten Winter alle erfroren!"'

"`Ach, wie oft hab ich an sie zur"uckgedacht. Da{\s} glauben Sie
mir gar nicht! Wie oft hab ich sie vor mir gesehen, wie damal{\s}
im Sommer, wenn die Morgensonne auf Ihre Jalousien schien ... und
Sie mit blo"sen Armen Ihre Blumen begossen~..."'

"`Armer Freund!"' sagte sie und reichte ihm ihre Hand.

Leo beeilte sich, seine Lippen darauf zu pressen. Dann seufzte er
tief auf und sagte:

"`Damal{\s} "ubten Sie einen geheimni{\s}vollen Zauber auf mich
au{\s}. Ich war ganz in Ihrem Banne. Einmal zum Beispiel kam ich
zu Ihnen ... aber Sie werden sich wohl nicht mehr daran
erinnern?"'

"`Doch, fahren Sie nur fort!"'

"`Sie standen unten in der Hau{\s}flur, wo die Treppe aufh"ort,
gerade im Begriff au{\s}zugehen. Sie hatten einen Hut mit kleinen
blauen Blumen auf. Ohne da"s Sie mich dazu aufgefordert hatten,
begleitete ich Sie. Ich konnte nicht ander{\s}. Aber mir jeder
Minute trat e{\s} mir klarer in{\s} Bewu"stsein, wie ungezogen
da{\s} von mir war. "Angstlich und unsicher ging ich neben Ihnen
her und brachte e{\s} doch nicht "uber mich, mich von Ihnen zu
trennen. Wenn Sie in einen Laden traten, wartete ich drau"sen auf
der Stra"se und sah Ihnen durch da{\s} Schaufenster zu, wie Sie
die Handschuhe abstreiften und da{\s} Geld auf den Ladentisch
legten. Zuletzt klingelten Sie bei Frau T"uvache; man "offnete
Ihnen, und ich stand wie ein begossener Pudel vor der m"achtigen
Hau{\s}t"ure, die hinter Ihnen in{\s} Schlo"s gefallen war."'

Frau Bovary h"orte ihm zu, ganz verwundert. Wie lange war da{\s}
schon her! Alle diese Dinge, die au{\s} der Vergessenheit
heraufstiegen, erweckten in ihr da{\s} Gef"uhl, eine alte Frau zu
sein. Unendlich viele innere Erlebnisse lagen dazwischen. Ab und
zu sagte sie mit leiser Stimme und halbgeschlossenen Lidern:

"`Ja ... So war e{\s} ... So war e{\s} ... So war e{\s}!"'

Von den verschiedenen Uhren der Stadt schlug e{\s} acht, von den
Uhren der Schulen, Kirchen und verlassenen Pal"aste. Sie sprachen
nicht mehr, aber sie sahen einander an und sp"urten dabei ein
Brausen in ihren K"opfen, und jeder hatte da{\s} Gef"uhl,
diese{\s} Rauschen str"ome au{\s} den starren Augensternen de{\s}
anderen. Ihre H"ande hatten sich gefunden, und Vergangenheit und
Zukunft, Erinnerung und Tr"aume, alle{\s} ward ein{\s} mir der
z"artlichen Wonne de{\s} Augenblick{\s}. Die D"ammerung dichtete
sich an den W"anden, und halb im Dunkel verloren, schimmerten nur
noch die grellen Farbenflecke von vier dah"angenden Buntdrucken.
Durch da{\s} oben offene Fenster erblickte man zwischen spitzen
Dachgiebeln ein St"uck de{\s} schwarzen Himmel{\s}.

Emma erhob sich, um die Kerzen in den beiden Leuchtern auf der
Kommode anzuz"unden. Dann setzte sie sich wieder.

"`Wa{\s} ich sagen wollte~..."', begann Leo von neuem.

"`Wa{\s} war e{\s}?"'

Er suchte nach Worten, um die unterbrochene Unterhaltung wieder
anzukn"upfen, da fragte sie ihn:

"`Wie kommt e{\s}, da"s mir noch niemand solche innere Erlebnisse
anvertraut hat?"'

Leo erwiderte, ideale Naturen f"anden selten Wahlverwandte. Er
habe sie vorn ersten Augenblicke an geliebt, und der Gedanke
bringe ihn zur Verzweiflung, da"s sie miteinander f"ur immerdar
verbunden worden w"aren, wenn ein guter Stern sie fr"uher
zusammengef"uhrt h"atte.

"`Ich habe manchmal da{\s}selbe gedacht"', sagte sie.

"`Welch ein sch"oner Traum!"' murmelte Leo. Und w"ahrend er mit
der Hand "uber den blauen Saum der Schleife ihre{\s} wei"sen
G"urtel{\s} hinstrich, f"ugte er hinzu: "`Aber wa{\s} hindert
un{\s} denn, von vorn anzufangen?"'

"`Nein, mein Freund"', erwiderte sie. "`Dazu bin ich zu alt ...
und Sie zu jung ... Vergessen Sie mich! Andre werden Sie lieben
... und Sie werden sie wieder lieben!"'

"`Nicht so, wie ich Sie liebe!"'

"`Sie sind ein Kind! Seien Sie vern"unftig. Ich will e{\s}!"'

Sie setzte ihm au{\s}einander, da"s Liebe zwischen ihnen ein Ding
der Un\-m"oglich\-keit sei und da"s sie sich nur wie Schwester und
Bruder lieben k"onnten, wie ehemal{\s}.

Ob sie da{\s} wirklich im Ernst sagte, da{\s} wu"ste sie selbst
nicht. Sie f"uhlte nur, wie sie der Verf"uhrung zu unterliegen
drohte und da"s sie dagegen ank"ampfen m"usse. Sie sah Leo
z"artlich an und stie"s sanft seine zitternden H"ande zur"uck, die
sie sch"uchtern zu liebkosen versuchten.

"`Seien Sie mir nicht b"o{\s}!"' sagte er und wich zur"uck.

Emma empfand eine unbestimmte Furcht vor seiner Zaghaftigkeit, die
ihr viel gef"ahrlicher war al{\s} die K"uhnheit Rudolf{\s}, wenn
er mit au{\s}gebreiteten Armen auf sie zugekommen war. Niemal{\s}
war ihr ein Mann so sch"on erschienen. In seinem Wesen lag eine
k"ostliche Keuschheit. Seine Augen mit den langen, feinen, ein
wenig aufw"art{\s}gebogenen Wimpern waren halb geschlossen. Die
zarte Haut seiner Wangen war rot geworden, au{\s} Verlangen nach
ihr, wie sie glaubte, und sie vermochte dem Drange kaum zu
widerstehen, sie mit ihren Lippen zu ber"uhren. Da fiel ihr Blick
auf die Wanduhr.

"`Mein Gott, wie sp"at e{\s} schon ist!"' rief sie au{\s}. "`Wir
haben un{\s} verplaudert!"'

Er verstand den Wink und suchte nach seinem Hut.

"`Da{\s} Theater habe ich ganz vergessen"', fuhr Emma fort. "`Und
mein armer Mann hat mich doch de{\s}halb nur hiergelassen. Herr
und Frau Lormeaux au{\s} der Gro"senbr"uckenstra"se wollten mich
begleiten~..."'

Schade! Denn morgen m"usse sie wieder zu Hause sein.

"`So?"' fragte Leo.

"`Gewi"s!"'

"`Aber ich mu"s Sie noch einmal sehen. Ich hab Ihnen noch etwa{\s}
zu sagen!"'

"`Wa{\s} denn?"'

"`Etwa{\s} ... Wichtige{\s}, Ernste{\s}! Ach, Sie d"urfen noch
nicht heimfahren! Nein! Da{\s} ist unm"oglich! Wenn Sie w"u"sten
... H"oren Sie mich doch an ... Sie haben mich doch verstanden?
Ahnen Sie denn nicht~..."'

"`Sie haben e{\s} doch ziemlich deutlich gesagt!"'

"`Ach, scherzen Sie nicht! Da{\s} ertrag ich nicht! Haben Sie
Mitleid mit mir! Ich m"ochte Sie noch einmal sehen ... einmal ...
ein einzige{\s}~..."'

"`E{\s} sei!"' Sie hielt inne. Dann aber, al{\s} bes"anne sie sich
ander{\s}, sagte sie: "`Aber nicht hier!"'

"`Wo Sie wollen!"'

Sie dachte bei sich nach, dann sagte sie kurz:

"`Morgen um elf in der Kathedrale!"'

"`Ich werde dort sein"', rief er au{\s} und griff hastig nach
ihren H"anden. Sie ent\/zog sie ihm.

Und wie sie beide aufrecht dastanden, sie mit gesenktem Kopf vor
ihm, da beugte er sich "uber sie und dr"uckte einen langen Ku"s
auf ihren Nacken.

"`Sie sind toll! Ach, Sie sind toll!"' rief sie und lachte mit
einem eigent"umlichen tiefen Klange leise auf, w"ahrend er ihren
Hal{\s} immer noch mehr mit K"ussen bedeckte. Dann beugte er den
Kopf "uber ihre Schulter, al{\s} wolle er in den Augen ihre
Zustimmung suchen. Da traf ihn ein eisiger stolzer Blick.

Er trat drei Schritte zur"uck, der T"ure zu. Auf der Schwelle
blieb er stehen und stammelte mit zitternder Stimme:

"`Auf Wiedersehn morgen!"'

Sie nickte und verschwand, leise wie ein Vogel, im Nebenzimmer.

Am Abend schrieb sie Leo einen endlosen Brief, in dem sie die
Verabredung zur"ucknahm. E{\s} sei alle{\s} au{\s}, und e{\s}
w"are zum Wohle beider, wenn sie sich nicht wieders"ahen. Aber
al{\s} der Brief fertig war, fiel ihr ein, da"s sie doch seine
Adresse gar nicht wu"ste. Wa{\s} sollte sie tun?

"`Ich werde ihm den Brief selbst geben,"' sagte sie sich,
"`morgen, wenn er kommt."'

Am andern Morgen stand Leo schon fr"uh in der offnen Balkont"ure,
reinigte sich eigenh"andig seine Schuhe und sang leise vor sich
hin. Er machte e{\s} sehr sorgf"altig. Dann zog er ein wei"se{\s}
Beinkleid an, elegante Str"umpfe, einen gr"unen Rock, und
sch"uttete seinen ganzen Vorrat von Parf"um in sein Taschentuch.
Er ging zum Coiffeur, zerst"orte sich aber hinterher die Frisur
ein wenig, weil sein Haar nicht unnat"urlich au{\s}sehen sollte.

"`E{\s} ist noch zu zeitig"', sagte er, al{\s} er auf der
Kuckuck{\s}uhr de{\s} Friseur{\s} sah, da"s e{\s} noch nicht neun
Uhr war.

Er bl"atterte in einem alten Modejournal, dann verlie"s er den
Laden, z"undete sich eine Zigarre an, schlenderte durch drei
Stra"sen, und al{\s} er dachte, e{\s} sei Zeit, ging er langsam
zum Notre-Dame-Platze.

E{\s} war ein pr"achtiger Sommermorgen. In den Schaufenstern der
Juweliere glitzerten die Silberwaren, und da{\s} Licht, da{\s}
schr"ag auf die Kathedrale fiel, flimmerte auf den Bruchfl"achen
der grauen Quadersteine. Ein Schwarm V"ogel flatterte im Blau
de{\s} Himmel{\s} um die Kreuzblumen der T"urme. "Uber den
l"armigen Platz wehte Blumenduft au{\s} den Anlagen her, wo
Ja{\s}min, Nelken, Narzissen und Tuberosen bl"uhten, von saftigen
Gra{\s}fl"achen umrahmt und von Beeren tragenden B"uschen f"ur die
V"ogel. In der Mitte pl"atscherte ein Springbrunnen, und zwischen
Pyramiden von Melonen sa"sen H"okerinnen, barh"auptig unter
ungeheuren Schirmen, und banden kleine Veilchenstr"au"se.

Leo kaufte einen. E{\s} war da{\s} erstemal, da"s er Blumen f"ur
eine Frau kaufte; und da{\s} Herz schlug ihm h"oher, wie er den
Duft der Veilchen einatmete, al{\s} ob diese Huldigung, die er
Emma darbringen wollte, ihm selber g"olte. Er f"urchtete,
beobachtet zu werden, und rasch trat er in die Kirche.

Auf der Schwelle der linken T"ure de{\s} Hauptportal{\s} unter der
{\glq}Tanzenden Salome{\grq} stand der Schweizer, den Federhut auf
dem Kopf, den Degen an der Seite, den Stock in der Faust,
w"urdevoller al{\s} ein Kardinal und goldstrotzend wie ein
Hostienkelch. Er trat Leo in den Weg und fragte mit jenem
s"u"slich-g"utigen L"acheln, da{\s} Geistliche anzunehmen pflegen,
wenn sie mit Kindern reden:

"`Der Herr ist gewi"s nicht von hier? Will der Herr die
Sehen{\s}w"urdigkeiten der Kathedrale besichtigen?"'

"`Nein!"'

Leo machte zun"achst einen Rundgang durch die beiden Seitenschiffe
und kam zum Hauptportal zur"uck. Emma war noch nicht da. Er ging
abermal{\s} bi{\s} zum Chor.

Teile de{\s} Ma"swerk{\s} und der bunten Fenster spiegelten sich
in den gef"ullten Weihwasserbecken. Da{\s} durch die
Gla{\s}malerei einfallende Licht brach sich an den marmornen
Kanten und breitete bunte Teppichst"ucke "uber die Fliesen. Durch
die drei ge"offneten T"uren de{\s} Hauptportal{\s} flutete da{\s}
Tage{\s}licht in drei m"achtigen Lichtstr"omen in die Innenr"aume.
Dann und wann ging ein Sakristan hinten am Hochaltar vor"uber und
machte vor dem Heiligtum die "ubliche Kniebeugung der eiligen
Frommen. Die kristallenen Kronleuchter hingen unbeweglich herab.
Im Chor brannte eine silberne Lampe. Au{\s} den Seitenkapellen,
au{\s} den in Dunkel geh"ullten Teilen der Kirche vernahm man
zuweilen Schluchzen oder da{\s} Klirren einer zugeschlagenen
Gittert"ur, Ger"ausche, die in den hohen Gew"olben widerhallten.

Leo ging gemessenen Schritte{\s} hin. Niemal{\s} war ihm da{\s}
Leben so sch"on erschienen. Nun mu"ste sie bald kommen, reizend,
erregt und stolz auf die Blicke, die ihr folgten, in ihrem
volantbesetzten Kleid, mit ihrem goldnen Lorgnon, ihren zierlichen
Stiefeletten, in all der Eleganz, die er noch nie gekostet hatte,
und all dem unbeschreiblich Verf"uhrerischen einer unterliegenden
Tugend. Und um sie die Kirche, gleichsam ein ungeheure{\s}
Boudoir. Die Pfeiler neigten sich, um die im Dunkel gefl"usterte
Beichte ihrer Liebe entgegenzunehmen. Die farbigen Fenster
leuchteten, ihr sch"one{\s} Gesicht zu verkl"aren, und au{\s} den
Weihrauchgef"a"sen wirbelten die D"ampfe, damit sie wie ein Engel
in einer Wolke von Wohlger"uchen erscheine.

Aber sie kam nicht. Er setzte sich in einen der hohen St"uhle, und
seine Blicke fielen auf ein blaue{\s} Fenster, auf da{\s} Fischer
mit K"orben gemalt waren. Er betrachtete da{\s} Bild aufmerksam,
z"ahlte die Schuppen der Fische und die Knopfl"ocher an den
W"amsen, w"ahrend seine Gedanken auf der Suche nach Emma in die
Weite irrten~...

Der Schweizer "argerte sich im stillen "uber den Menschen, der
sich erlaubte, die Kathedrale allein zu bewundern. Er fand sein
Benehmen unerh"ort. Man bestahl ihn gewisserma"sen und beging
geradezu eine Tempelsch"andung.

Da raschelte Seide "uber die Fliesen. Der Rand eine{\s} Hute{\s}
tauchte auf, eine schwarze Mantille. Sie war e{\s}. Leo eilte ihr
entgegen.

Sie war bla"s und kam mit schnellen Schritten auf ihn zu.

"`Lesen Sie da{\s}!"' sagte sie und hielt ihm ein Briefchen hin.
"`Nicht doch!"'

Sie ri"s ihre Hand au{\s} der seinen und eilte nach der Kapelle
der Madonna, wo sie in einem Betstuhle zum Gebet niederkniete.

Leo war "uber diesen Anfall von Bigotterie zuerst emp"ort, dann
fand er einen eigent"umlichen Reiz darin, sie w"ahrend eine{\s}
Stelldichein{\s} in Gebete vertieft zu sehen wie eine andalusische
Marquise, schlie"slich aber, al{\s} sie gar nicht aufh"oren
wollte, langweilte er sich.

Emma betete, oder vielmehr sie zwang sich zum Beten in der
Hoffnung, da"s der Himmel sie mit einer pl"otzlichen Eingebung
begnaden w"urde. Um diese Hilfe de{\s} Himmel{\s} herabzuschw"oren,
starrte sie auf den Glanz de{\s} Tabernakel{\s}, atmete sie den
Duft der wei"sen Blumen in den gro"sen Vasen, lauschte sie auf die
tiefe Stille der Kirche, die ihre innere Aufregung nur noch
steigerte.

Sie erhob sich und wandte sich dem Au{\s}gang zu. Da trat der
Schweizer rasch auf sie zu:

"`Gn"adige Frau sind gewi"s hier fremd? Wollen Sie sich die
Sehen{\s}w"urdigkeiten der Kirche ansehen?"'

"`Aber nein!"' rief der Adjunkt au{\s}.

"`Warum nicht?"' erwiderte sie. Ihre wankende Tugend klammerte
sich an die Madonna, an die Bilds"aulen, die Grabm"aler, an jeden
Vorwand.

Programmgem"a"s f"uhrte sie der Schweizer nach dem Hauptportal
zur"uck und zeigte ihnen mit seinem Stock einen gro"sen Krei{\s}
von schwarzen Steinchen ohne irgendwelche Beigabe noch Inschrift.

"`Da{\s} hier"', sagte er salbung{\s}voll, "`ist der Umfang der
ber"uhmten Glocke de{\s} Amboise. Sie wog vierzigtausend Pfund und
hatte ihre{\s}gleichen nicht in Europa. Der Meister, der sie
gegossen, ist vor Freude gestorben~..."'

"`Weiter!"' dr"angte Leo.

Der Biedermann setzte sich in Bewegung. Vor der Kapelle der
Madonna blieb er stehen, machte eine Schulmeisterbewegung mit dem
Arm und wie{\s} mit dem Stolze eine{\s} Landmanne{\s}, der seine
Saaten zeigt, auf eine Grabplatte.

"`Hier unter diesem sichren Stein ruht Peter von Br\'ez\'e, Edler
Herr von Varenne und Brissac, Gro"sseneschall von Poitou und
Verweser der Normandie, gefallen in der Schlacht bei Montlh\'ery
am 16. Juli 1465."'

Leo bi"s sich in die Lippen und trat vor Ungeduld von einem Fu"se
auf den andern.

"`Und hier recht{\s}, dieser Ritter im Harnisch auf dem steigenden
Rosse, ist sein Enkel Ludwig von Br\'ez\'e, Edler Herr von Breval
und Montchauvet, Graf von Maulevrier, Baron von Mauny, Kammerherr
de{\s} K"onig{\s}, Orden{\s}ritter und ebenfall{\s} Verweser der
Normandie, gestorben am 23. Juli 1531, an einem Sonntag, wie die
Inschrift besagt. Und dieser Mann hier unten, der eben in{\s} Grab
steigen will, zeigt ihn ebenfall{\s}. Eine un"ubertreffliche
Darstellung der irdischen Verg"anglichkeit!"'

Frau Bovary nahm ihr Lorgnon. Leo stand unbeweglich dabei und sah
sie an. Er wagte weder ein Wort zu sprechen noch eine Geste zu
machen. So sehr entmutigte ihn da{\s} langweilige Geschw"atz auf
der einen und die Gleichg"ultigkeit auf der andern Seite.

Der unerm"udliche Cicerone fuhr fort:

"`Hier diese Frau, die weinend neben ihm kniet, ist seine Gemahlin
Diana von Poitier{\s}, Gr"afin von Br\'ez\'e, Herzogin von
Valentinoi{\s}, geboren 1499, gestorben Anno 1566. Und hier
link{\s} die weibliche Gestalt mit dem Kind auf dem Arm ist die
heilige Jungfrau. Jetzt bitte ich die Herrschaften hierher zu
sehen. Hier sind die Grabm"aler derer von Amboise! Sie waren beide
Kardin"ale und Erzbisch"ofe von Rouen. Dieser hier war Minister
K"onig Ludwig{\s} de{\s} Zw"olften. Die Kathedrale hat ihm sehr
viel zu verdanken. In seinem Testament vermachte er den Armen
drei"sigtausend Taler in Gold."'

Ohne stehen zu bleiben und fortw"ahrend redend, dr"angte er die
beiden in eine Kapelle, die durch ein Gel"ander abgesperrt war. Er
"offnete e{\s} und zeigte auf einen Stein in der Mauer, der einmal
eine schlechte Statue gewesen sein konnte.

"`Dieser Stein zierte dereinst"', sagte er mit einem tiefen
Seufzer, "`da{\s} Grab von Richard L"owenherz, K"onig von England
und Herzog von der Normandie. Die Kalvinisten haben ihn so
zugerichtet, meine Herrschaften. Sie haben ihn au{\s} Bo{\s}heit
hier eingesetzt. Hier sehen Sie auch die T"ur, durch die sich
Seine Eminenz in die Wohnung begibt. Jetzt kommen wir zu den
ber"uhmten Kirchenfenstern von Lagargouille!"'

Da dr"uckte ihm Leo hastig ein gro"se{\s} Silberst"uck in die Hand
und nahm Emma{\s} Arm. Der Schweizer war ganz verbl"ufft "uber die
Freigebigkeit de{\s} Fremden, der noch lange nicht alle
Sehen{\s}w"urdigkeiten gesehen hatte. Er rief ihm nach:

"`Meine Herrschaften, der Turm, der Turm!"'

"`Danke!"' erwiderte Leo.

"`Er ist wirklich sehen{\s}wert, meine Herrschaften! Er mi{\ss}t
vierhundertvierzig Fu"s, nur neun weniger al{\s} die gr"o"ste
"agyptische Pyramide, und ist vollst"andig au{\s} Eisen~..."'

Leo eilte weiter. Seine Liebe war seit zwei Stunden stumm wie die
Steine der Kathedrale. Er hatte keine Lust, sie nun auch noch
durch den grote{\s}ken k"afigartigen Schornstein zw"angen zu
lassen, den ein "uberspannter Eisengie"ser keck auf die Kirche
gesetzt hatte. Da{\s} w"are ihr Tod gewesen.

"`Wohin gehen wir nun?"' fragte Emma.

Ohne zu antworten, lief er rasch weiter, und Frau Bovary tauchte
schon ihren Finger in da{\s} Weihwasserbecken am Au{\s}gang,
al{\s} sie pl"otzlich hinter sich ein Schnaufen und da{\s}
regelm"a"sige Aufklopfen eine{\s} Stocke{\s} h"orten. Leo wandte
sich um.

"`Meine Herrschaften!"'

"`Wa{\s} gibt{\s}?"'

E{\s} war wieder der Schweizer, der ein paar Dutzend dicke
ungebundene B"ucher, mit seinem linken Arme gegen den Bauch
gedr"uckt, trug. E{\s} war die Literatur "uber die Kathedrale.

"`Troddel!"' murmelte Leo und st"urzte au{\s} der Kirche.

Ein Junge spielte auf dem Vorplatz.

"`Hol un{\s} eine Droschke!"'

Der Knabe rannte "uber den Platz, w"ahrend sie ein paar Minuten
allein dastanden. Sie sahen einander an und waren ein wenig
verlegen.

"`Leo ... wirklich ... ich wei"s nicht ... ob ich darf!"' E{\s}
klang wie Koketterie. In ernstem Tone setzte sie hinzu: "`E{\s}
ist sehr unschicklich, wissen Sie da{\s}?"'

"`Wieso?"' erwiderte der Adjunkt. "`In \so{Pari{\s}} macht man{\s}
so!"'

Diese{\s} eine Wort bestimmte sie wie ein unumst"o"sliche{\s}
Argument. Aber der Wagen kam nicht. Leo f"urchtete schon, sie
k"onne wieder in die Kirche gehen. Endlich erschien die Droschke.

"`Fahren Sie wenigsten{\s} noch an{\s} Nordportal!"' rief ihnen
der Schweizer nach. "`Und sehen Sie sich {\glq}Die
Auferstehung{\grq}, da{\s} {\glq}J"ungste Gericht{\grq}, den
{\glq}K"onig David{\grq} und {\glq}Die Verdammten in der
H"olle{\grq} an!"'

"`Wohin wollen die Herrschaften?"' fragte der Kutscher.

"`Fahren Sie irgendwohin!"' befahl Leo und schob Emma in den Wagen.

Da{\s} schwerf"allige Gef"ahrt setzte sich in Bewegung.

Der Kutscher fuhr durch die Gro"sebr"uckenstra"se, "uber den Platz
der K"unste, den Kai Napoleon hinunter, "uber die Neue Br"ucke und
machte vor dem Denkmal Corneille{\s} Halt.

"`Weiter fahren!"' rief eine Stimme au{\s} dem Inneren.

Der Wagen fuhr weiter, rasselte den Abhang zum Lafayette-Platz
hinunter und bog dann schneller werdend nach dem Bahnhof ab.

"`Nein, geradeau{\s}!"' rief dieselbe Stimme.

Der Wagen machte kehrt und fuhr nun, auf dem Ring angelangt, in
gem"achlichem Trabe zwischen den alten Ulmen hin. Der Kutscher
trocknete sich den Schwei"s von der Stirn, nahm seinen Lederhut
zwischen die Beine und lenkte sein Gef"ahrt durch eine Seitenallee
dem Seine-Ufer zu, bi{\s} an die Wiesen. Dann fuhr er den
Schifferweg hin, am Strom entlang, "uber schlechte{\s} Pflaster,
nach Oyssel zu, "uber die Inseln hinau{\s}.

Auf einmal fuhr er wieder flotter, durch Quatremare{\s},
Sotteville, die gro"se Chaussee hin, durch die Elbeuferstra"se und
machte zum drittenmal Halt vor dem Botanischen Garten.

"`So fahren Sie doch weiter!"' rief die Stimme, die{\s}mal
w"utend. Alsobald nahm der Wagen seine Fahrt wieder auf, fuhr
durch Sankt Sever "uber da{\s} Bleicher-Ufer und M"uhlstein-Ufer,
wiederum "uber die Br"ucke, "uber den Exerzierplatz, hinten um den
Spitalgarten herum, wo Greise in schwarzen Kitteln auf der von
Schlingpflanzen "uberwachsenen Terrasse in der Sonne spazieren
gingen. Dann f"uhrte die Fahrt zum Boulevard Bouvreuil hinauf,
nach dem Causer Boulevard und dann den ganzen Riboudet-Berg hinan
bi{\s} zur Deviller H"ohe.

Wiederum ward kehrt gemacht, und nun begann eine Kreuz- und
Querfahrt ohne Ziel und Plan durch die Stra"sen und Gassen, "uber
die Pl"atze und M"arkte, an den Kirchen und "offentlichen
Geb"auden und am Hauptfriedhof vor"uber.

Hin und wieder warf der Kutscher einen verzweifelten Blick vom
Bock herab nach den Kneipen. Er begriff nicht, welche
Bewegung{\s}wut in seinen Fahrg"asten steckte, so da"s sie
nirgend{\s} Halt machen wollten. Er versuchte e{\s} ein paarmal,
aber jede{\s}mal erhob sich hinter ihm ein zorniger Ruf. Von neuem
trieb er seine warmgewordenen Pferde an und fuhr wieder weiter,
unbek"ummert, ob er hier und dort anrannte, ganz au"ser Fassung
und dem Weinen nahe vor Durst, Erschlaffung und Traurigkeit.

Am Hafen, zwischen den Karren und F"assern, in den Strassen und an
den Ecken machten die B"urger gro"se Augen ob diese{\s} in der
Provinz ungewohnten Anblick{\s}: ein Wagen mir herabgelassenen
Vorh"angen, der immer wieder auftauchte, bald da, bald dort, immer
verschlossen wie ein Grab.

Einmal nur, im Freien, um die Mittag{\s}stunde, al{\s} die Sonne
am hei"sesten auf die alten versilberten Laternen brannte, langte
eine blo"se Hand unter den gelben Fenstervorhang herau{\s} und
streute eine Menge Papierschnitzel hinau{\s}, die im Winde
flatterten wie wei"se Schmetterlinge und auf ein Kleefeld
niederfielen.

Gegen sech{\s} Uhr abend{\s} hielt die Droschke in einem G"a"schen
der Vorstadt Beauvoisine. Eine dichtverschleierte Dame stieg
herau{\s} und ging, ohne sich umzusehen, weiter.


\newpage\begin{center}
{\large \so{Zweite{\s} Kapitel}}\bigskip\bigskip
\end{center}

Wieder im Gasthofe, war Frau Bovary sehr erstaunt, die Post nicht
mehr vorzufinden. Hivert hatte dreiundf"unfzig Minuten auf Emma
gewartet, schlie"slich aber war er abgefahren.

E{\s} war zwar nicht unbedingt erforderlich, da"s sie wieder zu
Hause sein mu"ste. Aber sie hatte versprochen, an diesem Abend
zur"uckzukehren. Karl erwartete sie also, und so f"uhlte sie jene
feige Untert"anigkeit im Herzen, die f"ur viele Frauen die Strafe
und zugleich der Prei{\s} f"ur den Ehebruch ist.

Sie packte schnell ihren Koffer, bezahlte die Rechnung und nahm
einen der zweir"adrigen Wagen, die im Hofe bereitstanden.
Unterweg{\s} trieb sie den Kutscher zu gr"o"ster Eile an, fragte
aller Augenblicke nach der Zeit und nach der zur"uckgelegten
Kilometerzahl und holte die Post endlich bei den ersten H"ausern
von Quincampoix ein.

Kaum sa"s sie drin, so schlo"s sie auch schon die Augen. Al{\s}
sie erwachte, waren sie schon "uber den Berg, und von weitem sah
sie Felicie, die vor dem Hause de{\s} Schmiede{\s} auf sie
wartete. Hivert hielt seine Pferde an, und da{\s} M"adchen, da{\s}
sich bi{\s} zum Fenster hinaufreckte, fl"usterte ihr
geheimni{\s}voll zu:

"`Gn"adige Frau sollen gleich mal zu Herrn Apotheker kommen! E{\s}
handelt sich um etwa{\s} sehr Dringliche{\s}!"'

Da{\s} Dorf war still wie immer. Vor den H"ausern lagen kleine
dampfende, rosafarbige Haufen. E{\s} war die Zeit de{\s}
Fr"uchteeinmachen{\s}, und jedermann in Yonville bereitete sich am
selben Tag seinen Vorrat. Vor der Apotheke bewunderte man einen
besonder{\s} gro"sen Haufen dieser au{\s}gekochten "Uberreste. Man
sah, da"s hier mit f"ur die Allgemeinheit gesorgt wurde.

Emma trat in die Apotheke. Der gro"se Lehnstuhl war umgeworfen,
und sogar der "`Leuchtturm von Rouen"' lag am Boden zwischen zwei
M"orserkeulen. Sie stie"s die T"ur zur Flur auf und erblickte in
der K"uche -- inmitten von gro"sen braunen Einmachet"opfen voll
abgebeerter Johanni{\s}beeren und Sch"usseln mit geriebenem und
zerst"uckeltem Zucker, zwischen Wagen auf dem Tisch und Kesseln
"uber dem Feuer -- die ganze Familie Homai{\s}, gro"s und klein,
alle in Sch"urzen, die bi{\s} zum Kinn gingen, Gabeln in den
H"anden. Der Apotheker fuchtelte vor Justin herum, der gesenkten
Kopfe{\s} dastand, und schrie ihn eben an:

"`Wer hat dir gehei"sen, wa{\s} au{\s} dem Kapernaum zu holen?"'

"`Wa{\s} ist denn lo{\s}? Wa{\s} gibt{\s}?"' fragte die
Eintretende.

"`Wa{\s} lo{\s} ist?"' antwortete der Apotheker. "`Ich mache hier
Johanni{\s}beeren ein. Sie fangen an zu sieden, aber weil der Saft
zu dick ist, droht er mir "uberzukochen. Ich schicke nach einem
andern Kessel. Da geht dieser Mensch au{\s} Bequemlichkeit, au{\s}
Faulheit hin und nimmt au{\s} meinem Laboratorium den dort an
einem Nagel aufgeh"angten Schl"ussel zu meinem Kapernaum!"'

Kapernaum nannte er n"amlich eine Bodenkammer, in der er allerlei
Apparate und Material zu seinen Mixturen aufbewahrte. Oft
hantierte er da drinnen stundenlang ganz allein, mischte, klebte
und packte. Diese{\s} kleine Gemach betrachtete er nicht al{\s}
einen gew"ohnlichen Vorrat{\s}raum, sondern al{\s} ein wahre{\s}
Heiligtum, au{\s} dem, von seiner Hand hergestellt, alle die
verschiedenen Sorten von Pillen, Pasten, S"aften, Salben und
Arzneien hervorgingen, die ihn in der ganzen Gegend ber"uhmt
machten. Niemand durfte da{\s} Kapernaum betreten. Da{\s} ging
soweit, da"s er e{\s} selbst au{\s}fegte. Die Apotheke stand f"ur
jedermann offen. Sie war die St"atte, wo er w"urdevoll amtierte.
Aber da{\s} Kapernaum war der Zuflucht{\s}ort, wo sich Homai{\s}
selbst geh"orte, wo er sich seinen Liebhabereien und Experimenten
hingab. Justin{\s} Leichtsinn d"unkte ihn de{\s}halb eine
unerh"orte Respektlosigkeit, und r"oter al{\s} seine
Johanni{\s}beeren, wetterte er:

"`Nat"urlich! Au{\s}gerechnet in mein Kapernaum! Sich einfach den
Schl"ussel nehmen zu meinen Chemikalien! Und gar meinen
Reservekessel, den ich selber vielleicht niemal{\s} in Gebrauch
genommen h"atte! Meinen Deckelkessel! In unsrer peniblen Kunst hat
auch der geringste Umstand die gr"o"ste Wichtigkeit! Zum Teufel,
daran mu"s man immer denken! Man kann pharmazeutische Apparate
nicht zu K"uchenzwecken verwenden! Da{\s} w"are gradeso, al{\s} wenn
man sich mit einer Sense rasieren wollte oder al{\s} wenn~..."'

"`Aber so beruhige dich doch!"' mahnte Frau Homai{\s}.

Und Athalia zupfte ihn am Rock.

"`Papachen, Papachen!"'

"`La"st mich!"' erwiderte der Apotheker. "`Zum Donnerwetter, la"st
mich! Dann wollen wir doch lieber gleich einen Kramladen er"offnen!
Meinetwegen! Immer zu! Zerschlag und zerbrich alle{\s}! La"s die
Blutegel entwischen! Verbrenn den ganzen Krempel! Mach saure
Gurken in den Arzneib"uchsen ein! Zerrei"s die Bandagen!"'

"`Sie hatten mir doch~..."', begann Emma.

"`Einen Augenblick! -- Wei"st du, mein Junge, wa{\s} dir h"atte
passieren k"onnen? Hast du link{\s} in der Ecke auf dem dritten
Wandbrett nicht{\s} stehn sehn? Sprich! Antworte! Gib mal einen
Ton von dir!"'

"`Ich ... wei"s ... nicht"', stammelte der Lehrling.

"`Ah, du wei"st nicht! Freilich! Aber ich wei"s e{\s}! Du hast da
eine B"uchse gesehn, au{\s} blauem Gla{\s}, mit einem gelben
Deckel, gef"ullt mit wei"sem Pulver, und auf dem Schild steht, von
mir eigenh"andig draufgeschrieben: {\glq}Gift! Gift! Gift!{\grq}
Und wei"st du, wa{\s} da drin ist? Ar -- se -- nik! Und so wa{\s}
r"uhrst du an? Nimmst einen Kessel, der daneben steht!"'

"`Daneben!"' rief Frau Homai{\s} erschrocken und schlug die H"ande
"uber dem Kopfe zusammen. "`Arsenik! Du h"attest un{\s} alle
miteinander vergiften k"onnen!"'

Die Kinder fingen an zu schreien, al{\s} sp"urten sie bereit{\s}
die schrecklichsten Schmerzen in den Eingeweiden.

"`Oder du h"attest einen Kranken vergiften k"onnen"', fuhr der
Apotheker fort. "`Wolltest du mich gar auf die Anklagebank
bringen, vor da{\s} Schwurgericht? Wolltest du mich auf dem
Schafott sehen? Wei"st du denn nicht, da"s ich mich bei meinen
Arbeiten kolossal in acht nehmen mu"s, trotz meiner gro"sen
Routine darin? Oft wird mir selber angst, wenn ich an meine
Verantwortung denke. Denn die Regierung sieht un{\s} t"uchtig auf
die Finger, und die albernen Gesetze, denen wir unterstehen,
schweben unsereinem faktisch wie ein Damokle{\s}schwert
fortw"ahrend "uber dem Haupte!"'

Emma machte gar keinen Versuch mehr, zu fragen, wa{\s} man von ihr
wolle, denn der Apotheker fuhr in atemlosen S"atzen fort:

"`So vergiltst du also die Wohltaten, die dir zuteil geworden
sind? So dankst du mir die geradezu v"aterliche M"uhe und
Sorgfalt, die ich an dich verschwendet habe! Wo w"arst du denn
ohne mich? Wie ginge dir{\s} heute? Wer hat dich ern"ahrt,
erzogen, gekleidet? Wer erm"oglicht e{\s} dir, da"s du eine{\s}
Tage{\s} mit Ehren in die Gesellschaft eintreten kannst? Aber um
da{\s} zu erreichen, mu"st du noch feste zugreifen, mu"st, wie man
sagt, Blut schwitzen! \begin{antiqua}Fabricando sit faber, age,
quod agis\end{antiqua}!"'

Er war derma"sen aufgeregt, da"s er Lateinisch sprach. Er h"atte
Chinesisch oder Gr"onl"andisch gesprochen, wenn er da{\s} gekonnt
h"atte. Denn er befand sich in einem Seelenzustand, in dem der
Mensch sein geheimste{\s} Ich ohne Selbstkritik enth"ullt, wie
da{\s} Meer, da{\s} sich im Sturm an seinem Gestade bi{\s} auf den
Grund und Boden "offnet.

Er predigte immer weiter:

"`Ich fange an, e{\s} furchtbar zu bereuen, da"s ich dich in mein
Hau{\s} genommen habe. Ich h"atte besser getan, dich in dem Elend
Und dem Schmutz stecken zu lassen, in dem du geboren bist! Du
wirst niemal{\s} zu etwa{\s} Besserem zu gebrauchen sein al{\s}
zum Rindviehh"uten. Zur Wissenschaft hast du kein bi"schen Talent!
Du kannst kaum eine Etikette aufkleben. Und dabei lebst du bei mir
wie der liebe Gott in Frankreich, wie ein Hahn im Korb, und l"a"st
dir{\s} "uber die Ma"sen wohl gehn!"'

Emma wandte sich an Frau Homai{\s}:

"`Man hat mich hierher gerufen~..."'

"`Ach, du lieber Gott!"' unterbrach die gute Frau sie mit
trauriger Miene. "`Wie soll ich{\s} Ihnen nur beibringen? ...
E{\s} ist n"amlich ein Ungl"uck passiert~..."'

Sie kam nicht zu Ende. Der Apotheker "uberschrie sie:

"`Hier! Leer ihn wieder au{\s}! Mache ihn wieder rein! Bring ihn
wieder an Ort und Stelle! Und zwar fix!"'

Er packte Justin beim Kragen und sch"uttelte ihn ab. Dabei entfiel
Justin{\s} Tasche ein Buch.

Der Junge b"uckte sich, aber Homai{\s} war schneller al{\s} er,
hob den Band auf und betrachtete ihn mit weit aufgerissenen Augen
und offenem Mund.

"`Liebe und Ehe"', la{\s} er vor. "`Aha! Gro"sartig! Gro"sartig!
Wirklich nett! Mit Abbildungen! ... Da{\s} ist denn doch ein
bi"schen starker Tobak!"'

Frau Homai{\s} wollte nach dem Buche greifen.

"`Nein, da{\s} ist nicht{\s} f"ur dich!"' wehrte er sie ab.

Die Kinder wollten die Bilder sehn.

"`Geht hinau{\s}!"' befahl er gebieterisch.

Und sie gingen hinau{\s}.

Eine Weile schritt er zun"achst mit gro"sen Schritten auf und ab,
da{\s} Buch halb ge"offnet in der Hand, mit rollenden Augen, ganz
au"ser Atem, mit rotem Kopfe, al{\s} ob ihn der Schlag r"uhren
sollte. Dann ging er auf den Lehrling lo{\s} und stellte sich mit
verschr"ankten Armen vor ihn hin:

"`Bist du denn mit allen Lastern behaftet, du Ungl"uck{\s}wurm?
Nimm dich in acht, sag ich dir, du bist auf einer schiefen Ebene!
Hast du denn nicht bedacht, da"s diese{\s} sch"andliche Buch
meinen Kindern in die H"ande fallen konnte, den Samen der S"unde
in ihre Sinne streuen, die Unschuld Athalien{\s} tr"uben und
Napoleon verderben? Er ist kein Kind mehr! Kannst du wenigsten{\s}
beschw"oren, da"s die beiden nicht darin gelesen haben? Kannst du
mir da{\s} schw"oren?"'

"`Aber so sagen Sie mir doch endlich,"' unterbrach ihn Emma,
"`wa{\s} Sie mir mit\/zuteilen haben!"'

"`Ach so, Frau Bovary: Ihr Herr Schwiegervater ist gestorben!"'

In der Tat war der alte Bovary vor zwei Tagen just nach Tisch an
einem Schlaganfall verschieden. Au{\s} "ubertriebener
R"ucksichtnahme hatte Karl den Apotheker gebeten, seiner Frau die
schreckliche Nachricht schonend mit\/zuteilen.

Homai{\s} hatte sich die Worte, die er sagen wollte, genauesten{\s}
"uberlegt und au{\s}gekl"ugelt -- ein Meisterwerk voll Vorsicht,
Zartgef"uhl und feiner Wendungen. Aber der Zorn hatte "uber seine
Sprachkunst triumphiert.

Emma verzichtete auf Einzelheiten und verlie"s die Apotheke, da
Homai{\s} seine Strafpredigt wieder aufgenommen hatte, w"ahrend er
sich mit seinem K"appchen Luft zuf"achelte. Allm"ahlich beruhigte
er sich jedoch und ging in einen v"aterlicheren Ton "uber:

"`Ich will nicht sagen, da"s ich diese{\s} Buch g"anzlich ablehne.
Der Verfasser ist Arzt, und e{\s} stehen wissenschaftliche
Tatsachen darin, mit denen sich ein Mann vertraut machen darf, ja
die er vielleicht kennen mu"s. Aber da{\s} hat ja Zeit! Warte doch
wenigsten{\s}, bi{\s} du ein wirklicher Mann bist!"'

Al{\s} Emma an ihrem Hause klingelte, "offnete Karl, der sie
erwartet hatte, und ging ihr mit offenen Armen entgegen.

"`Meine liebe Emma!"'

Er neigte sich z"artlich zu ihr hernieder, um sie zu k"ussen. Aber
bei der Ber"uhrung ihrer Lippen mu"ste sie an den andern denken.
Da fuhr sie zusammenschaudernd mit der Hand "uber da{\s} Gesicht:

"`Ja ... ich wei"s ... ich wei"s~..."'

Er zeigte ihr den Brief, worin ihm seine Mutter da{\s} Ereigni{\s}
ohne jedwede sentimentale Heuchelei berichtete. Sie bedauerte nur,
da"s ihr Mann ohne den Segen der Kirche gestorben war. Der Tod
hatte ihn in Doudeville auf der Stra"se, an der Schwelle eine{\s}
Restaurant{\s}, getroffen, wo er mit ein paar Offizieren a.D. an
einem Liebe{\s}mahl teilgenommen hatte.

Emma reichte Karl den Brief zur"uck. Bei Tisch tat sie au{\s}
konventionellem Taktgef"uhl so, al{\s} h"atte sie keinen Appetit.
Al{\s} er ihr aber zuredete, langte sie tapfer zu, w"ahrend Karl
unbeweglich und mit betr"ubter Miene ihr gegen"uber dasa"s.

Hin und wieder hob er den Kopf und sah seine Frau mit einem
traurigen Blick an. Einmal seufzte er:

"`Ich wollt, ich h"atte ihn noch einmal gesehen!"'

Sie blieb stumm. Weil sie sich aber sagte, da"s sie etwa{\s}
entgegnen m"usse, fragte sie:

"`Wie alt war dein Vater eigentlich?"'

"`Achtundf"unfzig!"'

"`So!"'

Da{\s} war alle{\s}.

Eine Viertelstunde sp"ater fing er wieder an:

"`Meine arme Mutter! Wa{\s} soll nun au{\s} ihr werden?"'

Emma machte eine Geb"arde, da"s sie e{\s} nicht wisse.

Da sie so schweigsam war, glaubte Karl, da"s sie sehr betr"ubt
sei, und er zwang sich infolgedessen gleichfall{\s} zum Schweigen,
um ihren r"uhrenden Schmerz nicht noch zu vermehren. Sich
zusammenraffend, fragte er sie:

"`Hast du dich gestern gut am"usiert?"'

"`Ja!"'

Al{\s} der Tisch abgedeckt war, blieb Bovary sitzen und Emma
gleichfall{\s}. Je l"anger sie ihn in dieser monotonen Stimmung
ansah, um so mehr schwand da{\s} Mitleid au{\s} ihrem Herzen
bi{\s} auf den letzten Rest. Karl kam ihr erb"armlich, jammervoll,
wie eine Null vor. Er war wirklich in jeder Beziehung "`ein
trauriger Kerl"'. Wie konnte sie ihn nur lo{\s}werden? Welch
endloser Abend! Etwa{\s} Bet"aubende{\s} ergriff sie, wie Opium.

In der Hau{\s}flur ward ein schl"urfende{\s} Ger"ausch vernehmbar.
E{\s} war Hippolyt, der Emma{\s} Gep"ack brachte. E{\s} machte ihm
viel M"uhe, e{\s} abzulegen.

"`Karl denkt schon gar nicht mehr daran"', dachte Emma, al{\s} sie
den armen Teufel sah, dem da{\s} rote Haar in die
schwei"striefende Stirn herabhing.

Bovary zog einen Groschen au{\s} der Westentasche. Er hatte kein
Gef"uhl f"ur die Dem"utigung, die f"ur ihn in der blo"sen
Anwesenheit diese{\s} Kr"uppel{\s} lag. Lief er nicht wie ein
leibhaftiger Vorwurf der heillosen Unf"ahigkeit de{\s} Arzte{\s}
herum?

"`Ein h"ubscher Strau"s!"' sagte er, al{\s} er auf dem Kamin
Leo{\s} Veilchen bemerkte.

"`Ja!"' erwiderte sie gleichg"ultig. "`Ich habe ihn einer armen
Frau abgekauft."'

Karl nahm die Veilchen und hielt sie wie zur K"uhlung vor seine
von Tr"anen ger"oteten Augen und sog ihren Duft ein. Sie ri"s sie
ihm au{\s} der Hand und stellte sie in ein Wassergla{\s}.

Am andern Morgen traf die alte Frau Bovary ein. Sie und ihr Sohn
weinten lange. Emma verschwand unter dem Vorwand, sie habe in der
Wirtschaft zu tun.

Am Tage nachher besch"aftigten sich die beiden Frauen mit den
Trauerkleidern. Sie setzten sich mit ihrem N"ahzeug in die Laube
hinten im Garten am Bachrande.

Karl dachte an seinen Vater und wunderte sich "uber seine gro"se
Liebe zu diesem Mann, die ihm bi{\s} dahin gar nicht weiter zum
Bewu"stsein gekommen war. Auch Frau Bovary gr"ubelte "uber den
Toten nach. Jetzt fand sie die schlimmen Tage von einst
begehren{\s}wert. Ihr Joch war ihr so zur alten Gewohnheit
geworden, da"s sie nun Sehnsucht darnach empfand. Ab und zu rann
eine dicke Tr"ane "uber ihre Nase und blieb einen Augenblick daran
h"angen. Dabei n"ahte sie ununterbrochen weiter.

Emma dachte, da"s kaum achtundvierzig Stunden vor"uber waren, seit
sie und der Geliebte zusammengewesen waren, weltentr"uckt, ganz
trunken und nimmer satt, einander zu sehen. Sie versuchte sich die
kleinsten und allerkleinsten Z"uge diese{\s} entschwundenen
Tage{\s} in{\s} Ged"achtni{\s} zur"uckzurufen. Aber die Anwesenheit
ihre{\s} Manne{\s} und ihrer Schwiegermutter st"orte sie. Sie
h"atte nicht{\s} h"oren und nicht{\s} sehn m"ogen, um nicht in
ihren Liebestr"aumereien gest"ort zu werden, die gegen ihren
Willen unter den "au"seren Eindr"ucken zu verwehen drohten.

Sie trennte da{\s} Futter eine{\s} Kleide{\s} ab, da{\s} sie um
sich au{\s}gebreitet hatte. Die alte Frau Bovary handhabte Schere
und Nadel, ohne die Augen zu erheben. Karl stand, beide H"ande in
den Taschen, in seinen Tuchpantoffeln und seinem alten braunen
"Uberrock, der ihm al{\s} Hau{\s}anzug diente, bei ihnen und
sprach auch kein Wort. Berta, die ein wei"se{\s} Sch"urzchen
umhatte, spielte mit ihrer Schaufel im Sande.

Pl"otzlich sahen sie Lheureux, den Modewarenh"andler, kommen.

Er bot in Anbetracht de{\s} "`betr"ublichen Ereignisse{\s}"' seine
Dienste an. Emma erwiderte, sie glaube darauf verzichten zu
k"onnen, aber der H"andler wich nicht so leicht.

"`Ich bitte tausendmal um Verzeihung,"' sagte er, "`aber ich mu"s
Herrn Doktor um eine private Unterredung bitten."' Und fl"usternd
f"ugte er hinzu: "`E{\s} ist wegen dieser Sache ... Sie wissen
schon~..."'

Karl wurde rot bi{\s} "uber die Ohren.

"`Gewi"s ... freilich ... nat"urlich!"'

In seiner Verwirrung wandte er sich an seine Frau:

"`K"onntest du da{\s} nicht mal ... meine Liebe~...?"'

Sie verstand ihn offenbar und erhob sich. Karl sagte zu seiner
Mutter:

"`E{\s} ist nicht{\s} weiter! Wahrscheinlich irgend eine
Kleinigkeit, die den Hau{\s}halt betrifft."'

Er f"urchtete ihre Vorw"urfe und wollte nicht, da"s sie die
Vorgeschichte de{\s} Wechsel{\s} erf"uhre.

Sobald sie allein waren, begl"uckw"unschte Lheureux Emma in
ziemlich eindeutigen Worten zur Erbschaft und schwatzte dann von
gleichg"ultigen Dingen, vom Spalierobst, von der Ernte und von
seiner Gesundheit, die immer "`so lala"' sei. Er m"u"ste sich
wirklich h"ollisch anstrengen und, wa{\s} die Leute auch sagten,
ihm fehle doch die Butter zum Brote.

Emma lie"s ihn reden. Seit zwei Tagen langweilte sie sich
entsetzlich.

"`Und sind Sie v"ollig wiederhergestellt?"' fuhr er fort. "`Ich
sag Ihnen, ich habe Ihren armen Mann in einer sch"onen Verfassung
gesehn! Ja, ja, er ist ein guter Mensch, wenn wir un{\s} auch
ordentlich einander in die Haare gefahren sind."'

Sie fragte, wa{\s} da{\s} gewesen sei. Karl hatte ihr n"amlich die
Streitigkeit wegen der gelieferten Waren verschwiegen.

"`Aber Sie wissen doch! E{\s} handelte sich um Ihre Sachen zur
Reise~..."'

Er hatte den Hut tief in die Stirn hereingezogen, die H"ande auf
den R"ucken genommen und sah ihr, l"achelnd und leise redend, mit
einem unertr"aglichen Blick in{\s} Gesicht. Vermutete er etwa{\s}?
Emma verlor sich in allerlei Bef"urchtungen. Inzwischen fuhr er
fort:

"`Aber wir haben un{\s} schlie"slich geeinigt, und ich bin
gekommen, ihm ein Arrangement vorzuschlagen~..."'

E{\s} handelte sich darum, den Wechsel, den Bovary au{\s}gestellt
hatte, zu erneuern. "Ubrigen{\s} k"onne der Herr Doktor die Sache
ganz nach seinem Belieben regeln; er brauche sich gar nicht zu
"angstigen, noch dazu jetzt, wo er gewi"s mit Sorgen "uberh"auft
sei.

"`Da{\s} beste w"are ja, wenn die Schuld jemand ander{\s}
"ubern"ahme. Sie zum Beispiel. Durch eine Generalvollmacht. Da{\s}
w"are da{\s} Bequemste. Wir k"onnten dann unsere kleinen
Gesch"afte miteinander abmachen."'

Sie begriff nicht recht, aber er sagte nicht{\s} weiter. Dann kam
er auf sein Gesch"aft zu sprechen und erkl"arte ihr, sie m"usse
unbedingt etwa{\s} nehmen. Er wolle ihr zw"olf Meter Barege
schicken, zu einem neuen schwarzen Kleide.

"`Da{\s}, wa{\s} Sie da haben, ist gut f"ur{\s} Hau{\s}. Sie
brauchen noch noch ein andre{\s} f"ur die Besuche. Gleich beim
Eintreten habe ich da{\s} bemerkt. Ja, ja, ich habe Augen wie ein
Amerikaner!"'

Er schickte den Stoff nicht, sondern brachte ihn selbst. Dann kam
er nochmal{\s}, um Ma"s zu nehmen, und dann unter allen m"oglichen
anderen Vorw"anden wieder und wieder, wobei er sich so gef"allig
und dienstbeflissen wie nur m"oglich stellte. Er stand
"`gehorsamst zur Verf"ugung"', wie Homai{\s} zu sagen pflegte.
Dabei fl"usterte er Emma immer wieder irgendwelche Ratschl"age
wegen der Generalvollmacht zu. Den Wechsel erw"ahnte er nicht
mehr, und Emma dachte auch nicht daran. Karl hatte wohl kurz nach
ihrer Genesung mit ihr dar"uber gesprochen, aber e{\s} war ihr
seitdem so viel durch den Kopf gegangen, da"s sie da{\s} vergessen
hatte. Sie h"utete sich "uberhaupt, Geldinteressen an den Tag zu
legen. Frau Bovary wunderte sich dar"uber, aber sie schrieb da{\s}
der Fr"ommigkeit zu, die zur Zeit der Krankheit in ihr erstanden
sei.

Sobald die alte Frau jedoch abgereist war, setzte Emma ihren
Gatten durch ihren Gesch"aft{\s}sinn in Erstaunen. Man m"usse
Erkundigungen einholen, die Hypotheken pr"ufen und feststellen, ob
nicht vielleicht ein Nachla"skonkur{\s} n"otig sei. Sie gebrauchte
auf gut Gl"uck allerhand juristische Au{\s}dr"ucke, sprach von
Ordnung de{\s} Nachlasse{\s}, Nachla"sverbindlichkeiten, Haftung
usw., und "ubertrieb immerfort die Schwierigkeiten der
Erbschaft{\s}regelung. Eine{\s} Tage{\s} zeigte sie ihm sogar den
Entwurf einer Generalvollmacht, die ihr da{\s} Recht "ubertrug,
da{\s} Verm"ogen zu verwalten, Darlehen aufzunehmen, Wechsel
au{\s}zustellen und zu akzeptieren, jederlei Zahlung zu leisten
und zu empfangen usw.

Lheureux war ihr Lehrmeister.

Karl fragte sie naiv, wer ihr die Urkunde au{\s}gestellt habe.

"`Notar Guillaumin."' Und mit der gr"o"sten Kaltbl"utigkeit f"ugte
sie hinzu: "`Ich habe nur nicht da{\s} rechte Vertrauen zur Sache.
Die Notare stehn in so schlechtem Ruf! Vielleicht m"u"ste man noch
einen Recht{\s}anwalt um Rat fragen. Wir kennen aber nur ... nein
... keinen."'

"`H"ochsten{\s} Leo"', meinte Karl nachdenklich. Aber e{\s} sei
schwierig, sich brieflich zu verst"andigen.

Da erbot sich Emma, die Reise zu machen. Er dankte. Sie bot e{\s}
nochmal{\s} an. Kein{\s} wollte dem andern an Zuvorkommenheit
nachstehen. Schlie"slich rief sie mit gut gespieltem Eigensinn
au{\s}:

"`Ich will aber! Ich bitte dich, la"s mich{\s} machen!"'

"`Wie gut du bist!"' sagte er und k"u"ste sie auf die Stirn.

Am andern Morgen stieg sie in die Post, um nach Rouen zu fahren
und Leo zu konsultieren. Sie blieb drei Tage fort.


\newpage\begin{center}
{\large \so{Dritte{\s} Kapitel}}\bigskip\bigskip
\end{center}

E{\s} waren drei erlebni{\s}volle, k"ostliche, wunderbare wahre
Flitterwochentage.

Die beiden wohnten im Boulogner Hof am Hafen. Dort hausten sie bei
verschlossenen T"uren und herabgelassenen Fensterl"aden, unter
"uberallhin gestreuten Blumen und bei Fruchtei{\s}, da{\s} man
ihnen alle Morgen in der Fr"uhe brachte.

Abend{\s} mieteten sie einen "uberdeckten Kahn und a"sen auf einer
der Inseln.

E{\s} war die Stunde, da man von den Werften her die H"ammer gegen
die Schiff{\s}w"ande schlagen h"orte. Der Dampf von siedendem Teer
stieg zwischen den B"aumen empor, und auf dem Strome sah man
breite "olige, ungleich gro"se Flecken, die im Purpurlichte der
Sonne wie schwimmende Platten au{\s} Florenzer Bronze gl"anzten.

Sie fuhren zwischen den vielen vor Anker liegenden Flu"sk"ahnen
hindurch, und bi{\s}weilen streifte ihre Barke die langen
Ankertaue. Da{\s} Ger"ausch der Stadt, da{\s} Rasseln der Wagen,
da{\s} Stimmengewirr, da{\s} Bellen der Hunde auf den Schiffen
wurde ferner und ferner. Emma kn"upfte ihre Hutb"ander auf.

Sie landeten an "`ihrer Insel"'. Sie setzten sich in eine
Herberge, vor deren T"ur schwarze Netze hingen, und a"sen
gebackene Fische, Omeletten und Kirschen. Dann lagerten sie sich
in{\s} Gra{\s}, k"u"sten einander im Schatten der hohen Pappeln
und h"atten am liebsten wie zwei Robinson{\s} immer auf diesem
Erdenwinkel leben m"ogen, der ihnen in ihrer Gl"uckseligkeit
al{\s} da{\s} sch"onste Fleckchen der ganzen Welt erschien. Sie
sahn die B"aume, den blauen Himmel und da{\s} Gra{\s} nicht zum
ersten Male, sie lauschten nicht zum erstenmal dem Pl"atschern der
Wellen und dem Wind, der durch die Bl"atter rauschte, aber e{\s}
war ihnen, al{\s} h"atten sie da{\s} alle{\s} niemal{\s} so
genossen, al{\s} w"are die Natur vorher gar nicht dagewesen oder
al{\s} w"are sie erst sch"on, seitdem ihr Begehren gestillt war.

Wenn e{\s} dunkel ward, kehrten sie heim. Der Kahn fuhr am Gestade
von Inseln entlang. Die beiden sa"sen im Dunkeln auf der Bank
unter dem h"olzernen Verdeck und sprachen kein Wort. Die
vierkantigen Ruder knirschten durch die Stille in ihren eisernen
Gabeln, taktm"a"sig wie ein Uhrwerk. Hinter ihnen rauschte da{\s}
Wasser leise um da{\s} herrenlose Steuer.

Einmal erschien der Mond. Da schw"armten sie nat"urlich vom
stillen Nebelglanz "uber Busch und Tal und seinen Melodien. Und
Emma begann sogar zu singen: 
\begin{verse}
"`Wei"st du, eine{\s} Abend{\s} \\
Fuhren wir dahin~..."'
\end{verse}

Ihre metallische, aber schwache Stimme verhallte "uber der Flut,
vom Wind entf"uhrt. Wie sanfter Fl"ugelschlag streifte der Sang
Leo{\s} Ohr.

Emma sa"s an die R"uckwand der kleinen Kabine gelehnt. Durch eine
offene Luke im Dache fiel der Mondenschein herein und in ihr
Gesicht. Ihr schwarze{\s} Kleid, dessen faltiger Rock sich wie ein
F"acher au{\s}breitete, lie"s sie schlanker und gr"o"ser
erscheinen. Die H"ande gefaltet, hob sie den Kopf und schaute zum
Himmel empor. Von Zeit zu Zeit verschwand sie im Schatten der
Weiden, an denen der Kahn vor"uberglitt, und dann tauchte sie
pl"otzlich wieder auf, im Lichte de{\s} Monde{\s}, wie eine
Geistererscheinung.

Leo, der sich ihr zu F"u"sen am Boden de{\s} Fahrzeuge{\s}
gelagert hatte, hob ein Band au{\s} roter Seide auf. Der
Boot{\s}mann sah e{\s} und meinte:

"`Da{\s} ist von gestern! Da hab ich eine kleine Gesellschaft
spazierengefahren, lauter lustige Leute, Herren und Damen. Sie
hatten Kuchen und Champagner mit und Waldh"orner. Da{\s} war ein
Rummel! Da war einer dabei, ein gro"ser h"ubscher Mann mit einem
schwarzen Schnurrb"artchen, der war riesig fidel! Sie baten ihn
immer: {\glq}Du, erz"ahl un{\s} mal einen Schwank au{\s} deinem
Leben, Adolf!{\grq} Oder hie"s er Rudolf? Ich wei"s nicht mehr~..."'

Emma fuhr zusammen.

"`Ist dir nicht wohl?"' fragte Leo und legte ihr die Hand um den
Nacken.

"`Ach nein, e{\s} ist nicht{\s}! E{\s} ist ein bi"schen k"uhl."'

"`Er mochte auch viel Gl"uck bei den Frauen haben"', redete der
Boot{\s}mann leise weiter. Er wollte seinem Fahrgaste offenbar
eine Schmeichelei sagen. Dann spuckte er sich in die H"ande und
begann von neuem zu rudern.

Endlich kam die Trennung{\s}stunde. Der Abschied war sehr traurig.
Sie verabredeten, Leo solle durch die Adresse der Frau Rollet
schreiben. Emma gab ihm genaue Anweisungen. Er solle doppelte
Umschl"age verwenden. Er wunderte sich "uber ihre Schlauheit in
Liebe{\s}dingen.

"`Und da{\s} andre ist doch auch alle{\s} in Ordnung, nicht
wahr?"' fragte sie nach dem letzten Kusse.

"`Aber gewi"s!"'

Al{\s} er dann allein durch die Stra"sen heimging, dachte er bei
sich:

"`Warum macht sie denn eigentlich so viel Wesen{\s} mit ihrer
Generalvollmacht?"'


\newpage\begin{center}
{\large \so{Vierte{\s} Kapitel}}\bigskip\bigskip
\end{center}

Leo begann vor seinen Kameraden den "Uberlegenen zu spielen. Er
mied ihre Gesellschaft und vernachl"assigte seine Akten. Er
wartete nur immer auf Emma{\s} Briefe, la{\s} wieder und wieder in
ihnen und schrieb ihr alle Tage. Er verweilte in Gedanken und in
der Erinnerung immerdar voller Sehnsucht bei ihr. Sein hei"se{\s}
Begehren k"uhlte sich durch da{\s} Getrenntsein nicht ab, im
Gegenteil, sein Verlangen, sie wiederzusehen, wuch{\s} derma"sen,
da"s er an einem Sonnabendvormittag seiner Kanzlei entrann.

Al{\s} er von der H"ohe herab unten im Tale den Kirchturm mit
seiner sich im Winde drehenden blechernen Wetterfahne erblickte,
durchschauerte ihn ein sonderbare{\s} Gef"uhl von Eitelkeit und
R"uhrung, wie e{\s} vielleicht ein Milliard"ar empfindet, der sein
Heimatdorf wieder aufsucht.

Er ging um Emma{\s} Hau{\s}. In der K"uche war Licht. Er wartete,
ob nicht ihr Schatten hinter den Gardinen sichtbar w"urde. E{\s}
erschien nicht{\s}.

Al{\s} Mutter Franz ihn gewahrte, stie"s sie Freudenschreie
au{\s}. Sie fand ihn "`gr"o"ser und schlanker geworden"', w"ahrend
Artemisia im Gegensatze dazu meinte, er s"ahe "`st"arker und
brauner"' au{\s}.

Wie einst nahm er seine Mahlzeit in der kleinen Gaststube ein,
aber allein, ohne den Steuereinnehmer. Binet hatte e{\s} n"amlich
"`satt bekommen"', immer auf die Post warten zu sollen, und hatte
seine Tischzeit ein f"ur allemal auf Punkt f"unf Uhr verlegt,
wa{\s} ihn indessen nicht hinderte, dar"uber zu r"asonieren, da"s
der "`alte Klapperkasten egal zu sp"at"' k"ame.

Endlich fa"ste Leo Mut und klingelte an der Hau{\s}t"ure de{\s}
Arzte{\s}. Frau Bovary war in ihrem Zimmer. Erst nach einer
Viertelstunde kam sie herunter. Karl schien sich zu freuen, ihn
wiederzusehen; aber weder am Abend noch andern Tag{\s} wich er von
Emma{\s} Seite. Erst nacht{\s} kam sie allein mit Leo zusammen,
auf dem Wege hinter dem Garten, an der kleinen Treppe zum Bach,
wie einst mit dem andern.

Da ein Gewitterregen niederging, plauderten sie unter einem
Regenschirm, bei Donner und Blitz.

Die Trennung war ihnen unertr"aglich.

"`Lieber sterben!"' sagte Emma.

Sie entwand sich seinen Armen und weinte.

"`Lebwohl! Lebwohl! Wann werd ich dich wiedersehn?"'

Sie wandten sich noch einmal um und umarmten sich von neuem. Da
versprach ihm Emma, sie wolle demn"achst Mittel und Wege finden,
damit sie sich wenigsten{\s} einmal jede Woche sehen k"onnten.
Emma zweifelte nicht an der M"oglichkeit. Sie war "uberhaupt
voller Zuversicht. Lheureux hatte ihr f"ur die n"achste Zeit Geld
in Au{\s}sicht gestellt.

Sie schaffte ein Paar cremefarbige Store{\s} f"ur ihr Zimmer an.
Lheureux r"uhmte ihre Billigkeit. Dann bestellte sie einen
Teppich, den der H"andler bereitwillig zu besorgen versprach,
wobei er versicherte, er werde "`die Welt nicht kosten"'. Lheureux
war ihr unentbehrlich geworden. Zwanzigmal am Tage schickte sie
nach ihm, und immer lie"s er alle{\s} stehen und liegen und kam,
ohne auch nur zu murren. Man begriff ferner nicht, warum die alte
Frau Rollet t"aglich zum Fr"uhst"uck und auch au"serdem noch
h"aufig kam.

Gegen Anfang de{\s} Winter{\s} entwickelte Emma pl"otzlich einen
ungemein regen Eifer im Musizieren.

Eine{\s} Abend{\s} spielte sie da{\s}selbe St"uck viermal
hintereinander, ohne "uber eine bestimmte schwierige Stelle glatt
hinwegzukommen. Karl, der ihr zuh"orte, bemerkte den Fehler nicht
und rief:

"`Bravo! Au{\s}gezeichnet! Fehlerlo{\s}! Spiele nur weiter!"'

"`Nein, nein! Ich st"umpere. Meine Finger sind zu steif
geworden."'

Am andern Tag bat er sie, ihm wieder etwa{\s} vorzuspielen.

"`Meinetwegen! Wenn e{\s} dir Spa"s macht."'

Karl gab zu, da"s sie ein wenig au{\s} der "Ubung sei. Sie griff
daneben, blieb stecken, und pl"otzlich h"orte sie auf zu spielen.

"`Ach, e{\s} geht nicht, ich m"u"ste wieder Stunden nehmen,
aber~..."' Sie bi"s sich in die Lippen und f"ugte hinzu: "`Zwanzig
Franken f"ur die Stunde, da{\s} ist zu teuer."'

"`Allerding{\s} ... ja~..."', sagte Karl und l"achelte einf"altig,
"`aber e{\s} gibt doch auch unbekannte K"unstler, die billiger und
manchmal besser sind al{\s} die Ber"uhmtheiten."'

"`Such mir einen!"' sagte Emma.

Am andern Tag, al{\s} er heimkam, sah er sie mit pfiffiger Miene
an und sagte schlie"slich:

"`Wa{\s} du dir so manchmal in den Kopf setzt! Ich war heute in
Barfeuch\`ere{\s}, und da hat mir Frau Li\'egeard erz"ahlt, da"s
ihre drei T"ochter f"ur zw"olf Groschen die Stunde bei einer ganz
vortrefflichen Lehrerin Klavierunterricht haben."'

Emma zuckte mit den Achseln und "offnete fortan nicht mehr da{\s}
Klavier. Aber wenn sie in Karl{\s} Gegenwart daran vorbeiging,
seufzte sie allemal:

"`Ach, mein arme{\s} Klavier!"'

Wenn Besuch da war, erz"ahlte sie jedermann, da"s sie die Musik
aufgegeben und h"oheren R"ucksichten geopfert habe. Dann beklagte
man sie. E{\s} sei schade. Sie h"atte soviel Talent. Man machte
ihrem Manne geradezu Vorw"urfe, und der Apotheker sagte ihm
eine{\s} Tage{\s}:

"`E{\s} ist nicht recht von Ihnen. Man darf die Gaben, die einem
die Natur verliehen, nicht brachliegen lassen. Au"serdem sparen
Sie, wenn Sie Ihre Frau jetzt Stunden nehmen lassen, sp"ater bei
der musikalischen Erziehung Ihrer Tochter. Ich finde, die M"utter
sollten ihre Kinder immer selbst unterrichten. Da{\s} hat schon
Rousseau gesagt, so neu un{\s} diese Forderung auch anmutet. Aber
da{\s} wird dermaleinst doch Sitte, genau wie die Ern"ahrung der
S"auglinge durch die eigenen M"utter und wie die
Schutzpockenimpfung! Davon bin ich "uberzeugt!"'

Infolgedessen kam Karl noch einmal gespr"ach{\s}weise auf diese
Angelegenheit zur"uck. Emma erwiderte "argerlich, da"s e{\s}
besser w"are, da{\s} Instrument zu verkaufen. Dagegen verwahrte
sich Bovary. Da{\s} kam ihm wie die Prei{\s}gabe eine{\s}
St"ucke{\s} von sich selbst vor. Da{\s} brave Klavier hatte ihm so
oft Vergn"ugen bereitet und ihn einst so stolz und eitel gemacht!

"`Wie w"are e{\s} denn,"' schlug er vor, "`wenn du hin und wieder
eine Stunde n"ahmst? Da{\s} wird un{\s} wohl nicht gleich
ruinieren!"'

"`Unterricht hat nur Zweck, wenn er regelm"a"sig erfolgt"',
entgegnete sie.

Und so kam e{\s} schlie"slich dahin, da"s sie von ihrem Gatten die
Erlaubni{\s} erhielt, jede Woche einmal in die Stadt zu fahren, um
den Geliebten zu besuchen. Schon nach vier Wochen fand man, sie
habe bedeutende Fortschritte gemacht.


\newpage\begin{center}
{\large \so{F"unfte{\s} Kapitel}}\bigskip\bigskip
\end{center}

An jedem Donnerstag stand Emma zeitig auf und zog sich
ger"auschlo{\s} an, um Karl nicht aufzuwecken, der ihr Vorw"urfe
wegen ihre{\s} zu fr"uhen Aufstehen{\s} gemacht h"atte. Dann lief
sie in ihrem Zimmer herum, stellte sich an{\s} Fenster und sah auf
den Marktplatz hinau{\s}. Da{\s} Morgengrauen huschte um die
Pfeiler der Hallen und um die Apotheke, deren Fensterl"aden noch
geschlossen waren. Die gro"sen Buchstaben de{\s} Ladenschilde{\s}
lie"sen sich durch da{\s} fahle D"ammerlicht erkennen.

Wenn die Stutzuhr ein viertel acht Uhr zeigte, ging Emma nach dem
Goldnen L"owen. Artemisia "offnete ihr g"ahnend die T"ur und
fachte der gn"adigen Frau wegen im Herde die gl"uhenden Kohlen an.
Ganz allein sa"s Emma dann in der K"uche.

Von Zeit zu Zeit ging sie hinau{\s}. Hivert spannte h"ochst
gem"achlich die Postkutsche an, wobei er der Witwe Franz zuh"orte,
die in der Nachthaube oben zu ihrem Schlafstubenfenster
herau{\s}sah und ihm tausend Auftr"age und Verhaltung{\s}ma"sregeln
erteilte, die jeden andern Kutscher verr"uckt gemacht h"atten. Die
Abs"atze von Emma{\s} Stiefeletten klapperten laut auf dem
Pflaster de{\s} Hofe{\s}.

Nachdem Hivert seine Morgensuppe eingenommen, sich den Mantel
angezogen, die Tabak{\s}pfeife angez"undet und die Peitsche in die
Hand genommen hatte, kletterte er saumselig auf seinen Bock.

Langsam fuhr die Post endlich ab. Anfang{\s} machte sie
allerort{\s} Halt, um Reisende aufzunehmen, die an der Stra"se vor
den Hoftoren standen und warteten. Leute, die sich Pl"atze
vorbestellt hatten, lie"sen meist auf sich warten; ja e{\s} kam
vor, da"s sie noch in ihren Betten lagen. Dann rief, schrie und
fluchte Hivert, stieg von seinem Sitz herunter und pochte mit den
F"austen laut gegen die Fensterl"aden. Inzwischen pfiff der Wind
durch die schlecht schlie"senden Wagenfenster.

Allm"ahlich f"ullten sich die vier B"anke. Der Wagen rollte jetzt
schneller hin. Die Apfelb"aume an den Stra"senr"andern folgten
sich rascher. Aber zwischen den beiden mit gelblichem Wasser
gef"ullten Gr"aben dehnte sich die Chaussee noch endlo{\s} hin
bi{\s} in den Horizont.

Emma kannte jede Einzelheit de{\s} Wege{\s}. Sie wu"ste genau,
wann eine Wiese oder eine Wegs"aule kam oder eine Ulme, eine
Scheune, da{\s} H"auschen eine{\s} Stra"senw"arter{\s}. Manchmal
schlo"s sie die Augen eine Weile, um sich "uberraschen zu lassen.
Aber sie verlor niemal{\s} da{\s} Gef"uhl f"ur Zeit und Ort.

Endlich erschienen die ersten Backsteinh"auser. Der Boden dr"ohnte
unter den R"adern, recht{\s} und link{\s} lagen G"arten, durch
deren Gitter man Bilds"aulen, Lauben, beschnittene Taxu{\s}hecken
und Schaukeln erblickte. Dann, mit einemmal, tauchte die Stadt auf.

Sie lag vor Emma wie ein Amphitheater in der von leichtem Dunst
erf"ullten Tiefe. Jenseit{\s} der Br"ucken verlief da{\s}
H"ausermeer in undeutlichen Grenzen. Dahinter dehnte sich
flache{\s} Land in eint"onigen Linien, bi{\s} e{\s} weit in der
Ferne im fahlen Grau de{\s} Himmel{\s} verschwamm. So au{\s} der
Vogelschau sah die ganze Landschaft leblo{\s} wie ein Gem"alde
au{\s}. Die vor Anker liegenden Zillen dr"angten sich in einem
Winkel zusammen. Der Strom wand sich im Bogen um gr"une H"ugel,
und die l"anglichen Inseln in seinen Fluten glichen gro"sen
schwarzen, tot daliegenden Fischen. Au{\s} den hohen Fabrikessen
quollen dichte braune Rauchwolken, die sich oben in der Luft
aufl"osten. In da{\s} Dr"ohnen der Dampfh"ammer mischte sich
da{\s} helle Glockengel"aut der Kirchen, die au{\s} dem Dunste
hervorragten. Die bl"atterlosen B"aume auf den Boulevard{\s}
wuchsen au{\s} den H"ausermassen herau{\s} wie violette Gew"achse,
und die vom Regen nassen D"acher glitzerten st"arker oder
schw"acher, je nach der h"oheren oder tieferen Lage der
Stadtteile. Bi{\s}weilen trieb ein frischer Windsto"s da{\s}
dunstige Gew"olk nach der Sankt Katharinen-H"ohe hin, an deren
steilen H"angen sich die luftige Flut ger"auschlo{\s} brach.

Emma empfand jede{\s}mal eine Art Schwindel, wenn sie die Stadt,
diese Ansammlung von Existenzen, so vor sich sah. Da{\s} Blut
st"urmte ihr heftiger durch die Adern, al{\s} ob ihr die
hundertundzwanzigtausend Herzen, die da unten schlugen, den Brodem
der Leidenschaften, die in ihnen lodern mochten, in einem einzigen
Hauche entgegensandten. Vor der Gewalt diese{\s} Anblick{\s}
wuch{\s} ihre eigene Liebe, und da{\s} dumpfe Rauschen de{\s}
Stra"senl"arm{\s}, da{\s} zu ihr heraufdrang, hob ihre Stimmung.
Die Pl"atze, die Stra"sen, die Promenaden erweiterten und
vergr"o"serten sich vor ihr, und die alte Normannenstadt ward ihr
zur Ko{\s}mopoli{\s}, zu einem zweiten Babylon, in da{\s} sie
Einzug hielt.

Sie lehnte sich au{\s} dem Wagenfenster hinau{\s} und sog die
frische Luft ein. Die drei Pferde liefen schneller, die Steine der
schmutzigen Landstra"se knirschten, der Wagen schwankte. Hivert
rief die Fuhrwerke und Karren an, die vor ihm fuhren. Die B"urger,
die au{\s} ihren Landh"ausern im Wilhelm{\s}walde zur"uckkehrten,
wo sie die Nacht "uber geblieben waren, wichen mit ihren
Familienkutschen gem"achlich au{\s}.

Am Eingang der Stadt hielt die Post. Emma entledigte sich ihrer
"Uberschuhe, zog andre Handschuhe an, zupfte ihren Schal zurecht
und stieg au{\s}.

In der Stadt wurde e{\s} lebendig. Die Lehrjungen putzten die
Schaufenster der L"aden. Marktweiber mit K"orben schrien an den
Stra"senecken ihre Waren au{\s}. Emma dr"uckte sich mit
niedergeschlagenen Augen an den H"ausermauern entlang. Unter ihrem
herabgezogenen schwarzen Schleier l"achelte sie vergn"ugt. Um
nicht beobachtet zu werden, machte sie Umwege. Durch d"ustre
Gassen hindurch gelangte sie endlich ganz erhitzt zu dem Brunnen
am Ende der Rue Nationale. Wegen der N"ahe de{\s} Theater{\s} gibt
e{\s} dort die meisten Kneipen. E{\s} wimmelt von Frauenzimmern.
Ein paarmal fuhren Karren mit B"uhnendekorationen an Emma
vor"uber. Besch"urzte Kellner streuten Sand auf da{\s} Trottoir,
zwischen K"asten mit gr"unen Gew"achsen. E{\s} roch nach Absinth,
Zigarren und Austern.

Emma bog in die verabredete Stra"se ein. Da stand Leo. Sie
erkannte ihn schon von weitem an dem welligen Haar, da{\s} sich
unter seinem Hute zeigte. Er ging ruhig weiter. Sie folgte ihm
nach dem Boulogner Hof. Er stieg vor ihr die Treppe hinauf,
"offnete die T"ur und trat ein~...

Eine leidenschaftliche Umarmung! Liebe{\s}worte und K"usse ohne
Ende! Sie erz"ahlten sich vom Leid der vergangenen Woche, von
ihrem Hangen und Bangen, von ihrem Warten auf die Briefe. Aber
dann war da{\s} alle{\s} vergessen. Sie sahen sich von Auge zu
Auge, unter dem L"acheln der Wollust und unter dem Gefl"uster der
Z"artlichkeit.

Da{\s} Bett war au{\s} Mahagoni und sehr gro"s. Zu beiden Seiten
de{\s} Kopfkissen{\s} hingen rotseidne weitbauschige Vorh"ange
herab. Wenn sich Emma{\s} braune{\s} Haar und ihre wei"se Haut von
diesem Purpurrot abhoben, wenn sie ihre beiden nackten Arme
versch"amt hob und ihr Gesicht in den H"anden verbarg: wa{\s}
h"atte Leo Sch"onre{\s} schauen k"onnen?

Da{\s} warme Zimmer mit seinem weichen Teppich, seiner netten
Einrichtung und seinem traulichen Lichte war wie geschaffen zu
einer heimlichen Liebe. Wenn die Sonne hereinschien, funkelte
alle{\s}, wa{\s} blank im Gemache war, hell auf: die
Messingbeschl"age an der T"ur, an den Gardinenhaltern und am
Kamin.

Sie liebten diesen Raum, wenn seine Herrlichkeit auch ein wenig
verblichen war. Jede{\s}mal, wenn sie kamen, fanden sie alle{\s}
so vor, wie sie e{\s} verlassen. Mitunter lagen sogar die
Haarnadeln noch auf dem Sockel der Standuhr, wo Emma sie am
Donnerstag vorher liegen gelassen hatte.

Da{\s} Fr"uhst"uck pflegten sie am Kamin an einem kleinen
eingelegten Tisch au{\s} Polisanderholz einzunehmen. Emma machte
alle{\s} zurecht und legte Leo jeden Bissen einzeln auf den
Teller, unter tausend s"u"sen Torheiten. Wenn der Sekt ihr "uber
den Rand de{\s} d"unnen Kelche{\s} auf die Finger perlte, lachte
sie lustig auf. Sie waren beide in den gegenseitigen Genu"s
versunken und verga"sen v"ollig, da"s sie in einer Mietwohnung
hausten. E{\s} war Ihnen, al{\s} w"aren sie Jungverm"ahlte und
h"atten ein gemeinsame{\s} Heim, da{\s} sie nie wieder zu
verlassen brauchten. Sie sagten "`unser Zimmer, unser Teppich,
unsre St"uhle,"' wie sie "`unsre Pantoffeln"' sagten, wobei sie
die meinten, die Leo Emma geschenkt hatte: Pantoffeln au{\s} rosa
Atla{\s} mit Schwanflaumbesatz. Emma trug sie "uber den nackten
F"u"sen. Wenn sie sich Leo auf die Knie setzte, pendelte sie mir
ihren Beinen und balancierte die zierlichen Schuhe mit den gro"sen
Zehen.

Zum ersten Male in seinem Leben geno"s er den unbeschreiblichen
Reiz einer mond"anen Liebschaft. Alle{\s} war ihm neu: diese
ent\/z"uckende Art zu plaudern, diese{\s} versch"amte Sichentbl"o"sen,
diese{\s} schmachtende Girren. Er bewunderte ihre verz"uckte
Sinnlichkeit und zugleich die Spitzen ihre{\s} Unterrocke{\s}. Er
hatte eine schicke Dame der Gesellschaft zur Geliebten, eine
verheiratete Frau ... Wa{\s} h"atte er mehr haben wollen?

Durch den fortw"ahrenden Wechsel in ihren Launen, die sie bald
tiefsinnig, bald au{\s}gelassen machten, bald redselig, bald
schweigsam, bald "uberschwenglich, bald blasiert, rief und reizte
Emma in ihm tausend L"uste, Gef"uhle und Reminis\/zenzen. Die
Heldinnen aller Romane, die er je gelesen, aller Dramen, die er je
gesehen, erstanden in ihr wieder. Ihr galten alle Gedichte der
Welt. Ihre Schultern hatten den Bernsteinteint der "`Badenden
Odali{\s}ke"', ihr schlanker Leib gemahnte ihn an die edlen
Vrouwen der Minnes"anger, und ihr blasse{\s} Gesicht glich denen,
die spanische Meister verewigt hatten. Sie war ihm mehr al{\s}
alle{\s} da{\s}: sie war sein "`Engel"'.

Oft, wenn er sie anblickte, war e{\s} ihm, al{\s} erg"osse sich
seine Seele "uber sie und flie"se wie eine Welle "uber ihr Antlitz
und von da herab wie ein Strom auf ihre wei"se Brust. Er sank ihr
zu F"u"sen auf den Teppich, schlang beide Arme um ihre Knie, sah
zu ihr empor und schaute sie l"achelnd an. Und sie neigte sich zu
ihm herab und fl"usterte wie im Rausche:

"`O r"uhr dich nicht! Sprich nicht! Sieh mich an! E{\s} ist
etwa{\s} Liebe{\s}, S"u"se{\s} in deinen Augen, da{\s} ich so gern
habe!"'

Sie nannte ihn "`mein Junge"'.

"`Mein Junge, liebst du mich?"'

Er best"urmte sie mit K"ussen. Eine andre Antwort begehrte sie
nicht.

Auf der Stutzuhr spreizte sich ein kleiner kecker Amor au{\s}
Bronze, der in seinen erhobenen Armen eine vergoldete Girlande
trug. Er machte ihnen viel Spa"s. Nur wenn die Trennung{\s}stunde
schlug, kam ihnen alle{\s} ernsthaft vor.

Unbeweglich standen sie einander gegen"uber, und immer
wiederholten sie:

"`Auf Wiedersehn! N"achsten Donnerstag!"'

Pl"otzlich nahm sie seinen Kopf zwischen ihre beiden H"ande,
k"u"ste ihn rasch auf die Stirn, und mit einem "`Adieu!"' st"urmte
sie die Treppe hinunter.

Zun"achst ging sie jede{\s}mal zum Friseur in der Theaterstra"se
und lie"s sich ihr Haar in Ordnung bringen. E{\s} war schon sp"at.
Im Laden brannten bereit{\s} die Ga{\s}flammen. Sie h"orte da{\s}
Klingeln dr"uben im Theater, da{\s} dem Personal den Beginn der
Vorstellung anzeigte. Durch die Scheiben sah sie, wie M"anner mit
bleichen Gesichtern und Frauen in abgetragenen Kleidern im
hinteren Eingang de{\s} Theatergeb"aude{\s} verschwanden.

Der sehr niedrige Raum war "uberheizt. Mitten unter den Per"ucken
und Pomaden prasselte ein Ofen. Der Geruch der hei"sen
Brennscheren und der fettigen H"ande, die sich mit ihrem Haar zu
schaffen machten, bet"aubte sie beinahe. E{\s} fehlte nicht viel,
so w"are sie unter ihrem Frisiermantel eingeschlafen.

Wiederholt bot ihr der Friseur Billette zum Ma{\s}kenball an.

Dann ging sie fort, die Stra"sen wieder hinan, zur"uck in{\s}
"`Rote Kreuz"'. Sie suchte ihre "Uberschuhe hervor, die sie am
Vormittag unter einem Sitz der Postkutsche versteckt hatte, und
nahm ihren Platz ein, unter den bereit{\s} ungeduldigen
Mitfahrenden. Wo die steile Strecke begann, stiegen alle au{\s}.
Emma blieb allein im Wagen zur"uck.

Von Serpentine zu Serpentine sah sie in der Tiefe, unten in der
Stadt, immer mehr Lichter. Sie bildeten zusammen ein weite{\s}
Lichtermeer, in dem die H"auser verschwanden. Auf dem Sitzpolster
kniend, tauchte sie ihre Blicke in diesen Glanz. Schluchzend
fl"usterte sie den Namen Leo{\s} vor sich hin, k"u"ste ihn in
Gedanken und rief ihm leise Koseworte nach, die der Wind
verschlang.

Oben auf der H"ohe trieb sich ein Bettler herum, der die Postwagen
ablauerte. Er war in Lumpen geh"ullt, und ein alter verwetterter
Filzhut, rund wie ein Becken, verdeckte sein Gesicht. Wenn er ihn
abnahm, sah man in seinen Augenh"ohlen zwei blutige Aug"apfel mit
L"ochern an Stelle der Pupillen. Da{\s} Fleisch sch"alte sich in
roten Fetzen ab, und eine gr"unliche Fl"ussigkeit lief herau{\s},
die an der Nase gerann, deren schwarze Fl"ugel nerv"o{\s} zuckten.
Wenn man ihn ansprach, grinste er einen bl"od an. Dann rollten
seine bl"aulichen Aug"apfel fortw"ahrend in ihrem wunden Lager.

Er sang ein Lied, in dem folgende Stelle vorkam:
\begin{verse}
"`Wenn{\s} Sommer worden weit und breit, \\
Wird hei"s da{\s} Herze mancher Maid~..."'
\end{verse}

Manchmal erschien der Ungl"uckliche ohne Hut ganz pl"otzlich
hinter Emma{\s} Sitz. Sie wandte sich mit einem Aufschrei weg.

Hivert pflegte den Bettler zu verh"ohnen. Er riet ihm, sich auf
dem n"achsten Jahrmarkt in einer Bude sehen zu lassen, oder er
fragte ihn, wie e{\s} seiner Liebsten ginge.

Einmal streckte der Bettler seinen Hut w"ahrend der Fahrt durch
da{\s} Wagenfenster herein. Er war drau"sen auf da{\s}
kotbespritzte Trittbrett gesprungen und hielt sich mit einer Hand
fest. Sein erst schwacher und kl"aglicher Gesang ward schrill. Er
heulte durch die Nacht, ein Klagelied von namenlosem Elend. Da{\s}
Schellengel"aut der Pferde, da{\s} Rauschen der B"aume und da{\s}
Rasseln de{\s} Wagen{\s} t"onten in diese Jammerlaute hinein, so
da"s sie wie au{\s} der Ferne zu kommen schienen. Emma war
tiefersch"uttert. Empfindungen brausten ihr durch die Seele wie
wilder Wirbelsturm durch eine Schlucht. Grenzenlose Melancholie
ergriff sie.

Inzwischen hatte Hivert bemerkt, da"s eine fremde Last seinen
Wagen beschwerte. Er schlug mit seiner Peitsche mehrere Male auf
den Blinden ein. Die Schnur traf seine Wunden; er fiel in den
Stra"senkot und stie"s ein Schmerzen{\s}geheul au{\s}.

Die Insassen de{\s} Wagen{\s} waren nach und nach eingenickt. Die
einen schliefen mit offenem Munde; andern war da{\s} Kinn auf die
Brust gesunken; der lag mit seinem Kopfe an der Schulter de{\s}
Nachbar{\s}, und jener hatte den Arm in dem H"angeriemen, der je
nach den Bewegungen de{\s} Wagen{\s} hin und her schaukelte. Der
Schein der Laterne drang durch die schokoladenbraunen
Kattunvorh"ange und bedeckte die unbeweglichen Gestalten mit
blutroten Lichtstreifen. Emma war wie krank vor Traurigkeit. Sie
fror unter ihren Kleidern. Ihre F"u"se wurden ihr k"alter und
k"alter. Sie f"uhlte sich sterben{\s}ungl"ucklich.

Zu Hause wartete Karl auf sie. Donnerstag{\s} hatte die Post immer
Versp"atung. Endlich kam sie. Da{\s} Essen war noch nicht fertig,
aber wa{\s} k"ummerte sie da{\s}? Da{\s} Dienstm"adchen konnte
jetzt machen, wa{\s} e{\s} wollte.

E{\s} geschah oft, da"s Karl, dem Emma{\s} Bl"asse auffiel, sie
fragte, ob ihr etwa{\s} fehle.

"`Nein!"' antwortete sie.

"`Aber du bist so sonderbar heute abend?"'

"`Ach nein, nicht im geringsten!"'

Manchmal ging sie sofort nach ihrer Ankunft in ihr Zimmer. Oft war
gerade Justin da und bediente sie stumm und behutsam, besser
al{\s} eine Kammerzofe. Er stellte den Leuchter und die
Streichh"olzer zurecht, legte ihr ein Buch hin und da{\s}
Nachthemd und deckte da{\s} Bett auf.

"`Gut!"' sagte sie. "`Du kannst gehn."'

Er blieb n"amlich immer noch eine Weile an der T"ure stehen und
blickte Emma mit starren Augen wie verzaubert an.

Der Morgen nach der Heimkehr war ihr immer gr"a"slich, und noch
qualvoller wurden ihr die folgenden Tage durch die Ungeduld, mit
der sie nach ihrem Gl"ucke lechzte. Sie verging fast vor
L"usternheit, unter woll"ustigen Erinnerungen, bi{\s} alle ihre
Sehnsucht am siebenten Tage in Leo{\s} z"artlichen Armen
befriedigt wurde. Seine eigne, hei"se Sinnlichkeit verbarg sich
unter leidenschaftlicher Bewunderung und inniger Dankbarkeit.
Seine anbetung{\s}volle stille Liebe war Emma{\s} Ent\/z"ucken. Sie
hegte und pflegte sie mit tausend Liebkosungen, immer in Angst,
sein Herz zu verlieren.

Oft sagte sie ihm mit weicher, melancholischer Stimme:

"`Ach du! Du wirst mich verlassen! Du wirst dich verheiraten!
Wirst e{\s} machen wie alle andern!"'

"`Welche andern?"'

"`Wie alle M"anner, meine ich."'

Ihn sanft zur"ucksto"send, f"ugte sie hinzu:

"`Ihr seid alle gemein!"'

Eine{\s} Tage{\s} f"uhrten sie ein philosophische{\s} Gespr"ach
"uber die menschlichen Entt"auschungen, al{\s} sie pl"otzlich, um
seine Eifersucht auf die Probe zu stellen oder auch au{\s} allzu
starkem Mitteilung{\s}bed"urfni{\s}, da{\s} Gest"andni{\s} machte,
da"s sie vor ihm einen andern geliebt habe.

"`Nicht wie dich!"' f"ugte sie schnell hinzu und schwor beim Haupte
ihre{\s} Kinde{\s}, da"s e{\s} "`zu nicht{\s} gekommen"' sei.

Der junge Mann glaubte ihr, fragte sie aber doch, wo der
Betreffende jetzt sei.

"`Er war Schiff{\s}kapit"an, mein Lieber!"'

Log sie da{\s}, um jede Nachforschung zu vereiteln oder um sich
ein gewisse{\s} Ansehen zu verleihen, dieweil ein kriegerischer
und gewi"s vielumworbener Mann zu ihren F"u"sen gelegen haben
sollte?

In der Tat empfand der Adjunkt etwa{\s} wie da{\s} Bewu"stsein der
Inferiorit"at. Am liebsten h"atte er gleichfall{\s} Epauletten,
Orden und Titel getragen. Alle diese Dinge mu"sten ihr gefallen,
da{\s} sah er deutlich an ihrem Hang zum Luxu{\s}.

Dabei verschwieg ihm Emma noch einen gro"sen Teil ihrer in{\s}
Gro"sartige gehenden W"unsche; zum Beispiel, da"s sie gern einen
blauen Tilbury mit einem englischen Vollbl"uter und einem Groom in
schicker Livree gehabt h"atte, um in Rouen spazieren zu fahren.
Diesen Einfall verdankte sie Justin, der sie einmal flehentlich
gebeten hatte, ihn al{\s} Diener in ihren Dienst zu nehmen. Wenn
die Nichterf"ullung dieser Laune ihr auch die Seligkeit de{\s}
Wiedersehn{\s} nicht weiter tr"ubte, so versch"arfte sie doch
zweifello{\s} die Bitterkeit der Trennung.

Oft, wenn sie zusammen von Pari{\s} plauderten, sagte sie leise:

"`Ach, wenn wir dort leben k"onnten!"'

"`Sind wir denn nicht gl"ucklich?"' erwiderte Leo z"artlich und
strich mit der Hand liebkosend "uber ihr Haar.

"`Doch! Du hast recht! Ich bin t"oricht. K"usse mich!"'

Gegen ihren Gatten war sie jetzt lieben{\s}w"urdiger denn je. Sie
bereitete ihm seine Liebling{\s}gerichte und spielte ihm nach
Tisch Walzer vor. Er hielt sich f"ur den gl"ucklichsten Mann der
Welt. Emma lebte in v"olliger Sorglosigkeit. Aber eine{\s}
Abend{\s} sagte er pl"otzlich:

"`Nicht wahr, du hast doch bei Fr"aulein Lempereur Stunden?"'

"`Ja!"'

"`Merkw"urdig! Ich habe sie heute bei Frau Li\'egeard getroffen
und sie nach dir gefragt. Sie kennt dich gar nicht."'

Da{\s} traf sie wie ein Blitzstrahl. Trotzdem erwiderte sie
unbefangen:

"`Mein Name wird ihr entfallen sein."'

"`Oder e{\s} gibt mehrere Lehrerinnen diese{\s} Namen{\s} in
Rouen, die Klavierstunden geben"', meinte Karl.

"`Da{\s} ist auch m"oglich!"'

Pl"otzlich sagte Emma:

"`Aber ich habe ja ihre Quittungen. Wart mal! Ich werde dir gleich
eine bringen."'

Sie ging an ihren Schreibtisch, ri"s alle Schubf"acher auf,
w"uhlte in ihren Papieren herum und suchte so eifrig, da"s Karl
sie bat, sich wegen der dummen Quittungen doch nicht soviel M"uhe
zu machen.

"`Ich werde sie schon finden!"' beharrte sie.

In der Tat f"uhlte Karl am Freitag darauf, al{\s} er sich die
Stiefel anzog, die bei seinen Kleidern in einem finsteren Gela"s
zu stehen pflegten, zwischen Stiefelleder und Strumpf ein St"uck
Papier. Er zog e{\s} hervor und la{\s}:

\begin{center}
"`\so{Quittung.}
\end{center}
Honorar f"ur drei Monate Klavierstunden, nebst Au{\s}lagen f"ur
verschiedene beschaffte Musikalien: 65,--~Frkn.

\hfill Dankend erhalten \hspace{4em}

\hfill Friederike Lempereur,

\hfill Musiklehrerin."'
\bigskip

"`Zum Kuckuck! Wie kommt denn da{\s} in meinen Stiefel?"'

"`Wahrscheinlich"', erwiderte Emma, "`ist e{\s} au{\s} dem Karton
mit den alten Rechnungen gefallen, der auf dem obersten Regal
steht."'

Von nun an war ihre ganze Existenz nicht{\s} al{\s} ein Netz von
L"ugen. Sie h"ullte ihre Liebe darein wie in einen Schleier, damit
niemand sie s"ahe. Aber auch sonst wurde ihr da{\s} L"ugen
geradezu zu einem Bed"urfni{\s}. Sie log zu ihrem Vergn"ugen. Wenn
sie erz"ahlte, da"s sie auf der rechten Seite der Stra"se gegangen
sei, konnte man wetten, da"s e{\s} auf der linken gewesen war.

Eine{\s} Donnerstag{\s} war sie fr"uh, wie gew"ohnlich ziemlich
leicht gekleidet, abgefahren, al{\s} e{\s} pl"otzlich zu schneien
begann. Karl hielt am Fenster Umschau, da bemerkte er Bournisien
in der Kutsche de{\s} B"urgermeister{\s}. Sie fuhren zusammen nach
Rouen. Er ging hinunter und vertraute dem Priester einen dicken
Schal an mit der Bitte, ihn seiner Frau einzuh"andigen, sobald er
im "`Roten Kreuz"' angekommen sei. Bournisien fragte im Gasthofe
sogleich nach Frau Bovary, erhielt aber von der Wirtin die
Antwort, da"s sie da{\s} "`Rote Kreuz"' sehr selten aufsuche.
Abend{\s} traf er sie in der Postkutsche und erz"ahlte ihr von
seinem Mi"serfolge, dem er "ubrigen{\s} keine sonderliche
Bedeutung beizumessen schien, denn er begann al{\s}bald eine
Lobrede auf einen jungen Geistlichen, der in der Kathedrale so
wunderbar predige, da"s die Frauen in Scharen hingingen.

Wenn sich auch Bournisien ohne weitere{\s} zufrieden gegeben
hatte, so konnte doch ein andermal irgendwer nicht so di{\s}kret
sein. Und so hielt e{\s} Emma f"ur besser, fortan im "`Roten
Kreuz"' abzusteigen, damit die guten Leute au{\s} Yonville sie hin
und wieder auf der Treppe de{\s} Gasthofe{\s} sahen und nicht{\s}
argw"ohnten.

Eine{\s} Tage{\s} traf sie Lheureux, gerade al{\s} sie an Leo{\s}
Arm den Boulogner Hof verlie"s. Sie f"urchtete, er k"onne
schwatzen; aber er war nicht so t"oricht. Daf"ur trat er drei Tage
sp"ater in ihr Zimmer und erkl"arte, da"s er Geld brauche.

Sie erwiderte ihm, sie k"onne ihm nicht{\s} geben. Lheureux fing
zu jammern an und z"ahlte alle Dienste auf, die er ihr erwiesen.

In der Tat hatte Emma nur einen der von Karl au{\s}gestellten
Wechsel bezahlt, den zweiten hatte Lheureux auf ihre Bitte hin
verl"angert und dann abermal{\s} prolongiert. Jetzt zog er au{\s}
seiner Tasche eine Anzahl unbezahlter Rechnungen f"ur die
Store{\s}, den Teppich, f"ur M"obelstoff, mehrere Kleider und
verschiedene Toilettenst"ucke, im Gesamtbetrag von ungef"ahr
zweitausend Franken.

Sie lie"s den Kopf h"angen, und er fuhr fort:

"`Aber wenn Sie kein Geld haben, so haben Sie doch Immobilien."'

Und nun machte er sie auf ein halbverfallene{\s} alte{\s} Hau{\s}
in Barneville aufmerksam, da{\s} sie mit geerbt hatten. E{\s}
brachte nicht viel ein. E{\s} hatte urspr"unglich zu einem kleinen
Pachtgute geh"ort, da{\s} der alte Bovary vor Jahren verkauft
hatte. Lheureux wu"ste genau Bescheid "uber da{\s} Grundst"uck; er
kannte sogar die Anzahl der Hektare und die Namen der Nachbarn.

"`An Ihrer Stelle"', sagte er, "`versuchte ich, e{\s}
lo{\s}zuwerden. Sie bek"amen dann sogar noch bar Geld herau{\s}!"'

Sie entgegnete, e{\s} sei schwer, einen K"aufer zu finden, aber
Lheureux meinte, da{\s} lie"se sich schon machen. Da fragte sie,
wa{\s} sie tun m"usse, um da{\s} Hau{\s} zu verkaufen.

"`Sie haben doch die Vollmacht"', antwortete er.

Diese{\s} Wort belebte sie.

"`Lassen Sie mir die Rechnung hier!"' sagte sie.

"`O, da{\s} eilt ja nicht!"' erwiderte Lheureux.

In der kommenden Woche stellte er sich wiederum ein und berichtete,
e{\s} sei ihm mit vieler M"uhe gelungen, einen gewissen Langloi{\s}
au{\s}findig zu machen, der schon lange ein Auge auf da{\s}
Grundst"uck geworfen habe und wissen m"ochte, wa{\s} e{\s} koste.

"`Der Prei{\s} ist mir gleichg"ultig!"' rief Emma au{\s}.

Lheureux erkl"arte, man m"usse den K"aufer eine Weile zappeln
lassen. Die Sache sei aber schon eine Reise dahin wert. Da sie
selbst nicht gut verreisen k"onne, bot er sich dazu an, um da{\s}
Gesch"aft mit Langloi{\s} zu besprechen. Er kam mit der Mitteilung
zur"uck, der K"aufer habe viertausend Franken geboten.

Emma war hocherfreut.

"`Offen gestanden,"' f"ugte der H"andler hinzu, "`da{\s} ist
anst"andig bezahlt!"'

Die erste H"alfte der Summe z"ahlte er ihr sofort auf. Al{\s} Emma
sagte, damit solle ihre Rechnung beglichen werden, meinte
Lheureux:

"`Auf Ehre, e{\s} ist doch schade, da"s Sie ein so sch"one{\s}
S"ummchen gleich wieder au{\s} der Hand geben wollen!"'

Sie sah auf die Banknoten und dachte an die unbegrenzte Zahl der
Stelldichein, die ihr diese zweitausend Franken bedeuteten.

"`Wie? Wie meinen Sie?"' stammelte sie.

"`O,"' erwiderte er mit gutm"utigem L"acheln, "`man kann ja wa{\s}
ganz Beliebige{\s} auf die Rechnung setzen. Ich wei"s ja, wie
da{\s} in einem Hau{\s}halte so ist."'

Er sah sie scharf an, w"ahrend er die beiden Tausendfrankenscheine
langsam durch die Finger hin und her gleiten lie"s. Endlich machte
er seine Brieftasche auf und legte vier vorbereitete Wechsel zu je
tausend Franken auf den Tisch.

"`Unterschreiben Sie!"' sagte er, "`und behalten Sie die ganze
Summe!"'

Sie fuhr erschrocken zur"uck.

"`Na, wenn ich Ihnen den "Uberschu"s bar au{\s}zahle,"' sagte
Lheureux frech, "`erweise ich Ihnen dann nicht einen Dienst?"'

Er schrieb unter die Rechnung:

"`Von Frau Bovary viertausend Franken erhalten zu haben,
bescheinigt

\hfill Lheureux."'

"`So! Sie k"onnen unbesorgt sein. In sech{\s} Monaten erhalten Sie
die weiteren zweitausend Franken f"ur Ihre alte Bude! Eher ist
auch der letzte Wechsel nicht f"allig."'

Emma fand sich in der Rechnerei nicht mehr ganz zurecht. In den
Ohren klang e{\s} ihr, al{\s} w"urden S"acke voll Goldst"ucke vor
ihr au{\s}gesch"uttet, die nur so "uber die Diele kollerten.
Lheureux sagte noch, er habe einen Freund Vin\c{c}ard, Bankier in
Rouen, der die vier Wechsel di{\s}kontieren wolle. Die
"ubersch"ussige Summe werde er der gn"adigen Frau pers"onlich
bringen.

Aber statt zweitausend Franken brachte er nur eintausendachthundert.
Freund Vin\c{c}ard habe "`wie "ublich"' zweihundert Franken f"ur
Provision und Di{\s}kont abgezogen. Dann forderte er nachl"assig
eine Empfang{\s}best"atigung.

"`Sie verstehen! Gesch"aft ist Gesch"aft! Und da{\s} Datum! Bitte!
Da{\s} Datum!"'

Tausend nun erf"ullbare W"unsche umgaukelten Emma. Aber sie war so
vorsichtig, dreitausend Franken beiseite zu legen, womit sie dann
die ersten drei Wechsel prompt bezahlen konnte.

Der F"alligkeit{\s}tag de{\s} vierten Papiere{\s} fiel zuf"allig
auf einen Donnerstag. Karl war zwar arg betroffen, wartete aber
geduldig auf Emma{\s} R"uckkehr. Die Sache w"urde sich schon
aufkl"aren.

Sie log ihm vor, von dem Wechsel nur nicht{\s} gesagt zu haben, um
ihm h"au{\s}liche Sorgen zu ersparen. Sie setzte sich ihm auf die
Knie, liebkoste ihn, umgirrte ihn und z"ahlte ihm tausend
unentbehrliche Sachen auf, die sie auf Borg h"atte anschaffen
m"ussen.

"`Nicht wahr, du mu"st doch zugeben: f"ur so viele Dinge ist
tausend Franken nicht zuviel?"'

In seiner Ratlosigkeit lief Karl nun selber zu dem unvermeidlichen
Lheureux. Dieser verschwor sich, die Geschichte in Ordnung zu
bringen, wenn der Herr Doktor ihm zwei Wechsel au{\s}stelle, einen
davon zu siebenhundert Franken auf ein Vierteljahr. Daraufhin
schrieb Bovary seiner Mutter einen kl"aglichen Brief. Statt einer
Antwort kam sie pers"onlich. Al{\s} Emma wissen wollte, ob sie
etwa{\s} herau{\s}r"ucke, gab er ihr zur Antwort:

"`Ja! Aber sie will die Rechnung sehen!"'

Am andern Morgen lief Emma zu Lheureux und ersuchte ihn um eine
besondre Rechnung auf rund tausend Franken. Sonst k"ame die ganze
Geschichte und auch die Ver"au"serung de{\s} Grundst"uck{\s}
herau{\s}. Letztere hatte der H"andler so geschickt betrieben,
da"s sie erst viel sp"ater bekannt wurde.

Obgleich die aufgeschriebenen Preise sehr niedrig waren, konnte
die alte Frau Bovary nicht umhin, die Au{\s}gaben unerh"ort zu
finden.

"`Ging{\s} denn nicht auch ohne den Teppich? Wozu mu"sten die
Lehnst"uhle denn neu bezogen werden? Zu meiner Zeit gab e{\s} in
keinem Hause mehr al{\s} einen einigen Lehnstuhl, den
Gro"svaterstuhl! Die jungen Leute hatten keine n"otig. So war
e{\s} wenigsten{\s} bei meiner Mutter, und da{\s} war eine ehrbare
Frau! Da{\s} kann ich dir versichern! E{\s} sind nun einmal nicht
alle Menschen reich. Und Verschwendung ruiniert jeden! Ich w"urde
mich zu Tode sch"amen, wenn ich mich so verw"ohnen wollte wie du!
Und ich bin doch eine alte Frau, die wahrlich ein bi"schen der
Pflege n"otig h"atte ... Da schau mal einer diesen Luxu{\s} an!
Lauter Kinkerlitzchen! Seidenfutter, da{\s} Meter zu zwei Franken!
Wo man ganz sch"onen Futterstoff f"ur vier Groschen, ja schon f"ur
dreie bekommt, der seinen Zweck vollkommen erf"ullt!"'

Emma lag auf der Chaiselongue und erwiderte mit erzwungener Ruhe:

"`Ich finde, e{\s} ist nun gut!"'

Aber die alte Frau predigte immer weiter und prophezeite, sie
w"urden alle beide im Armenhause enden. "Ubrigen{\s} sei Karl der
Hauptschuldige. E{\s} sei ein wahre{\s} Gl"uck, da"s er ihr
versprochen habe, die unselige Generalvollmacht zu vernichten~...

"`Wa{\s}?"' unterbrach Emma ihre Rede.

"`Jawohl! Er hat mir sein Wort gegeben!"'

Emma "offnete ein Fenster und rief ihren Mann. Der
Ungl"uck{\s}mensch mu"ste zugeben, da"s ihm die Mutter da{\s}
Ehrenwort abgen"otigt hatte. Da ging Emma au{\s} dem Zimmer, kam
sehr bald wieder und h"andigte ihrer Schwiegermutter mit der
Geb"arde einer F"urstin ein gro"se{\s} Schriftst"uck ein.

"`Ich danke dir!"' sagte die alte Frau und steckte die Urkunde in
den Ofen.

Emma brach in eine rauhe, scharfe, andauernde Lache au{\s}. Sie
hatte einen Nervenchok bekommen.

"`Ach du mein Gott!"' rief Karl au{\s}. "`Siehst du, Mutter, e{\s}
war doch nicht recht von dir! Du darfst ihr nicht so zusetzen!"'

Sie zuckte mit den Achseln. Da{\s} sei alle{\s} "`blo"s Tuerei!"'

Da lehnte sich Karl zum ersten Male in seinem Leben gegen sie auf
und vertrat Emma so nachdr"ucklich, da"s die alte Frau erkl"arte,
sie werde abreisen. In der Tat tat sie da{\s} andern Tag{\s}.
Al{\s} Karl sie noch einmal auf der Schwelle zum Bleiben
"uberreden wollte, erwiderte sie:

"`Nein, nein! Du liebst sie mehr al{\s} mich, und da{\s} ist ja
ganz in der Ordnung! Wenn e{\s} auch dein Nachteil ist. Du wirst
ja sehen ... La"s dir{\s} wohl gehn! Ich werde ihr nicht sogleich
wieder -- sozusagen -- zusetzen!"'

Nicht weniger al{\s} armer S"under stand er dann vor Emma, die ihm
erbittert vorwarf, er habe kein Vertrauen mehr zu ihr. Er mu"ste
erst lange bitten, ehe sie sich herablie"s, eine neue
Generalvollmacht anzunehmen. Er begleitete sie zu Guillaumin, der
sie au{\s}stellen sollte.

"`Sehr begreiflich!"' meinte der Notar. "`Ein Mann der
Wissenschaft darf sich durch die Alltag{\s}dinge nicht ablenken
lassen."'

Karl f"uhlte sich durch diese im v"aterlichen Tone vorgebrachte
Wei{\s}heit wieder aufgerichtet. Sie bem"antelte seine Schwachheit
mit der schmeichelhaften Entschuldigung, er sei mit h"oheren
Dingen besch"aftigt.

Am Donnerstag darauf, in ihrem Zimmer im Boulogner Hofe, in
Leo{\s} Armen war sie "uber die Ma"sen au{\s}gelassen. Sie lachte,
weinte, sang, tanzte, lie"s sich Sorbett heraufbringen und rauchte
Zigaretten. So "uberschwenglich sie ihm auch vorkam, er fand sie
doch k"ostlich und bezaubernd. Er ahnte nicht, da"s e{\s} in ihrem
Innern g"arte und da"s sie sich au{\s} diesem Motiv kopf"uber in
den Strudel de{\s} Leben{\s} st"urzte. Sie war reizbar,
uners"attlich, woll"ustig geworden. Erhobenen Haupte{\s} ging sie
mit Leo durch die Stra"sen der Stadt spazieren, ohne die geringste
Angst, da"s sie in{\s} Gerede kommen k"onnte. So sagte sie
wenigsten{\s}. In{\s}geheim erzitterte sie freilich mitunter bei
dem Gedanken, Rudolf k"onne ihr einmal begegnen. Wenn sie auch auf
immerdar von ihm geschieden war, so f"uhlte sie sich doch noch
immer in seinem Banne.

Eine{\s} Abend{\s} kam sie nicht nach Yonville zur"uck. Karl war
au"ser sich vor Unruhe, und die kleine Berta, die ohne ihre
"`Mama"' nicht in{\s} Bett gehen wollte, schluchzte herzzerrei"send.
Justin wurde auf der Poststra"se entgegengesandt, und selbst
Homai{\s} verlie"s seine Apotheke.

Al{\s} e{\s} elf Uhr schlug, hielt e{\s} Karl nicht mehr au{\s}.
Er spannte seinen Wagen an, sprang auf den Bock, hieb auf sein
Pferd lo{\s} und langte gegen zwei Uhr morgen{\s} im "`Roten
Kreuz"' an. Emma war nicht da. Er dachte, vielleicht k"onne der
Adjunkt sie gesehen haben, aber wo wohnte er? Gl"ucklicherweise
fiel ihm die Adresse de{\s} Notar{\s} ein, bei dem Leo in der
Kanzlei arbeitete. Er eilte hin.

E{\s} begann zu d"ammern. Er erkannte da{\s} Wappenschild "uber
der T"ur und klopfte an. Ohne da"s ihm ge"offnet ward, erteilte
ihm jemand die gew"unschte Au{\s}kunft, nicht ohne auf den
n"achtlichen Ruhest"orer zu schimpfen.

Da{\s} Hau{\s}, in dem der Adjunkt wohnte, besa"s weder einen
T"urklopfer noch eine Klingel noch einen Pf"ortner. Karl schlug
mit der Faust gegen einen Fensterladen. Ein Schutzmann ging
vor"uber. Karl bekam Angst und ging davon.

"`Ich bin ein Narr!"' sagte er zu sich. "`Wahrscheinlich haben
Lormeaux' sie gestern abend zu Tisch dabehalten!"'

Die Familie Lormeaux wohnte gar nicht mehr in Rouen.

"`Vielleicht ist sie bei Frau D"ubreuil. Die ist vielleicht krank
... Ach nein, Frau D"ubreuil ist ja schon vor einem halben Jahre
gestorben ... Aber wo mag dann Emma nur sein?"'

Pl"otzlich fiel ihm etwa{\s} ein. Er lie"s sich in einem Caf\'e
da{\s} Adre"sbuch geben und suchte rasch nach dem Namen von
Fr"aulein Lempereur. Sie wohnte Rue de la Renelle de{\s}
Maroquinier{\s} Nummer 74.

Al{\s} er in diese Stra"se einbog, tauchte Emma am andern Ende
auf. Er st"urzte auf sie lo{\s} und fiel ihr um den Hal{\s}.

"`Wa{\s} hat dich denn gestern hier zur"uckgehalten?"' rief er.

"`Ich war krank."'

"`Wa{\s} fehlte dir denn? ... Na und wo ... Wie?"'

Sie fuhr mit der Hand "uber die Stirn und antwortete:

"`Bei Fr"aulein Lempereur."'

"`Da{\s} dachte ich mir doch gleich. Ich war auf dem Weg zu ihr."'

"`Die M"uhe kannst du dir nun ersparen. Sie ist "ubrigen{\s} schon
au{\s}gegangen. In Zukunft rege dich aber nicht wieder so auf! Du
kannst dir denken, da"s ich mich nicht gar frei f"uhle, wenn ich
wei"s, da"s dich die geringste Versp"atung derma"sen au{\s} dem
Gleichgewicht bringt!"'

Da{\s} war eine Art Erlaubni{\s}, die sie sich selbst gab, in
Zukunft mit aller Ruhe "uber den Strang hauen zu k"onnen, wie man
zu sagen pflegt. In der Tat machte sie nunmehr den
au{\s}giebigsten Gebrauch davon. Sobald sie Lust versp"urte, Leo
zu sehen, fuhr sie unter irgendeinem Vorwand nach Rouen. Da dieser
sie an solchen Tagen nicht erwartete, suchte sie ihn in seiner
Kanzlei auf.

Die ersten Male war ihm da{\s} eine gro"se Freude, aber
allm"ahlich verhehlte er ihr die Wahrheit nicht. Seinem Chef waren
diese St"orungen durchau{\s} nicht angenehm.

"`Ach wa{\s}, komm nur mit!"' sagte sie.

Und er verlie"s ihretwegen seine Arbeit.

Sie sprach den Wunsch au{\s}, er solle sich immer in Schwarz
kleiden und sich eine sogenannte Fliege stehen lassen, damit er
au{\s}s"ahe wie Ludwig der Dreizehnte auf dem bekannten Bilde. Er
mu"ste ihr seine Wohnung zeigen, die sie ziemlich armselig fand.
Er sch"amte sich, aber sie achtete nicht darauf und riet ihm,
Vorh"ange zu kaufen, wie sie welche hatte. Al{\s} er meinte, die
seien sehr teuer, sagte sie lachend:

"`Ach, h"angst du an deinen paar Groschen!"'

Jede{\s}mal mu"ste ihr Leo genau berichten, wa{\s} er seit dem
letzten Stelldichein erlebt hatte. Einmal bat sie ihn um ein
Gedicht, um ein Liebe{\s}gedicht ihr zu Ehren. Aber die Reimerei
lag ihm nicht, und er schrieb schlie"slich ein Sonett au{\s} einem
alten Almanach ab.

Er tat da{\s} keine{\s}weg{\s} au{\s} Eitelkeit. Er kannte kein
andre{\s} Bed"urfni{\s}, al{\s} ihr zu gefallen. Er war in allen
Dingen ihrer Ansicht und hatte stet{\s} denselben Geschmack wie
sie. Mit einem Worte: sie tauschten allm"ahlich ihre Rollen. Leo
wurde der feminine Teil in diesem Liebe{\s}verh"altnisse. Sie
verstand auf eine Art zu kosen und zu k"ussen, da"s er die
Empfindung hatte, al{\s} sauge sie ihm die Seele au{\s} dem Leibe.
E{\s} steckte, im Kerne ihre{\s} Wesen{\s} verborgen, eine
eigent"umliche, geradezu unk"orperliche Verderbni{\s} in Emma,
eine geheimni{\s}volle Erbschaft.


\newpage\begin{center}
{\large \so{Se{ch}{st}e{\s} Kapitel}}\bigskip\bigskip
\end{center}

Wenn Leo nach Yonville kam, um Emma zu besuchen, a"s er h"aufig
bei dem Apotheker zu Mittag. Au{\s} H"oflichkeit lud er ihn ein,
ihn nun auch einmal in Rouen zu besuchen.

"`Gern!"' gab Homai{\s} zur Antwort. "`Ich mu"s sowieso einmal
au{\s}spannen, sonst roste ich hier noch ganz und gar ein. Wir
wollen zusammen in{\s} Theater gehen, ein bi"schen kneipen und ein
paar Dummheiten lo{\s}lassen!"'

"`Aber Mann!"' mahnte Frau Homai{\s} besorgt. Die undefinierbaren
Gefahren, denen er entgegenlief, "angstigten sie im vorau{\s}.

"`Wa{\s} ist da weiter dabei? Hab ich meine Gesundheit nicht schon
genug ruiniert in den fortw"ahrenden Au{\s}d"unstungen der Drogen?
Ja, ja, so sind die Frauen! Vergr"abt man sich in die
Wissenschaften, so sind sie eifers"uchtig; und will man sich
gelegentlich in harmlosester Weise ein bi"schen erholen, dann
ist{\s} ihnen auch wieder nicht recht. Aber lassen wir{\s} gut
sein! Rechnen Sie auf mich! In allern"achster Zeit tauch ich in
Rouen auf: und dann wollen wir mal zusammen eine Kiste "offnen!"'

Fr"uher h"atte sich Homai{\s} geh"utet, einen derartigen
Au{\s}druck zu gebrauchen, aber seit einiger Zeit gefiel er sich
ungemein darin, den jovialen Gro"s\-st"adter zu spielen. "Ahnlich
wie seine Nachbarin, Frau Bovary, fragte er den Adjunkt auf da{\s}
neugierigste nach den Pariser Sitten und Unsitten au{\s}. Er
begann sogar in seiner Redeweise den Jargon der Pariser
anzunehmen, um den Philistern zu imponieren.

Eine{\s} Donnerstag{\s} fr"uh traf ihn Emma zu ihrer "Uberraschung
in der K"uche de{\s} Goldnen L"owen im Reiseanzug, da{\s} hei"st,
in einen alten Mantel gemummt, in dem man ihn noch nie gesehen
hatte, eine Reisetasche in der einen Hand, einen Fu"ssack in der
andern. Er hatte sein Vorhaben keinem Menschen verraten, au{\s}
Furcht, die Kundschaft k"onne an seiner Abwesenheit Ansto"s
nehmen.

Der Gedanke, die Orte wiedersehen zu sollen, wo er seine Jugend
verlebt hatte, regte ihn sichtlich auf, denn w"ahrend der ganzen
Fahrt redete er in einem fort. Kaum war man in Rouen angekommen,
so st"urzte er au{\s} dem Wagen, um Leo aufzusuchen. Dem Adjunkt
half kein Widerstreben: Homai{\s} schleppte ihn mit in da{\s}
"`Grand Caf\'e zur Normandie"', wo er, bedeckten Haupte{\s}, stolz
wie ein F"urst eintrat. Er hielt e{\s} n"amlich f"ur h"ochst
provinzlerhaft, in einem "offentlichen Lokal den Hut abzunehmen.

Emma wartete drei Viertelstunden lang auf Leo. Schlie"slich eilte
sie in seine Kanzlei. Unter allen m"oglichen Mutma"sungen, wobei
sie ihm den Vorwurf der Gleichg"ultigkeit und sich selber den der
Schw"ache machte, verbrachte sie dann den Nachmittag, die Stirn
gegen die Scheiben gepre"st, im Boulogner Hofe.

Um zwei Uhr sa"sen Leo und Homai{\s} immer noch bei Tisch. Der
gro"se Saal de{\s} Restaurant{\s} leerte sich. Sie sa"sen am Ofen,
der die Form eine{\s} hochragenden Palmenstamme{\s} hatte, dessen
innen vergoldete F"acher sich unter der wei"sen Decke
au{\s}breiteten. Neben ihnen, im hellen Sonnenlichte, hinter
Gla{\s}w"anden, sprudelte ein kleiner Springbrunnen "uber einem
Marmorbecken. An seinem Rande hockten zwischen Brunnenkresse und
Spargel drei schl"afrige Hummern; daneben lagen Wachteln, zu einem
Haufen aufgeschichtet.

Der Apotheker tat sich sozusagen eine G"ute. Wenngleich ihn die
Pracht noch mehr ent\/z"uckte al{\s} da{\s} vortreffliche Mahl, so
tat der Burgunder doch seine Wirkung. Und al{\s} da{\s} Omelett
mit Rum aufgetragen ward, da offenbarte er unmoralische Theorien
"`"uber die Weiber"'. Am meisten rege ihn eine "`schicke"' Frau
auf, und nicht{\s} ginge "uber eine elegante Robe in einem vornehm
eingerichteten Raume. Wa{\s} die k"orperlichen Reize anbelange, da
sei viel Fleisch "`nicht ohne"'.

Leo sah verzweifelt auf die Uhr. Der Apotheker trank, a"s und
schmatzte weiter.

"`Sie m"ussen sich "ubrigen{\s} ziemlich einsam f"uhlen hier in
Rouen"', sagte er pl"otzlich. "`Aber schlie"slich wohnt ja Ihr
Liebchen nicht allzuweit."' Da Leo err"otete, setzte er hinzu:
"`Na, gestehen Sie nur! Wollen Sie leugnen, da"s Sie in
Yonville~..."'

Der junge Mann stammelte etwa{\s} Unverst"andliche{\s}.

"`... im Hause Bovary jemanden poussieren~..."'

"`Aber wen denn?"'

"`Na, da{\s} Dienstm"adel!"'

E{\s} war sein Ernst. Aber Leo{\s} Eitelkeit war st"arker al{\s}
alle Vorsicht. Ohne sich{\s} zu "uberlegen, widersprach er. Er
liebe nur br"unette Frauen.

"`Da haben Sie nicht unrecht"', meinte der Apotheker. "`Die haben
mehr Temperament!"'

Homai{\s} begann zu fl"ustern und verriet seinem Freunde die
Symptome, an denen man erkennen k"onne, ob eine Frau Feuer habe.
Er geriet sogar auf eine ethnographische Abschweifung. Die
Deutschen seien schw"armerisch, die Franz"osinnen woll"ustig, die
Italienerinnen leidenschaftlich.

"`Und die Negerinnen?"' fragte der Adjunkt.

"`Da{\s} ist etwa{\s} f"ur Kenner! Kellner! Zwei Tassen Kaffee!"'

"`Gehen wir?"' fragte Leo ungeduldig.

"`\begin{antiqua}Yes!\end{antiqua}"'

Aber zuvor wollte er den Besitzer de{\s} Restaurant{\s} sprechen
und ihm seine Zufriedenheit au{\s}sprechen.

De{\s} weiteren sch"utzte der junge Mann einen gesch"aftlichen
Gang vor. Er wollte nun endlich allein sein.

"`Ich begleite Sie nat"urlich!"' sagte Homai{\s}.

Unterweg{\s} erz"ahlte er unaufh"orlich von seiner Frau, von
seinen Kindern, von ihrem Gedeihen, von seiner Apotheke, vom
verwahrlosten Zustand, in dem er sie "ubernommen, und wie er sie
in die H"ohe gebracht habe.

Vor dem Boulogner Hofe verabschiedete sich Leo kurzerhand von ihm,
eilte die Treppe hinan und fand seine Geliebte in der gr"o"sten
Erregung. Bei der Erw"ahnung de{\s} Apotheker{\s} geriet sie in
Wut. Leo versuchte, sie durch allerlei vern"unftige Gr"unde zu
beruhigen. E{\s} sei wirklich nicht seine Schuld gewesen. Sie
kenne Homai{\s} doch. Wie habe sie nur glauben k"onnen, da"s er
lieber mit ihm statt mit ihr zusammen sei? Aber sie wollte gar
nicht{\s} h"oren und schickte sich an, fort\/zugehen. Er hielt sie
zur"uck, sank vor ihr auf die Knie, umschlang sie mit beiden Armen
und sah sie mit einem r"uhrenden Blick voller Begehrlichkeit und
Unterw"urfigkeit an.

Sie stand aufrecht vor ihm. Mit gro"sen flammenden Augen sah sie
ihn ernst, fast drohend an. Dann aber verschwamm dieser Au{\s}druck
in Tr"anen. Ihre ger"oteten Lider schlossen sich, sie "uberlie"s
ihm ihre H"ande, die er an seine Lippen zog. Da erschien der
Hau{\s}diener. Ein Herr w"unsche ihn dringend zu sprechen.

"`Du kommst doch wieder?"' fragte Emma.

"`Gewi"s!"'

"`Aber wann?"'

"`Sofort!"'

E{\s} war der Apotheker.

"`Ein feiner Trick, nicht?"' schmunzelte er, al{\s} er Leo
erblickte.

"`Ich wollte Ihnen Ihre Unterredung verk"urzen. Sie war Ihnen doch
offensichtlich unangenehm. So! Jetzt gehen wir zu meinem Freund
Bridoux, einen Bittern genehmigen!"'

Leo beteuerte, er m"usse in seine Kanzlei. Aber der Apotheker
lachte ihn au{\s} und machte seine Witze "uber die Juristerei.

"`Lassen Sie doch den Aktenkram Aktenkram sein! Zum Teufel, warum
nur nicht? Seien Sie kein Frosch! Kommen Sie, wir gehn zu Bridoux!
Seinen Terrier m"ussen Sie mal sehen! Der ist zu spa"sig!"' Und da
der Adjunkt immer noch widerstrebte, fuhr er fort: "`Na, da
begleite ich Sie wenigsten{\s}! Werde in Ihrem Laden eine Zeitung
lesen oder in irgendeinem alten Schm"oker bl"attern."'

Leo war wie bet"aubt durch Emma{\s} Unwillen, durch de{\s}
Apotheker{\s} Geschw"atz und vielleicht auch durch die Nachwirkung
de{\s} reichlichen Fr"uh\-st"uck{\s}. Unentschlossen stand er da,
w"ahrend Homai{\s} immer wieder in ihn drang:

"`Kommen Sie nur mit! Wir gehn zu Bridoux! Er wohnt keine hundert
Schritte von hier! Rue Malpalu!"'

Diese Aufforderung wirkte wie eine Suggestion. Au{\s} Feigheit
oder Narrheit oder au{\s} jenem merkw"urdigen Drange, der den
Menschen mitunter zu Handlungen bewegt, die seinem eigentlichen
Willen zuwiderlaufen, lie"s sich Leo zu Bridoux f"uhren. Sie
fanden ihn in dem kleinen Hofe seine{\s} Hause{\s}, wo er drei
Burschen beaufsichtigte, die da{\s} gro"se Rad einer
Selterwasserzubereitung{\s}maschine drehten. Nach einer herzlichen
Begr"u"sung gab Homai{\s} seinem Kollegen Ratschl"age. Dann trank
man den Bittern. Leo war hundertmal im Begriffe, sich zu
empfehlen, aber Homai{\s} hielt ihn immer wieder fest, indem er
sagte:

"`Gleich! Gleich! Ich gehe ja mit! Wir wollen nun mal in den
{\glq}Leuchtturm von Rouen{\grq}! Dem Redakteur guten Tag sagen.
Ich mache Sie mit ihm bekannt, mit Herrn Thomassin."'

Trotzdem machte sich Leo endlich lo{\s} und eilte wiederum in den
Boulogner Hof. Emma war nicht mehr da. Im h"ochsten Grade
aufgebracht, war sie fortgegangen. Jetzt ha"ste sie Leo. Da{\s}
Stelldichein zu vers"aumen, da{\s} fa"ste sie al{\s} Beschimpfung
auf! Nun suchte sie nach noch andern Gr"unden, mit ihm zu brechen.
Er sei eine{\s} h"oheren Aufschwung{\s} unf"ahig, schwach, banal,
feminin, dazu knickerig und kleinm"utig.

Dann wurde sie ruhiger; sie sah ein, da"s sie ihn schlechter
machte, al{\s} er war. Aber da{\s} Herabzerren eine{\s} Geliebten
hinterl"a"st immer gewisse Spuren. Man darf ein G"otzenbild nicht
ber"uhren: die Vergoldung bleibt einem an den Fingern kleben.

Fortan unterhielten sie sich immer h"aufiger von Dingen, die
nicht{\s} mit ihrer Liebe zu tun hatten. In den Briefen, die ihm
Emma schrieb, war die Rede von Blumen, Versen, vom Mond und den
Sternen, mit einem Worte von allen den primitiven Requisiten, die
eine mattgewordne Leidenschaft aufbaut, um den Schein aufrecht zu
erhalten. Immer wieder erhoffte sie sich von dem n"achsten
Beieinandersein die alte Gl"uckseligkeit, aber hinterher gestand
sie sich jede{\s}mal, da"s sie nicht{\s} davon gesp"urt hatte.
Diese Entt"auschung wandelte sich trotzdem in neue{\s} Hoffen.
Emma kam immer wieder zu Leo voll Begehren und sinnlicher
Erregung. Sie warf die Kleider ab und ri"s da{\s} Korsett
herunter, dessen Schnuren ihr um die H"uften schlugen wie
zischende Schlangen. Mit nackten F"u"sen lief sie an die T"ur und
"uberzeugte sich, da"s sie verriegelt war. Mit einer hastigen
Bewegung entledigte sie sich dann de{\s} Hemde{\s} -- und bleich,
stumm, ernst und von Schauern durchstr"omt, warf sie sich in seine
Arme.

Aber auf ihrer von kaltem Schwei"s beperlten Stirn, auf ihren
st"ohnenden Lippen, in ihren irren Augen, in ihrer wilden Umarmung
lebte etwa{\s} Unheimliche{\s}, Feindselige{\s}, Todtraurige{\s}.
Leo f"uhlte e{\s}. E{\s} hatte sich eingeschlichen, um sie zu
trennen.

Ohne da"s er darnach zu fragen wagte, kam er ferner zu der
Erkenntni{\s}, da"s die Geliebte alle Pr"ufungen der Lust und
de{\s} Leid{\s} schon einmal an sich selber erfahren haben mu"ste.
Wa{\s} ihn dereinst ent\/z"uckt hatte, da{\s} fl"o"ste ihm jetzt
Grauen ein.

Dazu kam, da"s er gegen die t"aglich zunehmende Vergewaltigung
seiner Person rebellierte. Er grollte ihr ob ihrer immer neuen
Siege. Oft zwang er sich, kalt zu bleiben, aber wenn er sie dann
auf sich zukommen sah, ward er doch wieder schwach, wie ein
Absinthtrinker, den da{\s} gr"une Gift immer wieder verf"uhrt.

Allerding{\s} wandte sie alle Liebe{\s}k"unste an: von
au{\s}gesuchten Gen"ussen bei Tisch bi{\s} zu den Raffinement{\s}
der Kleidung und den schmachtendsten Z"artlichkeiten. Sie brachte
au{\s} ihrem Garten Rosen mit, die sie an der Brust trug und ihm
in{\s} Gesicht warf. Sie sorgte sich um seine Gesundheit und gab
ihm gute Ratschl"age, wie er leben solle. Abergl"aubisch schenkte
sie ihm ein Amulett mit einem Madonnenbildchen. Wie eine ehrsame
Mutter erkundigte sie sich nach seinen Freunden und Bekannten.

"`La"s sie! Geh nicht au{\s}! Denk nur an mich und bleib mir treu!"'

Am liebsten h"atte sie ihn "uberwacht oder gar "uberwachen lassen.
Mitunter kam ihr letztere{\s} in den Sinn. E{\s} trieb sich in der
N"ahe de{\s} Boulogner Hofe{\s} regelm"a"sig ein Tagedieb herum,
der die{\s} wohl "ubernommen h"atte. Aber ihr Stolz hielt sie
davon ab.

"`Mag er mich hintergehen! Dann ist er eben nicht{\s} wert! Wa{\s}
tut{\s}? Ich halte ihn nicht!"'

Eine{\s} Tage{\s} ging sie zeitiger von ihm weg al{\s}
gew"ohnlich. Al{\s} sie allein den Boulevard hinschlenderte,
bemerkte sie die Mauer ihre{\s} Kloster{\s}. Da setzte sie sich
auf eine schattige Bank unter den Ulmen. Wie friedsam hatte sie
damal{\s} gelebt! Sie bekam Sehnsucht nach den jungfr"aulichen
Vorstellungen von der Liebe, die sie sich damal{\s} au{\s}
B"uchern ertr"aumt hatte~...

Dann erinnerte sie sich an ihre Flitterwochen ... an den Vicomte,
mit dem sie Walzer getanzt hatte, ... an die Ritte durch den Wald
... an den Tenor Lagardy ... Alle{\s} da{\s} zog wieder an ihr
vor"uber ... Und mit einem Male stand ihr auch Leo so fern wie
alle{\s} andre.

"`Aber ich liebe ihn doch!"' fl"usterte sie.

Sie war dennoch nicht gl"ucklich, und nie war sie da{\s} gewesen!
Warum reichte ihr da{\s} Leben nie etwa{\s} Ganze{\s}? Warum kam
immer gleich Moder in alle Dinge, die sie an ihr Herz zog?

Wenn e{\s} irgendwo auf Erden ein Wesen gab, stark und sch"on und
tapfer, begeisterung{\s}f"ahig und liebe{\s}erfahren zugleich, mit
einem Dichterherzen und einem Engel{\s}k"orper, ein Schw"armer und
S"anger, warum war sie ihm nicht zuf"allig begegnet? Ach, weil
da{\s} eine Unm"oglichkeit ist! Weil e{\s} vergeblich ist, ihn zu
suchen! Weil alle{\s} Lug und Trug ist! Jede{\s} L"acheln verbirgt
immer nur da{\s} G"ahnen der Langweile, jede Freude einen Fluch,
jeder Genu"s den Ekel, der ihm unvermeidlich folgt! Die hei"sesten
K"usse hinterlassen dem Menschen nicht{\s} al{\s} die unstillbare
Begierde nach der Wollust der G"otter!

Eherne Kl"ange dr"ohnten durch die Luft. Die Klosterglocke schlug
viermal. Vier Uhr! E{\s} d"unkte Emma, sie s"a"se schon eine
Ewigkeit auf ihrer Bank. Unendlich viel Leidenschaft kann sich in
einer Minute zusammendr"angen, wie eine Menschenmenge in einem
kleinen Raume~...

Emma lebte nur noch f"ur sich selbst. Die Geldangelegenheiten
k"ummerten sie nicht mehr. Aber eine{\s} Tage{\s} erschien ein
Mann von sch"abigem Au{\s}sehen und erkl"arte, Herr Vin\c{c}ard in
Rouen schicke ihn her. Er zog die Stecknadeln herau{\s}, mit denen
er die eine Seitentasche seine{\s} langen gr"unen Rocke{\s}
verschlossen hatte, steckte sie im "Armelaufschlag fest und
"uberreichte ihr h"oflich ein Papier. E{\s} war ein Wechsel auf
siebenhundert Franken, den sie au{\s}gestellt hatte. Lheureux
hatte ihn seinem Versprechen entgegen an Vin\c{c}ard
weitergegeben.

Sie schickte Felicie zu dem H"andler. Er k"onne nicht abkommen,
lie"s er zur"ucksagen. Der Unbekannte hatte stehend gewartet und
dabei hinter seinen dichten blonden Augenlidern neugierige Blicke
auf Hau{\s} und Hof gerichtet. Jetzt fragte er einf"altig:

"`Wa{\s} soll ich Herrn Vin\c{c}ard au{\s}richten?"'

"`Sagen Sie ihm nur"', gab Emma zur Antwort, "`... ich h"atte kein
Geld! Vielleicht in acht Tagen ... Er solle warten ... Ja, ja, in
acht Tagen!"'

Der Mann ging, ohne etwa{\s} zu erwidern. Aber am Tage darauf
erhielt sie eine Wechselklage. Auf der gestempelten
Zustellung{\s}urkunde starrten ihr mehrfach die Worte "`Hareng,
Gericht{\s}vollzieher in B"uchy"' entgegen. Dar"uber erschrak sie
derma"sen, da"s sie spornstreich{\s} zu Lheureux lief.

Er stand in seinem Laden und schn"urte gerade ein Paket zu.

"`Ihr Diener!"' begr"u"ste er sie. "`Ich stehe Ihnen sogleich zur
Verf"ugung!"'

Im "ubrigen lie"s er sich in seiner Besch"aftigung nicht st"oren,
bei der ihm ein etwa dreizehnj"ahrige{\s} M"adchen half. E{\s} war
ein wenig verwachsen und versah bei dem H"andler zugleich die
Stelle de{\s} Ladenm"adchen{\s} und der K"ochin.

Al{\s} er fertig war, f"uhrte er Frau Bovary hinauf in den ersten
Stock. Er ging ihr in seinen schl"urfenden Holzschuhen auf der
Treppe voran. Oben "offnete er die T"ur zu einem engen Gemach, in
dem ein gro"ser Schreibtisch mit einem Aufsatz voller
Rechnung{\s}b"ucher stand, die durch eine eiserne, mit einem
Vorh"angeschlo"s versehene Stange verwahrt waren. An der Wand
stand ein Geldschrank von solcher Gr"o"se, da"s er sichtlich noch
andre Dinge al{\s} blo"s Geld und Banknoten enthalten mu"ste. In
der Tat lieh Lheureux Geld auf Pf"ander au{\s}. In diesem Schrank
lagen unter anderm die Kette der Frau Bovary und die Ohrringe
de{\s} alten Tellier. Der ehemalige Besitzer de{\s} Caf\'e
Fran\c{c}ai{\s} hatte inzwischen sein Grundst"uck verkaufen
m"ussen und in Quincampoix einen kleinen Kramladen er"offnet. Dort
ging er seiner Schwindsucht langsam zugrunde, inmitten seiner
Talglichte, die weniger gelb waren al{\s} sein Gesicht.

Lheureux setzte sich in seinen gro"sen Rohrstuhl und fragte:

"`Na, wa{\s} gibt{\s} Neue{\s}?"'

Emma hielt ihm die Vorladung hin.

"`Hier, lesen Sie!"'

"`Ja, wa{\s} geht denn mich da{\s} an?"'

Diese Antwort emp"orte sie. Sie erinnerte ihn an sein Versprechen,
ihre Wechsel nicht in Umlauf zu bringen. Er gab da{\s} zu.

"`Aber notgedrungen hab ich{\s} doch tun m"ussen! Mir sa"s selber
da{\s} Messer an der Kehle!"'

"`Und wa{\s} wird jetzt geschehn?"'

"`Ganz einfach! Erst kommt ein gerichtlicher Schuldtitel und dann
die Zwang{\s}\-voll\-stre"ckung! Schwapp! Ab!"'

Emma konnte sich nur mit M"uhe beherrschen. Sie h"atte ihm beinahe
in{\s} Gesicht geschlagen. Ruhig fragte sie, ob e{\s} denn kein
Mittel gebe, Herrn Vin\c{c}ard zu vertr"osten.

"`Den und vertr"osten! Da kennen Sie Vin\c{c}ard schlecht! Da{\s}
ist ein Bluthund!"'

Dann m"usse eben Lheureux einspringen.

"`H"oren Sie mal,"' entgegnete er, "`mir scheint, da"s ich schon
genug f"ur Sie eingesprungen bin! Sehen Sie!"' Er schlug seine
B"ucher auf: "`Hier! Am 3. August zweihundert Franken ... am 17.
Juni hundertundf"unfzig Franken ... am 23. M"arz sech{\s}undvierzig
Franken ... am 10. April~..."'

Er hielt inne, al{\s} f"urchte er eine Dummheit zu sagen.

"`Dazu kommen noch die Wechsel, die mir Ihr Mann au{\s}gestellt
hat, einen zu siebenhundert und einen zu dreihundert Franken! Von
Ihren ewigen kleinen Rechnungen und den r"uckst"andigen Zinsen gar
nicht zu reden! Da{\s} ist ja endlo{\s}! Da findet sich ja gar
niemand mehr hinein! Ich will nicht{\s} mehr mit der Sache zu tun
haben!"'

Emma fing an zu weinen, nannte ihn sogar ihren lieben guten
Lheureux, aber er verschanzte sich immer wieder hinter "`diesen
Schweinehund, den Vin\c{c}ard"'. "Ubrigen{\s} verf"uge er selber
"uber keinen roten Heller in bar. Kein Mensch bezahle ihn. Man
z"oge ihm da{\s} Fell "uber die Ohren. Ein armer H"andler, wie er,
k"onne nicht{\s} borgen.

Emma schwieg. Lheureux nagte an einem Federhalter. Durch ihr
Schweigen sichtlich beunruhigt, sagte er schlie"slich:

"`Na, vielleicht ... wenn dieser Tage wa{\s} einkommt~..."'

Sie unterbrach ihn:

"`Wenn ich die letzte Rate f"ur da{\s} Grundst"uck in Barneville
bekomme~..."'

"`Wieso?"'

Er tat so, al{\s} sei er sehr "uberrascht, da"s Langloi{\s} noch
nicht gezahlt habe. Mit honigs"u"ser Stimme sagte er:

"`Na, da machen Sie mal einen Vorschlag!"'

"`Ach, den m"ussen Sie machen!"'

Er schlo"s die Augen, al{\s} ob er sich etwa{\s} "uberlegte.
Hierauf schrieb er ein paar Ziffern, und dann erkl"arte er, er
k"ame sehr schlecht dabei weg, die Geschichte sei faul und er
schneide sich in sein eigne{\s} Fleisch. Schlie"slich f"ullte er
vier Wechsel au{\s}, jeden zu zweihundertundf"unfzig Franken, mit
F"alligkeit{\s}tagen, die je vier Wochen au{\s}einanderlagen.

"`Vorau{\s}gesetzt nat"urlich, da"s Vin\c{c}ard darauf eingeht!"'
sagte er. "`Mir soll{\s} ja recht sein! Ich fackle nicht lange!
Bei mir geht alle{\s} wie geschmiert!"'

Er zeigte ihr im Vorbeigehen schnell noch ein paar Neuigkeiten.

"`E{\s} ist aber nicht{\s} f"ur Sie darunter, gn"adige Frau!"'
meinte er. "`Wenn ich bedenke: dieser Stoff, da{\s} Meter zu drei
Groschen und angeblich sogar waschecht! Die Leute rei"sen sich
drum! Man sagt ihnen nat"urlich nicht, wa{\s} wirklich dran ist
... Sie k"onnen{\s} sich ja denken!"'

Durch derlei Gest"andnisse seiner Unreellit"at andern gegen"uber
sollte er sich bei ihr al{\s} desto ehrlicher hinstellen. Emma war
bereit{\s} an der T"ur, al{\s} er sie zur"uckrief und ihr drei
Meter Brokatstickerei zeigte, einen "`Gelegenheit{\s}kauf"', wie
er sagte.

"`Prachtvoll! Nicht?"' sagte er. "`Man nimmt e{\s} jetzt vielfach
zu Sofabeh"angen. Da{\s} ist hochmodern!"'

Mit der Geschicklichkeit eine{\s} Taschenspieler{\s} hatte er den
Spitzenstoff bereit{\s} in blaue{\s} Papier eingeschlagen und Emma
in die H"ande gedr"uckt.

"`Ich mu"s doch aber wenigsten{\s} wissen, wa{\s}~..."'

"`Ach, da{\s} eilt ja nicht!"' unterbrach er sie und wandte sich
einem andern Kunden zu.

Noch an dem n"amlichen Abend best"urmte sie Karl, er solle doch
seiner Mutter schreiben, da"s sie den Rest der Erbschaft schicke.
E{\s} kam die Antwort, e{\s} sei nicht{\s} mehr da. Nach
Erledigung aller Verbindlichkeiten verblieben ihm -- abgesehen von
dem Grundst"uck in Barneville -- j"ahrlich sech{\s}hundert
Franken, die ihm p"unktlich zugehen w"urden.

Nunmehr verschickte sie an ein paar von Karl{\s} Patienten
Rechnungen; und da die{\s} von Erfolg war, machte sie da{\s}
h"aufiger. Der Vorsicht halber schrieb sie darunter: "`Ich bitte,
e{\s} meinem Manne nicht zu sagen. Sie wissen, wie stolz er in
dieser Beziehung ist. Verzeihen Sie g"utigst. Ihre sehr
ergebene~..."' Hie und da liefen Beschwerden ein, die sie
unterschlug.

Um sich Geld zu verschaffen, verkaufte sie ihre alten Handschuhe,
ihre abgelegten H"ute, alte{\s} Eisen. Dabei handelte sie wie ein
Jude. Hier kam ihr gewinns"uchtige{\s} Bauernblut zum Vorschein.
Auf ihren Au{\s}fl"ugen nach Rouen erstand sie allerhand Tr"odel,
den Lheureux an Zahlung{\s} Statt annehmen sollte. Sie kaufte
Strau"senfedern, chinesische{\s} Porzellan, altert"umliche Truhen.
Sie lieh sich Geld von Felicie, von Frau Franz, von der Wirtin vom
"`Roten Kreuz"', von aller Welt. Darin war sie skrupello{\s}. Mit
dem Geld, da{\s} sie noch f"ur da{\s} Barneviller Hau{\s} bekam,
bezahlte sie zwei von den vier Wechseln. Die "ubrigen
f"unfzehnhundert Franken waren im Handumdrehen weg. Sie ging neue
Verpflichtungen ein und immer wieder welche.

Manchmal versuchte sie allerding{\s} zu rechnen, aber wa{\s} dabei
herau{\s}kam, erschien ihr unglaublich. Sie rechnete und rechnete,
bi{\s} ihr wirr im Kopfe wurde. Dann lie"s sie e{\s} und dachte
gar nicht mehr daran.

Um ihr Hau{\s} war e{\s} traurig bestellt. Oft sah man Lieferanten
mit w"utenden Gesichtern herau{\s}kommen. Am Ofen trocknete
W"asche. Und die kleine Berta lief zum gr"o"sten Entsetzen von
Frau Homai{\s} in zerrissenen Str"umpfen einher. Wenn sich Karl
gelegentlich eine bescheidene Bemerkung erlaubte, antwortete ihm
Emma barsch, e{\s} sei nicht ihre Schuld.

"`Warum ist sie so reizbar?"' fragte er sich und suchte die
Erkl"arung daf"ur in ihrem alten Nervenleiden. Er machte sich
Vorw"urfe, da"s er nicht gen"ugend R"ucksicht auf ihr
k"orperliche{\s} Leiden genommen habe. Er schalt sich einen
Egoisten und w"are am liebsten zu ihr gelaufen und h"atte sie
gek"u"st.

"`Lieber nicht!"' sagte er sich. "`E{\s} k"onnte ihr l"astig sein!"'

Und er ging nicht zu ihr.

Nach dem Essen schlenderte er allein im Garten umher. Er nahm die
kleine Berta auf seine Knie, schlug seine Medizinische Wochenschrift
auf und versuchte dem Kind da{\s} Lesen beizubringen. E{\s} war
noch g"anzlich unwissend. Sehr bald machte e{\s} gro"se, traurige
Augen und begann zu weinen. Da tr"ostete er e{\s}. Er holte Wasser
in der Gie"skanne und legte ein B"achlein im Kie{\s} an, oder er
brach Zweige von den Ja{\s}minstr"auchern und pflanze sie al{\s}
B"aumchen in die Beete. Dem Garten schadete da{\s} nur wenig, er
war schon l"angst von Unkraut "uberwuchert. Lestiboudoi{\s} hatte
schon wer wei"s wie lange keinen Lohn erhalten! Dann fror da{\s}
Kind, und e{\s} verlangte nach der Mutter.

"`Ruf Felicie!"' sagte Karl. "`Du wei"st, mein Herzchen, Mama will
nicht gest"ort werden!"'

E{\s} wurde wieder Herbst, und schon fielen die Bl"atter. Jetzt
war e{\s} genau zwei Jahre her, da"s Emma krank war! Wann w"urde
da{\s} endlich wieder in Ordnung sein? Er setzte seinen Weg fort,
die H"ande auf dem R"ucken.

Frau Bovary war in ihrem Zimmer. Kein Mensch durfte sie st"oren.
Sie hielt sich dort den ganzen Tag auf, im Halbschlafe und kaum
bekleidet. Von Zeit zu Zeit z"undete sie ein{\s} der
R"aucherkerzchen an, die sie in Rouen im Laden eine{\s}
Algerier{\s} gekauft hatte. Um in der Nacht nicht immer ihren
schnarchenden Mann neben sich zu haben, brachte sie e{\s} durch
allerlei Grimassen so weit, da"s er sich in den zweiten Stock
zur"uckzog. Nun la{\s} sie bi{\s} zum Morgen "uberspannte B"ucher,
die von Orgien und von Mord und Totschlag erz"ahlten. Oft bekam
sie davon Angstanf"alle. Dann schrie sie auf, und Karl kam eiligst
herunter.

"`Ach, geh nur wieder!"' sagte sie.

Manchmal wieder lief sie, vom heimlichen Feuer de{\s} Ehebruch{\s}
durch\-gl"uht, schwer atmend und in hei"ser sinnlicher Erregung
an{\s} Fenster, sog die k"uhle Nachtluft ein und lie"s sich den
Wind um da{\s} schwere Haar wehen. Zu den Gestirnen aufblickend,
w"unschte sie sich die Liebe eine{\s} F"ursten~...

Leo trat ihr vor die Phantasie. Wa{\s} h"atte sie in diesem
Augenblick darum gegeben, ihn bei sich zu haben und sich von ihm
sattk"ussen zu lassen.

Die Tage de{\s} Stelldichein{\s} waren ihre Sonntage, Tage der
Verschwendung! Und wenn Leo nicht imstande war, alle{\s} allein zu
bezahlen, steuerte sie auf da{\s} freigebigste dazu bei, wa{\s}
beinahe jede{\s}mal der Fall war. Er versuchte, sie zu
"uberzeugen, da"s sie ebensogut in einem einfacheren Gasthofe
zusammen kommen k"onnten. Sie wollte jedoch nicht{\s} davon
h"oren.

Eine{\s} Tage{\s} brachte sie in ihrer Reisetasche ein halbe{\s}
Dutzend vergoldete Teel"offel mit, da{\s} Hochzeit{\s}geschenk
ihre{\s} Vater{\s}. Sie bat Leo, sie im Leihhause zu versetzen. Er
gehorchte, obgleich ihm dieser Gang sehr peinlich war. Er
f"urchtete, sich blo"szustellen. Al{\s} er hinterher noch einmal
dar"uber nachdachte, fand er, da"s seine Geliebte "uberhaupt recht
seltsam geworden sei und da"s e{\s} vielleicht ratsam w"are, mit
ihr zu brechen. Seine Mutter hatte "ubrigen{\s} einen langen
anonymen Brief bekommen, in der ihr von irgendwem mitgeteilt
worden war, ihr Sohn "`ruiniere sich mit einer verheirateten
Frau."' Der guten alten Dame stand sofort der konventionelle
Familienpopanz vor Augen: der Vampir, die Sirene, die Teufelin,
die im Hexenreiche der Liebe ihr Wesen treibt. Sie wandte sich
brieflich an Leo{\s} Chef, den Justizrat D"ubocage, dem die
Geschichte l"angst schon zu Ohren gekommen war. Er nahm Leo
dreiviertel Stunden lang ordentlich in{\s} Gebet, "offnete ihm die
Augen, wie er sich au{\s}dr"uckte, und zeigte ihm den Abgrund, dem
er zusteuere. Wenn e{\s} zum "offentlichen Skandal k"ame, sei
seine weitere Karriere gef"ahrdet! Er bat ihn dringend, da{\s}
Verh"altni{\s} abzubrechen, wenn nicht im eignen Interesse, so
doch in seinem, de{\s} Notar{\s}.

Leo gab zu guter Letzt sein Ehrenwort, Emma nicht wiederzusehen.
Er hielt e{\s} nicht. Aber sehr bald bereute er diesen Wortbruch,
indem er sich klar ward, in welche Mi"shelligkeiten und in wa{\s}
f"ur Gerede ihn diese Frau noch bringen konnte, ganz abgesehen von
den Anz"uglichkeiten, die seine Kollegen allmorgendlich
lo{\s}lie"sen, wenn sie sich am Kamine w"armten. Er sollte
demn"achst in die erste Adjunktenstelle r"ucken. E{\s} ward also
Zeit, ein gesetzter Mensch zu werden. Au{\s} diesem Grunde gab er
auch da{\s} Fl"otespielen auf. Die Tage der Schw"armereien und
Phantastereien waren f"ur ihn vor"uber! Jeder Philister hat in
seiner Jugend seinen Sturm und Drang, und wenn der auch nur einen
Tag, nur eine Stunde w"ahrt. Einmal ist jeder der
ungeheuerlichsten Leidenschaft und himmelst"urmender Pl"ane
f"ahig. Den spie"serlichsten Mann gel"ustet e{\s} einmal nach
einer gro"sen Kurtisane, und selbst im n"uchternen Juristen hat
sich irgendwann einmal der Dichter geregt.

E{\s} verstimmte Leo jetzt, wenn Emma ohne besondre Veranlassung
an seiner Brust schluchzte. Und wie e{\s} Leute gibt, die Musik
nur in gewissen Grenzen vertragen, so hatte er f"ur die
"Uberschwenglichkeiten ihrer Liebe kein Gef"uhl mehr. Die wilde
Sch"onheit dieser Herzen{\s}st"urme begriff er nicht.

Sie kannten einander zu gut, al{\s} da"s der gegenseitige Besitz
sie noch zu berauschen vermochte. Ihre Liebe hatte die
Entwicklung{\s}f"ahigkeit verloren. Sie waren beide einander
"uberdr"ussig, und Emma fand im Ehebruche alle Banalit"aten der
Ehe wieder.

Wie sollte sie sich aber Leo{\s} entledigen? So ver"achtlich ihr
die Verflachung ihre{\s} Gl"ucke{\s} auch vorkam: au{\s}
Gewohnheit oder Verderbtheit klammerte sie sich doch daran. Der
Sinnengenu"s ward ihr immer unentbehrlicher, so sehr sie sich auch
nach h"oheren Wonnen sehnte. Sie warf Leo vor, er habe sie genarrt
und betrogen. Sie w"unschte sich eine Katastrophe herbei, die ihre
Ent\/zweiung zur Folge h"atte, weil sie nicht den Mut hatte, sich
au{\s} freien St"ucken von ihm zu trennen.

Sie h"orte nicht auf, ihn mit verliebten Briefen zu
"ubersch"utten. Ihrer Meinung nach war e{\s} die Pflicht einer
Frau, ihrem Geliebten alle Tage zu schreiben. Aber beim Schreiben
stand vor ihrer Phantasie ein ganz anderer Mann: nicht Leo,
sondern ein Traumgebilde, die Au{\s}geburt ihrer z"artlichsten
Erinnerungen, eine Reminis\/zenz an die herrlichsten Romanhelden,
da{\s} leibhaft gewordne Idol ihrer hei"sesten Gel"uste.
Allm"ahlich ward ihr dieser imagin"are Liebling so vertraut,
al{\s} ob er wirklich existiere, und sie empfand die seltsamsten
Schauer, wenn sie sich in ihn versenkte, obgleich sie eigentlich
gar keine bestimmte Idee von ihm hatte. Er war ihr ein Gott, in
der F"ulle seiner Eigenschaffen unsichtbar. Er wohnte irgendwo
hinter den Bergen, in einer Heimat romantischer Abenteuer, unter
Rosend"uften und Mondenschein. Sie f"uhlte, er war ihr nahe. Er
umarmte und k"u"ste sie~...

Nach solchen Traumzust"anden war sie kraftlo{\s} und gebrochen.
Die Raserei diese{\s} Liebe{\s}wahne{\s} erschlaffte sie mehr
al{\s} die wildeste Au{\s}schweifung.

Mehr und mehr verfiel sie in dauernde Mattheit. Gerichtliche
Zustellungen und Vorladungen kamen. E{\s} war ihr unm"oglich, sie
zu lesen. Leben war ihr eine Last. Am liebsten h"atte sie immerdar
geschlafen.

Am Fastnacht{\s}abend kam sie nicht nach Yonville zur"uck. Sie
nahm am Ma{\s}kenballe teil. In seidnen Kniehosen und roten
Str"umpfen, eine Rokokoper"ucke auf dem Kopfe und einen Dreimaster
auf dem linken Ohr, tollte und tanzte sie durch die laute Nacht.
E{\s} bildete sich eine Art Gefolge um sie, und gegen Morgen stand
sie unter der Vorhalle de{\s} Theater{\s}, umringt von einem
halben Dutzend Ma{\s}ken, Bekannten von Leo: Matrosen und
Fischerinnen. Man wollte irgendwo soupieren. Die Restaurant{\s} in
der N"ahe waren alle "uberf"ullt. Schlie"slich entdeckte man einen
bescheidenen Gasthof, in dem sie im vierten Stock ein kleine{\s}
Zimmer bekamen.

Die m"annlichen Ma{\s}ken tuschelten in einer Ecke; wahrscheinlich
einigten sie sich "uber die Kosten. E{\s} waren zwei Studenten der
medizinischen Hochschule, ein Adjunkt und ein Verk"aufer. Wa{\s}
f"ur eine Gesellschaft f"ur eine Dame! Und die weiblichen Wesen?
An ihrer Au{\s}druck{\s}weise merkte Emma gar bald, da"s sie fast
alle der untersten Volk{\s}schicht angeh"oren mu"sten. Nun begann
sie sich zu "angstigen. Sie r"uckte mit ihrem Sessel beiseite und
schlug die Augen nieder.

Die andern begannen zu tafeln. Emma a"s nicht{\s}. Ihre Stirn
gl"uhte, ihre Augenlider zuckten, und ein kalter Schauer rieselte
ihr "uber die Haut. In ihrem Hirn dr"ohnte noch der L"arm de{\s}
Tanzsaal{\s}; e{\s} war ihr, al{\s} stampften tausend F"u"se im
Takte um sie herum. Dazu bet"aubte sie der Zigarrenrauch und der
Duft de{\s} Punsche{\s}. Sie wurde ohnm"achtig. Man trug sie
an{\s} Fenster.

Der Morgen d"ammerte. Hinter der Sankt-Katharinen-H"ohe stand ein
breiter Purpurstreifen auf dem bleichen Himmel. Vor ihr rann der
graue Strom, im Winde erschauernd. Kein Mensch war auf den
Br"ucken. Die Laternenlichter verblichen.

Sie erholte sich allm"ahlich und dachte an ihre Berta, die fern in
Yonville schlief, im Zimmer de{\s} M"adchen{\s}. Ein Wagen voll
langer Eisenstangen fuhr unten vor"uber; da{\s} Metall vibrierte
in eigent"umlichen T"onen~...

Da stahl sie sich in pl"otzlichem Entschlusse fort. Sie lie"s Leo
und kam allein zur"uck in den Boulogner Hof. Alle{\s}, selbst ihr
eigner K"orper war ihr unertr"aglich. Sie h"atte fliegen m"ogen,
sich wie ein Vogel hoch emporschwingen und sich rein baden im
kristallklaren "Ather.

Nachdem sie sich ihre{\s} Kost"um{\s} entledigt hatte, verlie"s
sie den Gasthof und ging "uber den Boulevard, den Causer Platz,
durch die Vorstadt, bi{\s} zu einer freien Stra"se mit G"arten.
Sie ging rasch. Die frische Luft beruhigte sie. Nach und nach
verga"s sie die l"armende Menge, die Ma{\s}ken, die Tanzmusik,
da{\s} Lampenlicht, da{\s} Souper, die Dirnen. Alle{\s} war weg
wie der Nebel im Winde. Im "`Roten Kreuz"' angekommen, warf sie
sich auf{\s} Bett. E{\s} war in demselben Zimmer de{\s} zweiten
Stock{\s}, wo ihr Leo damal{\s} seinen ersten Besuch gemacht
hatte. Um vier Uhr nachmittag{\s} ward sie von Hivert geweckt.

Zu Hau{\s} zeigte ihr Felicie ein Schriftst"uck, da{\s} hinter der
Uhr steckte. Emma la{\s}:

"`Beglaubigte Abschrift. Urteil{\s}au{\s}fertigung~..."' Sie hielt
inne. "`Wa{\s} f"ur ein Urteil?"' Sie besann sich.

Etliche Tage vorher war ein andre{\s} Schriftst"uck abgegeben
worden, da{\s} sie ungelesen beiseitegelegt hatte. Erschrocken
la{\s} sie weiter:

"`\so{Im Namen de{\s} K"onig{\s}!}~..."' Sie "ubersprang einige
Zeilen. "`... binnen einer Frist von vierundzwanzig Stunden ...
achttausend Franken~..."' Und unten: "`Vorstehende Au{\s}fertigung
wird ... zum Zwecke der Zwang{\s}vollstreckung erteilt~..."'

Wa{\s} sollte sie dagegen tun? Binnen vierundzwanzig Stunden!

"`Die sind morgen abgelaufen!"' sagte sie sich. "`Unsinn! Lheureux
will mir nur angst machen!"'

Mit einem Male aber durchschaute sie alle seine Machenschaften,
den Endzweck aller seiner Gef"alligkeiten. Da{\s} einzige, wa{\s}
sie etwa{\s} beruhigte, war gerade die enorme H"ohe der
Schuldsumme. Durch ihre fortw"ahrenden K"aufe, ihr Nichtbarbezahlen,
die Darlehen, da{\s} Au{\s}stellen von Wechseln, die Zinsen, die
Prolongationen, Provisionen usw. waren ihre Schulden bi{\s} zu
dieser H"ohe angelaufen. Lheureux wartete auf diese{\s} Geld
ungeduldig. Er brauchte e{\s} zu neuen Gesch"aften.

Mit unbefangener Miene trat Emma in sein Kontor.

"`Wissen Sie, wa{\s} mir da zugefertigt worden ist? Da{\s} ist
wohl ein Scherz!"'

"`Bewahre!"'

"`Wieso aber?"'

Er wandte sich ihr langsam zu, verschr"ankte die Arme und sagte:

"`Haben Sie sich wirklich eingebildet, meine Verehrteste, da"s ich
bi{\s} zum J"ungsten Tage Ihr Hoflieferant und Bankier bliebe?
F"ur nicht{\s} und wieder nicht{\s}? E{\s} ist vielmehr die
h"ochste Zeit, da"s ich mein Geld zur"uckkriege! Da{\s} werden Sie
doch einsehen!"'

Sie bestritt die H"ohe der Schuldsumme.

"`Ja, da{\s} tut mir leid!"' erwiderte der H"andler. "`Da{\s}
Gericht hat die Forderung anerkannt. Gegen den Schuldtitel ist
nicht{\s} zu machen. Sie haben ja die Vorladung bekommen!
"Ubrigen{\s} bin ich nicht der Kl"ager, sondern Vin\c{c}ard."'

"`K"onnten Sie denn nicht~..."'

"`Ich kann gar nicht{\s}!"'

"`Aber ... sagen Sie ... "uberlegen wir un{\s} einmal~..."'

Sie redete hin und her. Sie habe nicht gewu"st, sie sei
"uberrascht worden~...

"`Ist da{\s} denn meine Schuld?"' fragte Lheureux mit einer
h"ohnischen Geste. "`W"ahrend ich mich hier abplagte, haben Sie
herrlich und in Freuden gelebt!"'

"`Wollen Sie mir eine Moralpredigt halten?"'

"`Da{\s} k"onnte nicht{\s} schaden!"'

Sie wurde feig und legte sich auf{\s} Bitten. Dabei ging sie so
weit, da"s sie den H"andler mit ihrer schmalen wei"sen Hand
ber"uhrte.

"`Lassen Sie mich zufrieden!"' wehrte er ab. "`Am Ende wollen Sie
mich gar noch verf"uhren!"'

"`Sie sind ein gemeiner Mensch!"' rief sie au{\s}.

"`Na, na!"' lachte er. "`Werden Sie nur nicht gleich ungn"adig!"'

"`Ich werde allen Leuten erz"ahlen, wa{\s} f"ur ein Mensch Sie
sind! Ich werde meinem Manne sagen~..."'

"`Und ich werde Ihrem Manne wa{\s} zeigen~..."'

Er entnahm seinem Geldschranke Emma{\s} Empfang{\s}best"atigung
der Summe f"ur da{\s} verkaufte Grundst"uck.

"`Glauben Sie, da"s er da{\s} nicht f"ur einen kleinen Diebstahl
halten wird, der arme gute Mann?"'

Sie brach zusammen, wie von einem Keulenschlage getroffen.
Lheureux lief zwischen seinem Schreibtisch und dem Fenster hin und
her und sagte immer wieder:

"`Jawohl, da{\s} zeig ich ihm ... da{\s} zeig ich ihm~..."'

Pl"otzlich trat er vor Emma hin und sagte in wieder friedlichem
Tone:

"`'{\s} ist grade kein Vergn"ugen -- da{\s} wei"s ich wohl! --
aber e{\s} ist noch niemand dran gestorben, und da e{\s} der
einzige Weg ist, der Ihnen bleibt, um mich zu bezahlen~..."'

"`Aber wo soll ich denn da{\s} viele Geld hernehmen?"' jammerte
Emma und rang die H"ande.

"`Na, wenn man Freunde hat wie Sie!"'

Er sah sie scharf und so t"uckisch an, da"s ihr dieser Blick durch
Mark und Bein ging.

"`Ich will Ihnen einen neuen Wechsel geben~..."'

"`Danke! Habe genug von den alten!"'

"`K"onnte ich nicht wa{\s} verkaufen?"'

"`Wa{\s} denn?"' fragte er achselzuckend. "`Sie besitzen doch gar
nicht{\s}!"' Dann rief er durch da{\s} kleine Schiebfensterchen in
seinen Laden hinein: "`Anna, vergi"s nicht die drei St"uck Tuch
Nummer vierzehn!"'

Da{\s} M"adchen trat ein. Emma begriff, wa{\s} da{\s} hei"sen
sollte. Sie machte einen letzten Versuch.

"`Wieviel Geld w"are dazu n"otig, die Zwang{\s}vollstreckung
aufzuhalten?"'

"`E{\s} ist schon zu sp"at!"' antwortete Lheureux.

"`Wenn ich nun aber ein paar Tausend Franken br"achte? Ein Viertel
der Summe? ... Ein Drittel? ... Und noch mehr?"'

"`Da{\s} h"atte alle{\s} keinen Zweck!"'

Er dr"angte sie sanft dem Au{\s}gange zu.

"`Ich beschw"ore Sie, bester Herr Lheureux! Nur ein paar Tage
Zeit!"'

Sie schluchzte.

"`Donnerwetter! Gar noch Tr"anen!"'

"`Sie bringen mich zur Verzweiflung!"' jammerte sie.

"`Mir auch egal!"'

Er machte die T"ure zu.


\newpage\begin{center}
{\large \so{Siebente{\s} Kapitel}}\bigskip\bigskip
\end{center}

Mit stoischem Gleichmut empfing Emma am andern Tage den
Gericht{\s}vollzieher Hareng und seine zwei Zeugen, al{\s} sie
sich einstellten, um da{\s} Pf"andung{\s}protokoll aufzusetzen.

Sie begannen in Bovary{\s} Sprechzimmer. Den phrenologischen
Sch"adel schrieben sie indessen nicht mit in da{\s}
Sachenverzeichni{\s}. Sie erkl"arten ihn al{\s} zur
Beruf{\s}au{\s}"ubung n"otig. Aber in der K"uche z"ahlten sie die
Sch"usseln, T"opfe, St"uhle und Leuchter, und in ihrem
Schlafzimmer die Nippsachen auf dem Wandbrette. Sie
durchst"oberten ihren Kleidervorrat, ihre W"asche. Sogar der
Klosettraum war vor ihnen nicht sicher. Emma{\s} Existenz ward
bi{\s} in die heimlichsten Einzelheiten -- wie ein Leichnam in der
Anatomie -- den Blicken der drei M"anner prei{\s}gegeben. Der
Gericht{\s}vollzieher, der einen fadenscheinigen schwarzen Rock,
eine wei"se Krawatte und Stege an den straffen Beinkleidern trug,
wiederholte immer wieder:

"`Sie erlauben, gn"adige Frau! Sie erlauben!"'

Mitunter entfuhren ihm auch Worte wie:

"`Wunderh"ubsch! Sehr nett!"'

Gleich darauf aber schrieb er von neuem an seinem Verzeichni{\s},
wobei er seinen Federhalter in sein Taschentintenfa"s au{\s} Horn
tauchte, da{\s} er in der linken Hand hielt.

Al{\s} man in den Wohnr"aumen fertig war, ging e{\s} hinauf in die
Bodenkammern. Al{\s} der Gericht{\s}vollzieher ein Schreibpult
bemerkte, in dem Rudolf{\s} Briefe aufbewahrt waren, ordnete er
an, da"s e{\s} ge"offnet werde.

"`Ah! Briefe!"' meinte er, geheimni{\s}voll l"achelnd. "`Sie
erlauben wohl! Ich mu"s mich n"amlich "uberzeugen, ob nicht sonst
noch wa{\s} drinnen steckt!"'

Er bl"atterte die B"undel fl"uchtig durch, al{\s} sollten
Goldst"ucke herau{\s}fallen. Emma war emp"ort, al{\s} sie sah, wie
seine plumpe rote Hand mit den mollu{\s}kenhaften Fettfingern
diese Bl"atter anfa"ste, bei deren Empfang ihr Herz einst h"oher
geschlagen hatte.

Endlich gingen sie. Felicie kam zur"uck. Sie hatte den Auftrag
gehabt, aufzupassen und Bovary vom Hause fernzuhalten. Den
Beamten, der zur Beaufsichtigung der gepf"andeten Gegenst"ande
zur"uckblieb, quartierten sie hurtig in einer Bodenkammer ein.

Karl schien an diesem Abend ernster denn sonst zu sein. Emma
beobachtete ihn "angstlich. E{\s} kam ihr vor, al{\s} st"unden in
den Falten seiner Stirn stumme Anklagen wider sie. Aber wenn ihre
Blicke den chinesischen Ofenschirm streiften oder die breiten
Gardinen oder die Lehnsessel, kurz alle die Dinge, mit denen sie
sich die Armseligkeit ihre{\s} Leben{\s} versch"ont hatte, f"uhlte
sie kaum einen Moment Reue, hingegen ein grenzenlose{\s} Mitleid
mit sich selber, da{\s} ihre W"unsche eher noch anfachte al{\s}
unterdr"uckte.

Karl sa"s friedlich am Kamin und f"uhlte sich h"ochst behaglich.
Einmal rumorte der Gericht{\s}diener, der sich in seinem K"afige
langweilte.

"`Ging da nicht oben einer?"' fragte Karl.

"`Nein!"' beschwichtigte sie ihn. "`Da war wahrscheinlich ein
Dachfenster offen, und der Wind hat e{\s} zugeschlagen."'

Am andern Tag, einem Sonntag, fuhr sie fr"uh nach Rouen, wo sie
alle Bankier{\s} aufsuchte, die sie dem Namen nach kannte. Die
meisten waren auf dem Lande oder auf Reisen. Aber sie lie"s sich
nicht abschrecken und ging die Anwesenden um Geld an, indem sie
beteuerte, sie brauche e{\s} und wolle e{\s} p"unktlich
zur"uckzahlen. Einige lachten ihr in{\s} Gesicht. Alle wiesen sie
ab.

Um zwei Uhr lief sie zu Leo und klopfte an seiner T"ure. E{\s}
"offnete niemand. Endlich kam er von der Stra"se her.

"`Wa{\s} f"uhrt dich her?"'

"`St"ore ich dich?"'

"`Nein ... aber~..."'

Er gestand, sein Wirt s"ahe e{\s} nicht gern, wenn man "`Damen"'
bei sich empfinge.

"`Ich mu"s dich sprechen!"' sagte sie.

Da nahm er den Schl"ussel, aber sie hinderte ihn am Aufschlie"sen.

"`Nein! Nicht hier! Bei un{\s}!"'

Sie gingen nach dem Boulogner Hof in ihr Zimmer.

Emma trank zun"achst ein gro"se{\s} Gla{\s} Wasser. Sie war ganz
bleich. Dann sagte sie:

"`Leo, du wirst mir einen Dienst erweisen!"'

Sie fa"ste seine H"ande, dr"uckte sie fest und f"ugte hinzu:

"`H"or mal: ich brauche achttausend Franken!"'

"`Du bist verr"uckt!"'

"`Noch nicht!"'

Nun erz"ahlte sie ihm rasch die Geschichte der Pf"andung und
klagte ihm ihre Notlage. Karl wisse von nicht{\s}; mit ihrer
Schwiegermutter stehe sie auf gespanntem Fu"se, und ihr Vater
k"onne ihr wirklich nicht helfen. Doch er, Leo, m"usse ihr diese
unbedingt n"otige Summe schleunigst verschaffen.

"`Wie soll ich da{\s}?"'

"`Du willst blo"s nicht!"' sagte sie aufgeregt.

Er stellte sich dumm:

"`E{\s} wird nicht so gef"ahrlich sein! Mit tausend Talern wird
der Biedermann schon zufrieden sein!"'

"`Vielleicht. Schaff sie mir nur!"' sagte sie. Dreitausend Franken
seien allemal aufzutreiben! Leo m"oge sie doch einstweilen auf
seinen Namen aufnehmen.

"`Geh! Versuch{\s}! E{\s} mu"s sein! Schnell! Schnell! Ich will
dich daf"ur auch recht liebhaben!"'

Er ging und kam nach einer Stunde zur"uck. Mit einem Gesicht,
al{\s} ob er wer wei"s wa{\s} zu verk"unden h"atte, sagte er:

"`Ich war bei drei Personen ... umsonst!"'

Darauf sa"sen sie einander gegen"uber am Kamin, regung{\s}lo{\s},
ohne zu sprechen. Emma zuckte mit den Achseln und trippelte vor
Ungeduld mit den F"u"sen. Er h"orte, wie sie ganz leise sagte:

"`Wenn ich an deiner Stelle w"are, ich w"u"ste, wo ich da{\s} Geld
auftriebe!"'

"`Wo denn?"'

"`In eurer Kanzlei!"'

Sie sah ihn starr an.

Au{\s} ihren fiebernden Augen sprach ein wilder D"amon. Zwischen
ihren sich ber"uhrenden Wimpern lockten Sinnlichkeit und S"unde so
stark, da"s der junge Mann unter der stummen Verf"uhrung{\s}kraft
diese{\s} Weibe{\s}, da{\s} ihn zum Verbrecher machen wollte, nahe
daran war, zu erliegen. Er f"uhlte seine Schwachheit. J"ahe Furcht
ergriff ihn, und um jeder weiteren Er"orterung zu entgehen, schlug
er sich vor die Stirn und rief au{\s}:

"`Morel kommt ja heute nacht zur"uck!"' Morel war ein Freund von
ihm, der Sohn eine{\s} sehr wohlhabenden Kaufmann{\s}. "`Der
schl"agt{\s} mir nicht ab! Ich werde dir da{\s} Geld morgen
vormittag bringen."'

Offenbar machte seine Zuversicht auf Emma einen viel weniger
freudigen Eindruck, al{\s} er erwartet hatte. Durchschaute sie
seine L"uge?

Err"otend fuhr er fort:

"`Wenn ich morgen bi{\s} drei Uhr nicht bei dir sein sollte, dann
warte nicht l"anger auf mich, Schatz! Jetzt mu"s ich aber wirklich
fort! Entschuldige mich! Lebwohl!"'

Er dr"uckte ihr die Hand, die schlaff in der seinen lag. Emma
hatte alle Kraft verloren~...

Al{\s} e{\s} vier Uhr schlug, stand sie auf, um nach Yonville
zur"uckzufahren. Nicht{\s} mehr trieb sie al{\s} die Gewohnheit.

Da{\s} Wetter war pr"achtig. Ein klarer kalter M"arztag. Die Sonne
strahlte auf einem kristallreinen Himmel. Sonnt"aglich gekleidete
B"urger gingen mit zufriedenen Gesichtern spazieren. Al{\s} Emma
den Notre-Dame-Platz "uberschritt, war die Vesper gerade zu Ende.
Die Menge str"omte au{\s} den drei T"uren de{\s} Hauptportal{\s}
wie ein Strom au{\s} einer dreibogigen Br"ucke.

Emma dachte zur"uck an den Tag, da sie mit Hangen und Bangen in
da{\s} Mittelschiff eingetreten war, da{\s} sich so hoch vor ihr
w"olbte und ihr damal{\s} doch klein erschien im Vergleich zu
ihrer grenzenlosen Liebe ... Sie ging weiter. Unter ihrem Schleier
str"omten die Tr"anen "uber ihre Wangen. Sie war wie bet"aubt, sie
schwankte und war einer Ohnmacht nahe.

"`Vorsehen!"' rief eine Stimme au{\s} einem Torwege.

Sie blieb stehen, um einen hochtretenden Rappen vorbeizulassen,
der, in der Gabel eine{\s} Dogcart{\s}, au{\s} dem Hause
herau{\s}kam. Ein Herr in einem Zobelpelz kutschierte~...

"`Wer war da{\s} doch?"' fragte sie sich. Er kam ihr bekannt vor.
Da{\s} Gef"ahrt fuhr im Trabe fort und war bald verschwunden.

"`Aber da{\s} war doch der Vicomte!"'

Emma wandte sich um, aber die Stra"se war leer. Sie f"uhlte sich
so niedergeschlagen, so traurig, da"s sie sich an die Wand
eine{\s} Hause{\s} lehnen mu"ste, um nicht umzusinken. Sie
gr"ubelte dar"uber nach, ob e{\s} wirklich der Vicomte gewesen
war. Vielleicht, vielleicht auch nicht! Wa{\s} lag daran? Sie war
eine Verlassene, vor sich selber und vor andern! Eine Verlorene,
vom Geratewohl gegen die Klippen de{\s} Leben{\s} getrieben ...
Und so empfand sie beinahe Freude, al{\s} sie, am "`Roten Kreuz"'
angelangt, den trefflichen Homai{\s} traf, der da{\s} Aufladen
einer gro"sen Kiste voll Apothekerwaren in die Post "uberwachte.
In der Hand hielt er, in ein Hal{\s}tuch eingewickelt, sech{\s}
St"uck Pumpernickel, die er seiner Frau mitbringen wollte.

Frau Homai{\s} liebte diese kleinen schweren Brote sehr, die in
der Normandie seit uralten Zeiten in Form eine{\s} Turban{\s}
gebacken und in der Fastenzeit mit gesalzner Butter gegessen
werden. Man buk sie bereit{\s} zur Zeit der Kreuzz"uge. Die
wetterfesten alten Normannen stopften sich voll davon, und wenn
sie diese Brote beim gelben Fackellicht vor sich auf dem Tische
liegen sahen, zwischen riesigen Beefsteaken und Methumpen, mochten
sie sich einbilden, Sarazenenk"opfe zu vertilgen. Die
Apotheker{\s}frau verzehrte sie mit nicht geringerem Heldenmute;
sie hatte n"amlich abscheulich schlechte Z"ahne.

"`Bin ent\/z"uckt, Sie zu sehen!"' rief Homai{\s}, bot Emma die Hand
und half ihr beim Einsteigen in die Postkutsche.

Dann legte er seine Pumpernickel hinauf in da{\s} Gep"acknetz,
nahm seinen Hut ab und setzte sich mit verschr"ankten Armen und
einer napoleonischen Denkermiene in die Ecke. Al{\s} unterweg{\s}
wie immer der Blinde am Stra"sengraben auftauchte, bemerkte er:

"`E{\s} ist mir unverst"andlich, da"s die Beh"orde nach wie vor
diese{\s} schandbare Gewerbe duldet! Solche Vagabunden sollte man
einsperren und zur Arbeit zwingen! Auf Ehre, die Kultur schleicht
bei un{\s} im Schneckengange vorw"art{\s}! Wir waten noch in
Barbarei!"'

Der Blinde steckte seinen Hut so durch{\s} Wagenfenster, da"s er
wie eine halb abgerissene Wagentasche auf und nieder wippte.

"`Er hat eine skroful"ose Affektion"', dozierte der Apotheker.

Obgleich er den armen Schelm schon l"angst kannte, tat er doch,
al{\s} s"ahe er ihn zum ersten Male. Er murmelte etwa{\s} von
Hornhaut, Star, Sklerotika, Facie{\s} vor sich hin. Dann riet er
ihm in salbung{\s}vollem Tone:

"`Hast du diese{\s} schreckliche Gebrechen schon lange, mein Sohn?
Du solltest vor allem Di"at halten, statt dich in der Kneipe zu
betanken! Gut essen und gut trinken ist immer die Hauptsache."'

Der Blinde leierte sein Lied ab. Er war zweifello{\s} geistig
beschr"ankt.

Schlie"slich zog Homai{\s} seine B"orse.

"`Hier hast du einen F"unfer, gib mir einen Dreier wieder rau{\s}
und vergi"s nicht, wa{\s} ich dir verordnet habe! E{\s} wird dir
gut bekommen!"'

Hivert erlaubte sich, ganz laut die Wirksamkeit seine{\s}
Rezept{\s} zu bezweifeln. Da versicherte Homai{\s} dem Manne,
lediglich eine "`antiphlogistische Salbe eignen Fabrikat{\s}"'
k"onne ihn heilen. Er gab ihm seine Adresse:

"`Apotheker Homai{\s}, am Markt, allgemein bekannt!"'

"`So, nun zeig mal zum Dank den Herrschaften, wa{\s} du
Sch"one{\s} kannst!"' rief ihm Hivert zu.

Der Blinde lie"s sich in die Knie nieder, warf den Kopf zur"uck,
rollte mit seinen gr"unlichen Augen und streckte die Zunge
herau{\s}. Dazu rieb er sich die Magengegend mit den H"anden und
stie"s ein dumpfe{\s} Geheul au{\s} wie ein halbverhungerter Hund.

Emma ward "ubel. Sie warf ihm "uber die Schulter ein
F"unf\/frankenst"uck zu. E{\s} war ihr ganze{\s} Geld. E{\s} kam
ihr edel vor, e{\s} so wegzuwerfen.

Der Wagen war schon ein ziemliche{\s} St"uck weiter, al{\s} sich
Homai{\s} pl"otzlich au{\s} dem Fenster lehnte und hinau{\s}rief:

"`Und keine Mehlspeisen und keine Milch! Wolle auf dem Leibe
tragen! Und Wacholderd"ampfe auf die kranken Teile!"'

Der Anblick der wohlbekannten Gegend, die an Emma vor"uberzog,
lenkte sie ein wenig von ihrem Schmerz ab. Eine unbezwingliche
M"udigkeit "uberkam sie. Ganz ersch"opft, leben{\s}m"ude und
verschlafen langte sie in Yonville an.

"`Mag nun kommen, wa{\s} will!"' dachte sie beim Au{\s}steigen.
"`Zu guter Letzt, wer wei"s? Kann nicht jeden Augenblick ein
unerwartete{\s} Ereigni{\s} eintreten? Sogar Lheureux kann
sterben~..."'

Am andern Morgen wurde sie durch ein Ger"ausch auf dem Markt wach.
E{\s} war ein Gedr"ange um ein gro"se{\s} Plakat entstanden,
da{\s} an einem der Pfeiler der Hallen angeschlagen war. Sie sah,
wie Justin auf einen Prellstein stieg und e{\s} abri"s. Aber im
selben Moment fa"ste ihn der Schutzmann am Kragen. In diesem
Augenblick trat Homai{\s} au{\s} seiner Apotheke, und auch Frau
Franz tauchte laut redend mitten in der Volk{\s}menge auf.

"`Gn"adige Frau! Gn"adige Frau!"' rief Felicie, die in{\s} Zimmer
st"urzte.

Da{\s} arme Ding war au"ser sich. Sie hielt einen gelben Zettel in
der Hand, den sie von der Hau{\s}t"ure abgerissen hatte. Emma
"uberflog ihn. E{\s} war die Versteigerung{\s}ank"undigung.

Dann sahen sich beide wortlo{\s} an. Herrin und Dienerin hatten
l"angst keine Geheimnisse mehr voreinander. Seufzend sagte Felicie
nach einer Weile:

"`An der Stelle der gn"adigen Frau ging ich mal zum Notar
Guillaumin."'

"`Meinst du?"'

Diese Frage bedeutete: "`Durch dein Verh"altni{\s} mit dem Diener
diese{\s} Hause{\s} wei"st du doch Bescheid. Interessiert sich
dieser Junggeselle f"ur mich?

"`Ja, gehn Sie nur, gn"adige Frau! E{\s} wird Ihnen n"utzen!"'

Emma kleidete sich an. Sie zog ihr schwarze{\s} Kleid an und
setzte einen Kapotthut mit Jettbesatz auf. Damit man sie nicht
s"ahe -- e{\s} standen immer noch eine Menge Leute auf dem Markte
--, ging sie zur Gartenpforte hinau{\s} und den Weg am Bache hin.

Atemlo{\s} erreichte sie da{\s} Gittertor de{\s} Notar{\s}. Der
Himmel war grau. E{\s} schneite ein wenig. Auf ihr Klingeln hin
erschien Theodor in einer roten Jacke auf der Freitreppe. Dann kam
er und "offnete ihr. Er behandelte sie mit einer gewissen
Vertraulichkeit, al{\s} ob sie in{\s} Hau{\s} geh"orte, und
f"uhrte sie in da{\s} E"szimmer.

Emma{\s} Blick fiel fl"uchtig auf den breiten Porzellanofen, vor
dem ein m"achtiger Kaktu{\s} stand. An den braun tapezierten
W"anden hingen in schwarzen Holzrahmen ein paar Kupferstiche:
woll"ustige Frauengestalten. Der gedeckte Tisch, die silbernen
Sch"usselw"armer, der Kristallgriff der T"urklinke, der
Parkettboden, die M"obel, alle{\s} blinkte in reinlicher,
germanischer Sauberkeit.

"`So ein E"szimmer m"u"ste ich haben!"' dachte Emma.

Der Notar trat ein. Er dr"uckte seinen mit Palmenblattstickerei
verzierten Schlafrock mit dem linken Arm gegen den Leib; mit der
andern Hand nahm er sein braunsamtne{\s} Hau{\s}k"appchen zum
Gru"se ab und setzte e{\s} rasch wieder auf. E{\s} sa"s ihm kokett
etwa{\s} auf der rechten Seite seine{\s} kahlen Sch"adel{\s},
"uber den drei lange blonde Haarstr"ahnen liefen.

Nachdem er Emma einen Stuhl angeboten hatte, setzte er sich an den
Tisch, um zu fr"uhst"ucken. Er entschuldigte sich ob dieser
Unh"oflichkeit.

"`Herr Notar,"' sagte sie, "`ich m"ochte Sie bitten~..."'

"`Um wa{\s} denn, gn"adige Frau? Ich bin ganz Ohr!"'

Sie begann ihm ihre Lage zu schildern.

Guillaumin wu"ste bereit{\s} alle{\s}, da er in geheimer
Gesch"aft{\s}verbindung mit Lheureux stand, der ihm die
Hypothekengelder zu verschaffen pflegte, die man dem Notar zu
besorgen Auftrag gab. Somit kannte er -- und besser al{\s} Emma --
die lange Geschichte ihrer Wechsel, die erst unbedeutend gewesen,
von den verschiedensten Leuten di{\s}kontiert, auf lange Fristen
au{\s}gestellt und dann immer wieder prolongiert worden waren.
Jetzt hatte sie der H"andler allesamt protestieren lassen und auf
seinen Freund Vin\c{c}ard abgeschoben, der die Angelegenheit nun
in seinem Namen verfolgte, damit der andre bei seinen Mitb"urgern
nicht in den Ruf eine{\s} Hal{\s}abschneider{\s} gerate.

Sie unterbrach ihre Erz"ahlung h"aufig durch Beschuldigungen gegen
Lheureux, auf die der Notar ab und zu mit ein paar
nicht{\s}sagenden Worten antwortete. Er verzehrte sein Kotelett
und trank seinen Tee, -- wobei er da{\s} Kinn gegen seine
himmelblaue, mit einer Brillantnadel geschm"uckte Krawatte einzog.
Ein sonderbare{\s}, s"u"sliche{\s} und zweideutige{\s} L"acheln
spielte um seine Lippen. Al{\s} er sah, da"s Emma nasse Schuhe
hatte, sagte er:

"`Kommen Sie doch n"aher an den Ofen heran! Halten Sie die Schuhe
doch an die Kacheln ... h"oher!"'

Sie bef"urchtete, die Porzellankacheln zu beschmutzen. Aber der
Notar sagte galant:

"`Sch"one Sachen verderben nie etwa{\s}!"'

Sie machte einen Versuch, ihn zu r"uhren. Da{\s} brachte sie aber
nur selbst in R"uhrung. Sie erz"ahlte ihm von der Enge ihre{\s}
h"au{\s}lichen Leben{\s}, von ihrem Unbefriedigtsein, von ihren
Bed"urfnissen. Der Notar verstand da{\s}: eine elegante Frau! Und
ohne sich vom Essen abhalten zu lassen, drehte er seinen Stuhl
nach ihr um. Er ber"uhrte mit einem Knie ihren Schuh, dessen Sohle
am hei"sen Ofen zu dampfen begann.

Al{\s} sie ihn aber um tausend Taler anging, bi"s er sich auf die
Lippen und erkl"arte, e{\s} tue ihm ungemein leid, da"s er die
Verwaltung ihre{\s} Verm"ogen{\s} nicht rechtzeitig in die H"ande
bekommen habe. E{\s} g"abe tausend M"oglichkeiten, selbst f"ur
eine Dame, ihr Geld gewinnbringend anzulegen. Beispiel{\s}weise
w"aren die Torfgruben von Gr"ume{\s}nil oder Bauland in Havre
bombensichere Spekulationen. Er machte Emma rasend vor Wut,
angesicht{\s} der enormen Summen, die sie zweifello{\s} dabei
gewonnen h"atte.

"`We{\s}halb sind Sie denn nicht zu mir gekommen?"'

"`Da{\s} wei"s ich selber nicht"', erwiderte sie.

"`Na, warum denn nicht? Sie haben wohl Angst vor mir gehabt? Ich
sollte Ihnen wirklich de{\s}halb b"ose sein! Wir h"atten un{\s}
schon l"angst kennen lernen sollen! Ich bin aber trotzdem Ihr
gehorsamster Diener! Da{\s} werden Sie mir doch glauben, hoffe
ich!"'

Er fa"ste nach ihrer Hand, dr"uckte einen gierigen Ku"s darauf und
behielt sie dann auf seinem Knie. Er liebkoste ihre Finger und
sagte ihr tausend Schmeicheleien. Seine fade Stimme gurgelte wie
Wasser im Rinnstein. Seine stechenden Augen funkelten durch die
spiegelnden Brillengl"aser; w"ahrend seine H"ande in die
"Armel"offnung von Emma{\s} Kleid fuhren, um ihren Arm zu
betasten. Sie f"uhlte seinen schnaubenden Atem auf ihrer Wange.

Sie sprang auf und sagte:

"`Herr Guillaumin, ich warte~..."'

"`Worauf?"' sagte der Notar, pl"otzlich ganz bleich geworden.

"`Auf da{\s} Geld!"'

"`Aber~..."' In seiner L"usternheit lie"s er sich bewegen zu
sagen: "`Na ja~..."'

Trotz seine{\s} Schlafrocke{\s} fiel er vor Emma auf die Knie und
keuchte:

"`Bitte, bleiben! Ich liebe Sie!"'

Er umschlang ihre Taille.

Ein Blutstrom scho"s Emma in die Wangen. Emp"ort machte sie sich von
dem Manne lo{\s} und rief:

"`Sie n"utzen mein Ungl"uck au{\s}! Da{\s} ist schamlo{\s}! Ich
bin beklagen{\s}wert, aber nicht k"auflich!"'

Damit eilte sie hinau{\s}.

Der Notar sah ihr ganz verdutzt nach. Sein Blick fiel auf seine
sch"onen gestickten Pantoffeln. Sie waren ein Geschenk von zarter
Hand. Dieser Anblick tr"ostete ihn schlie"slich. "Uberdie{\s} fiel
ihm ein, da"s ihn ein derartige{\s} Abenteuer zu wer wei"s wa{\s}
h"atte verleiten k"onnen.

"`Ein gemeiner Mensch! Ein Lump! Ein ehrloser Kerl!"' sagte Emma
bei sich, al{\s} sie hastigen Schritt{\s} an den Pappeln hinging.
Ihre Entt"auschung "uber den Mi"serfolg verst"arkte die Emp"orung
ihre{\s} Schamgef"uhl{\s}. E{\s} war ihr, al{\s} verfolge sie ein
unselige{\s} Geschick, und diese{\s} Gef"uhl erf"ullte sie von
neuem mit Stolz. Nie in ihrem Leben war sie hochm"utiger und
selbstbewu"ster gewesen und noch nie so voller Menschenverachtung.
Ein wilder Trotz entflammte sie. Sie h"atte alle M"anner schlagen,
ihnen in{\s} Gesicht speien, sie niedertreten m"ogen. W"ahrend sie
weitereilte, bleich, zitternd, verbittert, irrten ihre
tr"anenreichen Augen den grauen Horizont hin. Mit einer gewissen
Wollust bohrte sie sich in Ha"s hinein.

Al{\s} sie ihr Hau{\s} von weitem wiedersah, erstarrte sie. Die
Beine versagten ihr. Sie konnte nicht weiter ... Aber e{\s} mu"ste
sein! Wohin h"atte sie fliehen k"onnen?

Felicie erwartete sie an der kleinen Pforte.

"`Gn"adige Frau?"'

"`E{\s} war umsonst!"'

Eine Viertelstunde lang gingen sie zusammen alle Yonviller durch,
die vielleicht ihr zu helfen geneigt w"aren. Aber bei jedem Namen,
den Felicie nannte, wandte Emma ein:

"`Unm"oglich! Die tun e{\s} nicht!"'

"`Der Herr Doktor mu"s jeden Augenblick nach Hause kommen!"'

"`Ich wei"s e{\s}! La"s mich allein!"'

Sie hatte alle{\s} versucht. Nun mu"ste sie den Dingen ihren Lauf
lassen. Karl w"urde heimkommen. Sie mu"ste ihm sagen:

"`Geh wieder! Der Teppich, auf dem du stehst, ist nicht mehr
unser. In diesem Hau{\s} geh"ort un{\s} kein Stuhl mehr, kein
Nagel, kein Halm Stroh! Und ich, ich habe dich zugrunde gerichtet.
Armer Mann!"'

Dann w"urde e{\s} eine gro"se Szene geben, sie w"urde ma"slo{\s}
weinen, und wenn sich die erste Best"urzung gelegt h"atte, w"urde
er ihr verzeihen!

"`Ja! Er wird mir verzeihen!"' murmelte sie in verhaltener Wut.
"`Er! Er, dem ich nicht f"ur eine Million verzeihen kann, da"s ich
die Seine geworden bin! Niemal{\s}! Niemal{\s}!"'

Der Gedanke, Bovary k"onnte die "Uberlegenheit "uber sie erringen,
emp"orte sie. Ob sie ihm ein Gest"andni{\s} machte oder nicht,
jetzt sofort, nach ein paar Stunden oder morgen: er mu"ste doch
alle{\s} erfahren. Und dann war die gr"a"sliche Szene da, und sie
hatte die Zentnerlast seiner Gro"smut zu tragen!

Wiederum "uberlegte sie, ob sie nicht noch einmal zu Lheureux
gehen solle? Aber da{\s} n"utzte ja nicht{\s}! Oder ihrem Vater
schreiben? Dazu war e{\s} zu sp"at! Beinahe bereute sie e{\s}, dem
Notar nicht gef"ugig gewesen zu sein, -- da h"orte sie den
Hufschlag eine{\s} Pferde{\s} in der Allee. E{\s} war Karl. Er
"offnete da{\s} Hoftor. Sie sah ihn: er war wei"ser al{\s} Kalk.

Da lief sie eilend{\s} die Treppe hinunter und au{\s} der
Hau{\s}t"ur hinau{\s} nach dem Markt. Die Frau B"urgermeister
stand vor der Kirchent"ur und sprach mit dem Kirchendiener. Sie
beobachtete, wie Emma in dem Hause verschwand, wo der
Steuereinnehmer wohnte. Schnell ging sie zu Frau Caron, die
ihm gegen"uber in der Ecke de{\s} Markte{\s} wohnte, und klatschte
ihr diese Neuigkeit. Die beiden Frauen stiegen zusammen auf den
Oberboden, wo sie sich, gedeckt durch aufgeh"angte W"asche, so
aufstellten, da"s sie bequem in Binet{\s} Dachst"ubchen sehen
konnten.

Er war allein und sa"s an seiner Drehbank, gerade dabei
besch"aftigt, eine v"ollig zwecklose Spielerei au{\s} Holz
fertigzustellen. Im Halbdunkel seiner Werkstatt spr"uhte der helle
Holzstaub au{\s} seiner Maschine hervor, wie Funkenb"uschel unter
den Eisen eine{\s} galoppierenden Pferde{\s}. Die beiden R"ader
schnurrten und kreisten. Binet l"achelte mit aufmerksamer Miene,
den Kopf etwa{\s} vorgebeugt. Er war sichtlich v"ollig versunken
in sein Sch"opfergl"uck. Gerade da{\s} Handwerk{\s}m"a"sige,
da{\s} der Intelligenz nur leichte Schwierigkeiten bietet,
befriedigt den Menschen ungemein, wenn e{\s} vollendet ist, denn
e{\s} gibt dabei ja kein ideale{\s} Dar"uberhinau{\s}, da{\s} man
ersehnen k"onnte.

"`Ah, da ist sie!"' sagte Frau T"uvache.

Infolge de{\s} Ger"ausche{\s} der Drehbank vermochten sie nicht zu
verstehen, wa{\s} dr"uben gesprochen wurde. Nur einmal glaubten
sie, da{\s} Wort "`Taler"' zu h"oren, worauf Frau Caron
fl"usterte:

"`Sie bittet ihn um Aufschub der Steuern."'

"`E{\s} scheint so"', meinte die andre.

Sie beobachteten, wie Emma in Binet{\s} Stube hin und her ging und
die Serviettenringe, die Leuchter und all seinen andern zur Schau
au{\s}gelegten Krim{\s}kram besichtigte, w"ahrend sich der
Steuereinnehmer wohlgef"allig den Bart strich.

"`Will sie bei ihm etwa{\s} bestellen?"' fragte Frau T"uvache.

"`Er verkauft doch nie etwa{\s}!"'

Dann sah man, da"s Binet ihr aufmerksam zuh"orte. Er ri"s die
Augen weit auf. Offenbar verstand er sie nicht. Sie redete weiter,
eindringlich, flehend. Sie n"aherte sich ihm. Sie war sichtlich
erregt. Jetzt schwiegen sie beide.

"`Macht sie ihm gar einen Antrag?"' fl"usterte Frau T"uvache.
Binet bekam einen roten Kopf. Emma erfa"ste seine H"ande.

"`Nein, da{\s} ist doch stark!"' zischelte Frau Caron.

In der Tat mu"ste Emma etwa{\s} Sch"andliche{\s} von Binet
gefordert haben, denn dieser tapfere Veteran, der bei Dre{\s}den
und Leipzig mitgek"ampft hatte und dekoriert worden war, wich
pl"otzlich vor ihr zur"uck, al{\s} ob ihn eine Natter stechen
wollte, und rief au{\s}:

"`Frau Bovary, wa{\s} muten Sie mir zu!"'

"`Solche Frauenzimmer sollte man "offentlich au{\s}peitschen!"' eiferte
Frau T"uvache.

"`Wo ist sie denn mit einem Male hin?"' erwiderte die andre.

Wenige Augenblicke sp"ater sahen sie Emma die Hauptstra"se
hinau{\s}gehen und dann link{\s} verschwinden, wo der Weg zum
Friedhof abzweigt. Die beiden Horcherinnen ersch"opften sich in
allerhand Vermutungen.

Emma lief zur alten Frau Rollet.

"`Machen Sie mir da{\s} Korsett auf! Ich ersticke!"'

Mit diesen Worten trat sie bei ihr ein. Dann sank sie auf da{\s}
Bett und begann zu schluchzen. Die Frau deckte sie mit einem Rocke
zu und blieb vor ihr stehen. Da Emma auf keine ihrer Fragen
antwortete, ging sie schlie"slich hinau{\s}, holte ihr Spinnrad
und begann zu spinnen.

"`Ach, h"oren Sie auf!"' sagte Emma leise. E{\s} war ihr, al{\s}
h"ore sie noch Binet{\s} Drehbank.

"`Wa{\s} mag sie nur haben?"' fragte sich Frau Rollet. "`Warum ist
sie hergekommen?"'

Wa{\s} ahnte sie von der Angst, die Frau Bovary au{\s} ihrem Hause
gejagt hatte?

Emma lag auf dem R"ucken, regung{\s}lo{\s}, mit stieren Augen, die
keinen Gegenstand deutlich sahen, so sehr sie sich mit idiotischer
Beharrlichkeit bem"uhte, scharf zu beobachten. Sie starrte auf die
br"uchigen Stellen der Mauer, auf da{\s} armselige bi"schen Holz,
da{\s} im Kamine qualmte, auf eine gro"se Spinne, die gerade "uber
ihr an einem rissigen Deckenbalken hinkroch~...

Endlich kam Ordnung in ihre Gedanken. Erinnerungen tauchten auf
... der Tag, an dem sie mit Leo hier gewesen war ... Ach, wie weit
lag da{\s} zur"uck! Die Sonne hatte im Bache geglitzert, und die
Klemati{\s}ranken hatten sie im Vor"ubergehen gestreift ...
Tausend andre Erinnerungen umwirbelten sie wie ein brodelnder
Katarakt, und mit einem Male war sie wieder bei ihren j"ungsten
Erlebnissen.

"`Wieviel Uhr ist e{\s}?"' fragte sie.

Mutter Rollet ging vor da{\s} Hau{\s}, schaute nach der lichten
Stelle de{\s} Himmel{\s}, die den Stand der Sonne verriet, und kam
gem"achlich wieder herein.

"`Bald drei Uhr!"' sagte sie.

"`Sch"on! Ich danke!"'

Jetzt mu"ste Leo bald da sein! Sicherlich kam er. Er hatte da{\s}
Geld aufgetrieben. Aber er suchte sie in ihrer Wohnung. Da"s sie
hier war, konnte er doch nicht wissen. De{\s}halb bat sie Frau
Rollet, sofort einmal nachzusehen und ihn herzubringen.

"`Machen Sie recht schnell!"'

"`Aber beste Frau Bovary, ich gehe ja schon! Ich fliege!"'

Emma verwunderte sich, da"s ihr Leo jetzt erst wieder eingefallen
war. Er hatte ihr doch gestern sein Wort gegeben! Da{\s} brach er
gewi"s nicht! Schon sah sie sich im Geiste in Lheureux' Kontor und
z"ahlte ihm die drei Tausendfrankenscheine auf seinen
Schreibtisch. Nun brauchte sie nur noch ein M"archen zu ersinnen,
um ihrem Manne die ganze Geschichte harmlo{\s} hinzustellen.
Da{\s} war nicht weiter schlimm!

Frau Rollet h"atte l"angst wieder zur"uck sein m"ussen. E{\s}
schien der Wartenden wenigsten{\s} so. Aber da sie keine Uhr bei
sich hatte, redete sie sich ein, sie irre sich. Sie ging hinau{\s}
in da{\s} G"artchen und wanderte langsam hin und her. Dann schritt
sie ein St"uck den Pfad entlang der Hecke hin, kehrte aber
pl"otzlich wieder um, weil sie sich sagte, die Frau k"onne auch
auf einem andern Wege nach Hause kommen. Schlie"slich war sie
de{\s} Warten{\s} m"ude. Bange Ahnungen qu"alten sie. Sie hatte
kein Zeitgef"uhl mehr. Wartete sie seit ein paar Minuten oder seit
einem Jahrhundert?

Sie kauerte sich in einen Winkel, schlo"s die Augen und hielt sich
die Ohren zu. Die Zaunt"ure knarrte. Emma sprang auf. Ehe sie eine
Frage tat, vermeldete Frau Rollet:

"`E{\s} war niemand da!"'

"`Niemand?"'

"`Nein, niemand! Der Herr Doktor weint. Er l"a"st Sie suchen.
Alle{\s} ist auf den Beinen!"'

Emma blieb stumm. Sie atmete schwer. Ihre Augen irrten im Zimmer
umher. Frau Rollet sah ihr erschrocken in{\s} Gesicht.
Unwillk"urlich lief sie davon. Sie dachte, Emma sei wahnsinnig
geworden.

Pl"otzlich schlug sie sich auf die Stirn und tat einen lauten
Schrei. Rudolf war ihr in{\s} Ged"achtni{\s} gekommen, wie ein
heller Stern in stockfinsterer Nacht! Er war immer gutm"utig,
r"ucksicht{\s}voll und freigebig gewesen! Und selbst wenn er
z"ogerte, ihr diesen Dienst zu leisten, mu"ste ihn nicht ein
einziger voller Blick ihrer Augen an die verlorene Liebe mahnen
und ihn dazu zwingen!

So ging sie denn nach der H"uchette, ohne da{\s} Bewu"stsein zu
haben, da"s sie damit doch da{\s} tun wollte, wa{\s} ihr eben noch
so ver"achtlich vorgekommen war. Nicht im entferntesten dachte sie
daran, da"s sie sich prostituierte.


\newpage\begin{center}
{\large \so{A{ch}te{\s} Kapitel}}\bigskip\bigskip
\end{center}

Auf dem Wege fragte sie sich:

"`Wa{\s} werde ich ihm sagen? Womit soll ich anfangen?"'

Je n"aher sie kam, um so bekannter erschienen ihr die B"usche und
B"aume, der Ginster am Hange und schlie"slich da{\s} Herrenhau{\s}
vor ihr. Die z"artliche Liebe{\s}stimmung von damal{\s} tauchte
wieder auf, und ihr arme{\s} gequ"alte{\s} Herz schwoll im
Nachhall der vergangenen Seligkeit. Ein lauer Wind strich ihr
"uber{\s} Gesicht. Schmelzender Schnee fiel, Tropfen auf Tropfen,
von den knospenden B"aumen hernieder in{\s} Gra{\s}.

Wie einst schl"upfte sie durch die kleine Gartenpforte und ging
"uber den von einer doppelten Lindenreihe durchschnittenen
Herrenhof. Die B"aume wiegten s"auselnd ihre langen Zweige.
S"amtliche Hunde im Zwinger schlugen an, aber trotz ihre{\s}
Gebell{\s} erschien niemand.

Sie stieg die breite, mit einem h"olzernen Gel"ander versehene
Treppe hinauf. Die f"uhrte zu einem mit Steinfliesen belegten
staubigen Gang, auf den eine lange Reihe verschiedener Zimmer
m"undete, wie in einem Kloster oder in einem Hotel. Rudolf{\s}
Zimmer lag link{\s} ganz am Ende. Al{\s} sie die Finger um die
T"urklinke legte, verlie"sen sie pl"otzlich die Kr"afte. Sie
f"urchtete, er m"ochte nicht zu Hau{\s} sein, ja, sie w"unschte
e{\s} beinah, und doch war e{\s} ihre einzige Hoffnung, der letzte
Versuch zu ihrer Rettung. Einen Augenblick sammelte sie sich noch,
dachte an ihre Not, fa"ste Mut und trat ein.

Er sa"s vor dem Feuer, beide F"u"se gegen den Kaminsim{\s}
gestemmt, und rauchte eine Pfeife.

"`Mein Gott, Sie!"' rief er au{\s} und sprang rasch auf.

"`Ja, ich! Rudolf! Ich komme, Sie um einen Rat zu bitten!"'

Weiter brachte sie trotz aller Anstrengung nicht{\s} herau{\s}.

"`Sie haben sich nicht ver"andert! Sie sind noch immer reizend."'

"`So,"' wehrte sie voll Bitterni{\s} ab, "`da{\s} m"ussen traurige
Reize sein, mein Freund, da Sie sie verschm"aht haben!"'

Und nun begann er sein damalige{\s} Benehmen zu erkl"aren. Er
entschuldigte sich in halbsch"urigen Au{\s}dr"ucken, da er
etwa{\s} Ordentliche{\s} nicht vorzubringen hatte. Emma lie"s sich
durch seine Worte fangen, mehr noch durch den Klang seiner Stimme
und durch seine Gegenwart. Die{\s} war so m"achtig, da"s sie sich
stellte, al{\s} schenke sie seinen Au{\s}fl"uchten Glauben.
Vielleicht glaubte sie ihm auch wirklich. Er deutete ein
Geheimni{\s} an, von dem die Ehre und da{\s} Leben eine{\s}
dritten Menschen abgehangen h"atte.

"`Da{\s} ist ja nun gleichg"ultig"', sagte sie und sah ihn traurig
an. "`Ich habe schwer gelitten!"'

Rudolf meinte philosophisch:

"`So ist da{\s} Leben!"'

"`Hat e{\s} wenigsten{\s} Ihnen Gute{\s} gebracht, nach unserer
Trennung?"' fragte sie.

"`Ach, nicht{\s} Gute{\s} und nicht{\s} Schlechte{\s}!"'

"`Dann w"are e{\s} vielleicht besser gewesen, wenn wir damal{\s}
nicht voneinander gegangen w"aren?"'

"`Ja! Vielleicht!"'

"`Glaubst du da{\s}?"' fragte sie, indem sie aufseufzend ihm
n"aher trat. "`Ach Rudolf! Wenn du w"u"stest! Ich habe dich sehr
lieb gehabt!"'

Jetzt war sie e{\s}, die seine Hand ergriff. Eine Zeitlang sa"sen
sie mit verschlungenen H"anden da wie damal{\s}, am Bunde{\s}tage
der Landwirte. In einer sichtlichen Regung seine{\s} Stolze{\s}
k"ampfte er gegen seine eigene R"uhrung. Da schmiegte sich Emma an
seine Brust und sagte:

"`Wie hast du nur glauben k"onnen, da"s ich ohne dich leben
sollte! Ein Gl"uck, da{\s} man besessen, vergi"st man nie! Ich war
ganz verzweifelt! Dem Tode nahe! Ich will dir alle{\s} erz"ahlen,
du sollst alle{\s} erfahren. Aber du! Du hast mich nicht einmal
sehen m"ogen!"'

In der Tat war er ihr seit drei Jahren "angstlich au{\s} dem Wege
gegangen, in jener nat"urlichen Feigheit, die f"ur da{\s} starke
Geschlecht charakteristisch ist. Emma sprach weiter, unter
zierlichen Sendungen ihre{\s} Kopfe{\s}, schmeichlerischer al{\s}
eine verliebte Katze.

"`Du liebst andre! Gesteh e{\s} nur! Ach, ich begreife da{\s} ja
auch und entschuldige diese anderen! Du hast sie verf"uhrt, wie du
mich verf"uhrt hast. Du bist der geborene Verf"uhrer! Hast
alle{\s}, wa{\s} un{\s} Frauen verr"uckt macht. Aber sag! Wollen
wir von neuem beginnen? Ja? Sieh, ich lache! Ich bin gl"ucklich!
... So rede doch!"'

Sie sah ent\/z"uckend au{\s}. Eine Tr"ane zitterte in ihrem Auge,
wie eine Wasserperle nach einem Gewitter im Kelch einer blauen
Blume.

Er zog sie auf seine Knie und strich mit der Hand liebkosend ihr
Haar, "uber da{\s} der letzte Sonnenstrahl wie ein goldner Pfeil
hinwegflog, funkelnd im D"ammerlicht. Sie senkte die Stirn, und er
k"u"ste sie leise und sanft auf die Augenlider.

"`Du hast geweint?"' fragte er. "`Warum?"'

Da schluchzte sie laut auf. Rudolf hielt da{\s} f"ur einen
Au{\s}bruch ihrer Liebe, und da sie kein Wort sagte, nahm er ihr
Schweigen f"ur eine letzte Scham und rief au{\s}:

"`O, verzeih mir! Du bist die einzige, die mir gef"allt. Ich war
ein Tor, ein Schw"achling! Ein Elender! Ich liebe dich! Ich werde
dich immer lieben! Aber wa{\s} hast du? Sag e{\s} mir doch!"'

Er sank ihr zu F"u"sen.

"`So h"ore! ... Ich bin zugrunde gerichtet, Rudolf! Du mu"st mir
dreitausend Franken leihen."'

"`Ja ... aber~..."'

Er erhob sich langsam, und sein Gesicht nahm einen ernsten
Au{\s}druck an.

"`Du mu"st n"amlich wissen,"' fuhr sie schnell fort, "`da"s mein
Mann sein ganze{\s} Verm"ogen einem Notar anvertraut hatte. Der
ist fl"uchtig geworden. Wir haben un{\s} Geld geliehen. Die
Patienten bezahlten nicht. "Ubrigen{\s} ist der Nachla"skonkur{\s}
meine{\s} Schwiegervater{\s} noch nicht zu Ende. Wir werden bald
wieder Geld haben. Aber heute fehlen un{\s} dreitausend Franken.
De{\s}wegen sollen wir gepf"andet werden. Und zwar gleich, in
einer Stunde! Ich baue auf deine Freundschaft, und de{\s}halb bin
zu dir gekommen!"'

"`Aha!"' dachte Rudolf und ward pl"otzlich bla"s. "`Also darum ist
sie gekommen!"' Nach einer kleinen Weile sagte er gelassen:
"`Verehrteste, soviel habe ich nicht!"'

Er log nicht. Er w"urde ihr die Summe wohl gegeben haben, wenn er
sie da gehabt h"atte, obgleich e{\s} ihm wie den meisten Menschen
unangenehm gewesen w"are, sich gro"sm"utig zeigen zu m"ussen. Von
allen Feinden, die "uber die Liebe herfallen k"onnen, ist eine
Bitte um Geld der hartherzigste und gef"ahrlichste.

Sie sah ihn erst lange fest an; dann sagte sie:

"`Du hast sie nicht!"' Und mehrere Male wiederholte sie: "`Du hast
sie nicht! ... Ich h"atte mir diese letzte Schmach also ersparen
k"onnen! Du hast mich nie geliebt! Du bist nicht mehr wert al{\s}
die andern!"'

Sie verriet sich und ihre Frauenehre.

Rudolf unterbrach sie und versicherte, er sei selbst in
Verlegenheit.

"`Ach! Du tust mir sehr leid~..."', sagte Emma. "`Ja, ungemein!"'

Ihre Augen blieben an einer damas\/zierten B"uchse h"angen, die im
Gewehrschrank blinkte.

"`Aber wenn man arm ist, dann kauft man sich keine Flinten mit
Silberbeschlag, kauft man sich keine Stutzuhr mit
Schildpatteinlagen, keine Reitst"ocke mit goldnen Griffen!"' Sie
ber"uhrte einen, der auf dem Tische lag. "`Und tr"agt keine solche
Berlocken an der Uhrkette!"' Ach, er lie"s sich sichtlich
nicht{\s} abgehen. Da{\s} bewie{\s} allein da{\s}
Lik"orschr"ankchen im Zimmer. "`Ja, dich selber, dich liebst du!
Dich und ein gute{\s} Leben! Du hast ein Schlo"s, Pachth"ofe,
W"alder! Du reitest die Jagden mit, machst Reisen nach Pari{\s}!
Und wenn du mir nur \so{da{\s}} gegeben h"attest!"' Sie sprach
immer lauter und nahm seine mit Brillanten geschm"uckten
Manschettenkn"opfe vom Kamin. "`Diesen und andern entbehrlichen
Tand! Geld l"a"st sich schnell schaffen! Aber nun nicht mehr! Ich
will nicht{\s} davon haben! Behalt alle{\s}!"' Sie schleuderte die
beiden Kn"opfe weit von sich. Sie schlugen gegen die Wand. Ein
Goldkettchen zerbrach.

"`Ich, ach, ich h"atte dir alle{\s} gegeben, h"atte alle{\s}
verkauft. Mit meinen H"anden h"atte ich f"ur dich gearbeitet, auf
der Stra"se h"atte ich gebettelt, nur um von dir ein L"acheln,
einen Blick, ein einzige{\s} Dankwort zu erhaschen. Aber du! Du
bleibst gem"utlich in deinem Lehnstuhl sitzen, al{\s} ob du mir
nicht schon genug Leid zugef"ugt h"attest! Ohne dich -- da{\s}
wei"st du sehr wohl! -- h"atte ich gl"ucklich sein k"onnen! Wer
zwang dich dazu? Wolltest du eine Wette gewinnen? Und dabei hast
du mir eben noch gesagt, da"s du mich liebtest! Ach, h"attest du
mich doch lieber davongejagt! Meine H"ande sind noch warm von
deinen K"ussen, und hier auf dem Teppich, hier auf dieser Stelle
hast du gekniet und mir ewige Liebe geschworen! Du hast mich immer
belogen und betrogen! Mich zwei Jahre lang in dem s"u"sen Wahn
de{\s} herrlichsten Gef"uhl{\s} gelassen! Und dann der Plan unsrer
Flucht! Erinnerst du dich daran? An deinen Brief, deinen Brief! Er
hat mir da{\s} Herz zerrissen! Und heute, wo ich zu diesem Manne
zur"uckkehre, zu ihm, der reich, gl"ucklich und frei ist, und ihn
um eine Hilfe bitte, die der erste beste gew"ahren w"urde, wo ich
ihn unter Tr"anen bitte und ihm meine ganze Liebe wiederbringe, da
st"o"st er mich zur"uck, -- weil{\s} ihn dreitausend Franken
kosten k"onnte!"'

"`Ich habe sie nicht"', wiederholte Rudolf mit der Gelassenheit,
hinter die sich zornige Naturen wie hinter einen Schild zu bergen
pflegen.

Sie ging.

Die W"ande schwankten, die Decke drohte sie zu erdr"ucken. Wieder
nahm sie ihren Weg durch den langen Lindengang, "uber Haufen
welken Laub{\s}, da{\s} der Wind aufw"uhlte. Endlich stand sie vor
dem Gittertor. Sie zerbrach sich die N"agel an seinem Schlo"s, so
hastig wollte sie e{\s} "offnen. Hundert Schritte weiter blieb sie
v"ollig au"ser Atem stehn und konnte sich kaum noch aufrecht
halten. Wie sie sich umwandte, sah sie noch einmal auf da{\s}
still daliegende Herrenhau{\s} mit seinen langen Fensterreihen,
auf den Park, die H"ofe und die G"arten.

Wie in einer Bet"aubung stand sie da. Sie empfand kaum noch
etwa{\s} andre{\s} al{\s} da{\s} Pochen und Pulsen de{\s}
Blute{\s} in ihren Adern, da{\s} ihr au{\s} dem K"orper zu
springen und wie laute Musik da{\s} ganze Land ring{\s} um sie zu
durchrauschen schien. Der Boden unter ihren F"u"sen kam ihr
weicher vor al{\s} Wasser, und die Furchen der Felder am Wege
erschienen ihr wie lange braune Wellen, die auf und nieder wogten.
Alle{\s}, wa{\s} ihr im Kopfe lebte, alle Erinnerungen und
Gedanken sprangen auf einmal herau{\s}, mit tausend Funken wie ein
Feuerwerk. Sie sah ihren Vater vor sich, dann da{\s} Kontor de{\s}
Wucherer{\s}, ihr Zimmer zu Hau{\s}, dann irgendeine Landschaft,
immer wieder etwa{\s} andre{\s}. Da{\s} war heller Wahnsinn! Ihr
ward bange. Da raffte sie ihre letzten Kr"afte zusammen. E{\s} war
nur noch wenig Verstand in ihr, denn sie erinnerte sich nicht mehr
an die Ursache ihre{\s} schrecklichen Zustande{\s}, da{\s} hei"st
an die Geldfrage. Sie litt einzig an ihrer Liebe, und sie f"uhlte,
wie ihr durch die alten Erinnerungen die Seele dahinschwand, so
wie zu Tode Verwundete ihr Leben mit dem Blute ihrer Wunde
hinstr"omen f"uhlen.

Die Nacht brach herein. Raben flogen.

E{\s} schien ihr pl"otzlich, al{\s} sausten feurige Kugeln durch
die Luft. Sie kreisten und kreisten, um schlie"slich im Schnee
zwischen den kahlen "Asten der B"aume zu zergehen. In jeder
erschien Rudolf{\s} Gesicht. Sie wurden immer zahlreicher; sie
kamen immer n"aher; sie bedrohten sie. Da, pl"otzlich waren sie
alle verschwunden ... Jetzt erkannte sie die Lichter der H"auser,
die von ferne durch den Nebel schimmerten.

Nun ward sie sich auch wieder ihrer Not bewu"st, ihre{\s} tiefen
Elend{\s}. Ihr klopfende{\s} Herz schien ihr die Brust zersprengen
zu wollen ... Aber mit einem Male f"ullte sich ihre Seele mit
einem beinahe freudigen Heldenmut, und so schnell sie konnte, lief
sie den Abhang hinunter, "uberschritt die Planke "uber dem Bach,
eilte durch die Allee, an den Hallen vorbei, bi{\s} sie vor der
Apotheke stand.

E{\s} war niemand im Laden. Sie wollte eintreten, aber da{\s}
Ger"ausch der Klingel h"atte sie verraten k"onnen. De{\s}halb ging
sie durch die Hau{\s}t"ure; kaum atmend, tastete sie an der Wand
der Hau{\s}flur hin bi{\s} zur K"uchent"ure. Drinnen brannte eine
Kerze "uber dem Herd. Justin, in Hemd{\s}"armeln, trug gerade eine
Sch"ussel durch die andere T"ur hinau{\s}.

"`So! Man ist bei Tisch. Ich will warten"', sagte sie sich.

Al{\s} er zur"uckkam, klopfte sie gegen die Scheibe der
K"uchent"ure.

Er kam herau{\s}.

"`Den Schl"ussel! Den von oben, wo die~..."'

Er sah sie an und erschrak "uber ihr blasse{\s} Gesicht, da{\s}
sich vom Dunkel der Nacht grell abhob. Sie kam ihm "uberirdisch
sch"on vor und hoheit{\s}voll wie eine Fee. Ohne zu begreifen,
wa{\s} sie wollte, ahnte er doch etwa{\s} Schreckliche{\s}.

Sie begann wieder, hastig, aber mit sanfter Stimme, die ihm da{\s}
Herz r"uhrte:

"`Ich will ihn haben! Gib ihn mir!"'

Durch die d"unne Wand h"orte man da{\s} Klappern der Gabeln auf
den Tellern im E"szimmer.

Sie gebrauche etwa{\s}, um die Ratten zu t"oten, die sie nicht
schlafen lie"sen.

"`Ich m"u"ste den Herrn Apotheker rufen."'

"`Nein! Nicht!"' Und in gleichg"ultigem Tone setzte sie hinzu:
"`Da{\s} ist nicht n"otig. Ich werd e{\s} ihm nachher selber
sagen. Leucht mir nur!"' Sie trat in den Gang, von dem au{\s} man
in da{\s} Laboratorium gelangte. An der Wand hing ein Schl"ussel
mit einem Schildchen: "`Kapernaum."'

"`Justin!"' rief drinnen der Apotheker, dem der Lehrling zu lange
wegblieb.

"`Gehn wir hinauf!"' befahl Emma.

Er folgte ihr.

Der Schl"ussel drehte sich im Schlo"s. Sie st"urzte nach link{\s},
griff nach dem dritten Wandbrett -- ihr Ged"achtni{\s} f"uhrte sie
richtig --, hob den Deckel der blauen Gla{\s}b"uchse, fa"ste mit
der Hand hinein und zog die Faust voll wei"sen Pulver{\s}
herau{\s}, da{\s} sie sich schnell in den Mund sch"uttete.

"`Halten Sie ein!"' schrie Justin, ihr in die Arme fallend.

"`Still! Man k"onnte kommen!"'

Er war verzweifelt und wollte um Hilfe rufen.

"`Sag nicht{\s} davon! Man k"onnte deinen Herrn zur Verantwortung
ziehen!"'

Dann ging sie hinau{\s}, pl"otzlich voller Frieden, im seligen
Gef"uhle, eine Pflicht erf"ullt zu haben.


\newpage\begin{center}
{\large \so{Neunte{\s} Kapitel}}\bigskip\bigskip
\end{center}

Emma hatte eben da{\s} Hau{\s} verlassen, al{\s} Karl heimkam. Die
Nachricht von der Pf"andung traf ihn wie ein Keulenschlag. Dazu
seine Frau fort! Er schrie, weinte und fiel in Ohnmacht. Wa{\s}
n"utzte da{\s}? Wo konnte sie nur sein? Er schickte Felicie zu
Homai{\s}, zu T"uvache, zu Lheureux, nach dem Goldenen L"owen,
"uberallhin. Und mitten in seiner Angst um Emma qu"alte ihn der
Gedanke, da"s sein guter Ruf vernichtet, ihr gemeinsame{\s}
Verm"ogen verloren und die Zukunft Berta{\s} zerst"ort sei. Und
warum? Keine Erkl"arung! Er wartete bi{\s} sech{\s} Uhr abend{\s}.
Endlich hielt er{\s} nicht mehr au{\s}, und da er vermutete, sie
sei nach Rouen gefahren, ging er ihr auf der Landstra"se eine
halbe Wegstunde weit entgegen. Niemand kam. Er wartete noch eine
Weile und kehrte dann zur"uck.

Sie war zu Hau{\s}.

"`Wa{\s} ist da{\s} f"ur eine Geschichte? Wie ist da{\s} gekommen?
Erkl"ar e{\s} mir!"'

Sie sa"s an ihrem Schreibtisch und beendete gerade einen Brief,
den sie langsam versiegelte, nachdem sie Tag und Stunde darunter
gesetzt hatte. Dann sagte sie in feierlichem Tone:

"`Du wirst ihn morgen lesen! Bi{\s} dahin bitte ich dich, keine
einzige Frage an mich zu richten! Keine, bitte!"'

"`Aber~..."'

"`Ach, la"s mich!"'

Sie legte sich lang auf ihr Bett.

Ein bitterer Geschmack im Munde weckte sie auf. Sie sah Karl ...
verschwommen ... und schlo"s die Augen wieder.

Sie beobachtete sich aufmerksam, um Schmerzen fest\/zustellen. Nein,
sie f"uhlte noch keine! Sie h"orte den Pendelschlag der Uhr,
da{\s} Knistern de{\s} Feuer{\s} und Karl{\s} Atemz"uge, der neben
ihrem Bett stand.

"`Ach, der Tod ist gar nicht{\s} Schlimme{\s}!"' dachte sie. "`Ich
werde einschlafen, und dann ist alle{\s} vor"uber!"'

Sie trank einen Schluck Wasser und drehte sich der Wand zu.

Der abscheuliche Tintengeschmack war immer noch da.

"`Ich habe Durst! Gro"sen Durst!"' seufzte sie.

"`Wa{\s} fehlt dir denn?"' fragte Karl und reichte ihr ein
Gla{\s}.

"`E{\s} ist nicht{\s}! ... Mach da{\s} Fenster auf! ... Ich
ersticke!"'

Ein Brechreiz "uberkam sie jetzt so pl"otzlich, da"s sie kaum noch
Zeit hatte, ihr Taschentuch unter dem Kopfkissen hervorzuziehen.

"`Nimm{\s} weg!"' sagte sie nerv"o{\s}. "`Wirf{\s} weg!"'

Er fragte sie au{\s}, aber sie antwortete nicht. Sie lag
unbeweglich da, au{\s} Furcht, sich bei der geringsten Bewegung
erbrechen zu m"ussen. Inzwischen f"uhlte sie eine eisige K"alte
von den F"u"sen zum Herzen hinaufsteigen.

"`Ach,"' murmelte sie, "`jetzt f"angt e{\s} wohl an?"'

"`Wa{\s} sagst du?"'

Sie warf den Kopf in unterdr"uckter Unruhe hin und her.
Fortw"ahrend "offnete sie den Mund, al{\s} l"age etwa{\s}
Schwere{\s} auf ihrer Zunge. Um acht Uhr fing da{\s} Erbrechen
wieder an.

Karl bemerkte auf dem Boden de{\s} Napfe{\s} einen wei"sen
Niederschlag, der sich am Porzellan ansetzte.

"`Sonderbar! Sonderbar!"' wiederholte er.

Aber sie sagte mit fester Stimme:

"`Nein, du irrst dich!"'

Da fuhr er ihr mit der Hand zart, wie liebkosend, bi{\s} in die
Magengegend und dr"uckte da. Sie stie"s einen schrillen Schrei
au{\s}. Er wich erschrocken zur"uck.

Dann begann sie zu wimmern, zuerst nur leise. Ein Sch"uttelfrost
"uberfiel sie. Sie wurde bleicher al{\s} da{\s} Bettuch, in da{\s}
sich ihre Finger krampfhaft einkrallten. Ihr unregelm"a"siger
Pul{\s}schlag war kaum noch f"uhlbar. Kalte Schwei"stropfen rannen
"uber ihr bl"aulich gewordne{\s} Gesicht; etwa{\s} wie ein
metallischer Au{\s}schlag lag "uber ihren erstarrten Z"ugen. Die
Z"ahne schlugen ihr klappernd aufeinander. Ihre erweiterten Augen
blickten au{\s}druck{\s}lo{\s} umher. Alle Fragen, die man an sie
richtete, beantwortete sie nur mit Kopfnicken. Zwei- oder dreimal
l"achelte sie freilich. Allm"ahlich wurde da{\s} St"ohnen
heftiger. Ein dumpfe{\s} Geheul entrang sich ihr. Dabei behauptete
sie, da"s e{\s} ihr besser gehe und da"s sie sofort aufstehen
w"urde.

Sie verfiel in Zuckungen. Sie schrie:

"`Mein Gott, ist da{\s} gr"a"slich!"'

Karl warf sich vor ihrem Bett auf die Knie.

"`Sprich! Wa{\s} hast du gegessen? Um Gotte{\s} willen, antworte
mir!"'

Er sah sie an mit Augen voller Z"artlichkeit, wie Emma keine je
geschaut hatte.

"`Ja ... da ... da ... lie{\s}!"' stammelte sie mit versagender
Stimme.

Er st"urzte zum Schreibtisch, ri"s den Brief auf und la{\s} laut:

"`Man klage niemanden an~..."' Er hielt inne, fuhr sich mit der
Hand "uber die Augen und la{\s} stumm weiter~...

"`Vergiftet!"'

Er konnte immer nur da{\s} eine Wort herau{\s}bringen:

"`Vergiftet! Vergiftet!"'

Dann rief er um Hilfe.

Felicie lief zu Homai{\s}, der e{\s} aller Welt au{\s}posaunte.
Frau Franz im Goldenen L"owen erfuhr e{\s}. Manche standen au{\s}
ihren Betten auf, um e{\s} ihren Nachbarn mit\/zuteilen. Die ganze
Nacht hindurch war der halbe Ort wach.

Halb von Sinnen, vor sich hinredend, nahe am Hinfallen, lief Karl
im Zimmer umher, wobei er an die M"obel anrannte und sich Haare
au{\s}raufte. Der Apotheker hatte noch nie ein so f"urchterliche{\s}
Schauspiel gesehen.

Er ging nach Hause, um an den Doktor Canivet und den Professor
Larivi\`ere zu schreiben. Er hatte selber den Kopf verloren. Er
brachte keinen vern"unftigen Brief zustande. Schlie"slich mu"ste
sich Hippolyt nach Neufch\^atel aufmachen, und Justin ritt auf
Bovary{\s} Pferd nach Rouen. Am Wilhelm{\s}walde lie"s er den Gaul
lahm und halbtot zur"uck.

Karl wollte in seinem Medizinischen Lexikon nachschlagen, aber er
war nicht imstande zu lesen. Die Buchstaben tanzten ihm vor den
Augen.

"`Ruhe!"' sagte der Apotheker. "`E{\s} handelt sich einzig und
allein darum, ein wirksame{\s} Gegenmittel anzuwenden. Wa{\s} war
e{\s} f"ur ein Gift?"'

Karl zeigte den Brief. E{\s} w"are Arsenik gewesen.

"`Gut!"' versetzte Homai{\s}. "`Wir m"ussen eine Analyse machen!"'

Er hatte n"amlich gelernt, da"s man bei allen Vergiftungen eine
Analyse machen m"usse. Bovary hatte in seiner Angst alle
Gelehrsamkeit vergessen. Er erwiderte ihm:

"`Ja! Machen Sie eine. Tun Sie e{\s}! Retten Sie sie!"'

Dann kehrte er in ihr Zimmer zur"uck, warf sich auf die Diele,
lehnte den Kopf gegen den Rand ihre{\s} Bette{\s} und schluchzte.

"`Weine nicht!"' fl"usterte sie. "`Bald werde ich dich nicht mehr
qu"alen!"'

"`Warum hast du da{\s} getan? Wa{\s} trieb dich dazu?"'

"`E{\s} mu"ste sein, mein Lieber!"'

"`Warst du denn nicht gl"ucklich? Bin ich schuld? Ich habe dir
doch alle{\s} zuliebe getan, wa{\s} ich konnte!"'

"`Ja ... freilich ... Du bist gut ... du!"'

Sie strich ihm langsam mit der Hand "uber da{\s} Haar. Die s"u"se
Empfindung vermehrte seine Traurigkeit. Er f"uhlte sich bi{\s} in
den tiefsten Grund seiner verzweifelten Seele ersch"uttert, da"s
er sie verlieren sollte, jetzt, da sie ihm mehr Liebe bewie{\s}
denn je. Er fand keinen Au{\s}weg; er wu"ste keinen Zusammenhang;
er wagte keine Frage. Und die Dringlichkeit eine{\s}
Entschlusse{\s} machte ihn vollend{\s} wirr.

Sie dachte bei sich: "`Nun ist e{\s} zu Ende mit dem vielfachen
Verrat, mit allen den Erniedrigungen und den unz"ahligen,
qualvollen Sehns"uchten!"' Nun ha"ste sie keinen mehr. Ihre
Gedanken verschwammen wie in D"ammerung, und von allen Ger"auschen
der Erde h"orte Emma nur noch die versagende Klage eine{\s} armen
Herzen{\s}, matt und verklungen wie der leise Nachhall einer
Symphonie.

"`Bring mir die Kleine"', sagte sie und st"utzte sich leicht auf.

"`E{\s} ist nicht schlimmer, nicht wahr?"' fragte Karl.

"`Nein, nein!"'

Da{\s} Dienstm"adchen trug da{\s} Kind auf dem Arm herein. E{\s}
hatte ein lange{\s} Nachthemd an, au{\s} dem die nackten F"u"se
hervorsahen. E{\s} war ernst und noch halb im Schlaf. Erstaunt
betrachtete e{\s} die gro"se Unordnung im Zimmer. Geblendet vom
Licht der Kerzen, die da und dort brannten, zwinkerte e{\s} mit
den Augen. Offenbar dachte e{\s}, e{\s} sei
Neujahr{\s}tag{\s}morgen, an dem e{\s} auch so fr"uh wie heute
geweckt wurde und beim Kerzenschein zur Mutter an{\s} Bett kam, um
Geschenke zu bekommen. Und so fragte e{\s}:

"`Wo ist e{\s} denn, Mama?"' Und da niemand antwortete, redete
e{\s} weiter: "`Ich seh doch meine Schuhchen gar nicht!"'

Felicie hielt die Kleine "uber{\s} Bett, die immer noch nach dem
Kamin hinsah.

"`Hat Frau Rollet sie mir genommen?"'

Bei diesem Namen, der an ihre Ehebr"uche und all ihr Mi"sgeschick
erinnerte, wandte sich Frau Bovary ab, al{\s} f"uhle sie den
ekelhaften Geschmack eine{\s} noch viel st"arkeren Gifte{\s} auf
der Zunge. Berta sa"s noch auf ihrem Bette.

"`Wa{\s} f"ur gro"se Augen du hast, Mama! Wie bla"s du bist! Wie
du schwitzest!"'

Die Mutter sah sie an.

"`Ich f"urchte mich!"' sagte die Kleine und wollte fort.

Emma wollte die Hand de{\s} Kinde{\s} k"ussen, aber e{\s}
str"aubte sich.

"`Genug! Bringt sie weg!"' rief Karl, der im Alkoven schluchzte.

Dann lie"sen die Symptome einen Augenblick nach. Emma schien
weniger aufgeregt, und bei jedem unbedeutenden Worte, bei jedem
etwa{\s} ruhigeren Atemzug sch"opfte er neue Hoffnung. Al{\s}
Canivet endlich erschien, warf er sich weinend in seine Arme.

"`Ach, da sind Sie! Ich danke Ihnen! E{\s} ist g"utig von Ihnen!
E{\s} geht ja besser! Da! Sehen Sie mal~..."'

Der Kollege war keine{\s}weg{\s} dieser Meinung, und da er, wie er
sich au{\s}dr"uckte, "`immer auf{\s} Ganze"' ging, verordnete er
Emma ein ordentliche{\s} Brechmittel, um den Magen zun"achst
einmal v"ollig zu entleeren.

Sie brach al{\s}bald Blut au{\s}. Ihre Lippen pre"sten sich
krampfhaft aufeinander. Sie zog die Gliedma"sen ein. Ihr K"orper
war bedeckt mit braunen Flecken, und ihr Pul{\s} glitt unter ihren
Fingern hin wie ein d"unne{\s} F"adchen, da{\s} jeden Augenblick
zu zerrei"sen droht.

Dann begann sie, gr"a"slich zu schreien. Sie verfluchte und
schm"ahte da{\s} Gift, flehte, e{\s} m"oge sich beeilen, und
stie"s mit ihren steif gewordnen Armen alle{\s} zur"uck, wa{\s}
Karl ihr zu trinken reichte. Er war der v"olligen Aufl"osung noch
n"aher al{\s} sie. Sein Taschentuch an die Lippen gepre"st, stand
er vor ihr, st"ohnend, weinend, von ruckweisem Schluchzen
ersch"uttert und am ganzen Leib durchr"uttelt. Felicie lief im
Zimmer hin und her, Homai{\s} stand unbeweglich da und seufzte
tief auf, und Canivet begann sich, trotz seiner ihm zur Gewohnheit
gewordnen selbstbewu"sten Haltung, unbehaglich zu f"uhlen.

"`Zum Teufel!"' murmelte er. "`Der Magen ist nun doch leer! Und
wenn die Ursache beseitigt ist, so~..."'

"`... mu"s die Wirkung aufh"oren!"' erg"anzte Homai{\s}. "`Da{\s}
ist klar!"'

"`Rettet sie mir nur!"' rief Bovary.

Der Apotheker ri{\s}kierte die Hypothese, e{\s} sei vielleicht ein
heilsamer Paroxi{\s}mu{\s}. Aber Canivet achtete nicht darauf und
wollte ihr gerade Theriak eingeben, da knallte drau"sen eine
Peitsche. Alle Fensterscheiben klirrten. Eine Extrapost mit drei
bi{\s} an die Ohren von Schmutz bedeckten Pferden raste um die
Ecke der Hallen. E{\s} war Professor Larivi\`ere.

Die Erscheinung eine{\s} Gotte{\s} h"atte keine gr"o"sere Erregung
hervorrufen k"onnen. Bovary streckte ihm die H"ande entgegen,
Canivet stand bewegung{\s}lo{\s} da, und Homai{\s} nahm sein
K"appchen ab, noch ehe der Arzt eingetreten war.

Larivi\`ere geh"orte der ber"uhmten Chirurgenschule Bichat{\s} an,
da{\s} hei"st, einer Generation philosophischer Praktiker, die
heute au{\s}gestorben ist, begeisterter, gewissenhafter und
scharfsichtiger J"unger ihrer Kunst. Wenn er in Zorn geriet, wagte
in der ganzen Klinik niemand zu atmen. Seine Sch"uler verehrten
ihn so, da"s sie ihn, sp"ater in ihrer eigenen Praxi{\s}, mit
m"oglichster Genauigkeit kopierten. So kam e{\s}, da"s man bei den
"Arzten in der Umgegend von Rouen allerort{\s} seinen langen
Schaf{\s}pelz und seinen weiten schwarzen Gehrock wiederfand. Die
offenen "Armelaufschl"age daran reichten ein St"uck "uber seine
fleischigen H"ande, sehr sch"one H"ande, die niemal{\s} in
Handschuhen steckten, al{\s} wollten sie immer schnell bereit
sein, wo e{\s} Krankheit und Elend anzufassen galt. Er war ein
Ver"achter von Orden, Titeln und Akademien, gastfreundlich,
freidenkend, den Armen ein v"aterlicher Freund, Pessimist, selbst
aber edel in Wort und Tat. Man h"atte ihn al{\s} einen Heiligen
gepriesen, wenn man ihn nicht wegen seine{\s} Witze{\s} und
Verstande{\s} gef"urchtet h"atte wie den Teufel. Sein Blick war
sch"arfer al{\s} sein Messer; er drang einem bi{\s} tief in die
Seele, durch alle Heucheleien, L"ugen und Au{\s}fl"uchte hindurch.
So ging er seine{\s} Wege{\s} in der schlichten W"urde, die ihm
da{\s} Bewu"stsein seiner gro"sen T"uchtigkeit, seine{\s}
materiellen Verm"ogen{\s} und seiner vierzigj"ahrigen
arbeit{\s}reichen und unanfechtbaren Wirksamkeit verlieh.

Al{\s} er da{\s} leichenhafte Antlitz Emma{\s} sah, zog er schon
von weitem die Brauen hoch. Sie lag mit offnem Munde auf dem
R"ucken au{\s}gestreckt da. W"ahrend er Canivet{\s} Bericht
scheinbar aufmerksam anh"orte, strich er sich mit dem Zeigefinger
um die Nasenfl"ugel und sagte ein paarmal:

"`Gut! ... Gut!"'

Dann aber zuckte er bedenklich mit den Achseln. Bovary beobachtete
ihn "angstlich. Sie sahen einander in die Augen, und der Gelehrte,
der an den Anblick menschlichen Elend{\s} so gew"ohnt war, konnte
eine Tr"ane nicht zur"uckhalten, die ihm auf die Krawatte
herablief.

Er wollte Canivet in da{\s} Nebenzimmer ziehen. Karl folgte ihnen.

"`E{\s} steht wohl nicht gut mit meiner Frau? Wie w"ar e{\s}, wenn
man ihr ein Senfpflaster auflegte? Ich wei"s nicht{\s}. Finden Sie
doch etwa{\s}! Sie haben ja schon so viele gerettet!"'

Karl legte beide Arme auf Larivi\`ere{\s} Schultern und starrte
ihn verst"ort und flehend an. Beinahe w"are er ihm ohnm"achtig an
die Brust gesunken.

"`Mut! Mein armer Junge! E{\s} ist nicht{\s} mehr zu machen!"'
Larivi\`ere wandte sich ab.

"`Sie gehn?"'

"`Ich komme wieder."'

Larivi\`ere ging hinau{\s}, angeblich um dem Postillion eine
Anweisung zu geben. Canivet folgte ihm. Auch er wollte nicht Zeuge
de{\s} Tode{\s}kampfe{\s} sein.

Der Apotheker holte die beiden auf dem Marktplatz ein. Nicht{\s}
fiel ihm von jeher schwerer, al{\s} sich von ber"uhmten Menschen
zu trennen. So beschwor er denn Larivi\`ere, er m"oge ihm die hohe
Ehre erweisen, zum Fr"uhst"uck sein Gast zu sein.

Man schickte ganz rasch nach dem Goldnen L"owen nach Tauben, zu
T"uvache nach Sahne, zu Lestiboudoi{\s} nach Eiern und zum
Fleischer nach Kotelett{\s}. Der Apotheker war selbst bei den
Vorbereitungen zum Mahle behilflich, und Frau Homai{\s}, sich ihre
Jacke zurechtzupfend, sagte:

"`Sie m"ussen schon entschuldigen, Herr Professor, man ist in so
einer weggesetzten Gegend nicht immer gleich vorbereitet~..."'

"`Die Weingl"aser!"' fl"usterte Homai{\s}.

"`Wer in der Stadt wohnt, der kann sich schnell helfen ... mit
Wurst und~..."'

"`Sei doch still! -- Zu Tisch, bitte, Herr Professor!"'

Er hielt e{\s} f"ur angebracht, nach den ersten Bissen ein paar
Einzelheiten "uber die Katastrophe zum besten zu geben:

"`Zuerst "au"serte sich Trockenheit im Pharynx, darauf
unertr"agliche gastrische Schmerzen, Neigung zum Vomieren,
Schlafsucht~..."'

"`Wie hat sich denn die Vergiftung eigentlich ereignet?"'

"`Habe keine Ahnung, Herr Professor! Ich wei"s nicht einmal recht,
wo sie da{\s} \begin{antiqua}acidum arsenicum\end{antiqua}
herbekommen hat."'

Justin, der einen Sto"s Teller hereinbrachte, begann am ganzen
K"orper zu zittern.

"`Wa{\s} hast du?"' fuhr ihn der Apotheker an.

Bei dieser Frage lie"s der Bursche alle{\s}, wa{\s} er trug,
fallen. E{\s} gab ein gro"se{\s} Gekrache.

"`Tolpatsch!"' schrie Homai{\s}. "`Ungeschickter Kerl! Tranlampe!
Alberner Esel!"'

Dann aber beherrschte er sich pl"otzlich:

"`Ich habe gleich daran gedacht, eine Analyse zu machen, Herr
Professor, und de{\s}halb \begin{antiqua}primo\end{antiqua} ganz
vorsichtig in ein Reagenzgl"aschen~..."'

"`Dienlicher w"are e{\s} gewesen,"' sagte der Chirurg, "`wenn Sie
ihr Ihre Finger in den Hal{\s} gesteckt h"atten."'

Kollege Canivet sagte gar nicht{\s} dazu, dieweil er soeben unter
vier Augen eine energische Belehrung wegen seine{\s}
Brechmittel{\s} eingesteckt hatte. Er, der bei Gelegenheit de{\s}
Klumpfu"se{\s} so hochfahrend und redselig gewesen war, verhielt
sich jetzt m"auschenstill. Er l"achelte nur unau{\s}gesetzt, um
seine Zustimmung zu markieren.

Homai{\s} strahlte vor Hau{\s}herrenstolz. Selbst der betr"ubliche
Gedanke an Bovary trug -- in egoistischer Kontrastwirkung --
unbestimmt zu seiner Freude bei. Die Anwesenheit de{\s} ber"uhmten
Arzte{\s} stieg ihm in den Kopf. Er kramte seine ganze
Gelehrsamkeit au{\s}. Kunterbunt durcheinander schwatzte er von
Kanthariden, Pflanzengiften, Manzanilla, Schlangengift usw.

"`Ich habe sogar einmal gelesen, Herr Professor, da"s mehrere
Personen nach dem Genusse von zu stark ger"aucherter Wurst
erkrankt und pl"otzlich gestorben sind. So berichtet wenigsten{\s}
ein hochinteressanter Aufsatz eine{\s} unserer hervorragendsten
Pharmazeuten, eine{\s} Klassiker{\s} meiner Wissenschaft, ... ein
Aufsatz de{\s} ber"uhmten Cadet de Gassicourt!"'

Frau Homai{\s} erschien mit der Kaffeemaschine. Homai{\s} pflegte
sich n"amlich den Kaffee nach Tisch selbst zu bereiten. Er hatte
ihn auch eigenh"andig gemischt, gebrannt und gemahlen.

"`\begin{antiqua}Saccharum\end{antiqua} gef"allig, Herr
Professor?"' fragte er, indem er ihm den Zucker anbot.

Dann lie"s er alle seine Kinder herunterkommen, da er neugierig
war, die Ansicht de{\s} Chirurgen "uber ihre "`Konstitution"' zu
h"oren.

Al{\s} Larivi\`ere im Begriffe stand aufzubrechen, bat ihn Frau
Homai{\s} noch um einen "arztlichen Rat in betreff ihre{\s}
Manne{\s}. Er schlief n"amlich allabendlich nach Tisch ein. Davon
bek"ame er dicke{\s} Blut.

Der Arzt antwortete mit einem Scherze, dessen doppelten Sinn sie
nicht verstand, dann ging er zur T"ure. Aber die Apotheke war
voller Leute, die ihn konsultieren wollten, und e{\s} gelang ihm
nur schwer, sie lo{\s}zuwerden. Da war T"uvache, der seine Frau
f"ur schwinds"uchtig hielt, weil sie "ofter{\s} in die Asche
spuckte; Binet, der bi{\s}weilen an Hei"shunger litt; Frau Caron,
die e{\s} am ganzen Leibe juckte; Lheureux, der Schwindelanf"alle
hatte; Lestiboudoi{\s}, der rheumatisch war; Frau Franz, die "uber
Magenbeschwerden klagte. Endlich brachten ihn die drei Pferde von
dannen. Man fand aber allgemein, da"s er sich nicht besonder{\s}
lieben{\s}w"urdig gezeigt habe.

Nunmehr wurde die Aufmerksamkeit auf den Pfarrer Bournisien
gelenkt, der mit dem Sterbesakrament an den Hallen hinging.

Seiner Weltanschauung treu, verglich Homai{\s} die Geistlichen mit
den Raben, die der Leichengeruch anlockt. Der Anblick eine{\s}
"`Pfaffen"' war ihm ein Greuel. Er mu"ste bei einer Soutane immer
an ein Leichentuch denken, und so verw"unschte er jene schon
de{\s}halb, weil er diese{\s} f"urchtete.

Trotzdem verzichtete er nicht auf die gewissenhafte Erf"ullung
seiner "`Mission"', wie er e{\s} nannte, und kehrte mit Canivet,
dem die{\s} von Larivi\`ere dringend an{\s} Herz gelegt worden
war, in da{\s} Bovarysche Hau{\s} zur"uck. Wenn seine Frau nicht
v"ollig dagegen gewesen w"are, h"atte er sogar seine beiden Knaben
mitgenommen, damit sie da{\s} gro"se Ereigni{\s}, da{\s} der Tod
eine{\s} Menschen ist, kennen lernten. E{\s} sollte ihnen eine
Lehre, ein Beispiel, ein ernster Eindruck sein, eine Erinnerung
f"ur ihr ganze{\s} weitere{\s} Leben.

Sie fanden da{\s} Zimmer voll d"ustrer Feierlichkeit. Auf dem mit
einem wei"sen Tischtuch bedeckten N"ahtische stand zwischen zwei
brennenden Wach{\s}kerzen ein hohe{\s} Kruzifix; daneben eine
silberne Sch"ussel und f"unf oder sech{\s} St"uck Watte. Emma{\s}
Kinn war ihr auf die Brust hinabgesunken, ihre Augen standen
unnat"urlich weit offen, und ihre armen H"ande tasteten "uber den
Bett"uberzug hin, mit einer jener r"uhrend-schrecklichen
Geb"arden, die Sterbenden eigen sind. Man hat die Empfindung,
al{\s} bereiteten sie sich selber ihr Totenbett. Karl stand am
Fu"sende de{\s} Lager{\s}, ihrem Antlitz gegen"uber, bleich wie
eine Bilds"aule, tr"anenlo{\s}, aber mit Augen, die rot waren wie
gl"uhende Kohlen. Der Priester kniete und murmelte leise Worte.

Emma wandte langsam ihr Haupt und empfand beim Anblick der
violetten Stola sichtlich Freude. Offenbar f"uhlte sie einen
seltsamen Frieden, eine Wiederholung derselben mystischen Wollust,
die sie schon einmal erlebt hatte. Etwa{\s} wie eine Vision von
himmlischer Gl"uckseligkeit bet"aubte ihre letzten Leiden.

Der Priester erhob sich und ergriff da{\s} Kruzifix. Da reckte sie
den Kopf in die H"ohe, wie ein Durstiger, und pre"ste auf da{\s}
Symbol de{\s} Gott-Menschen mit dem letzten Rest ihrer Kraft den
innigsten Liebe{\s}ku"s, den sie jemal{\s} gegeben hatte. Dann
sprach der Geistliche da{\s}
\begin{antiqua}Misereatur\end{antiqua} und
\begin{antiqua}Indulgentiam\end{antiqua}, tauchte seinen rechten
Daumen in da{\s} "Ol und nahm die letzte "Olung vor. Zuerst salbte
er die Augen, die e{\s} nach allem Herrlichen auf Erden so hei"s
gel"ustet; dann die Nasenfl"ugel, die so gern die lauen L"ufte und
die D"ufte der Liebe eingesogen; dann den Mund, der so oft zu
L"ugen sich aufgetan, oft hoff"artig gezuckt und in s"undigem
Girren geseufzt hatte; dann die H"ande, die sich an vergn"uglichen
Ber"uhrungen erg"otzt hatten; und endlich die Sohlen der F"u"se,
die einst so flink waren, wenn sie zur Stillung von Begierden
liefen, und die jetzt keinen Schritt mehr tun sollten.

Der Priester trocknete sich die H"ande, warf da{\s} "olgetr"ankte
St"uck Watte in{\s} Feuer und setzte sich wieder zu der
Sterbenden. Er sagte ihr, da"s ihre Leiden nunmehr mit denen Jesu
Christi ein{\s} seien. Sie solle der g"ottlichen Barmherzigkeit
vertrauen.

Al{\s} er mit seiner Tr"ostung zu Ende war, versuchte er, ihr eine
geweihte Kerze in die Hand zu dr"ucken, da{\s} Symbol der
himmlischen Glorie, von der sie nun bald umstrahlt sein sollte.
Aber Emma war zu schwach, um die Finger zu schlie"sen, und wenn
Bournisien nicht rasch wieder zugegriffen h"atte, w"are die Kerze
zu Boden gefallen.

Emma war nicht mehr so bleich wie erst. Ihr Gesicht hatte den
Au{\s}druck heiterer Gl"uckseligkeit angenommen, al{\s} ob da{\s}
Sakrament sie wieder gesund gemacht h"atte.

Der Priester verfehlte nicht, die Umstehenden darauf hinzuweisen,
ja er gemahnte Bovary daran, da"s der Herr zuweilen da{\s} Leben
Sterbender wieder verl"angere, wenn er e{\s} zum Heil ihrer Seele
f"ur notwendig erachte. Karl dachte an den Tag zur"uck, an dem sie
schon einmal, dem Tode nahe, die letzte "Olung empfangen hatte.

"`Vielleicht brauche ich noch nicht zu verzweifeln!"' dachte er.

Wirklich sah sie sich langsam um wie jemand, der au{\s} einem
Traum erwacht. Dann verlangte sie mit deutlicher Stimme ihren
Spiegel und betrachtete darin eine Weile ihr Bild, bi{\s} ihr die
Tr"anen au{\s} den Augen rollten. Darnach legte sie den Kopf
zur"uck, stie"s einen Seufzer au{\s} und sank in da{\s} Kissen.

Ihre Brust begann al{\s}bald heftig zu keuchen. Die Zunge trat
weit au{\s} dem Munde. Die Augen begannen zu rollen und ihr Licht
zu verlieren wie zwei Lampenglocken, hinter denen die Flammen
verl"oschen. Man h"atte glauben k"onnen, sie sei schon tot, wenn
ihre Atmung{\s}organe nicht so f"urchterlich heftig gearbeitet
h"atten. E{\s} war, al{\s} sch"uttle sie ein wilder innerer Sturm,
al{\s} ringe da{\s} Leben gewaltig mit dem Tode.

Felicie kniete vor dem Kruzifix, und sogar der Apotheker knickte
ein wenig die Beine, w"ahrend Canivet gleichg"ultig auf den Markt
hinau{\s}starrte. Bournisien hatte wieder zu beten begonnen, die
Stirn gegen den Rand de{\s} Bette{\s} geneigt, weit hinter sich
die lange schwarze Soutane. An der andern Seite de{\s} Bette{\s}
kniete Karl und streckte beide Arme nach Emma au{\s}. Er ergriff
ihre H"ande und dr"uckte sie! Bei jedem Schlag ihre{\s} Pulse{\s}
zuckte er zusammen, al{\s} st"urze eine Ruine auf ihn.

Je st"arker da{\s} R"ocheln wurde, um so mehr beschleunigte der
Priester seine Gebete. Sie mischten sich mit dem erstickten
Schluchzen Bovary{\s}, und zuweilen vernahm man nicht{\s} al{\s}
da{\s} dumpfe Murmeln der lateinischen Worte, da{\s} wie
Totengel"aut klang.

Pl"otzlich klapperten drau"sen auf der Stra"se Holzschuhe. Ein
Stock schlug mehrere Male auf, und eine Stimme erhob sich, eine
rauhe Stimme, und sang:

\begin{verse}
{\glq}Wenn{\s} Sommer worden weit und breit, \\
Wird hei"s da{\s} Herze mancher Maid~...{\grq}
\end{verse}

Emma richtete sich ein wenig auf, wie eine Leiche, durch die ein
elektrischer Strom geht. Ihr Haar hatte sich gel"ost, ihre
Augensterne waren starr, ihr Mund stand weit auf.

\begin{verse}
{\glq}Nanette ging hinau{\s} in{\s} Feld, \\
Zu sammeln, wa{\s} die Sense f"allt. \\
Al{\s} sie sich in der Stoppel b"uckt, \\
Da ist passiert, wa{\s} sich nicht schickt~...{\grq}
\end{verse}

"`Der Blinde!"' schrie sie.

Sie brach in Lachen au{\s}, in ein furchtbare{\s}, wahnsinnige{\s},
verzweifelte{\s} Lachen, weil sie in ihrer Phantasie da{\s}
scheu"sliche Gesicht de{\s} Ungl"ucklichen sah, wie ein
Schreckgespenst au{\s} der ewigen Nacht de{\s} Jenseit{\s}~...

\begin{verse}
{\glq}Der Wind, der war so stark ... O weh! \\
Hob ihr die R"ockchen in die H"oh.{\grq}
\end{verse}

Ein letzter Krampf warf sie in da{\s} Bett zur"uck. Alle traten
hinzu. Sie war nicht mehr.



\newpage\begin{center}
{\large \so{Zehnte{\s} Kapitel}}\bigskip\bigskip
\end{center}

Nach dem Tode eine{\s} Menschen sind die Umstehenden immer wie
bet"aubt. So schwer ist e{\s}, den Hereinbruch de{\s} ewigen
Nicht{\s} zu begreifen und sich dem Glauben daran zu ergeben. Karl
aber, al{\s} er sah, da"s Emma unbeweglich dalag, warf sich "uber
sie und schrie:

"`Lebwohl! Lebwohl!"'

Homai{\s} und Canivet zogen ihn au{\s} dem Zimmer.

"`Fassen Sie sich!"'

"`Ja!"' rief er und machte sich von ihnen lo{\s}. "`Ich will
vern"unftig sein! Ich tue ja nicht{\s}. Aber lassen Sie mich! Ich
mu"s sie sehen! E{\s} ist meine Frau!"'

Er weinte.

"`Weinen Sie nur!"' sagte der Apotheker. "`Lassen Sie der Natur
freien Lauf! Da{\s} wird Sie erleichtern!"'

Da wurde Karl schwach wie ein Kind und lie"s sich in die Gro"se
Stube im Erdgescho"s hinunterf"uhren. Homai{\s} ging bald darnach
in sein Hau{\s} zur"uck.

Auf dem Markte wurde er von dem Blinden angesprochen, der sich
bi{\s} Yonville geschleppt hatte, um die Salbe zu holen. Jeden
Vor"ubergehenden hatte er gefragt, wo der Apotheker wohne.

"`Gro"sartig! Al{\s} wenn ich gerade jetzt nicht schon genug zu
tun h"atte! Bedaure! Komm ein andermal!"'

Er verschwand schnell in seinem Hause.

Er hatte zwei Briefe zu schreiben, einen beruhigenden Trank f"ur
Bovary zu brauen und ein M"archen zu ersinnen, um Frau Bovary{\s}
Vergiftung auf eine m"oglichst harmlose Weise zu erkl"aren. Er
wollte einen Artikel f"ur den "`Leuchtturm von Rouen"' darau{\s}
machen. Au"serdem wartete eine Menge neugieriger Leute auf ihn.
Alle wollten Genauere{\s} wissen. Nachdem er mehreremal{\s}
wiederholt hatte, Frau Bovary habe bei der Zubereitung von
Vanillecreme au{\s} Versehen Arsenik statt Zucker genommen, begab
er sich abermal{\s} zu Bovary.

Er fand ihn allein. Canivet war eben fortgefahren. Karl sa"s im
Lehnstuhl am Fenster und starrte mit bl"odem Blick auf die Dielen.

"`Wir m"ussen die Stunde f"ur die Feierlichkeit festsetzen!"'
sagte der Apotheker.

"`Wozu? F"ur wa{\s} f"ur eine Feierlichkeit?"' Stammelnd und voll
Grauen f"ugte er hinzu: "`Nein, nein ... nicht wahr? Ich darf sie
dabehalten?"'

Um seine Haltung zu bewahren, nahm Homai{\s} die Wasserflasche vom
Tisch und bego"s die Geranien.

"`O, ich danke Ihnen!"' sagte Karl. "`Sie sind sehr g"utig~..."'

Er wollte noch mehr sagen, aber die F"ulle von Erinnerungen, die
de{\s} Apotheker{\s} Tun in ihm wachrief, "uberw"altigte ihn.
E{\s} waren Emma{\s} Blumen!

Homai{\s} gab sich M"uhe, ihn zu zerstreuen, und begann "uber die
G"artnerei zu plaudern. Die Pflanzen h"atten die Feuchtigkeit sehr
n"otig. Karl nickte zustimmend.

"`Jetzt werden auch bald sch"one Tage kommen~..."'

Bovary seufzte.

Der Apotheker wu"ste nicht mehr, wovon er reden sollte, und schob
behutsam eine Scheibengardine beiseite.

"`Sehn Sie, da dr"uben geht der B"urgermeister!"'

Karl wiederholte mechanisch:

"`Da dr"uben geht der B"urgermeister!"'

Homai{\s} wagte nicht, auf die Vorbereitungen zum Begr"abni{\s}
zur"uckzukommen. Erst der Pfarrer brachte Bovary zu einem
Entschlusse hier"uber.

Karl schlo"s sich in sein Sprechzimmer ein, ergriff die Feder, und
nachdem er eine Zeitlang geschluchzt hatte, schrieb er:

\begin{quotation}
"`Ich bestimme, da"s man meine Frau in ihrem Hochzeit{\s}kleid
begrabe, in wei"sen Schuhen, einen Kranz auf dem Haupte. Da{\s}
Haar soll man ihr "uber die Schultern legen. Drei S"arge: einen
au{\s} Eiche, einen au{\s} Mahagoni, einen von Blei. Man soll mich
nicht tr"osten wollen! Ich werde stark sein. Und "uber den Sarg
soll man ein gro"se{\s} St"uck gr"unen Samt breiten. So will ich
e{\s}! Tut e{\s}!"'
\end{quotation}

Man war "uber Bovary{\s} Romantik arg erstaunt, und der Apotheker
ging sofort zu ihm hinein, um ihm zu sagen:

"`Da{\s} mit dem Samt scheint mir "ubertrieben. Allein die
Kosten~..."'

"`Wa{\s} geht Sie da{\s} an!"' schrie Karl. "`Lassen Sie mich! Sie
haben sie nicht geliebt! Gehn Sie!"'

Der Priester fa"ste Karl unter den Arm und f"uhrte ihn in den
Garten. Er sprach von der Verg"anglichkeit alle{\s} Irdischen.
Gott sei gut und weise. Man m"usse sich ohne Murren seinem
Ratschlu"s unterwerfen. Man m"usse ihm sogar daf"ur danken.

Aber Karl brach in Gotte{\s}l"asterungen au{\s}.

"`Ich verfluche ihn, euren Gott!"'

"`Der Geist de{\s} Aufruhr{\s} steckt noch in Ihnen!"' seufzte der
Priester.

Bovary lie"s ihn stehen. Mit gro"sen Schritten ging er die
Gartenmauer entlang, an den Spalieren hin. Er knirschte mit den
Z"ahnen und sah mit Blicken zum Himmel, die Verw"unschungen waren.
Aber auch nicht ein Blatt wurde davon bewegt.

E{\s} begann zu regnen. Karl{\s} Weste stand offen. Nach einer
Weile fror ihn. Er ging in{\s} Hau{\s} zur"uck und setzte sich an
den Herd in der K"uche.

Um sech{\s} Uhr h"orte er Wagengerassel drau"sen auf dem Markte.
E{\s} war die Post, die von Rouen zur"uckkehrte. Er pre"ste die
Stirn gegen die Scheiben und sah zu, wie die Reisenden
nacheinander au{\s}stiegen. Felicie legte ihm eine Matratze in
da{\s} Wohnzimmer, er warf sich darauf und schlief ein.

Herr Homai{\s} war ein Freigeist, aber er ehrte die Toten. Er trug
dem armen Karl auch nicht{\s} nach und kam abend{\s}, um
Totenwache zu halten. Er brachte drei B"ucher und ein Notizbuch
mit. Er pflegte sich Au{\s}z"uge zu machen.

Bournisien fand sich gleichfall{\s} ein. Zwei hohe Wach{\s}kerzen
brannten am Kopfende de{\s} Bette{\s}, da{\s} man au{\s} dem
Alkoven hervorger"uckt hatte.

Der Apotheker, dem da{\s} Schweigen unheimlich vorkam, drechselte
Jeremiaden "uber die "`ungl"uckliche junge Frau"'. Der Priester
unterbrach ihn. E{\s} sei nicht{\s} am Platze, al{\s} f"ur sie zu
beten.

"`Immerhin"', versetzte Homai{\s}, "`sind nur zwei F"alle
m"oglich. Entweder ist sie, wie sich die Kirche au{\s}dr"uckt,
selig verschieden. Dann bedarf sie unsrer Gebete nicht. Oder sie
ist al{\s} S"underin von hinnen gegangen ... Oder wie lautet hier
der kirchliche Au{\s}druck? Dann~..."'

Bournisien unterbrach ihn und erkl"arte in m"urrischem Tone, man
m"usse in jedem Falle beten.

"`Aber sagen Sie mir,"' wandte der Apotheker ein, "`da Gott
stet{\s} wei"s, wa{\s} un{\s} not tut, wozu dann erst da{\s}
Gebet?"'

"`Wozu da{\s} Gebet?"' wiederholte der Priester. "`Ja, sind Sie
denn kein Christ?"'

"`Verzeihung! Ich bewundre da{\s} Christentum. E{\s} hat zuerst
die Sklaverei abgeschafft, e{\s} hat der Welt eine neue Moral
geschenkt, die~..."'

"`Davon reden wir nicht. In der Heiligen Schrift~..."'

"`Gehen Sie mir mit der Bibel! Lesen Sie in der Geschichte nach!
Man wei"s, da"s sie von den Jesuiten gef"alscht ist~..."'

Karl trat ein, n"aherte sich dem Totenbette und zog langsam die
Vorh"ange beiseite.

Emma{\s} Kopf war ein wenig nach der rechten Schulter zu geneigt.
Ihr Mund stand offen und sah wie ein schwarze{\s} Loch im unteren
Teil ihre{\s} Gesichte{\s} au{\s}. Beide Daumen hatten sich fest
in die Handballen gedr"uckt. Etwa{\s} wie wei"ser Staub lag in
ihren Wimpern, und die Augen verschwammen bereit{\s} in blassem
Schleim, der wie ein d"unne{\s} Gewebe war, al{\s} h"atten Spinnen
ihr Netz dar"uber gesponnen. Da{\s} Bettuch senkte sich von ihren
Br"usten bi{\s} zu den Knien und hob sich von da an nach ihren
Fu"sspitzen. Karl hatte die Empfindung, ein schwere{\s} Etwa{\s},
ein ungeheure{\s} Gewicht laste auf ihr.

Die Turmuhr der Kirche schlug zwei Uhr. Vom Garten her drang
da{\s} dumpfe Murmeln de{\s} Bache{\s}, der in die dunkle Ferne
str"omte. Von Zeit zu Zeit schneuzte sich Bournisien
ger"auschvoll, und Homai{\s} kritzelte Notizen auf da{\s} Papier.

"`Lieber Freund,"' sagte er, "`gehn Sie nun! Dieser Anblick
zerrei"st Ihnen da{\s} Herz!"'

Sobald Karl da{\s} Zimmer verlassen hatte, begannen die beiden
ihre Er"orterung von neuem.

"`Lesen Sie Voltaire!"' sagte der eine. "`Lesen Sie Holbach! Die
Enzyklop"adisten!"'

"`Lesen Sie die {\glq}Briefe einiger portugiesischen Juden{\grq}"',
sagte der andre, "`lesen Sie die {\glq}Grundlagen de{\s}
Christentum{\s}{\grq} von Nicola{\s}!"'

Sie regten sich auf, bekamen rote K"opfe und sprachen gleichzeitig
ineinander hinein. Bournisien war entr"ustet "uber die Vermessenheit
de{\s} Apotheker{\s}, Homai{\s} erstaunt "uber die Beschr"anktheit
de{\s} Priester{\s}. Sie waren beide nahe daran, sich Beleidigungen
zu sagen, da kam pl"otzlich Karl abermal{\s} herein. Eine
unwiderstehliche Gewalt zog ihn her. Er mu"ste immer wieder die
Treppe hinauf.

Er setzte sich der Toten gegen"uber, so da"s er ihr voll in{\s}
Antlitz sehen konnte. Er verlor sich in ihren Anblick, mit einer
Innigkeit, die den Schmerz verscheuchte.

Er erinnerte sich an allerlei Legenden von Scheintoten und von den
Wundern de{\s} Magneti{\s}mu{\s}. Er bildete sich ein, er k"onne
sie wieder aufwecken, wenn er alle seine Willen{\s}kraft
konzentriere. Einmal beugte er sich sogar "uber sie und rief ganz
leise: "`Emma, Emma!"'

Er atmete so heftig, da"s die Flammen der Kerzen flackerten~...

Bei Tage{\s}anbruch traf die alte Frau Bovary ein. Karl umarmte
sie und brach von neuem in Tr"anen au{\s}. Ebenso wie der
Apotheker versuchte sie, ihm wegen de{\s} Aufwande{\s} beim
Begr"abnisse Vorstellungen zu machen, aber er brauste so auf, da"s
sie schwieg. Hinterher beauftragte er sie sogar, baldigst in die
Stadt zu fahren und da{\s} N"otige zu besorgen.

Karl blieb den ganzen Nachmittag allein. Berta war bei Frau
Homai{\s}. Felicie sa"s mit Frau Franz bei der Toten.

Am Abend empfing Karl Besuche. Er erhob sich jede{\s}mal, dr"uckte
dem Kommenden stumm die Hand, der sich dann zu den andern setzte,
die nach und nach einen gro"sen Halbkrei{\s} um den Kamin
bildeten. Alle hatten die K"opfe gesenkt. Die Knie aufeinander,
schaukelten sie mit den Beinen und stie"sen von Zeit zu Zeit einen
tiefen Seufzer au{\s}. Alle langweilten sich ma"slo{\s}, aber
keinem fiel e{\s} ein, wieder zu gehen.

Um neun Uhr kam Homai{\s} zur"uck, beladen mit einer Menge
Kampfer, Benzoe und aromatischen Kr"autern. Auch ein Gef"a"s voll
Chlor brachte er mit, um die Luft zu de{\s}infizieren. Felicie,
die L"owenwirtin und die alte Frau Bovary standen gerade um Emma
herum, damit besch"aftigt, die letzte Hand an{\s} Totenkleid zu
legen. Sie zupften den langen steifen Schleier zurecht, der bi{\s}
hinab an die Atla{\s}schuhe reichte.

Felicie wehklagte:

"`Ach, meine arme gute Herrin! Meine arme gute Herrin!"'

"`Sehn Sie nur!"' sagte die Witwe Franz seufzend, "`wie reizend
sie noch immer au{\s}schaut! Man m"ochte drauf schw"oren, da"s sie
gleich wieder aufst"unde!"'

Dann beugten sie sich "uber sie, um ihr den Kranz umzulegen. Dabei
mu"sten sie den Kopf etwa{\s} hochheben. Da quoll schwarze
Fl"ussigkeit au{\s} dem Munde hervor, al{\s} erbr"ache sie sich.

"`Mein Gott! Da{\s} Kleid! Geben Sie acht!"' schrie Frau Franz.
Und zum Apotheker gewandt: "`Helfen Sie un{\s} doch! Oder
f"urchten Sie sich vielleicht?"'

"`Ich mich f"urchten?"' erwiderte er achselzuckend. "`Nein, so
wa{\s}! Ich habe in den Spit"alern noch ganz andre{\s} gesehen und
erlebt, al{\s} ich Pharmazeutik studierte. Wir brauten un{\s}
unsern Punsch im Seziersaal! Der Tod erschreckt einen Philosophen
nicht. Ich habe sogar die Absicht -- wie ich schon oft gesagt habe
--, meinen K"orper der Anatomie zu vermachen, damit er dermaleinst
der Wissenschaft noch etwa{\s} n"utzt."'

Der Pfarrer kam und fragte nach Karl. Auf den Bescheid de{\s}
Apotheker{\s} erwiderte er:

"`Die Wunde, wissen Sie, ist noch zu frisch."'

Darauf prie{\s} Homai{\s} ihn gl"ucklich, weil er nicht darauf
gefa"st zu sein brauche, eine teure Gef"ahrtin zu verlieren,
worauf sich ein Di{\s}put "uber da{\s} Z"olibat entspann.

"`E{\s} ist unnat"urlich,"' sagte der Apotheker, "`da"s sich ein
Mann de{\s} Weibe{\s} enthalten soll. Manche Verbrechen~..."'

"`Aber, zum Kuckuck!"' rief der Priester. "`Kann denn ein
verheirateter Mensch da{\s} Beichtgeheimni{\s} wahren?"'

Nun griff Homai{\s} die Beichte an. Bournisien verteidigte sie. Er
z"ahlte ihre guten Wirkungen auf. Er wu"ste Geschichten von
Dieben, die auf einmal ehrliche Menschen geworden w"aren. Sogar
Soldaten seien, nachdem sie im Beichtstuhl ihrer S"unden ledig
gesprochen, fromme Menschen geworden. Und in Freiburg sei ein
Diener~..."'

Sein Partner war eingeschlafen. Al{\s} die schw"ule Luft im Zimmer
immer unertr"aglicher wurde, "offnete der Pfarrer da{\s} Fenster.
Da ward der Apotheker wieder wach.

"`Wie w"ar{\s} mit einer Prise?"' fragte er ihn. "`Hier! Da{\s}
h"alt munter!"'

In der Ferne bellte irgendwo fortw"ahrend ein Hund.

"`H"oren Sie, wie der Hund heult?"' fragte der Apotheker.

"`Man sagt, da"s sie die Toten wittern"', sagte der Priester.
"`"Ahnlich ist e{\s} bei den Bienen. Sie verlassen ihren Stock,
wenn im Hau{\s} ein Mensch stirbt."'

Homai{\s} erhob keinen Einwand gegen diesen Aberglauben, denn er
war bereit{\s} wieder eingeschlafen.

Bournisien, der widerstand{\s}f"ahiger war, bewegte noch eine
Zeitlang leise die Lippen. Dann senkte sich allm"ahlich sein Kinn,
sein dicke{\s} schwarze{\s} Buch entfiel ihm, und er begann zu
schnarchen.

So sa"sen sie einander gegen"uber, mit vorgestreckten B"auchen,
mit ihren aufgedunsenen Gesichtern voller Stirnrunzeln. Nach all
ihrem Zwist vereinte sie die gleiche menschliche Schw"ache. Sie
regten sich ebensowenig wie der Leichnam neben ihnen, der zu
schlummern schien.

Karl kam. Er weckte die beiden nicht. Er kam zum letzten Male. Um
Abschied von ihr zu nehmen.

Da{\s} R"aucherwerk qualmte noch. Die bl"auliche Wolke verm"ahlte
sich am Fensterkreuz mit dem Nebel, der hereindrang. Drau"sen
blinkten einige Sterne. Die Nacht war mild.

Da{\s} Wach{\s} der Kerzen tr"aufelte in langen Tr"anen herab auf
da{\s} Bettuch. Karl sah zu, wie die gelben Flammen flackerten.
Der Lichtschimmer machte ihm die Augen m"ude.

"Uber da{\s} Atla{\s}kleid huschten Reflexe; e{\s} war wei"s wie
Mondenschein. Emma verschwand darunter, und e{\s} schien ihm,
al{\s} gehe die Tote in alle die Dinge ring{\s}umher "uber, al{\s}
lebe sie nun in der Stille, in der Nacht, im leisen Winde, in dem
wirbelnden Kr"auterdufte~...

Und mit einem Male sah er sie wieder in Toste{\s} auf der
Gartenbank unter dem bl"uhenden Wei"sdornbusch ... dann in Rouen
auf dem Gange durch die Stra"se ... und dann auf der Schwelle
ihre{\s} Vaterhause{\s}, im Gut{\s}hofe, in Bertaux ... E{\s} war
ihm, al{\s} h"ore er da{\s} Jodeln der lustigen Burschen, die
unter den Apfelb"aumen tanzten bei seiner Hochzeit{\s}feier. Wie
hatte da{\s} Brautgemach nach ihrem Haar geduftet! Wie hatte ihr
Atla{\s}kleid in seinen Armen geknistert, wie spr"uhende Funken!
Da{\s}selbe Kleid! Damal{\s} und heute!

Langsam zog sein ganze{\s} einstige{\s} Gl"uck noch einmal an ihm
vor"uber. Er sah sie vor sich in ihren eigent"umlichen Bewegungen,
ihrer Haltung, ihrem Gang. Er h"orte den Klang ihrer Stimme. Immer
wieder brandete die Verzweiflung an ihn heran, unaufh"orlich,
unversiegbar wie die Flut de{\s} Meere{\s} am Strande.

Eine gr"a"sliche Neugier "uberkam ihn. Langsam und klopfenden
Herzen{\s} hob er mit den Fingerspitzen den Schleier. Aber da
schrie er vor Schrecken laut auf, und die beiden andern M"anner
erwachten. Sie zogen ihn fort und f"uhrten ihn hinunter in die
Gro"se Stube.

Bald darauf kam Felicie und richtete au{\s}, Bovary wolle vom Haar
der Toten haben.

"`Schneiden Sie ihr welche{\s} ab!"' befahl der Apotheker.

Da sie sich{\s} nicht getraute, trat er selbst mit der Schere
heran. Er zitterte so stark, da"s er die Haut an der Schl"afe an
mehreren Stellen ritzte. Endlich raffte er sich zusammen und
schnitt blindling{\s} zwei- oder dreimal zu. E{\s} entstanden ein
paar kahle Stellen mitten in dem sch"onen schwarzen Haar der
Toten.

Der Apotheker und der Pfarrer versenkten sich wieder in ihre
B"ucher, nicht ohne von Zeit zu Zeit einzunicken. Jede{\s}mal,
wenn sie wieder erwachten, warfen sie e{\s} sich gegenseitig vor.
Der Pfarrer besprengte da{\s} Zimmer mit Weihwasser, und Homai{\s}
sch"uttete ein wenig Chlor auf die Dielen.

Felicie hatte f"ur sie gesorgt und auf der Kommode eine Flasche
Branntwein, K"ase und ein lange{\s} Wei"sbrot bereitgestellt.
Gegen vier Uhr fr"uh hielt e{\s} der Apotheker nicht mehr au{\s}.
Er seufzte:

"`Wahrhaftig. Eine St"arkung w"are nicht "ubel!"'

Der Priester hatte durchau{\s} nicht{\s} dagegen. Er ging aber
erst die Messe lesen. Al{\s} er wieder zur"uckkam, a"sen und
tranken beide, wobei sie sich angrinsten, ohne recht zu wissen
warum, verf"uhrt von der sonderbaren Fr"ohlichkeit, die den
Menschen nach "uberstandenen Trauerakten ergreift. Beim letzten
Gl"aschen klopfte der Priester dem Apotheker auf die Schulter und
sagte:

"`Wir werden un{\s} am Ende noch verstehen!"'

In der Hau{\s}flur begegneten sie den Leuten, die den Sarg
brachten. Zwei Stunden lang mu"ste sich Karl von den
Hammerschl"agen martern lassen, die von den Brettern zu ihm
hallten. Dann legte man die Tote in den Sarg au{\s} Eichenholz und
diesen in die beiden andern. Aber da der letzte zu breit war,
f"ullte man die Hohlr"aume mit Werg au{\s} einer Matratze. Al{\s}
der letzte Deckel zurechtgehobelt und vernagelt war, stellte man
den Sarg vor die T"ur. Da{\s} Hau{\s} ward weit ge"offnet, und die
Leute von Yonville begannen herbeizustr"omen.

Der alte Rouault kam an. Al{\s} er da{\s} Sargtuch sah, wurde er
mitten auf dem Markte ohnm"achtig.


\newpage\begin{center}
{\large \so{Elfte{\s} Kapitel}}\bigskip\bigskip
\end{center}

Rouault hatte den Brief de{\s} Apotheker{\s} sech{\s}unddrei"sig
Stunden nach dem Ereigni{\s} erhalten. Um ihn zu schonen, hatte
Homai{\s} so geschrieben, da"s er gar nicht genau wissen konnte,
wa{\s} eigentlich geschehen war.

Der gute Mann war zun"achst wie vom Schlag ger"uhrt umgesunken.
Dann sagte er sich, sie k"onne wohl tot sein, aber sie k"onne auch
noch leben ... Schlie"slich hatte er seine Bluse angezogen, seinen
Hut aufgesetzt, Sporen an die Stiefel geschnallt und war im Galopp
weggeritten. Den ganzen Weg "uber verging er beinahe vor Angst.
Einmal mu"ste er sogar absitzen. Er sah nicht{\s} mehr, er h"orte
Stimmen ring{\s}um und glaubte, er verl"ore den Verstand.

Der Tag brach an. Er sah drei schwarze Hennen, die auf einem Baum
schliefen. Er erbebte vor Schreck "uber diese b"ose Vorbedeutung.
Schnell gelobte er der Madonna drei neue Me"sgew"ander f"ur ihre
Kirche und eine Wallfahrt in blo"sen F"u"sen vom heimatlichen
Kirchhof bi{\s} zur Kapelle von Vassonville.

In Maromme, wo er rastete, br"ullte er die Leute im Gasthof
munter, rannte mit der Schulter die Hau{\s}t"ur ein, st"urzte sich
auf einen Hafersack, go"s in die Krippe eine Flasche Apfelsekt,
setzte sich wieder auf seinen Gaul und trabte von neuem lo{\s},
da"s die Funken stoben.

Immer wieder sagte er sich, da"s man sie sicher retten w"urde. Die
"Arzte h"atten schon Mittel. Er erinnerte sich aller wunderbaren
Heilungen, die man ihm je erz"ahlt hatte. Dann aber sah er sie
tot. Sie lag auf dem R"ucken vor ihm, mitten auf der Stra"se. Er
ri"s in die Z"ugel. Da schwand die Erscheinung.

In Quincampoix trank er, um sich Mut zu machen, nacheinander drei
Tassen Kaffee.

E{\s} w"are auch m"oglich, sagte er sich, da"s sich der Absender
in der Adresse geirrt hatte. Er suchte in seiner Tasche nach dem
Briefe, f"uhlte ihn, wagte aber nicht, ihn noch einmal zu lesen.
Schlie"slich kam er auf die Vermutung, e{\s} sei vielleicht nur
ein schlechter Witz, irgendein Racheakt oder der Einfall eine{\s}
Betrunkenen. Und wenn sie wirklich schon tot w"are, dann m"u"ste
er e{\s} doch an irgend etwa{\s} merken! Aber die Fluren sahen
au{\s} wie alle Tage, der Himmel war blau, die B"aume wiegten ihre
Wipfel. Eine Herde Schafe trottete friedlich vor"uber.

Endlich erblickte er den Ort Yonville. Er kam im Galopp an, nur
noch im Sattel h"angend. Er hatte da{\s} Pferd mit Schl"agen
vorw"art{\s} gehetzt; au{\s} den Flanken de{\s} Tiere{\s} tropfte
Blut. Al{\s} der alte Mann wieder zu sich kam, warf er sich unter
heftigem Weinen in Bovary{\s} Arme.

"`Meine Tochter! Meine Emma! Mein Kind! Sag mir doch~..."'

Der andre antwortete schluchzend:

"`Ich wei"s nicht! Ich wei"s nicht! E{\s} ist so schrecklich!"'

Der Apotheker zog sie au{\s}einander.

"`Die gr"a"slichen Einzelheiten sind unn"utz! Ich werde dem Herrn
schon alle{\s} erz"ahlen. Da kommen Leute! W"urde! Fassung! Man
mu"s Philosoph sein!"'

Der arme Karl gab sich alle M"uhe, stark zu sein. Mehrere Male
wiederholte er:

"`Ja, ja ... Mut! Mut!"'

"`Na, wenn{\s} sein mu"s!"' sagte Rouault. "`Ich hab welchen!
Himmeldonnerwetter! Wir wollen unsrer Emma da{\s} Geleite geben,
und wenn{\s} noch so weit w"are!"'

Die Glocke begann zu l"auten. Alle{\s} war bereit. Der Zug setzte sich
in Bewegung.

Rouault und Bovary sa"sen nebeneinander in den Chorst"uhlen. Die
drei Chorknaben wandelten psalmodierend vor ihnen hin und her.
Musik brummte. Bournisien in vollem Ornat sang mit scharfer
Stimme. Er verbeugte sich vor dem Tabernakel, hob die H"ande empor
und breitete die Arme au{\s}. Der Kirchendiener hantierte. Vor dem
Chorpult stand der Sarg zwischen vier Kerzen. Karl bekam eine
Anwandlung, aufzustehn und sie au{\s}zublasen.

Er strengte sich an, Andacht zu empfinden, sich zum Glauben an ein
jenseitige{\s} Dasein aufzuschwingen, wo er Emma wiedersehen
w"urde. Er versuchte sich einzubilden, sie sei verreist, weit,
weit weg und schon seit langer Zeit. Aber wenn er daran dachte,
da"s sie dort unter dem Leichentuche lag, da"s alle{\s} zu Ende
war, da"s man sie nun in die Erde scharrte, da fa"ste ihn wilde
Wut und schwarze Verzweiflung. Und dann wieder war ihm, al{\s}
empf"ande er "uberhaupt nicht{\s} mehr. Er f"uhlte sich in seinem
Schmerze erleichtert, aber al{\s}bald warf er sich vor, eine
erb"armliche Kreatur zu sein.

Auf die Fliesen der Kirche schlug in gleichen Zeitr"aumen etwa{\s}
wie ein Eisenstab auf. Diese{\s} harte Ger"ausch drang au{\s} dem
Hintergrund, bi{\s} e{\s} mit einem Male im Winkel eine{\s}
Seitenschiffe{\s} aufh"orte. Ein Mensch in einem groben braunen
Rock kniete m"uhsam nieder. E{\s} war Hippolyt, der Knecht vom
Goldnen L"owen. Heute hatte er sein Bein erster Garnitur
angeschnallt.

Ein Chorknabe machte die Runde durch{\s} Kirchenschiff, um Geld
einzusammeln. Die gro"sen Kupferst"ucke klirrten ein{\s} nach dem
andern in der silbernen Schale.

"`Schnell weg! Ich leide!"' rief Bovary und warf zornig ein
F"unf\/frankenst"uck hinein.

Der Sammelnde bedankte sich mit einer tiefen Verbeugung.

Man sang, man kniete nieder, man richtete sich wieder auf ...
Da{\s} nahm kein Ende! Karl erinnerte sich, da"s er mit Emma in
der ersten Zeit ihre{\s} Hiersein{\s} einmal zur Messe dagewesen
war. Sie hatten recht{\s} an der Mauer gesessen ... Die Glocke
begann wieder zu l"auten. Ein allgemeine{\s} St"uhler"ucken fing
an. Die Sargtr"ager hoben die drei Stangen der Bahre in die H"ohe.
Man verlie"s die Kirche.

Justin stand an der T"ur der Apotheke. Er verschwand schleunigst,
bla"s und taumelnd.

Alle Fenster im Orte waren voller Neugieriger, um den Trauerzug
vorbeiziehen zu sehn. Karl ging voran, erhobenen Haupte{\s}. Er
trug eine tapfre Miene zur Schau und gr"u"ste kopfnickend jeden,
der au{\s} den Gassen oder den H"ausern trat, um sich dem Zuge
anzuschlie"sen.

Die sech{\s} Tr"ager, drei auf jeder Seite, schritten langsam
vorw"art{\s}. Sie keuchten. Die Priester, die S"anger und die
Chorknaben sangen da{\s} \begin{antiqua}De profundis\end{antiqua}.
Ihre bald lauten, bald leisen Stimmen verhallten im Feld. Wo der
Weg eine Biegung machte, verschwanden sie auf Augenblicke, aber
da{\s} hohe silberne Kreuz schimmerte immer zwischen den B"aumen.

Die Frauen schlossen sich hinten an, in schwarzen M"anteln mit
zur"uckgeschlagenen Kapuzen, in den H"anden dicke brennende
Wach{\s}kerzen. Karl f"uhlte, wie ihn seine Kr"afte verlie"sen
unter der ewigen Monotonie der Gebete und der Lichter, inmitten
de{\s} faden Geruch{\s} von Wach{\s} und Me"sgew"andern. Ein
frischer Wind wehte her"uber. Roggen und Rap{\s} gr"unten, und
Tautropfen zitterten auf den Dornenhecken am Wege. Allerlei
fr"ohliche Laute erf"ullten die Luft: da{\s} Quietschen eine{\s}
kleinen Wagen{\s} in der Ferne auf zerfahrener Stra"se, da{\s}
wiederholte Kr"ahen eine{\s} Hahne{\s} oder der Galopp eine{\s}
F"ullen{\s}, da{\s} sich unter den Apfelb"aumen au{\s}tobte. Der
klare Himmel war mit rosigen W"olkchen betupft. Bl"auliche Lichter
spielten um die Schwertlilien vor den H"ausern und H"utten. Karl
erkannte im Vorbeigehen jeden einzelnen Hof. Er entsann sich
eine{\s} bestimmten Morgen{\s}, an dem er, einen Kranken zu
besuchen, hier vor"ubergekommen war, erst hin und dann auf dem
R"uckwege zu "`ihr"'.

Manchmal flatterte da{\s} schwarze mit silbernen Tr"anen bestickte
Leichentuch auf und lie"s den Sarg sehen. Die erm"udeten Tr"ager
verlangsamten den Schritt. Die Bahre schwankte fortw"ahrend wie
eine Schaluppe auf bewegter See.

Endlich war man da.

Die Tr"ager gingen bi{\s} ganz hinter, bi{\s} zu einer Stelle im
Rasen, wo da{\s} Grab gegraben war. Man stellte sich im Krei{\s}
herum auf. W"ahrend der Priester sprach, rieselte die rote, an den
Seiten aufgeh"aufte Erde "uber die Kanten hinweg in die Grube,
lautlo{\s} und ununterbrochen.

Dann wurden die vier Seile zurechtgelegt und der Sarg darauf
gehoben. Karl sah ihn hinabgleiten ... tiefer ... immer tiefer.

Endlich h"orte man ein Aufschlagen. Die Seile kamen ger"auschvoll
wieder hoch. Bournisien nahm den Spaten, den ihm Lestiboudoi{\s}
reichte. Und w"ahrend er mit der rechten Hand den Weihwedel
schwang, warf er wuchtig mit der linken eine volle Schaufel Erde
in{\s} Grab. Der Sand und die Steinchen polterten auf den Sarg,
und da{\s} Ger"ausch dr"ohnte Karl in die Ohren, unheimlich wie
ein Widerhall au{\s} der Ewigkeit.

Der Priester gab die Schaufel an seinen Nachbar weiter. E{\s} war
Homai{\s}. W"urdevoll f"ullte und leerte er sie und reichte sie
dann Karl, der auf die Knie sank, mit vollen H"anden Erde
hinabwarf und "`Lebe wohl!"' rief. Er sandte ihr K"usse und beugte
sich "uber da{\s} Grab, al{\s} ob er sich hinabst"urzen wollte.

Man f"uhrte ihn fort. Er beruhigte sich sehr bald. Offenbar
empfand er gleich den andern eine merkw"urdige Befriedigung, da"s
alle{\s} "uberstanden war.

Auf dem Heimwege z"undete sich Vater Rouault ruhig seine Pfeife
an, wa{\s} Homai{\s} in{\s}geheim nicht besonder{\s} schicklich
fand. Er berichtete, da"s Binet nicht zugegen gewesen war, da"s
sich T"uvache nach der Messe "`gedr"uckt"' hatte und da"s Theodor,
der Diener de{\s} Notar{\s}, einen blauen Rock getragen hatte,
"`al{\s} ob nicht ein schwarzer aufzutreiben gewesen w"are, da
e{\s} nun einmal so "ublich ist, zum Teufel!"' So hechelte er
alle{\s} durch, wa{\s} er beobachtet hatte.

Alle andern beklagten Emma{\s} Tod, besonder{\s} Lheureux, der
nicht verfehlt hatte, zum Begr"abni{\s} zu erscheinen.

"`Die arme, liebe Frau! Welch ein Schlag f"ur ihren Mann!"'

Der Apotheker antwortete:

"`Wissen Sie, wenn ich nicht gewesen w"are, h"atte er au{\s}
Verzweiflung Selbstmord begangen."'

"`Sie war immer so lieben{\s}w"urdig! Wenn ich bedenke, da"s sie
vorigen Sonnabend noch in meinem Laden war!"'

"`Ich hatte nur keine Zeit,"' sagte der Apotheker, "`sonst h"atte
ich mich gern auf ein paar Worte vorbereitet, die ich ihr in{\s}
Grab nachgerufen h"atte!"'

Wieder im Hause, kleidete sich Karl um, und der alte Rouault zog
seine blaue Bluse wieder an. Sie war neu, und da er sich
unterweg{\s} "ofter{\s} die Augen mit dem "Armel gewischt hatte,
hatte sie Farbenspuren auf seinem staubbedeckten Gesicht
hinterlassen. Man sah, wo die Tr"anen herabgerollt waren.

Die alte Frau Bovary setzte sich zu ihnen. Alle drei schwiegen.
Endlich sagte Vater Rouault mit einem Seufzer:

"`Erinnerst du dich noch, mein lieber Karl, wie ich damal{\s} nach
Toste{\s} kam, al{\s} du deine erste Frau verloren hattest?
Damal{\s} tr"ostete ich dich, damal{\s} fand ich Worte! Jetzt
aber~..."' Er st"ohnte tief auf, wobei sich seine ganze Brust
hob. "`Ach, nun ist e{\s} au{\s} mit mir! Ich habe meine Frau
sterben sehen ... dann meinen Sohn ... und heute meine
Tochter!"'

Er bestand darauf, noch am selben Tage nach Bertaux
zur"uckzureiten. In diesem Hause k"onne er nicht schlafen. Auch
seine Enkelin wollte er nicht sehen.

"`Nein! Nein! Da{\s} w"urde mich zu traurig machen! Aber k"usse
sie mir ordentlich! Lebe wohl! Du bist ein braver Junge! Und
da{\s} hier,"' er schlug auf sein Bein, "`da{\s} werde ich dir nie
vergessen. Hab keine Bange! Und euren Truthahn bekommst du auch
noch jede{\s} Jahr!"'

Aber al{\s} er auf der H"ohe angelangt war, wandte er sich um,
ganz wie damal{\s} nach der Hochzeit, al{\s} er sich nach dem
Abschied auf der Landstra"se bei Sankt Viktor noch einmal nach
seiner Tochter umgedreht hatte. Die Fenster im Dorfe gl"uhten wie
im Feuer unter den Strahlen der Sonne, die in der Ebene unterging.
Er beschattete die Augen mit der Hand und gewahrte fern am
Horizont ein Mauerviereck und B"aume darinnen, die wie schwarze
B"uschel zwischen wei"sen Steinen hervorleuchteten. Dort lag der
Friedhof~...

Dann ritt er seinen Weg weiter, im Schritt, dieweil sein Gaul lahm
geworden war.

Karl und seine Mutter blieben bi{\s} in die sp"ate Nacht auf und
plauderten, obwohl sie beide sehr m"ude waren. Sie sprachen von
vergangenen Tagen und von dem, wa{\s} nun werden sollte. Die alte
Frau wollte nach Yonville "ubersiedeln, ihm die Wirtschaft f"uhren
und f"ur immer bei ihm bleiben. Sie fand immer neue Troste{\s}-
und Liebe{\s}worte. Im geheimen freute sie sich, eine Neigung
zur"uckzugewinnen, die sie so viele Jahre entbehrt hatte.

E{\s} schlug Mitternacht. Da{\s} Dorf lag in tiefer Stille. Da{\s}
war wie immer. Nur Karl war wach und dachte in einem fort an
"`sie"'.

Rudolf, der zu seinem Vergn"ugen den Tag "uber durch den Wald
geritten war, schlief ruhig in seinem Schlo"s. Ebenso schlummerte
Leo. Einer aber schlief nicht in dieser Stunde.

Am Grabe, unter den Fichten, kniete ein junger Bursche und weinte.
Seine vom Schluchzen wunde Brust st"ohnte im Dunkel unter dem
Druck einer unerme"slichen Sehnsucht, die s"u"s war wie der Mond
und geheimni{\s}voll wie die Nacht.

Pl"otzlich knarrte die Gittert"ur. Lestiboudoi{\s} hatte seine
Schaufel vergessen und kam sie zu holen. Er erkannte Justin,
al{\s} er sich "uber die Mauer schwang. Nun glaubte er zu wissen,
wer ihm immer Kartoffeln stahl.


\newpage\begin{center}
{\large \so{Le{tz}te{\s} Kapitel}}\bigskip\bigskip
\end{center}

Am Tage darauf lie"s Karl die kleine Berta wieder in{\s} Hau{\s}
kommen. Sie fragte nach der Mutter. Man antwortete ihr, sie sei
verreist und werde ihr h"ubsche Spielsachen mitbringen. Da{\s}
Kind tat noch ein paarmal die gleiche Frage, dann aber, mit der
Zeit, sprach sie nicht mehr von ihr. Die Sorglosigkeit de{\s}
Kinde{\s} bereitete Bovary Schmerzen. Ganz unertr"aglich aber
waren ihm die Trostreden de{\s} Apotheker{\s}.

Bald begannen die Geldsorgen von neuem. Lheureux lie"s seinen
Strohmann Vin\c{c}ard abermal{\s} vorgehen, und Karl "ubernahm
betr"achtliche Verpflichtungen, weil er e{\s} um keinen Prei{\s}
zulassen wollte, da"s von den M"obeln, die ihr geh"ort hatten,
auch nur da{\s} geringste verkauft w"urde. Seine Mutter war au"ser
sich dar"uber. Da{\s} emp"orte ihn wiederum ma"slo{\s}. Er war
"uberhaupt ein ganz andrer geworden. So verlie"s sie da{\s}
Hau{\s}.

Nun fingen alle m"oglichen Leute an, ihr "`Schnittchen"' zu
machen. Fr"aulein Lempereur forderte f"ur sech{\s} Monate
Stundengeld, obgleich Emma doch niemal{\s} Unterricht bei ihr
genommen hatte. Die quittierte Rechnung, die Bovary einmal gezeigt
bekommen hatte, war nur auf Emma{\s} Bitte hin au{\s}gestellt
worden. Der Leihbibliothekar verlangte Abonnement{\s}geb"uhren auf
eine Zeit von drei Jahren und Frau Rollet Botenlohn f"ur zwanzig
Briefe. Al{\s} Karl N"ahere{\s} wissen wollte, war sie
wenigsten{\s} so r"ucksicht{\s}voll, zu antworten:

"`Ach, ich wei"s von nicht{\s}! E{\s} waren wohl Rechnungen."'

Bei jedem Schuldbetrag, den er bezahlte, glaubte Karl, e{\s} sei
nun zu Ende, aber e{\s} meldeten sich immer wieder neue
Gl"aubiger.

Er schickte an seine Patienten Liquidationen au{\s}. Da zeigte man
ihm die Briefe seiner Frau, und so mu"ste er sich noch
entschuldigen.

Felicie trug jetzt die Kleider ihrer Herrin, aber nicht alle, denn
Karl hatte einige davon zur"uckbehalten. Manchmal schlo"s er sich
in ihr Zimmer und betrachtete sie. Felicie hatte ungef"ahr
Emma{\s} Figur. Wenn sie au{\s} dem Zimmer ging, hatte er manchmal
den Eindruck, e{\s} sei die Verstorbne. Dann war er nahe daran,
ihr nachzurufen: "`Emma, bleib, bleib!"'

Aber zu Pfingsten verlie"s sie Yonville, zusammen mit dem Diener
de{\s} Notar{\s}, wobei sie alle{\s} mitnahm, wa{\s} von Emma{\s}
Kleidern noch "ubrig war.

Um diese Zeit gab sich die Witwe D"upui{\s} die Ehre, ihm die
Verm"ahlung ihre{\s} Sohne{\s} Leo D"upui{\s}, Notar{\s} zu
Yvetot, mit Fr"aulein Leocadia Leboeuf au{\s} Bondeville ganz
ergebenst mit\/zuteilen. In Karl{\s} Gl"uckwunschbrief kam die
Stelle vor:

"`Wie h"atte sich meine arme Frau dar"uber gefreut!"'

Eine{\s} Tage{\s}, al{\s} Karl ohne bestimmte Absicht durch{\s}
Hau{\s} irrte, kam er in die Dachkammer und sp"urte pl"otzlich
unter einem seiner Pantoffel ein zusammengekn"ullte{\s} St"uck
Papier. Er entfaltete e{\s} und la{\s}: "`Liebe Emma! Sei tapfer!
Ich will Dir Deine Existenz nicht zertr"ummern~..."' E{\s} war
Rudolf{\s} Brief, der zwischen die Kisten gefallen und dort liegen
geblieben war, bi{\s} ihn der durch{\s} Dachfenster wehende
Luft\/zug an die T"ure getrieben hatte. Karl stand ganz starr da,
mit offnem Munde, just auf demselben Platz, wo dereinst Emma,
bleicher noch al{\s} er, au{\s} Verzweiflung in den Tod gehen
wollte. Am Ende der zweiten Seite stand al{\s} Unterschrift ein
kleine{\s} R. Wer war da{\s}? Er erinnerte sich der vielen Besuche
und Aufmerksamkeiten Rudolf Boulanger{\s}, seine{\s} pl"otzlichen
Au{\s}bleiben{\s} und der gezwungenen Miene, die er gehabt, wenn
er ihnen sp"ater -- e{\s} war zwei- oder dreimal gewesen --
begegnet war. Aber der achtung{\s}volle Ton de{\s} Briefe{\s}
t"auschte ihn.

"`Da{\s} scheint doch nur eine platonische Liebelei gewesen zu
sein!"' sagte er sich.

"Ubrigen{\s} geh"orte Karl nicht zu den Menschen, die den Dingen
bi{\s} auf den Grund gehen. Er war weit davon entfernt, Beweise zu
suchen, und seine vage Eifersucht ging auf in seinem ma"slosen
Schmerze.

"`Man mu"ste sie anbeten!"' sagte er bei sich. "`E{\s} ist ganz
nat"urlich, da"s alle M"anner sie begehrt haben!"' Nunmehr
erschien sie ihm noch sch"oner, und e{\s} "uberkam ihn ein
best"andige{\s} hei"se{\s} Verlangen nach ihr, da{\s} ihn
trostlo{\s} machte und da{\s} keine Grenzen kannte, weil e{\s}
nicht mehr zu stillen war.

Um ihr zu gefallen, al{\s} lebte sie noch, richtete er sich nach
ihrem Geschmack und ihren Liebhabereien. Er kaufte sich
Lackstiefel, trug feine Krawatten, pflegte seinen Schnurrbart und
-- unterschrieb Wechsel wie sie. So verdarb ihn Emma noch au{\s}
ihrem Grabe herau{\s}.

Karl sah sich gen"otigt, da{\s} Silberzeug zu verkaufen, ein
St"uck nach dem andern, dann die M"obel de{\s} Salon{\s}. Alle
Zimmer wurden kahl, nur "`ihr Zimmer"' blieb wie fr"uher. Nach dem
Essen pflegte Karl hinaufzugehen. Er schob den runden Tisch an den
Kamin und r"uckte ihren Sessel heran. Dem setzte er sich
gegen"uber. Eine Kerze brannte in einem der vergoldeten Leuchter.
Berta, neben ihm, tuschte Bilderbogen au{\s}.

E{\s} tat dem armen Manne weh, wenn er sein Kind so schlecht
gekleidet sah, mit Schuhen ohne Schn"ure, die N"ahte de{\s}
Kleidchen{\s} aufgerissen, denn darum k"ummerte sich die
Aufwartefrau nicht. Berta war sanft und allerliebst. Wenn sie
da{\s} K"opfchen grazi"o{\s} neigte und ihr die blonden Locken
"uber die rosigen Wangen fielen, dann sah sie so reizend au{\s},
da"s ihn unendliche Z"artlichkeit ergriff, eine Freude, die nach
Wehmut schmeckte, wie ungepflegter Wein nach Pech. Er besserte ihr
Spielzeug au{\s}, machte ihr Hampelm"anner au{\s} Pappe und
flickte sie aufgeplatzten B"auche ihrer Puppen. Wenn seine Augen
dabei auf Emma{\s} Arbeit{\s}k"astchen fielen, auf ein Band,
da{\s} liegengeblieben war, oder auf eine Stecknadel, die noch in
einer Ritze de{\s} N"ahtische{\s} steckte, dann verfiel er in
Tr"aumereien und sah so traurig au{\s}, da"s da{\s} Kind auch mit
traurig wurde.

Kein Mensch besuchte sie mehr. Justin war nach Rouen davongelaufen,
wo er Kr"amerlehrling geworden war, und die Kinder de{\s}
Apotheker{\s} lie"sen sich auch immer seltner sehen, da ihr Vater
bei der jetzigen Verschiedenheit der gesellschaftlichen
Verh"altnisse auf eine Fortsetzung de{\s} n"aheren Verkehr{\s}
keinen Wert legte.

Der Blinde, den Homai{\s} mit seiner Salbe nicht hatte heilen
k"onnen, war auf die H"ohe am Wilhelm{\s}walde zur"uckgekehrt und
erz"ahlte allen Reisenden den Mi"serfolg de{\s} Apotheker{\s}.
Wenn Homai{\s} zur Stadt fuhr, versteckte er sich infolgedessen
hinter den Vorh"angen der Postkutsche, um eine Begegnung mit ihm
zu vermeiden. Er ha"ste ihn, und da er ihn zugunsten seine{\s}
Rufe{\s} al{\s} Heilk"unstler um jeden Prei{\s} au{\s} dem Wege
r"aumen wollte, legte er ihm einen Hinterhalt. Die Art und Weise,
wie er da{\s} bewerkstelligte, enth"ullte ebenso seinen Scharfsinn
wie seine bi{\s} zur Verruchtheit gehende Eitelkeit. Sech{\s}
Monate hintereinander konnte man im "`Leuchtturm von Rouen"'
Nachrichten wie die folgenden lesen:

\begin{quotation}
"`Wer nach den fruchtbaren Gefilden der Pikardie reist, wird ohne
Zweifel auf der H"ohe am Wilhelm{\s}walde einen Vagabunden bemerkt
haben, der mit einem ekelhaften Augenleiden behaftet ist. Er
bel"astigt und verfolgt die Reisenden, erhebt von ihnen
gewisserma"sen einen Zoll. Leben wir denn noch in den
abscheulichen Zeiten de{\s} Mittelalter{\s}, wo e{\s} den
Landstreichern erlaubt war, auf den "offentlichen Pl"atzen die
Lepra und die Skrofeln zur Schau zu stellen, die sie von einem der
Kreuzz"uge mitgebracht hatten?"'
\end{quotation}

Oder:

\begin{quotation}
"`Ungeachtet der Gesetze gegen da{\s} Landstreichertum werden die
Zug"ange unsrer Gro"sst"adte noch unau{\s}gesetzt von
Bettlerscharen heimgesucht. Manche treten auch vereinzelt auf, und
da{\s} sind vielleicht nicht die ungef"ahrlichsten. Au{\s} welchem
Grunde duldet da{\s} eigentlich die Obrigkeit?"'
\end{quotation}

Daneben erfand Homai{\s} auch Anekdoten:

\begin{quotation}
"`Gestern ist auf der H"ohe am Wilhelm{\s}walde ein Pferd
durchgegangen~..."'
\end{quotation}

E{\s} folgte der Bericht eine{\s} durch da{\s} pl"otzliche
Auftauchen de{\s} Blinden verursachten Unfall{\s}.

Alle{\s} da{\s} hatte eine so treffliche Wirkung, da"s der
Ungl"uckliche in Haft genommen wurde. Aber man lie"s ihn wieder
frei. Er trieb e{\s} wie vorher. Ebenso Homai{\s}. E{\s} begann
ein Kampf. Der Apotheker blieb Sieger. Sein Gegner wurde zu
leben{\s}l"anglichem Aufenthalt in ein Krankenhau{\s} gesteckt.

Dieser Erfolg machte ihn immer k"uhner. Fortan konnte kein Hund
"uberfahren werden, keine Scheune abbrennen, keine Frau Pr"ugel
bekommen, ohne da"s er den Vorfall sofort ver"offentlicht h"atte
--, geleitet vom Fortschritt{\s}fanati{\s}mu{\s} und vom Ha"s
gegen die Priester.

Er stellte Vergleiche an zwischen den Volk{\s}schulen und den von
den "`Ignorantinern"' geleiteten, die nat"urlich zum Nachteil der
letzteren au{\s}fielen. Anl"a"slich einer staatlichen Bewilligung
von hundert Franken f"ur kirchliche Zwecke erinnerte er an die
Niedermetzelung der Hugenotten. Er denunzierte kirchliche
Mi"sbr"auche. Er la{\s} den Pfaffen die Leviten, wie er meinte.
Dabei wurde er ein gef"ahrlicher Intrigant.

Bald war ihm der Journali{\s}mu{\s} zu eng; er wollte ein Buch
Schreiben, ein "`Werk"'. So verfa"ste er eine "`Allgemeine
Statistik von Yonville und Umgebung nebst klimatologischen
Beobachtungen"'. Die damit verbundenen Studien f"uhrten ihn in{\s}
volk{\s}wirtschaftliche Gebiet. Er vertiefte sich in die sozialen
Fragen, in die Theorien "uber die Volk{\s}erziehung, in da{\s}
Verkehr{\s}wesen und andre{\s} mehr. Nun begann er sich seiner
kleinb"urgerlichen Ob{\s}kurit"at zu sch"amen; er bekam
genialische Anwandlungen.

Seinen Beruf vernachl"assigte er dabei keine{\s}weg{\s}, im
Gegenteil, er verfolgte alle neuen Entdeckungen seine{\s}
Fache{\s}. Beispiel{\s}weise interessierte ihn der gro"se
Aufschwung in der Schokoladenindustrie. Er war weit und breit der
erste, der den Schoka (eine Mischung von Kakao und Kaffee) und die
Eisenschokolade einf"uhrte. Er begeisterte sich f"ur die
hydro-elektrischen Ketten Pulvermacher{\s} und trug selbst eine.
Wenn er beim Schlafengehen da{\s} Hemd wechselte, staunte Frau
Homai{\s} diese goldene Spirale an, die ihn umschlang, und
entbrannte in verdoppelter Liebe f"ur diesen Mann, der wie ein
Magier gl"anzte.

F"ur Emma{\s} Grabmal hatte er sehr sch"one Ideen. Zuerst schlug
er einen S"aulenstumpf mit einer Draperie vor, dann eine Pyramide,
einen Vestatempel in Form einer Rotunde, zu guter Letzt eine
"`k"unstliche Ruine"'. Keine{\s}fall{\s} aber d"urfe die
Trauerweide fehlen, die er f"ur da{\s} "`traditionelle Symbol"'
der Trauer hielt.

Karl und er fuhren zusammen nach Rouen, um bei einem
Grabsteinfabrikanten etwa{\s} Passende{\s} zu suchen. Ein
Kunstmaler begleitete sie, namen{\s} Vaufrylard, ein Freund de{\s}
Apotheker{\s} Bridoux. Er ri"s die ganze Zeit "uber schlechte
Witze. Man besichtigte an die hundert Modelle, und Karl erbat sich
die Zusendung von Kostenanschl"agen. Er fuhr dann ein
zweite{\s}mal allein nach Rouen und entschlo"s sich zu einem
Grabstein, "uber dem ein Geniu{\s} mit gesenkter Fackel trauert.

Al{\s} Inschrift fand Homai{\s} nicht{\s} sch"oner al{\s}:
\begin{antiqua}STA VIATOR!\end{antiqua} Diese Worte schlug er
immer wieder vor. Er war richtig vernarrt in sie. Best"andig
fl"usterte er vor sich hin: "`\begin{antiqua}Sta
viator!\end{antiqua}"' Endlich kam er auf: \begin{antiqua}AMABILEM
CONJUGEM CALCAS!\end{antiqua} Da{\s} wurde angenommen.

Seltsamerweise verlor Bovary, obwohl er doch ununterbrochen an
Emma dachte, mehr und mehr die Erinnerung an ihre "au"sere
Erscheinung. Zu seiner Verzweiflung f"uhlte er, wie ihr Bild
seinem Ged"achtni{\s} entwich, w"ahrend er sich so viel M"uhe gab,
e{\s} zu bewahren. Dabei tr"aumte er jede Nacht von ihr. E{\s} war
immer derselbe Traum: er sah sie und n"aherte sich ihr, aber
sobald er sie umarmen wollte, zerfiel sie ihm in Staub und Moder.

Eine Woche lang sah man ihn jeden Abend in die Kirche gehen. Der
Pfarrer machte ihm zwei oder drei Besuche, dann aber gab er ihn
auf. Bournisien war neuerding{\s} "uberhaupt unduldsam, ja
fanatisch, wie Homai{\s} behauptete. Er wetterte gegen den Geist
de{\s} Jahrhundert{\s}, und aller vierzehn Tage pflegte er in der
Predigt vom schrecklichen Ende Voltaire{\s} zu erz"ahlen, der im
Tode{\s}kampfe seine eignen Exkremente verschlungen habe, wie
jedermann wisse.

Trotz aller Sparsamkeit kam Bovary nicht au{\s} den alten Schulden
herau{\s}. Lheureux wollte keinen Wechsel mehr prolongieren, und
so stand die Pf"andung abermal{\s} bevor. Da wandte er sich an
seine Mutter. Sie schickte ihm eine B"urgschaft{\s}erkl"arung.
Aber im Begleitbriefe erhob sie eine Menge Beschuldigungen gegen
Emma. Al{\s} Entgelt f"ur ihr Opfer erbat sie sich einen Schal,
der Felicie{\s} Raubgier entgangen war. Karl verweigerte ihn ihr.
Dar"uber ent\/zweiten sie sich.

Trotzdem reichte sie bald darauf selber die Hand zur Vers"ohnung.
Sie schlug ihrem Sohne vor, sie wolle die kleine Berta zu sich
nehmen; sie k"onne ihr im Hau{\s}halt helfen. Karl willigte ein.
Aber al{\s} da{\s} Kind abreisen sollte, war er nicht imstande
sich von ihm zu trennen. Die{\s}mal erfolgte ein endg"ultiger,
v"olliger Bruch.

Nun hatte er alle{\s} verloren, wa{\s} ihm lieb und wert gewesen
war, und er schlo"s sich immer enger an sein Kind an. Aber auch
die{\s} machte ihm Sorgen. Berta hustete manchmal und hatte rote
Flecken auf den Wangen.

Ihm gegen"uber machte sich in Gesundheit, Gl"uck und Frohsinn die
Familie de{\s} Apotheker{\s} breit. Wa{\s} Homai{\s} auch wollte,
gelang ihm. Napoleon half dem Vater im Laboratorium, Athalia
stickte ihm ein neue{\s} K"appchen, Irma schnitt
Pergamentpapierdeckel f"ur die Einmachegl"aser, und Franklin
bewie{\s} ihm bereit{\s} schlankweg den pythagoreischen Lehrsatz.
Der Apotheker war der gl"ucklichste Vater und der gl"ucklichste
Mensch.

Und doch nicht! Der Ehrgeiz nagte heimlich an seinem Herzen.
Homai{\s} sehnte sich nach dem Kreuz der Ehrenlegion. Verdient
h"atte er e{\s} zur Gen"uge, meinte er. Ersten{\s} hatte er sich
w"ahrend der Cholera durch grenzenlosen Opfermut au{\s}gezeichnet.
Zweiten{\s} hatte er -- und zwar auf seine eigenen Kosten --
verschiedene gemeinn"utzige Werke ver"offentlicht,
beispiel{\s}weise die Schrift "`Der Apfelwein. Seine Herstellung
und seine Wirkung"', sodann seine "`Abhandlung "uber die
Reblau{\s}"', die er dem Ministerium unterbreitet hatte, ferner
seine statistische Ver"offentlichung, ganz abgesehen von seiner
ehemaligen Pr"ufung{\s}arbeit. Er z"ahlte sich da{\s} alle{\s}
auf. "`Dazu bin ich auch noch Mitglied mehrerer wissenschaftlicher
Gesellschaften."' In Wirklichkeit war e{\s} nur eine einzige.

"`Eigentlich m"u"ste e{\s} schon gen"ugen,"' rief er und warf sich
selbstbewu"st in die Brust, "`da"s ich mich bei den
Feuer{\s}br"unsten hervorgetan habe!"'

Er begann F"uhlung mit der Regierung zu suchen. Zur Zeit der
Wahlen erwie{\s} er dem Landrat heimlich gro"se Dienste.
Schlie"slich verkaufte und prostituierte er sich regelrecht. Er
reichte ein Immediatgesuch an Seine Majest"at ein, worin er ihn
alleruntert"anigst bat, "`ihm Gerechtigkeit widerfahren zu
lassen."' Er nannte ihn "`unsern guten K"onig"' und verglich ihn
mit Heinrich dem Vierten.

Jeden Morgen st"urzte er sich auf die Zeitung, um seine Ernennung
zu lesen; aber sie wollte nicht kommen. Sein Orden{\s}koller ging
so weit, da"s er in seinem Garten ein Beet in Form de{\s}
Kreuze{\s} der Ehrenlegion anlegen lie"s, auf der einen Seite von
Geranien ums"aumt, die da{\s} rote Band vorstellten. Oft umkreiste
er diese{\s} bunte Beet und dachte "uber die Schwerf"alligkeit der
Regierung und "uber den Undank der Menschen nach.

Au{\s} Achtung f"ur seine verstorbene Frau, oder weil er au{\s}
einer Art Sinnlichkeit noch etwa{\s} Unerforschte{\s} vor sich
haben wollte, hatte Karl da{\s} geheime Fach de{\s}
Schreibtische{\s} au{\s} Polisanderholz, den Emma benutzt hatte,
noch nicht ge"offnet. Eine{\s} Tage{\s} setzte er sich endlich
davor, drehte den Schl"ussel um und zog den Kasten herau{\s}. Da
lagen s"amtliche Briefe Leo{\s}. Die{\s}mal war kein Zweifel
m"oglich. Er verschlang sie von der ersten bi{\s} zur letzten
Zeile. Dann st"oberte er noch in allen Winkeln, allen M"obeln,
allen Schiebf"achern, hinter den Tapeten, schluchzend, st"ohnend,
halbverr"uckt. Er entdeckte eine Schachtel und stie"s sie mit
einem Fu"stritt auf. Rudolf{\s} Bildni{\s} sprang ihm
buchst"ablich in{\s} Gesicht. E{\s} lag neben einem ganzen B"undel
von Liebe{\s}briefen.

Bovary{\s} Niedergeschlagenheit erregte allgemeine Verwunderung.
Er ging nicht mehr au{\s}, empfing niemanden und weigerte sich
sogar, seine Patienten zu besuchen. Dadurch entstand da{\s}
Ger"ucht, da"s er sich einschlie"se, um zu trinken. Neugierige
aber, die hin und nieder den Kopf "uber die Gartenhecke reckten,
sahen zu ihrer "Uberraschung, wie der Menschenscheue in seinem
langen Bart und in schmutziger Kleidung im Garten auf und ab ging
und laut weinte.

An Sommerabenden nahm er sein T"ochterchen mit sich hinau{\s} auf
den Friedhof. Erst sp"at in der Nacht kamen die beiden zur"uck,
wenn auf dem Marktpl"atze kein Licht mehr schimmerte, au"ser
au{\s} dem St"ubchen Binet{\s}.

Aber auf die Dauer befriedigte ihn die Wollust seine{\s}
Schmerze{\s} nicht mehr. Er brauchte jemanden, der sein Leid mit
ihm teilte. Au{\s} diesem Grunde suchte er Frau Franz auf, um von
"`ihr"' sprechen zu k"onnen. Aber die Wirtin h"orte nur mit halbem
Ohre zu, da auch sie ihre Sorgen hatte. Lheureux hatte n"amlich
seine Postverbindung zwischen Yonville und Rouen er"offnet, und
Hivert, der ob seiner Zuverl"assigkeit in Kommissionen
allenthalben gro"se{\s} Vertrauen geno"s, verlangte Lohnerh"ohung
und drohte, "`zur Konkurrenz"' "uberzugehen.

Eine{\s} Tage{\s}, al{\s} Karl nach Argueil zum Markt gegangen
war, um sein Pferd, sein letzte{\s} St"uck Besitz, zu verkaufen,
begegnete er Rudolf. Al{\s} sie einander sahn, wurden sie beide
bla"s. Rudolf, der bei Emma{\s} Tode sein Beileid nur durch seine
Visitenkarte bezeigt hatte, murmelte zun"achst einige Worte der
Entschuldigung, dann aber fa"ste er Mut und hatte sogar die
Dreistigkeit, -- e{\s} war ein hei"ser Augusttag -- Karl zu einem
Gla{\s} Bier in der n"achsten Kneipe einzuladen.

Er l"ummelte sich Karl gegen"uber auf der Tischplatte auf,
plauderte und schmauchte seine Zigarre. Karl verlor sich in
tausend Tr"aumen vor diesem Gesicht, da{\s} "`sie"' geliebt hatte.
E{\s} war ihm, al{\s} s"ahe er ein St"uck von ihr wieder. Da{\s}
war ihm selber sonderbar. Er h"atte der andre sein m"ogen.

Rudolf sprach unau{\s}gesetzt von landwirtschaftlichen Dingen, vom
Vieh, vom D"ungen und dergleichen. Wenn er einmal in seiner Rede
stockte, half er sich mit ein paar allgemeinen Reden{\s}arten. So
vermied er jedwede Anspielung auf da{\s} Einst. Karl h"orte ihm
gar nicht zu. Rudolf nahm da{\s} wahr; er ahnte, da"s hinter
diesem zuckenden Gesicht Erinnerungen heraufkamen. Karl{\s} Wangen
r"oteten sich mehr und mehr, seine Nasenfl"ugel bl"ahten sich,
seine Lippen bebten. Einen Augenblick lang sahen Karl{\s} Augen in
so d"usterem Groll auf Rudolf, da"s dieser erschrak und mitten im
Satz steckenblieb. Aber al{\s}bald erschien wieder die fr"uhere
Leben{\s}m"udigkeit auf Karl{\s} Gesicht.

"`Ich bin Ihnen nicht b"ose!"' sagte er.

Rudolf blieb stumm. Karl barg den Kopf zwischen seinen H"anden und
wiederholte mit erstickter Stimme im resignierten Tone namenloser
Schmerzen:

"`Nein, ich bin Ihnen nicht mehr b"ose!"'

Er f"ugte ein gro"se{\s} Wort hinzu, da{\s} einzige, da{\s} er je
in seinem Leben sprach:

"`Da{\s} Schicksal ist schuld!"'

Rudolf, der diese{\s} Schicksal gelenkt hatte, fand in{\s}geheim,
f"ur einen Mann in seiner Lage sei Bovary doch allzu gutm"utig,
eigentlich sogar komisch und ver"achtlich.

Am Tag darauf setzte Karl sich auf die Bank in der Laube. Die
Abendsonne leuchtete durch da{\s} Gitter, die Weinbl"atter
zeichneten ihren Schatten auf den Sand, der Ja{\s}min duftete
s"u"s, der Himmel war blau, Insekten summten um die bl"uhenden
Lilien. Karl atmete schwer; da{\s} Herz war ihm beklommen und
tieftraurig vor unsagbarer Liebe{\s}sehnsucht.

Um sieben Uhr kam Berta, die ihn den ganzen Nachmittag nicht
gesehen hatte, um ihn zum Essen zu holen.

Sein Kopf war gegen die Mauer gesunken. Die Augen waren ihm
zugefallen, sein Mund stand offen. In den H"anden hielt er eine
lange schwarze Haarlocke.

"`Papa, komm doch!"' rief die Kleine.

Sie glaubte, er wolle mit ihr spa"sen, und stie"s ihn sacht an. Da
fiel er zu Boden. Er war tot.

Sech{\s}unddrei"sig Stunden darnach eilte auf Veranlassung de{\s}
Apotheker{\s} Doktor Canivet herbei. Er "offnete die Leiche, fand
aber nicht{\s}.

Al{\s} aller Hau{\s}rat verkauft war, blieben zw"olf und
dreiviertel Franken "ubrig, die gerade au{\s}reichten, die Reise
der kleinen Berta Bovary zu ihrer Gro"smutter zu bestreiten. Die
gute alte Frau starb aber noch im selben Jahre, und da der Vater
Rouault gel"ahmt war, nahm sich eine Tante de{\s} Kinde{\s} an.
Sie ist arm und schickt Berta, damit sie sich da{\s} t"agliche
Brot verdient, in eine Baumwollspinnerei.

Seit Bovary{\s} Tode haben sich bereit{\s} drei "Arzte nacheinander
in Yonville niedergelassen, aber keiner hat sich dort halten
k"onnen. Homai{\s} hat sie alle au{\s} dem Feld geschlagen. Seine
Kurpfuscherei hat einen unheimlichen Umfang gewonnen. Die Beh"orde
duldet ihn, und die "offentliche Meinung empfiehlt ihn immer mehr.

K"urzlich hat er da{\s} Kreuz der Ehrenlegion erhalten.



\chapter{PROJECT GUTENBERG ``SMALL PRINT''}
\small \pagenumbering{gobble}
\begin{verbatim}


*** This file should be named 15711-t.txt or 15711-t.zip ***
This and all associated files of various formats will be found in:
        http://www.gutenberg.org/1/5/7/1/15711/

Produced by Gunter Hille, K.F. Greiner and the Online
Distributed Proofreading Team.


Updated editions will replace the previous one--the old editions
will be renamed.

Creating the works from public domain print editions means that no
one owns a United States copyright in these works, so the Foundation
(and you!) can copy and distribute it in the United States without
permission and without paying copyright royalties.  Special rules,
set forth in the General Terms of Use part of this license, apply to
copying and distributing Project Gutenberg-tm electronic works to
protect the PROJECT GUTENBERG-tm concept and trademark.  Project
Gutenberg is a registered trademark, and may not be used if you
charge for the eBooks, unless you receive specific permission.  If you
do not charge anything for copies of this eBook, complying with the
rules is very easy.  You may use this eBook for nearly any purpose
such as creation of derivative works, reports, performances and
research.  They may be modified and printed and given away--you may do
practically ANYTHING with public domain eBooks.  Redistribution is
subject to the trademark license, especially commercial
redistribution.



*** START: FULL LICENSE ***

THE FULL PROJECT GUTENBERG LICENSE
PLEASE READ THIS BEFORE YOU DISTRIBUTE OR USE THIS WORK

To protect the Project Gutenberg-tm mission of promoting the free
distribution of electronic works, by using or distributing this work
(or any other work associated in any way with the phrase "Project
Gutenberg"), you agree to comply with all the terms of the Full Project
Gutenberg-tm License (available with this file or online at
http://gutenberg.net/license).


Section 1.  General Terms of Use and Redistributing Project Gutenberg-tm
electronic works

1.A.  By reading or using any part of this Project Gutenberg-tm
electronic work, you indicate that you have read, understand, agree to
and accept all the terms of this license and intellectual property
(trademark/copyright) agreement.  If you do not agree to abide by all
the terms of this agreement, you must cease using and return or destroy
all copies of Project Gutenberg-tm electronic works in your possession.
If you paid a fee for obtaining a copy of or access to a Project
Gutenberg-tm electronic work and you do not agree to be bound by the
terms of this agreement, you may obtain a refund from the person or
entity to whom you paid the fee as set forth in paragraph 1.E.8.

1.B.  "Project Gutenberg" is a registered trademark.  It may only be
used on or associated in any way with an electronic work by people who
agree to be bound by the terms of this agreement.  There are a few
things that you can do with most Project Gutenberg-tm electronic works
even without complying with the full terms of this agreement.  See
paragraph 1.C below.  There are a lot of things you can do with Project
Gutenberg-tm electronic works if you follow the terms of this agreement
and help preserve free future access to Project Gutenberg-tm electronic
works.  See paragraph 1.E below.

1.C.  The Project Gutenberg Literary Archive Foundation ("the Foundation"
or PGLAF), owns a compilation copyright in the collection of Project
Gutenberg-tm electronic works.  Nearly all the individual works in the
collection are in the public domain in the United States.  If an
individual work is in the public domain in the United States and you are
located in the United States, we do not claim a right to prevent you from
copying, distributing, performing, displaying or creating derivative
works based on the work as long as all references to Project Gutenberg
are removed.  Of course, we hope that you will support the Project
Gutenberg-tm mission of promoting free access to electronic works by
freely sharing Project Gutenberg-tm works in compliance with the terms of
this agreement for keeping the Project Gutenberg-tm name associated with
the work.  You can easily comply with the terms of this agreement by
keeping this work in the same format with its attached full Project
Gutenberg-tm License when you share it without charge with others.

1.D.  The copyright laws of the place where you are located also govern
what you can do with this work.  Copyright laws in most countries are in
a constant state of change.  If you are outside the United States, check
the laws of your country in addition to the terms of this agreement
before downloading, copying, displaying, performing, distributing or
creating derivative works based on this work or any other Project
Gutenberg-tm work.  The Foundation makes no representations concerning
the copyright status of any work in any country outside the United
States.

1.E.  Unless you have removed all references to Project Gutenberg:

1.E.1.  The following sentence, with active links to, or other immediate
access to, the full Project Gutenberg-tm License must appear prominently
whenever any copy of a Project Gutenberg-tm work (any work on which the
phrase "Project Gutenberg" appears, or with which the phrase "Project
Gutenberg" is associated) is accessed, displayed, performed, viewed,
copied or distributed:

This eBook is for the use of anyone anywhere at no cost and with
almost no restrictions whatsoever.  You may copy it, give it away or
re-use it under the terms of the Project Gutenberg License included
with this eBook or online at www.gutenberg.net

1.E.2.  If an individual Project Gutenberg-tm electronic work is derived
from the public domain (does not contain a notice indicating that it is
posted with permission of the copyright holder), the work can be copied
and distributed to anyone in the United States without paying any fees
or charges.  If you are redistributing or providing access to a work
with the phrase "Project Gutenberg" associated with or appearing on the
work, you must comply either with the requirements of paragraphs 1.E.1
through 1.E.7 or obtain permission for the use of the work and the
Project Gutenberg-tm trademark as set forth in paragraphs 1.E.8 or
1.E.9.

1.E.3.  If an individual Project Gutenberg-tm electronic work is posted
with the permission of the copyright holder, your use and distribution
must comply with both paragraphs 1.E.1 through 1.E.7 and any additional
terms imposed by the copyright holder.  Additional terms will be linked
to the Project Gutenberg-tm License for all works posted with the
permission of the copyright holder found at the beginning of this work.

1.E.4.  Do not unlink or detach or remove the full Project Gutenberg-tm
License terms from this work, or any files containing a part of this
work or any other work associated with Project Gutenberg-tm.

1.E.5.  Do not copy, display, perform, distribute or redistribute this
electronic work, or any part of this electronic work, without
prominently displaying the sentence set forth in paragraph 1.E.1 with
active links or immediate access to the full terms of the Project
Gutenberg-tm License.

1.E.6.  You may convert to and distribute this work in any binary,
compressed, marked up, nonproprietary or proprietary form, including any
word processing or hypertext form.  However, if you provide access to or
distribute copies of a Project Gutenberg-tm work in a format other than
"Plain Vanilla ASCII" or other format used in the official version
posted on the official Project Gutenberg-tm web site (www.gutenberg.net),
you must, at no additional cost, fee or expense to the user, provide a
copy, a means of exporting a copy, or a means of obtaining a copy upon
request, of the work in its original "Plain Vanilla ASCII" or other
form.  Any alternate format must include the full Project Gutenberg-tm
License as specified in paragraph 1.E.1.

1.E.7.  Do not charge a fee for access to, viewing, displaying,
performing, copying or distributing any Project Gutenberg-tm works
unless you comply with paragraph 1.E.8 or 1.E.9.

1.E.8.  You may charge a reasonable fee for copies of or providing
access to or distributing Project Gutenberg-tm electronic works provided
that

- You pay a royalty fee of 20% of the gross profits you derive from
     the use of Project Gutenberg-tm works calculated using the method
     you already use to calculate your applicable taxes.  The fee is
     owed to the owner of the Project Gutenberg-tm trademark, but he
     has agreed to donate royalties under this paragraph to the
     Project Gutenberg Literary Archive Foundation.  Royalty payments
     must be paid within 60 days following each date on which you
     prepare (or are legally required to prepare) your periodic tax
     returns.  Royalty payments should be clearly marked as such and
     sent to the Project Gutenberg Literary Archive Foundation at the
     address specified in Section 4, "Information about donations to
     the Project Gutenberg Literary Archive Foundation."

- You provide a full refund of any money paid by a user who notifies
     you in writing (or by e-mail) within 30 days of receipt that s/he
     does not agree to the terms of the full Project Gutenberg-tm
     License.  You must require such a user to return or
     destroy all copies of the works possessed in a physical medium
     and discontinue all use of and all access to other copies of
     Project Gutenberg-tm works.

- You provide, in accordance with paragraph 1.F.3, a full refund of any
     money paid for a work or a replacement copy, if a defect in the
     electronic work is discovered and reported to you within 90 days
     of receipt of the work.

- You comply with all other terms of this agreement for free
     distribution of Project Gutenberg-tm works.

1.E.9.  If you wish to charge a fee or distribute a Project Gutenberg-tm
electronic work or group of works on different terms than are set
forth in this agreement, you must obtain permission in writing from
both the Project Gutenberg Literary Archive Foundation and Michael
Hart, the owner of the Project Gutenberg-tm trademark.  Contact the
Foundation as set forth in Section 3 below.

1.F.

1.F.1.  Project Gutenberg volunteers and employees expend considerable
effort to identify, do copyright research on, transcribe and proofread
public domain works in creating the Project Gutenberg-tm
collection.  Despite these efforts, Project Gutenberg-tm electronic
works, and the medium on which they may be stored, may contain
"Defects," such as, but not limited to, incomplete, inaccurate or
corrupt data, transcription errors, a copyright or other intellectual
property infringement, a defective or damaged disk or other medium, a
computer virus, or computer codes that damage or cannot be read by
your equipment.

1.F.2.  LIMITED WARRANTY, DISCLAIMER OF DAMAGES - Except for the "Right
of Replacement or Refund" described in paragraph 1.F.3, the Project
Gutenberg Literary Archive Foundation, the owner of the Project
Gutenberg-tm trademark, and any other party distributing a Project
Gutenberg-tm electronic work under this agreement, disclaim all
liability to you for damages, costs and expenses, including legal
fees.  YOU AGREE THAT YOU HAVE NO REMEDIES FOR NEGLIGENCE, STRICT
LIABILITY, BREACH OF WARRANTY OR BREACH OF CONTRACT EXCEPT THOSE
PROVIDED IN PARAGRAPH F3.  YOU AGREE THAT THE FOUNDATION, THE
TRADEMARK OWNER, AND ANY DISTRIBUTOR UNDER THIS AGREEMENT WILL NOT BE
LIABLE TO YOU FOR ACTUAL, DIRECT, INDIRECT, CONSEQUENTIAL, PUNITIVE OR
INCIDENTAL DAMAGES EVEN IF YOU GIVE NOTICE OF THE POSSIBILITY OF SUCH
DAMAGE.

1.F.3.  LIMITED RIGHT OF REPLACEMENT OR REFUND - If you discover a
defect in this electronic work within 90 days of receiving it, you can
receive a refund of the money (if any) you paid for it by sending a
written explanation to the person you received the work from.  If you
received the work on a physical medium, you must return the medium with
your written explanation.  The person or entity that provided you with
the defective work may elect to provide a replacement copy in lieu of a
refund.  If you received the work electronically, the person or entity
providing it to you may choose to give you a second opportunity to
receive the work electronically in lieu of a refund.  If the second copy
is also defective, you may demand a refund in writing without further
opportunities to fix the problem.

1.F.4.  Except for the limited right of replacement or refund set forth
in paragraph 1.F.3, this work is provided to you 'AS-IS', WITH NO OTHER
WARRANTIES OF ANY KIND, EXPRESS OR IMPLIED, INCLUDING BUT NOT LIMITED TO
WARRANTIES OF MERCHANTIBILITY OR FITNESS FOR ANY PURPOSE.

1.F.5.  Some states do not allow disclaimers of certain implied
warranties or the exclusion or limitation of certain types of damages.
If any disclaimer or limitation set forth in this agreement violates the
law of the state applicable to this agreement, the agreement shall be
interpreted to make the maximum disclaimer or limitation permitted by
the applicable state law.  The invalidity or unenforceability of any
provision of this agreement shall not void the remaining provisions.

1.F.6.  INDEMNITY - You agree to indemnify and hold the Foundation, the
trademark owner, any agent or employee of the Foundation, anyone
providing copies of Project Gutenberg-tm electronic works in accordance
with this agreement, and any volunteers associated with the production,
promotion and distribution of Project Gutenberg-tm electronic works,
harmless from all liability, costs and expenses, including legal fees,
that arise directly or indirectly from any of the following which you do
or cause to occur: (a) distribution of this or any Project Gutenberg-tm
work, (b) alteration, modification, or additions or deletions to any
Project Gutenberg-tm work, and (c) any Defect you cause.


Section  2.  Information about the Mission of Project Gutenberg-tm

Project Gutenberg-tm is synonymous with the free distribution of
electronic works in formats readable by the widest variety of computers
including obsolete, old, middle-aged and new computers.  It exists
because of the efforts of hundreds of volunteers and donations from
people in all walks of life.

Volunteers and financial support to provide volunteers with the
assistance they need, is critical to reaching Project Gutenberg-tm's
goals and ensuring that the Project Gutenberg-tm collection will
remain freely available for generations to come.  In 2001, the Project
Gutenberg Literary Archive Foundation was created to provide a secure
and permanent future for Project Gutenberg-tm and future generations.
To learn more about the Project Gutenberg Literary Archive Foundation
and how your efforts and donations can help, see Sections 3 and 4
and the Foundation web page at http://www.pglaf.org.


Section 3.  Information about the Project Gutenberg Literary Archive
Foundation

The Project Gutenberg Literary Archive Foundation is a non profit
501(c)(3) educational corporation organized under the laws of the
state of Mississippi and granted tax exempt status by the Internal
Revenue Service.  The Foundation's EIN or federal tax identification
number is 64-6221541.  Its 501(c)(3) letter is posted at
http://pglaf.org/fundraising.  Contributions to the Project Gutenberg
Literary Archive Foundation are tax deductible to the full extent
permitted by U.S. federal laws and your state's laws.

The Foundation's principal office is located at 4557 Melan Dr. S.
Fairbanks, AK, 99712., but its volunteers and employees are scattered
throughout numerous locations.  Its business office is located at
809 North 1500 West, Salt Lake City, UT 84116, (801) 596-1887, email
business@pglaf.org.  Email contact links and up to date contact
information can be found at the Foundation's web site and official
page at http://pglaf.org

For additional contact information:
     Dr. Gregory B. Newby
     Chief Executive and Director
     gbnewby@pglaf.org

Section 4.  Information about Donations to the Project Gutenberg
Literary Archive Foundation

Project Gutenberg-tm depends upon and cannot survive without wide
spread public support and donations to carry out its mission of
increasing the number of public domain and licensed works that can be
freely distributed in machine readable form accessible by the widest
array of equipment including outdated equipment.  Many small donations
($1 to $5,000) are particularly important to maintaining tax exempt
status with the IRS.

The Foundation is committed to complying with the laws regulating
charities and charitable donations in all 50 states of the United
States.  Compliance requirements are not uniform and it takes a
considerable effort, much paperwork and many fees to meet and keep up
with these requirements.  We do not solicit donations in locations
where we have not received written confirmation of compliance.  To
SEND DONATIONS or determine the status of compliance for any
particular state visit http://pglaf.org

While we cannot and do not solicit contributions from states where we
have not met the solicitation requirements, we know of no prohibition
against accepting unsolicited donations from donors in such states who
approach us with offers to donate.

International donations are gratefully accepted, but we cannot make
any statements concerning tax treatment of donations received from
outside the United States.  U.S. laws alone swamp our small staff.

Please check the Project Gutenberg Web pages for current donation
methods and addresses.  Donations are accepted in a number of other
ways including including checks, online payments and credit card
donations.  To donate, please visit: http://pglaf.org/donate


Section 5.  General Information About Project Gutenberg-tm electronic
works.

Professor Michael S. Hart is the originator of the Project Gutenberg-tm
concept of a library of electronic works that could be freely shared
with anyone.  For thirty years, he produced and distributed Project
Gutenberg-tm eBooks with only a loose network of volunteer support.

Project Gutenberg-tm eBooks are often created from several printed
editions, all of which are confirmed as Public Domain in the U.S.
unless a copyright notice is included.  Thus, we do not necessarily
keep eBooks in compliance with any particular paper edition.

Most people start at our Web site which has the main PG search facility:

     http://www.gutenberg.net

This Web site includes information about Project Gutenberg-tm,
including how to make donations to the Project Gutenberg Literary
Archive Foundation, how to help produce our new eBooks, and how to
subscribe to our email newsletter to hear about new eBooks.
\end{verbatim}

*** END: FULL LICENSE ***

\end{document}
